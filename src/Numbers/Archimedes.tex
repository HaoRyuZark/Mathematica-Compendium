\newpage
\section{The Archimedean Principle}

For any real number \( x \in \Reals \), there exists a natural number \( n \in \Naturals \) such 
that \( n > x \).
\vspace{\baselineskip}

In other words, no matter how large a real number you choose, there is always a natural number that 
is larger. Similarly, for any positive real number \( \epsilon > 0 \), there exists a natural number 
\( n \in \Naturals \) such that \( \frac{1}{n} < \epsilon \).
\vspace{\baselineskip}

\textbf{Proof:}

We prove the Archimedean Principle by contradiction.
\vspace{\baselineskip}

Assume that there exists some real number \( x \in \Reals \) such that \( n \leq x \) for all 
\( n \in \Naturals \). That is, \( x \) is an upper bound for the set \( \Naturals \subset \Reals\).
\vspace{\baselineskip}

Let \( S = \sup(\Naturals) \), the least upper bound of \( \Naturals \). Then 
\( S - 1 < \sup(\Naturals) \), so \( S - 1 \) is not an upper bound of \( \Naturals \). Hence, 
there exists \( n_0 \in \Naturals \) such that:

\[
	n_0 > S - 1 \Rightarrow n_0 + 1 > S
\]

But \( n_0 + 1 \in \Naturals \), which contradicts the assumption that \( S \) is an upper bound of \( \Naturals \). Therefore, our assumption must be false, and the theorem is proven.

\QED

\subsection{Equivalent Formulations}

The Archimedean Principle is often stated in different but equivalent ways:

\begin{itemize}
	\item For any \( \epsilon > 0 \), there exists \( n \in \Naturals \) such that \( \frac{1}{n} < \epsilon \).
	\item For any \( a, b \in \Reals \) with \( a > 0 \), there exists \( n \in \Naturals \) such that \( na > b \).
\end{itemize}

\subsection{Applications}

\begin{enumerate}
	\item \textbf{Density of Rational Numbers:} The Archimedean Principle helps in proving that between any two real numbers, there exists a rational number.

	\item \textbf{Limits and Infinitesimals:} It ensures that sequences like \( \left\{ \frac{1}{n} \right\} \) converge to 0, foundational in real analysis and calculus.

	\item \textbf{Bounding Functions:} It is used in analysis to show that functions do not grow faster than natural numbers in certain contexts.

	\item \textbf{Non-Existence of Infinitely Small Numbers:} The principle implies that real numbers do not contain infinitesimals 
	(nonzero numbers smaller than all \( \frac{1}{n} \)), distinguishing \( \Reals \) from non-standard number systems.
\end{enumerate}

