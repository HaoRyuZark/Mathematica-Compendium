\newpage
\section{Complex Numbers}

A complex number is a number of the form:

\[
	z = a + bi,
\]

where \( a, b \in \Reals \), and \emph{i} is the imaginary unit defined by \( i^2 = -1 \). 
The set of all complex numbers is denoted by \( \Complex \).

\subsection{The Complex Plane}

Complex numbers can be represented graphically in the \emph{complex plane}, where the horizontal axis 
represents the real part and the vertical axis the imaginary part.

\begin{center}
	\setlength{\unitlength}{0.8cm}
	\begin{picture}(6,6)
		\put(0,3){\vector(1,0){6}}
		\put(3,0){\vector(0,1){6}}
		\put(6.2,3){\makebox(0,0){Re}}
		\put(3,6.2){\makebox(0,0){Im}}
		\put(3,3){\circle*{0.15}}
		\put(4,4){\circle*{0.2}}
		\put(4.2,4.2){\makebox(0,0){$1+i$}}
		\put(3,3){\line(1,1){1}}
	\end{picture}
\end{center}

The point \( 1+i \) is located at (1,1), showing 1 unit on the real axis and 1 unit on the imaginary 
axis.

\subsection{Conjugate of a Complex Number}

The \emph{conjugate} of a complex number \( z = a + bi \) is denoted \( \overline{z} \) and is defined 
as:

\[
	\overline{z} = a - bi
\]

Geometrically, it reflects the point \emph{z} across the real axis in the complex plane. Conjugates are 
useful in division and in finding the modulus, since:

\[
	z \cdot \overline{z} = a^2 + b^2 = |z|^2
\]

\subsection{Operations in Cartesian Coordinates}

Let \( z_1 = a + bi \) and \( z_2 = c + di \) be two complex numbers.
\vspace{\baselineskip}

\emph{Addition:}

\[ 
	z_1 + z_2 = (a + c) + (b + d)i
\]

\emph{Multiplication:}
	      
\[
	z_1 \cdot z_2 = (ac - bd) + (ad + bc)i
\]

\emph{Quotient:}
	      
\[
	\frac{z_1}{z_2} = \frac{(a + bi)}{(c + di)} \frac{(c - di)}{(c - di)} = \frac{(a + bi)(c - di)}
	{c^2 + d^2} = \frac{(ac + bd) + (bc - ad)i}{c^2 + d^2}
\]

\subsection{Polar Coordinates}

A complex number can also be expressed in polar form as:

\[
	z = r(\cos \theta + i \sin \theta) = re^{i\theta},
\]

where:

\begin{align*}
	r      & = |z| = \sqrt{a^2 + b^2} \quad \text{(modulus)}                       \\
	\theta & = \arg(z) = \tan^{-1}\left(\frac{b}{a}\right) \quad \text{(argument)}
\end{align*}

The value of \( \theta \) depends on the quadrant where the complex number lies:

\begin{itemize}
	
	\item Quadrant I: \( a > 0, b > 0 \) - use \( \tan^{-1}(b/a) \)
	
	\item Quadrant II: \( a < 0, b > 0 \) — add \( \pi \) to \( \tan^{-1}(b/a) \)
	
	\item Quadrant III: \( a < 0, b < 0 \) — add \( \pi \) to \( \tan^{-1}(b/a) \)
	
	\item Quadrant IV: \( a > 0, b < 0 \) — use \( \tan^{-1}(b/a) \)

\end{itemize}

\subsection{Multiplication and Division in Polar Coordinates}

Given:

\[
	z_1 = r_1 e^{i\theta_1}, \quad z_2 = r_2 e^{i\theta_2},
\]

\emph{Multiplication:}

\[
	z_1 \cdot z_2 = r_1 r_2 e^{i(\theta_1 + \theta_2)}
\]

\emph{Division:}
	      
\[
	\frac{z_1}{z_2} = \frac{r_1}{r_2} e^{i(\theta_1 - \theta_2)}
\]

This polar form is especially useful in simplifying powers and roots of complex numbers using De 
Moivre’s Theorem.

\subsection{Exponentiation and Roots (De Moivre´s Theorem)}

Let \( z = r(\cos \theta + i \sin \theta) = re^{i\theta} \) be a complex number in polar form.

\subsubsection{Exponentiation}

To raise \emph{z} to the power \( n \in \Naturals \), we use De Moivre’s Theorem:

\[
	z^n = r^n (\cos(n\theta) + i \sin(n\theta)) = r^n e^{in\theta}
\]

\subsubsection{Roots of Complex Numbers}

To find the \emph{n}th roots of a complex number \( z = r e^{i\theta} \), we use the formula:

\[
	z^{1/n} = r^{1/n} \left( \cos\left( \frac{\theta + 2k\pi}{n} \right) + i \sin\left( \frac{\theta + 
	2k\pi}{n} \right) \right), \quad k = 0, 1, \ldots, n-1
\]

This yields \emph{n} distinct roots, each separated by an angle of \( \frac{2\pi}{n} \) in 
the complex plane.

\subsection{Example: Solve \texorpdfstring{\( z^4 = 1 + \sqrt{3}i \)}{}}

\textbf{Step 1: Convert RHS to polar form.}

Let \( w = 1 + \sqrt{3}i \). Real part: \( a = 1 \), Imaginary part: \( b = \sqrt{3} \)

\begin{align*}
	r      & = |w| = \sqrt{1^2 + {(\sqrt{3})}^2} = \sqrt{1 + 3} = 2                    \\
	\theta & = \arg(w) = \tan^{-1} \left( \frac{\sqrt{3}}{1} \right) = \frac{\pi}{3}
\end{align*}

So,

\[
	w = 2 \left( \cos\left( \frac{\pi}{3} \right) + i \sin\left( \frac{\pi}{3} \right) \right)
\]

\textbf{Step 2: Solve \( z^4 = w \to z = w^{1/4} \)}

Using the root formula:

\[
	z_k = 2^{1/4} \left( \cos\left( \frac{\pi + 2k\pi}{12} \right) + i \sin\left( \frac{\pi + 2k\pi}{12} 
	\right) \right), \quad k = 0, 1, 2, 3
\]

So the four roots are:

\begin{align*}
	z_0 & = 2^{1/4} \left( \cos\left( \frac{\pi}{12} \right) + i \sin\left( \frac{\pi}{12} \right) \right)                                                                                                    \\
	z_1 & = 2^{1/4} \left( \cos\left( \frac{5\pi}{12} \right) + i \sin\left( \frac{5\pi}{12} \right) \right)                                                                                                  \\
	z_2 & = 2^{1/4} \left( \cos\left( \frac{9\pi}{12} \right) + i \sin\left( \frac{9\pi}{12} \right) \right) = 2^{1/4} \left( \cos\left( \frac{3\pi}{4} \right) + i \sin\left( \frac{3\pi}{4} \right) \right) \\
	z_3 & = 2^{1/4} \left( \cos\left( \frac{13\pi}{12} \right) + i \sin\left( \frac{13\pi}{12} \right) \right)
\end{align*}

These represent the four complex 4th roots of \( 1 + \sqrt{3}i \), equally spaced around the circle of 
radius \( 2^{1/4} \) in the complex plane.

\subsection{Solving Equations with Complex Numbers}

Solving equations in \( \Complex \) can involve various forms. Here are the most common cases:

\subsubsection{Linear Equations:}

Solve for \emph{z} in \( az + b = 0 \), where \( a, b \in \Complex \), \( a \neq 0 \):
	
\[
	z = -\frac{b}{a}
\]

\subsubsection{Equations Involving the Conjugate:}

Solve for \emph{z} in equations like \( z + \overline{z} = 4 \).
Let \( z = x + iy \), then \( \overline{z} = x - iy \). So:
	      
\[
	z + \overline{z} = 2x \quad \to \quad x = 2 \quad \Rightarrow \quad z = 2 + iy
\]
	      
The imaginary part remains free unless further constraints are given.

\subsubsection{Modulus Equations:}

Solve \( |z| = r \). Let \( z = x + iy \), then:
	      
\[
	\sqrt{x^2 + y^2} = r \to x^2 + y^2 = r^2
\]

This is a circle of radius \( r \) centered at the origin in the complex plane.

\subsubsection{Equations Involving \texorpdfstring{\( z \cdot \overline{z} \)}{}}

Recall \( z \cdot \overline{z} = |z|^2 \). For example, solve:
	      
\[
	z \cdot \overline{z} = 9 \to |z| = 3
\]

Again, a circle in the complex plane of radius 3.

\subsubsection{Quadratic Equations:}

Complex roots occur naturally. For example:

\[
	z^2 + 1 = 0 \to z^2 = -1 \Rightarrow z = \pm i
\]

\subsubsection{General Polynomial Equations:}

 Use De Moivre’s Theorem or polar form. Example:
	      
\[
	z^n = w \to z_k = \sqrt[n]{|w|} \cdot e^{i\left( \frac{\arg(w) + 2k\pi}{n} \right)}, \quad k = 0, 1, 
	\dots, n-1
\]

\textbf{Example 1:}
\vspace{\baselineskip}

Solve:

\begin{align*}
	\left( \frac{2 + 3i}{1 + i} + \frac{4 + 5i}{2 - 2i}\right) \hat{z} &= \frac{i + 2}{i}\\
	\left( \frac{-3 -i}{4} \right) \hat{z} &= \frac{i + 2}{i}\\
	\hat{z} &= \frac{i + 2}{i} : \frac{-3 -i}{4}
\end{align*}

\textbf{Example 2:}
\vspace{\baselineskip}

Solve:

\[
	z - 3i + (2 -i)\hat{z} + 2 = 0
\]

In this case we let \( z = x + iy \) and \( \hat{z} = x - iy \).

\begin{align*}
	z &= 3i - (2 - i)\hat{z} - 2\\
	x + iy &= 3i - (2 - i)(x - iy) - 2\\
	x + iy &= 3i - [2x - 2yi -xi +yi^2] - 2\\
	x + yi &= 3i - 2 + 2x + 2yi + xi + y\\
	x + yi &= (y - 2 -2x) + i(3 + 2y + x)\\
	x &= y - 2 -2x\ y = 3 + 2y + x\\
	x &= \frac{y - 2}{3}\ \ y = 3 + 2y + \frac{y-2}{3} = -7\\
	x &= \frac{-7 -2}{3} = -3\\
	z &= -3 -7i\ \ \hat{z} = -3 + 7i\\
\end{align*}

\subsection{The Complex Logarithm}

The logarithm of a complex number is multivalued due to the periodic nature of the complex exponential.

Let \( z = re^{i\theta} \) with \( r > 0 \), \( \theta \in \Reals \). Then:

\[
	\log z = \ln r + i(\theta + 2\pi k), \quad k \in \Integers
\]

Here:

\begin{itemize}
	
	\item \( \ln r \) is the natural (real) logarithm of the modulus.
	
	\item \( \theta \) is the principal argument \( \arg(z) \in (-\pi, \pi] \).
	
	\item The term \( 2\pi k \) accounts for the infinitely many branches of the logarithm in 
		  \( \Complex \).

		\end{itemize}

\subsubsection*{Principal Value:}

The principal value of the complex logarithm is often written:

\[
	\mathrm{Log}\,z = \ln |z| + i\,\mathrm{Arg}(z), \quad \text{where } \mathrm{Arg}(z) \in (-\pi, \pi]
\]

\textbf{Example:}
\vspace{\baselineskip}

Let \( z = -1 \). Then:

\[
	|z| = 1, \quad \arg(z) = \pi, \quad \to \log(-1) = i(\pi + 2\pi k), \quad k \in \Integers
\]

\[
	\to \mathrm{Log}(-1) = i\pi
\]

The multivalued nature of \( \log z \) is crucial in advanced complex analysis, especially in defining 
analytic continuations and branch cuts.

\subsection{Complex Exponents}

\(a^x \approx 1 {\left(a + \alpha \frac{x}{N} \right)}^N \to e^z := \lim_{N \rightarrow \inf} 
{\left( 1 + \frac{z}{N}\right)}^N \) The process above is called linearization of the exponential 
function by zooming  \(\alpha \approx \frac{dy}{d}\)
\vspace{\baselineskip}

For  \(e^{ci} = \cos{\theta} + i\sin{\theta}\) every exponentiation of a complex number is a rotation 
in the complex plane.

\[
	e^{ic} = \lim_{N \to \inf} {\left( 1 + \frac{ic}{N}\right)}^N
\]

Now imagine that in a sector of a circumference you put triangles one above the other with base of 
length one and a height of \(\frac{C}{N}\) and 
an angle of \(\delta\)

\[
	\tan{\delta} \approx \delta \text{ for } \delta \ll 1
\]

\[
	1 + \frac{ci}{N} = 1 \angle \frac{c}{N}, N \gg 1
\]

\[
	e^{ic} = \lim_{N \to \inf} {\left( 1 + \frac{c}{N}\right)}^N \to e^{ci} = 1 \angle c = \cos{c} + 
	i\sin{c}
\]

\subsection{Euler's Formula Proof}

We know that

\[
	e^{i\pi} = -1 \text{ and } e^{i\theta} = |r|(\cos{\theta} + i\sin{\theta})
\]

\[
	e^{z} = \lim_{n \to \infty} {\left( 1 + \frac{z}{n}\right)}^n \implies\ e^{i\pi} = 
	\lim_{n \to \infty} {\left( 1 + \frac{i\pi}{n}\right)}^n = -1
\]

\[
	\implies \lim_{n \to \infty} |r_n| = 1\ \lim_{n \to \infty} \theta = 0+
\]
 Now we can demonstrate the formula.

\[
	|r_n| = \left( 1 + \left|\frac{z}{n}\right|^n \right) \implies 
	\left( \sqrt{1 + \frac{\pi^2}{n^2}}\right)
\]

\[
	\theta = \sum_{k = 1}^{n} n \arctan \frac{\pi}{2} = n \arctan \frac{\pi}{n}
\]

\[
	\lim_{n \to \infty} {\left( \sqrt{1 + \frac{\pi^2}{n^2}}\right)}^n = \lim_{n \to \infty} 
	{\left( 1 + \frac{\pi^2}{2n} \right) }^{\frac{n}{2}}
	= \lim_{n \to \infty} e^{\ln\left(1 +\frac{\pi^2}{2}\right) \frac{n}{2}} = e^0 = 1
\]

\[
	\lim_{n \to \infty} \theta = \lim_{n \to \infty} n \arctan \frac{\pi}{n} = \lim_{n \to \infty} 
	n^{-1} \arctan\frac{\pi}{n} = 0
\]

Thus, for \(e^{i\pi} = 1\) for \(x = \pi \forall x \in \lim r_n (x) = 1\) and \( \lim \theta (x) = x\)

\QED
