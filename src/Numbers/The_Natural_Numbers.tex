\newpage
\section{The Natural Numbers}

In this section we will take a look at the natural numbers, which are the numbers we use for counting. The natural numbers are defined as follows:
This not going not be a deep dive just a look at the axioms and the basic construction of the natural numbers. The natural numbers are defined as follows:
\vspace{\baselineskip}

We will now define the set of natural numbers, \( \mathbb{N} \), via the following 9 axioms. These axioms are known as the \textbf{Peano Axioms}. The first 4 axioms define equality on the set \( \mathbb{N} \).

\begin{enumerate}[label=\Roman*.]
	\item For every \( x \in \mathbb{N} \), we have \( x = x \).\hfill (Reflexivity)
	\item For every \( x, y \in \mathbb{N} \), if \( x = y \) then \( y = x \).\hfill (Symmetry)
	\item For every \( x, y, z \in \mathbb{N} \), if \( x = y \) and \( y = z \) then \( x = z \).\hfill (Transitivity)
	\item For all \( x, y \), if \( x \in \mathbb{N} \) and \( x = y \), then \( y \in \mathbb{N} \).\hfill (Closure of Equality)
\end{enumerate}

The remaining 5 axioms define the structure of \( \mathbb{N} \):

\begin{enumerate}[label=\Roman*.]
	\setcounter{enumi}{4}
	\item \( 1 \in \mathbb{N} \)
	\item If \( x \in \mathbb{N} \), then the successor \( S(x) \in \mathbb{N} \).
	\item There is no \( x \in \mathbb{N} \) such that \( S(x) = 1 \).
	\item For all \( x, y \in \mathbb{N} \), if \( S(x) = S(y) \), then \( x = y \).
	\item Let \( P(x) \) be a statement about the natural number \( x \). If:
		\begin{itemize}
			\item \( P(1) \) is true, and
			\item for all \( n \in \mathbb{N} \), if \( P(n) \) is true, then \( P(S(n)) \) is also true,
		\end{itemize}
		then \( P(x) \) is true for all \( x \in \mathbb{N} \).\hfill (Mathematical Induction)
\end{enumerate}

As shorthand, we denote:
\[
	S(1) = 2, \quad S(S(1)) = 3, \quad S(S(S(1))) = 4, \quad \text{and so on.}
\]

\subsection{Order in Fields}

An \emph{ordered field} is a field \( F \) equipped with a total order \( < \) that is compatible with the field operations \( + \) and \( \cdot \). That is, the order satisfies both algebraic and ordering properties.

\subsubsection{Order Axioms for Fields}

A field \( F \) is called an \emph{ordered field} if it satisfies the following properties for all \( a, b, c \in F \):

\begin{enumerate}[label=\Roman*.]
    \item \emph{Trichotomy:} Exactly one of the following holds:
    \[
    a < b, \quad a = b, \quad a > b
    \]

    \item \emph{Transitivity:} If \( a < b \) and \( b < c \), then \( a < c \).

    \item \emph{Additive Compatibility:} If \( a < b \), then \( a + c < b + c \).

    \item \emph{Multiplicative Compatibility:} If \( 0 < a \) and \( 0 < b \), then \( 0 < ab \).
\end{enumerate}

These properties ensure that arithmetic operations respect the order structure.

\subsubsection{Consequences of the Order Axioms}

\begin{itemize}[label=\(-\)]
    \item \( a < b \Rightarrow -b < -a \)
    \item \( a < b \land c < 0 \Rightarrow ac > bc \)
    \item Squares are always non-negative: \( a^2 \ge 0 \)
    \item The order is total: any two elements are comparable
\end{itemize}

\subsubsection{Examples of Ordered Fields}

\begin{itemize}[label=\(-\)]
    \item \( \mathbb{Q} \): Rational numbers with the usual order
    \item \( \mathbb{R} \): Real numbers with the usual order
    \item \( \mathbb{C} \): Complex numbers are \textbf{not} an ordered field, since \( i^2 = -1 < 0 \) would violate positivity of squares
\end{itemize}


\subsection{Propositions and Proofs}

\subsubsection{Proposition 1: \texorpdfstring{\(n \ne m \implies S(n) \ne S(m)\)}{n!= m implies S (n)!=S (m)}}

\textbf{Proof:} 

We will prove this by contradiction. Assume \( n \ne m \) and \( S(n) = S(m) \). By Axiom 8, we have \( n = m \), which is a contradiction. Therefore, \( S(n) \ne S(m) \).

\subsubsection{Proposition 2: \texorpdfstring{\text{For any} \(n \in \mathbb{N},\ n \ne S(n)\)}{For any n in N, n!= S (n)}}

\textbf{Proof:} 

\(M = \{n \in \mathbb{N} \| n \ne S(n) \}\ \)
By \(1 \ne S(n)\ \) for any \(n \in \mathbb{N}\), this implies that 1 is part of the set \(M\). This implies that \(S(n) \ne S(S(n)) \implies S(n) \in M\ \) By Axiom 9 \(M = \mathbb{N}\)

\subsubsection{Proposition 3: \texorpdfstring{\(n \ne 1\ \exists m \in \mathbb{N} \mid n = S(m)\)}{n!= 1, exists m in N | n = S (m)}}

\textbf{Proof:}
\[
	M = \{1\} \cup\{n \in \mathbb{N} | \text{Proposition 3 is true}\}
\]
We know that 1 ins in the set \(M\). An by proposition 1 we know that
\[
	S(n) = S(S(m)) \implies S(n) \in M
\]
And by Axiom 9 we know that \(M = \mathbb{N}\)

\subsection{Definition of Addition in \texorpdfstring{\(\mathbb{N}\)}{}}

For any pair n,m \(\in \mathbb{N}\) there is a unique way to define
\[
	Add(n , m) = n + m
\]
\begin{enumerate}
	\item \textbf{Base Case:} \(n + 1 = S(n)\)
	\item \textbf{Inductive Step:} \(n + S(m) = S(n + m) \iff S(n + m)\)
\end{enumerate}

\textbf{Uniqueness:} 
\vspace{\baselineskip}

Suppose: \textit{A} \& \textit{B} satisfy our conditions. Fix n and then let \(M = \{ m \in \mathbb{N} | A(n,m) = B(n, m)\}\). Then
\[
	A(n,1) = S(n) = B(n,1) \implies 1 \in M
\]
\[
	m \in M \implies A(n, m) = B(n, m) \implies A(n, S(m)) = S(A(n, m)) = S(B(n, m)) = B(n, S(m))
\]
\[
	\implies A(n , S(m)) = B(n, S(m))
\]

 by Axiom 9 we know that \(M = \mathbb{N}\) and A = B

\textbf{Construction:} For \(n = 1\) Define \(A(n, m) = S(m)\)
\begin{enumerate}
	\item \(A(n, 1) = S(n) = S(1)\)
	\item \(A(n, S(m)) = S(A(n, m)) = S(S(m))\)
\end{enumerate}

 Define: \(A(S(n), S(m)) = S(A(n, m))\)

\begin{enumerate}
	\item \(A(S(n), 1) = S(A(n, 1)) = S(S(n))\)
	\item \(A(S(n), S(m)) = S(A(n, m)) = S(S(m))\)
\end{enumerate}

\textbf{Commutativity of Addition:}
\vspace{\baselineskip}

The proposition says \(n + m = m + n\) for any \(n, m \in \mathbb{N}\). We will prove this by induction on \(m\).
 Fix n and consider \(M = \{ n \in \mathbb{N} | A(n, m) = A(n, m)\}\)

 Now recall that \(A(n, 1) = S(n)\)\\
For \(n = 1\):
\[
	A(n, 1) = S(1) = A(1, n) \implies 1 \in \mathbb{N}
\]
also \(A(n , k) = 1 + k \implies 1 + m = S(m) \implies 1 + m = m + 1 \implies 1 \in \mathbb{N}\)

 Suppose: \(n \in \mathbb{N} \implies n + m = m + n\) or \(A(n, m) = A(m, n)\)

 By construction \(A(S(n), m) = S(A(n, m))\) and by definition
\(A(S(n), m) = S(A(n, m)) = A(m, S(n)) = S(n) + m =  m + S(n) \implies S(n) \in M \implies \text{by induction } M = \mathbb{N}\).
