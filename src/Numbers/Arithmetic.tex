\section{Fundamental Theorem of Arithmetic}

\begin{itemize}
	\item The Fundamental Theorem of Arithmetic says that every integer greater than 1 can be factored uniquely into a product of primes.
	\item Euclid’s lemma says that if a prime divides a product of two numbers, it must divide at least one of the numbers.
	\item The least common multiple $[a, b]$ of nonzero integers $a$ and $b$ is the smallest positive integer divisible by both $a$ and $b$.
\end{itemize}

\textbf{Fundamental Theorem of Arithmetic:}
Every integer greater than 1 can be written in the form
\[
	p_1^{n_1}p_2^{n_2} \cdots p_k^{n_k}
\]
where $n_i \geq 0$ and the $p_i$ are distinct primes. The factorization is unique, except possibly for the order of the factors.

\textbf{Example.}
\[
	4312 = 2 \cdot 2156 = 2 \cdot 2 \cdot 1078 = 2 \cdot 2 \cdot 2 \cdot 539 = 2 \cdot 2 \cdot 2 \cdot 7 \cdot 77 = 2 \cdot 2 \cdot 2 \cdot 7 \cdot 7 \cdot 11
\]
That is,
\[
	4312 = 2^3 \cdot 7^2 \cdot 11
\]

\subsection{Lemmas}

\textbf{Lemma.} If $m \mid pq$ and $\gcd(m, p) = 1$, then $m \mid q$.

\textbf{Proof.} Write $1 = \gcd(m, p) = am + bp$ for some $a, b \in \mathbb{Z}$.
Then
\[
	q = amq + bpq
\]
Since $m \mid amq$ and $m \mid bpq$ (because $m \mid pq$), we conclude $m \mid q$.

\bigskip

\textbf{Lemma.} If $p$ is prime and $p \mid a_1a_2 \cdots a_n$, then $p \mid a_i$ for some $i$.

\textbf{Proof.} (Case $n=2$): Suppose $p \mid a_1a_2$, and $p \nmid a_1$.
Then $\gcd(p, a_1) = 1$, and by the previous lemma, $p \mid a_2$.

For general $n > 2$: Assume the result is true for $n-1$. Suppose $p \mid a_1a_2 \cdots a_n$.
Group as $(a_1a_2 \cdots a_{n-1})a_n$.

By the $n=2$ case, either $p \mid a_n$ or $p \mid a_1a_2 \cdots a_{n-1}$, and by induction, $p \mid a_i$ for some $i$.

\subsection{Proof of the Fundamental Theorem}

\textbf{Existence:}
Use induction on $n > 1$.
Base case: $n = 2$ is prime.

Inductive step: If $n$ is prime, done. Otherwise $n = ab$, with $1 < a, b < n$.
By induction, both $a$ and $b$ factor into primes, so $n$ does too.

\textbf{Uniqueness:}
Suppose:
\[
	p_1^{m_1} \cdots p_j^{m_j} = q_1^{n_1} \cdots q_k^{n_k}
\]
with all $p_i$ and $q_i$ distinct primes.

Since $p_1$ divides the LHS, it divides the RHS. So $p_1 \mid q_i^{n_i}$ for some $i$, hence $p_1 = q_i$.
Reorder so $p_1 = q_1$. Then:

If $m_1 > n_1$, divide both sides by $q_1^{n_1}$:
\[
	p_1^{m_1-n_1} \cdots p_j^{m_j} = q_2^{n_2} \cdots q_k^{n_k}
\]
But then $p_1$ divides LHS but not RHS, contradiction. So $m_1 = n_1$. Cancel and repeat.

Eventually, all $p_i$ match with some $q_i$, and the exponents are equal. So the factorizations are the same up to order.

\subsection{Least Common Multiple}

The least common multiple of $a$ and $b$, denoted $[a, b]$, is the smallest positive integer divisible by both.

\textbf{Example:}
\[
	[6, 4] = 12, \quad [33, 15] = 165
\]

\textbf{Fact:}
\[
	[a, b] \cdot \gcd(a, b) = ab
\]

Let:
\[
	a = p_1 \cdots p_lq_1 \cdots q_m, \quad b = q_1 \cdots q_mr_1 \cdots r_n
\]

Then:
\begin{align*}
	\gcd(a, b) & = q_1 \cdots q_m                                 \\
	[a, b]     & = p_1 \cdots p_lq_1 \cdots q_mr_1 \cdots r_n     \\
	ab         & = p_1 \cdots p_lq_1^2 \cdots q_m^2r_1 \cdots r_n
\end{align*}
So:
\[
	[a, b] \cdot \gcd(a, b) = ab
\]

\textbf{Example:}
\[
	\gcd(36, 90) = 18, \quad [36, 90] = 180, \quad 36 \cdot 90 = 32400 = 18 \cdot 180
\]

\newpage