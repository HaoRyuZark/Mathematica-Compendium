\newpage
\section{Geometry of Linear Maps}

In this section we are going to take a look at some special type of linear maps that are called \emph{orthogonal} 
and at the term \emph{isometry}. It will be useful to remember some concepts about the dot product such as 
\(\langle x, x \rangle = x^T \cdot x = \|x\|^2\).

\subsection{Isometry}

Given two vector spaces \((V, \|\centerdot\|_V)\) and \((W, \|\centerdot\|_W)\). Then a map 
\(f: V \to W\) is called \emph{Isometric} if and only if \(\|x - y\|_V = \|f(x) - f(y)\|_W \).
\vspace{\baselineskip} 

It also has the following traits:
\begin{enumerate}
	\item \emph{Length-Preserving}: \(\|x\|_V = \|y\|_W \quad \forall x \in V\).
	\item \emph{Angle-Preserving}: \(\angle (x,y) = \angle (f(x),f(y)) \quad \forall x,y \in V\)
	\item \emph{Linear}: If \(f(x + \lambda y) = f(x) + \lambda f(y)\)
\end{enumerate}

\subsection{The Rotation Matrix}

This matrix is a common example of an isometric linear map, because of the way it transforms the space.
\vspace{\baselineskip}

For the \(2D\) we have 

\[
\begin{pmatrix}
    \cos(\phi) & -\sin(\phi)\\
    \sin(\phi) & \cos(\phi)
\end{pmatrix}
\]

The derivation comes from the addition theorems for \(\sin(\phi) \) and \(\cos(\phi)\).
\vspace{\baselineskip}

Given two vectors \(\vec{v}\) and \(\vec{g}\) whose angles are \(\alpha\) and \(\theta = \alpha + \beta\). We 
can rotate the vector \(\vec{v}\) by some \(\beta\) to transform into \(\vec{g}\).

\begin{align*}
    x &= r \cos(\alpha)\\
    y &= r \sin(\alpha)
\end{align*}

Then 

\begin{align*}
    x' &= r' \cos(\alpha + \beta)\\
    y' &= r' \sin(\alpha + \beta)
\end{align*}

And therefore

\begin{align*}
    x' &= r' \cos(\alpha)\cos(\beta) - r'\sin(\alpha)\sin(\beta)\\
    y' &= r' \cos(\alpha)\sin(\beta) + r'\cos(\beta)\sin(\alpha)
\end{align*}

Because, a rotation does not change the magnitude we can say that \(r = r'\). Thus 

\begin{align*}
    x' &= r \cos(\alpha)\cos(\beta) - r\sin(\alpha)\sin(\beta)\\
    y' &= r \cos(\alpha)\sin(\beta) + r\cos(\beta)\sin(\alpha)
\end{align*}

Now notice that we have our \(x\) and \(y\) inside the expression

\begin{align*}
    x' &= x\cos(\beta) - y\sin(\beta)\\
    y' &= x\sin(\beta) + y\cos(\beta)
\end{align*}

which gives us 

\[
\begin{pmatrix}
    \cos(\phi) & -\sin(\phi)\\
    \sin(\phi) & \cos(\phi)
\end{pmatrix}
\begin{pmatrix}
    x\\
    y
\end{pmatrix}
\]

\QED

\subsubsection{Other rotation matrices}

\[
R_x (\phi) =
\begin{pmatrix}
    1 & 0 & 0 \\
    0 & \cos(\phi) & -\sin(\phi) \\
    0 & \sin(\phi) & \cos(\phi)
\end{pmatrix}
\]

\[
R_y (\phi) =
\begin{pmatrix}
    1 & 0 & \sin(\phi) \\
    0 & 1 & 0 \\
    -\sin(\phi) & 0 & \cos(\phi)
\end{pmatrix}
\]

\[
R_z (\phi) =
\begin{pmatrix}
    \cos(\phi) & -\sin(\phi) & 0 \\
    \sin(\phi) & \cos(\phi) & 0 \\
    0 & 0 & 1
\end{pmatrix}
\]


\subsection{Orthogonal Matrices}

A matrix \(A \in \mathbb{R}^{n \times n}\) is \emph{orthogonal} if its columns vectors form an orthonormal 
basis.
\vspace{\baselineskip}

The set of all \emph{orthogonal} matrices is \(O(n)\).

\subsubsection{Properties of orthogonal matrices}

\begin{enumerate}
    \item \(A \in O(n)\)
    \item If \(A\) has an inverse then \(A^{-1} = A^T\)
    \item \(A^T \in O(n)\)
\end{enumerate}

\textbf{Proof 1-2:}

\begin{align*}
    A^T = A^{-1} &\iff A^T A = E\\
                 &\iff (A^T A)_{ij} = \langle a_i, a_j\rangle = \delta_{ij} 1 \le i,j \le n\\
                 &\iff a_i \bot a_j, i \ne j, \text{ and } \|a_i\| = 1, 1 \le i \le n\\
                 &\iff A \in O(n)
\end{align*}

\textbf{Proof 1-3:}

\begin{align*}
    A \in O(n) &\iff \\
               &\iff A^{-1} = A^T\\
               &\iff (A^T)^{-1} = (A^T)^T \\
               &\iff (A^{1})^{T} = A\\
               &\iff A^T \in O(n)
\end{align*}

\textbf{Example of a proof}
\vspace{\baselineskip}

Prove that \(H_n\) is \emph{orthogonal} for all \(\vec{u}\in \mathbb{R}^n \backslash \{0\} \).

\[
H_n = I_n - 2 \frac{uu^T}{u^Tu}
\]

\(I_n\) is the identity matrix.
\vspace{\baselineskip}

\textbf{Proof:}

\begin{align*}
(H_n)^2 &= I \\
&= \left(I_n - 2 \frac{uu^T}{u^Tu}\right)^2\\
&= I_n^2 - 4\frac{uu^T}{u^Tu} + 4\left(\frac{uu^T}{u^Tu}\right)^2\\
&= I_n -4\frac{uu^T}{u^Tu} +4\frac{uu^T uu^T}{(u^Tu)^2}\\ 
&= I_n -4\frac{uu^T}{u^Tu} +4\frac{u(u^T u)u^T}{(u^Tu)^2}\\ 
&= I_n -4\frac{uu^T}{u^Tu} +4\frac{uu^T}{u^Tu}\\ 
&= I_n
\end{align*}

\QED

\subsection{The Determinants of Orthogonal Matrices}

The \emph{determinant} of this kind of matrices can only be either positioning 1 or negative 1.
\vspace{\baselineskip}

\textbf{Proof}:

We know that \(\det(I) = I \), therefore

\begin{align*}
    \det(I) &= \det(A A^{-1}) \\
            &= \det(A A^T) \\
            &= \det(A) \det(A^T) \\
            &= (\det(A))^2 \\
            &= \|\det (A) \| = 1
\end{align*}

\QED

\subsection{The Orthogonal Group}

\(O(n)\) with the operation matrix multiplication forms the \emph{orthogonal group}.
\vspace{\baselineskip}

\subsection{Dot Product}

Given \(f: \mathbb{R}^n \to \mathbb{R}^n\) then this two statements are equivalent.

\[
\langle f(x), f(y) \rangle = \langle x, y \rangle \quad \forall x,y \in \mathbb{R}^n
\]

and 
\vspace{\baselineskip}

\(f\) preserves the angles and length.
\vspace{\baselineskip}

\(f\) is also called \emph{orthogonal}.
\vspace{\baselineskip}

\textbf{Proof \(\Rightarrow\):}
 
\[
\|x\|^2 = \langle x, x \rangle = \langle f(x), f(x) \rangle = \|f(x)\|^2
\]

This implies the length preservation. The isometry gives us:

\begin{align*}
\|f(x) - f(y)\|^2 
&= \langle f(x) - f(y), f(x) - f(y) \rangle \\
&= \langle f(x), f(x) \rangle - 2\langle f(x), f(y) \rangle + \langle f(y), f(y) \rangle \\
&= \langle x, x \rangle - 2\langle x, y \rangle + \langle y, y \rangle \\
&= \langle x - y, x - y \rangle = \|x - y\|^2
\end{align*}

Then 
\begin{align*}
\angle(f(x), f(y)) 
&= \arccos\left( \frac{\langle f(x), f(y) \rangle}{\|f(x)\| \cdot \|f(y)\|} \right) \\
&= \arccos\left( \frac{\langle x, y \rangle}{\|x\| \cdot \|y\|} \right) 
= \angle(x, y)
\end{align*}

\QED
\vspace{\baselineskip}

\textbf{Proof \(\Leftarrow\):} 

With \(\alpha = \angle(x, y) = \angle(f(x), f(y))\) gilt:

\[
\langle f(x), f(y) \rangle = \cos(\alpha) \cdot \|f(x)\| \cdot \|f(y)\| = \cos(\alpha) \cdot \|x\| \cdot \|y\| = \langle x, y \rangle
\]

\QED

\subsection{Theorem of the Matrix Representation}

The mapping \( f : \mathbb{R}^n \to \mathbb{R}^n \) is orthogonal if and only if  
the associated matrix representation \( Q \) (with respect to the standard basis) is an orthogonal matrix, i.e., \( Q \in O(n) \).
\vspace{\baselineskip}

\textbf{Proof \(\Rightarrow\):} 

\( f \) is linear, and thus has a matrix representation \( Q \). It follows that

\[
\langle Qx, Qy \rangle = \langle Q^\top Qx, y \rangle = \langle x, y \rangle \quad \forall x, y \in \mathbb{R}^n,
\]

thus,

\[
\langle Q^\top Qx - x, y \rangle = 0 \quad \forall x, y.
\]

Choosing \( y = Q^\top Qx - x \), we get

\[
\langle Q^\top Qx - x, Q^\top Qx - x \rangle = 0 \quad \forall x,
\]

and hence \( (Q^\top Q - I)x = 0 \quad \forall x \).  
It follows that \( Q^\top Q = I \), and by Theorem 7.3, \( Q \in O(n) \).

\QED
\vspace{\baselineskip}

\textbf{Proof \(\Leftarrow\):} 

Follows by the same calculation as above.

\QED

\subsection{Change of Basis}

Let \( f \) be orthogonal. Then the matrix representation of \( f \)  
with respect to any orthonormal basis is orthogonal.
\vspace{\baselineskip}

\textbf{Proof:} 

Let \( E \) be the standard basis and \( B \) an orthonormal basis. Let  
\( Q = M_E^E(f) \) and \( S = T_E^B \) be the change-of-basis matrix from \( B \) to \( E \).  
Since the columns of \( S \) are the basis vectors of \( B \), we have \( S \in O(n) \).  

\[
M_B^B(f) = S^{-1} Q S \in O(n),
\]

due to the group property of \( O(n) \).

\QED

