\section{Linear Maps}
A linear map \( f: V \to W \) is a function that satisfies the following properties:

\begin{itemize}[label=\(-\)]
    \item \( f(v_1 + v_2) = f(v_1) + f(v_2) \) for all \( v_1, v_2 \in V \).
    \item \( f(\lambda v) = \lambda f(v) \) for all \( v \in V \) and \( \lambda \in \mathbb{R} \).
    \item \( f(0) = 0 \).
\end{itemize} 

\noindent This kind of function is also called \textbf{Homomorphism}.

\noindent If \(V = W\) is it called \textbf{Endomorphism}.

\subsection{Types of Linear Maps}
\begin{itemize}[label=\(-\)]
    \item \textbf{Injective or Monomorphism} (One-to-One): A linear map \( f: V \to W \) is injective if \( f(v_1) = f(v_2) \) implies \( v_1 = v_2 \).
    \item \textbf{Surjective or Epimorphism} (Onto): A linear map \( f: V \to W \) is surjective if for every \( w \in W \), there exists a \( v \in V \) such that \( f(v) = w \).
    \item \textbf{Bijective or Isomorphism}: A linear map \( f: V \to W \) is bijective if it is both injective and surjective.
\end{itemize}

\noindent\textbf{Note:} Two vector spcaces are called 
\textbf{isomorphic} if there exists a bijective linear map between them. In this case, we can say that the two vector spaces are \textbf{isomorphic} and we write \( V \cong W \).

\subsection{Properties of Linear Maps}
\begin{itemize}[label=\(-\)]
    \item The composition of two linear maps is a linear map.
    \item The inverse of a bijective linear map is also a linear map.
    \item The zero map \( f: V \to W \) defined by \( f(v) = 0 \) 
    for all \( v \in V \) is a linear map.
    \item The identity map \( \text{id}_V: V \to V \) defined by \( \text{id}_V(v) = v \) for all \( v \in V \) is a linear map.
    \item The sum of two linear maps \( f: V \to W \) 
    and \( g: V \to W \) is a linear map defined by \( (f + g)(v) = f(v) + g(v) \).
    \item The scalar multiplication of a linear map \( f: V \to W \) by a scalar \( c \) is a linear map defined by \( (cf)(v) = c(f(v)) \).
    \item The composition of linear maps is associative, i.e., \( (f \circ g) \circ h = f \circ (g \circ h) \).
    \item The composition of linear maps is distributive over addition, i.e., \( f \circ (g + h) = f \circ g + f \circ h \).
    \item The composition of linear maps is compatible with scalar multiplication, i.e., \( (cf) \circ g = c(f \circ g) \).
    \item Linear Maps compose a vector space 
    over the field of scalars.
    \[
    (Hom(V,W), +, \cdot )
    \]
\end{itemize}

\subsection{The Kernel of a Linear Map}
The kernel of a linear map \( f: V \to W \) is the set of all vectors in \( V \) that are mapped to the zero vector in \( W \):
    \[
    \ker(f) = \{ v \in V \mid f(v) = 0 \}.
    \]

\subsection{The Image of a Linear Map}
The image of a linear map \( f: V \to W \) is the set of all vectors in \( W \) that can be expressed as \( f(v) \) for some \( v \in V \):
    \[
    \text{Im}(f) = \{ w \in W \mid w = f(v) \text{ for some } v \in V \}.
    \]
 
\subsection{The Rank of a Linear Map}
The rank of a linear map \( f: V \to W \) is the ension of its image:
    \[
    \text{rank}(f) = \text{}(\text{Im}(f)).
    \]
\subsection{The Nullity of a Linear Map}
The nullity of a linear map \( f: V \to W \) is the ension of its kernel:
    \[
    \text{nullity}(f) = \text{}(\ker(f)).
    \]
\subsection{The Rank-Nullity Theorem}
The rank-nullity theorem states that for a linear map \( f: V \to W \):
\[
\text{}(\ker(f)) + \text{dim}(\text{Im}(f)) = \text{dim}(V).
\]

\subsection{Proof of the injectivity of the Kernel}

Suppose we have a linear map \(f\) and two vectors \(v_1\) and \(v_2\) which are not equal.
\noindent We are going to assume that they both map to the \(\vec{0}\) therefore:
\[
f(v_1) = f(v_2) = \vec{0}
\]

\noindent Because of the linearity of the map we can write:
\[
f(v_1 - v_2) = f(v_1) - f(v_2) = \vec{0} - \vec{0} = \vec{0}
\]

\noindent And that contradicts the assumption that \(v_1\) and \(v_2\) are not equal. Therefore, the kernel of a linear map is injective.

\subsection{Dimension Formula for Linear Mappings}
    Let \(f: V \to W\) be linear and \(\dim(V) = n\). Then
    \[\dim(\ker(f)) + \operatorname{rank}(f) = n.\]
    
    
   \noindent Remember that \(\ker(f)\) forms a subspace of \(V\) and therefore \(\dim(\ker(f)) := r \leq n\). We extend an arbitrary basis \((v_1, \ldots, v_r)\) of \(\ker(f)\) to a basis \((v_1, \ldots, v_r, v_{r+1}, \ldots, v_n)\) of \(V\). Setting \(w_{r+i} = f(v_{r+i})\) for \(i = 1, \ldots, n - r\), we have \(\forall v \in V\):
    \begin{align*}
    f(v) &= f(\lambda_1v_1 + \cdots + \lambda_rv_r + \lambda_{r+1}v_{r+1} + \cdots + \lambda_nv_n) \\
    &= \lambda_1 \underbrace{f(v_1)}_{=0} + \cdots + \lambda_r \underbrace{f(v_r)}_{=0} + \lambda_{r+1} f(v_{r+1}) + \cdots + \lambda_n f(v_n) \\
    &= \lambda_{r+1} f(v_{r+1}) + \cdots + \lambda_n f(v_n) \\
    &= \lambda_{r+1}w_{r+1} + \cdots + \lambda_nw_n
    \end{align*}
    
    Thus, \(\operatorname{Im}(f) = \operatorname{span}(w_{r+1}, \ldots, w_n)\). We now show that \(w_{r+1}, \ldots, w_n\) are linearly independent. Let 
    \[\lambda_{r+1}w_{r+1} + \cdots + \lambda_nw_n = 0.\]
    
    From
    \[0 = \lambda_{r+1}w_{r+1} + \cdots + \lambda_nw_n = f(\lambda_{r+1}v_{r+1} + \cdots + \lambda_nv_n)\]
    it follows that
    \[\lambda_{r+1}v_{r+1} + \cdots + \lambda_nv_n \in \ker(f).\]
    
    Therefore,
    \[\lambda_{r+1}v_{r+1} + \cdots + \lambda_nv_n = \lambda_1v_1 + \cdots + \lambda_rv_r\]
    for some \(\lambda_1, \ldots, \lambda_r\). Since \(v_1, \ldots, v_n\) are linearly independent, we have \(\lambda_1 = \cdots = \lambda_n = 0\). Thus, the vectors \(w_{r+1}, \ldots, w_n\) are linearly independent.
    
    It follows that \(\dim(\operatorname{Im}(f)) = \operatorname{rank}(f) = n - r\) and therefore
    \[\dim(\ker(f)) + \dim(\operatorname{Im}(f)) = \dim(\ker(f)) + \operatorname{rank}(f) = r + n - r = n.\]
  \(\mathfrak{QED}\)
  
\subsection{Identifying the type of linear map}

\begin{itemize}
    
\item If \(\dim(\ker(f)) = 0\) then the map is injective.

\item If \(\dim(\ker(f)) = \dim(V)\) then the map is surjective.

\item If \(\dim(\ker(f)) = \dim(V)\) and \(\dim(W) = 0\) then the map is bijective.
\end{itemize}

\subsection{Image of the basis}

Let \((V) = \dim(W)\), For the basis \((v_1, \cdots, v_n)\) of \(V\) we have the image \((w_1, \cdots, w_n)\) of \(W\). Then:
\begin{itemize}[label=\(-\)]
    \item If \(\dim(\ker(f)) = 0\) then the map is injective and \((w_1, \cdots, w_n)\) is a basis of \(W\).
    
    \item If \(\dim(\ker(f)) = \dim(V)\) then the map is surjective and \((w_1, \cdots, w_n)\) is a spanning set of \(W\).
    
    \item If \(\dim(\ker(f)) = \dim(V)\) and \(\dim(W) = 0\) then the map is bijective and \((w_1, \cdots, w_n)\) is a basis of \(W\).
\end{itemize}
\subsection{Linear Maps and Matrices}
\noindent Let \( V \) and \( W \) be finite-ensional vector spaces over the same field \( K \). If \( \dim V = n \) and \( \dim W = m \), then a linear map \( f: V \to W \) can be represented by an \( m \times n \) matrix. The action of the linear map on a vector can be expressed as:
\[
f(x) = A \cdot x,
\]
where \( A \) is the matrix representation of \( f \) and \( v \) is the vector represented in a column format. The columns of the matrix \( A \) are the images of the basis vectors of \( V \) under the linear map \( f \).


\newpage
