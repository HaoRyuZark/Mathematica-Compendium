\newpage
\section{The Determinant of a Matrix}

The determinant of a square matrix \(A \in \mathbb{R}^{n \times n}\), denoted \(\det(A)\) or \(|A|\), is a 
scalar value that provides important information about the matrix, including whether it is 
invertible and the volume scaling factor of the linear transformation represented by \(A\).
The determinant can be computed using various methods, including the Laplace 
expansion, row reduction, or the Leibniz formula.

The determinant of a \(2 \times 2\) matrix is given by:

\begin{equation*}
\det(A) =
\begin{vmatrix}
a & b \\
c & d
\end{vmatrix}
= ad - bc
\end{equation*}

For a \(3 \times 3\) matrix, the determinant can be computed using the rule of Sarrus or the cofactor expansion:

\begin{equation*}
\det(A) =
\begin{vmatrix}
a & b & c \\
d & e & f \\
g & h & i
\end{vmatrix}
= aei + bfg + cdh - ceg - bdi - afh
\end{equation*}

The determinant of larger matrices can be computed using cofactor expansion along any row or column:

\begin{equation*}
\det(A) = \sum_{j=1}^{n} (-1)^{i+j} a_{ij} \det(A_{ij})
\end{equation*}

where \(A_{ij}\) is the \((n-1) \times (n-1)\) submatrix obtained by deleting the \(i\)-th row and \(j\)-th column of \(A\).

\subsubsection{Properties}

\begin{itemize}[label=\(-\)]
    \item \(\det(A) = 0\) if and only if \(A\) is singular (not invertible).
    \item \(\det(AB) = \det(A) \cdot \det(B)\) for any square matrices \(A\) and \(B\) of the same size.
    \item \(\det(A^T) = \det(A)\).
    \item If a row (or column) of \(A\) is multiplied by a scalar \(\alpha\), 
    then \(\det(A)\) is multiplied by \(\alpha\).
    \item If two rows (or columns) of \(A\) are swapped, then \(\det(A)\) changes sign.
    \[\det(a,b,c) = - \det(b,a,c)\]
    \item If a row (or column) of \(A\) is added to another row (or column), then \(\det(A)\) remains unchanged.
    \item If one of the columns is a linear combination of the others, then \(\det(A) = 0\).
    \item The determinant of the identity matrix \(I_n\) is 1.
    \item The determinant of can splitet into the sum of more determinants:
    \[
    \det(a,b,c + d) = \det(a,b,c) + \det(a,b,d)
    \]
    \item The determinant of a diagonal matrix is the product of its diagonal entries.
    \item For elementary matrices \(\det C1 = 1\), \(\det C2 = -1\), \(\det C3 = \lambda\)
\end{itemize}

\subsection{Proof of \(\det(AB) = \det(A) \det(B)\)}

For \( A, B \in K^{n \times n} \), it holds that
\[
\det(AB) = \det(A)\det(B).
\]

\textbf{Proof:} \\
If \( \det(A) = 0 \) or \( \det(B) = 0 \), 
then \( \text{rank}(L_A) < n \) or \( \text{rank}(L_B) < n \). \\
Thus, \( \text{rank}(L_{AB}) < n \), 
and the statement is trivial.

So let us assume that \( A \) and \( B \) are 
invertible, i.e., \( \text{rank}(A) = \text{rank}(B) = n \). \\
We now assume that in such a case, a matrix can 
always be transformed into reduced row 
echelon form using the row operations 
from Gaussian elimination.

Then, \( A = C_k \cdot \ldots \cdot C_1 \) for certain elementary matrices \( C_i \), and therefore, by Corollary 5.13:

\begin{align*}
\det(AB) &= \det(C_k \cdot \ldots \cdot C_1 B) \\
&= \det(C_k)\det(C_{k-1} \cdot \ldots \cdot C_1 B) \\
&\quad \vdots \\
&= \det(C_k) \cdots \det(C_1)\det(B) \\
&= \det(C_k \cdot \ldots \cdot C_1)\det(B) \\
&= \det(A)\det(B).
\end{align*}

Therefore also if \( A \) is invertible, then
\[
\det(A^{-1}) = (\det(A))^{-1}.
\]

\textbf{Proof:} \\
It holds that
\[
\det(A)\det(A^{-1}) = \det(AA^{-1}) = \det(E) = 1.
\]

\subsection{The Leibniz Formula}

The determinant can be computed using the following formula where the sign
of a permutation is the number of swaps \(\mod 2\)

\[
\det(A) = \sum_{\sigma \in S_n} \text{sgn}(\sigma) a_{1\sigma(1)} a_{2\sigma(2)} \cdots a_{n\sigma(n)}
\]
    
where \(S_n\) is the set of all permutations of \(\{1, 2, \dots, n\}\) and \(\text{sgn}(\sigma)\) is the sign of the permutation \(\sigma\). 

\subsection{Laplace's Method (Cofactor Expansion)}

Here's how to find the determinant of a matrix using Laplace's method:

Laplace's method, also known as cofactor expansion, allows you to compute the determinant of a square matrix by expanding along any row or column.

\textbf{Steps:}

1.\textbf{Choose a Row or Column:} Select any row or column of the matrix.  It's often easiest to choose one with many zeros.

 2.\textbf{For Each Element:} For each element, \(a_{ij}\), in the chosen row or column:

    \textbf{Find the Minor, \(M_{ij}\):} The minor \(M_{ij}\) is the determinant of the submatrix formed by deleting the 
    \indent \(i\)-th row and the \(j\)-th column of the original matrix.

    \textbf{Find the Cofactor, \(C_{ij}\):} The cofactor \(C_{ij}\) is the minor multiplied by a sign factor:
        \[
        C_{ij} = (-1)^{i+j} M_{ij}
        \]
        The term  \((-1)^{i+j}\)  gives a checkerboard pattern of signs:
        \[
        \begin{pmatrix}
        + & - & + & - & \cdots \\
        - & + & - & + & \cdots \\
        + & - & + & - & \cdots \\
        - & + & - & + & \cdots \\
        \vdots & \vdots & \vdots & \vdots & \ddots
        \end{pmatrix}
        \]

 3.\textbf{Calculate the Determinant:} The determinant of the matrix, \(A\), is the sum of the products of the elements in the chosen row or column and their corresponding cofactors.

    \textbf{Expansion along the \(i\)-th row:}
        \[
        \det(A) = \sum_{j=1}^{n} a_{ij} C_{ij} = a_{i1}C_{i1} + a_{i2}C_{i2} + \cdots + a_{in}C_{in}
        \]

    \textbf{Expansion along the \(j\)-th column:}
        \[
        \det(A) = \sum_{i=1}^{n} a_{ij} C_{ij} = a_{1j}C_{1j} + a_{2j}C_{2j} + \cdots + a_{nj}C_{nj}
        \]
        Both expansions give the same result.

\textbf{Example (\(3\times3\) Matrix):}

Let
\[
A = \begin{pmatrix}
a_{11} & a_{12} & a_{13} \\
a_{21} & a_{22} & a_{23} \\
a_{31} & a_{32} & a_{33}
\end{pmatrix}
\]

Expanding along the first row:

1.\(a_{11}\):  \(M_{11} = \det \begin{pmatrix} a_{22} & a_{23} \\ a_{32} & a_{33} \end{pmatrix}\),  \(C_{11} = +M_{11}\)

2.\(a_{12}\):  \(M_{12} = \det \begin{pmatrix} a_{21} & a_{23} \\ a_{31} & a_{33} \end{pmatrix}\),  \(C_{12} = -M_{12}\)

3.\(a_{13}\):  \(M_{13} = \det \begin{pmatrix} a_{21} & a_{22} \\ a_{31} & a_{32} \end{pmatrix}\),  \(C_{13} = +M_{13}\)

Therefore,
\[
\det(A) = a_{11}C_{11} + a_{12}C_{12} + a_{13}C_{13}
\]


\textbf{Example of Determinant Calculation}

Let's calculate the determinant of the matrix:
\begin{equation*}
A =
\begin{pmatrix}
2 & 1 & 3 \\
1 & 0 & 2 \\
0 & 1 & 1
\end{pmatrix}
\end{equation*}
Using the rule of Sarrus for \(3 \times 3\) matrices:
\begin{align*}
\det(A) &= 2 \cdot 0 \cdot 1 + 1 \cdot 2 \cdot 3 + 3 \cdot 1 \cdot 1 - (3 \cdot 0 \cdot 0 + 1 \cdot 2 \cdot 2 + 2 \cdot 1 \cdot 1) \\
&= 0 + 6 + 3 - (0 + 4 + 2) \\
&= 9 - 6 = 3
\end{align*}
 Thus, the determinant of matrix \(A\) is \(\det(A) = 3\).

 Now consider the following \(4 \times 4\) matrix:

\begin{equation*}
A = 
\begin{pmatrix}
2 & 1 & 3 & 2 \\
4 & 0 & -1 & 3 \\
-2 & 3 & 1 & 5 \\
1 & -1 & 0 & 2
\end{pmatrix}
\end{equation*}

For a \(4 \times 4\) matrix, we can use Laplace Method along the first row:
\begin{align*}
\det(A) &= a_{11}C_{11} + a_{12}C_{12} + a_{13}C_{13} + a_{14}C_{14} \\
&= 2 \cdot \det\begin{pmatrix} 0 & -1 & 3 \\ 3 & 1 & 5 \\ -1 & 0 & 2 \end{pmatrix} 
- 1 \cdot \det\begin{pmatrix} 4 & -1 & 3 \\ -2 & 1 & 5 \\ 1 & 0 & 2 \end{pmatrix} \\
&\quad + 3 \cdot \det\begin{pmatrix} 4 & 0 & 3 \\ -2 & 3 & 5 \\ 1 & -1 & 2 \end{pmatrix} 
- 2 \cdot \det\begin{pmatrix} 4 & 0 & -1 \\ -2 & 3 & 1 \\ 1 & -1 & 0 \end{pmatrix}
\end{align*}

\[\det(A) = 145\]

