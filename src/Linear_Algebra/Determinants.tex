\newpage
\section{The Determinant of a Matrix}

The determinant of a square matrix \(A \in \Reals^{n \times n}\), denoted \(\det(A)\) or \(|A|\), is a 
scalar value that provides important information about the matrix, including whether it is 
invertible and the volume scaling factor of the linear transformation represented by \(A\).
The determinant can be computed using various methods, including the Laplace 
expansion, row reduction, or the Leibniz formula.

The determinant of a \(2 \times 2\) matrix is given by:

\[
    \det(A) =
    \begin{vmatrix}
    a & b \\
    c & d
    \end{vmatrix}
    = ad - bc
\]

For a \(3 \times 3\) matrix, the determinant can be computed using the rule of Sarrus or the co-factor expansion:

\[
    \det(A) =
    \begin{vmatrix}
    a & b & c \\
    d & e & f \\
    g & h & i
    \end{vmatrix}
    = aei + bfg + cdh - ceg - bdi - afh
\]

The determinant of larger matrices can be computed using co-factor expansion along any row or column:

\[
    \det(A) = \sum_{j=1}^{n} {(-1)}^{i+j} a_{ij} \det(A_{ij}),
\]

where \(A_{ij}\) is the \((n-1) \times (n-1)\) sub-matrix obtained by deleting the \(i\)-th row and \(j\)-th column of \(A\).

\subsection{Properties of the determinant}

\begin{itemize}
    
    \item \(\det(A) = 0\) if and only if \(A\) is singular (not invertible).
    
    \item \(\det(AB) = \det(A) \cdot \det(B)\) for any square matrices \(A\) and \(B\) of the same size.
    
    \item \(\det(A^T) = \det(A)\).
    
    \item If a row (or column) of \(A\) is multiplied by a scalar \(\alpha\),     
    then \(\det(A)\) is multiplied by \(\alpha\).
    
    \item If two rows (or columns) of \(A\) are swapped, then \(\det(A)\) changes sign.
    
    \[
        \det(a,b,c) = - \det(b,a,c)
    \]
    
    \item If a row (or column) of \(A\) is added to another row (or column), then \(\det(A)\) remains unchanged.

    \item If one of the columns is a linear combination of the others, then \(\det(A) = 0\).
    
    \item The determinant of the identity matrix \(I_n\) is 1.
    
    \item The determinant of can split into the sum of more determinants:

    \[
        \det(a,b,c + d) = \det(a,b,c) + \det(a,b,d)
    \]
    
    \item The determinant of a diagonal matrix is the product of its diagonal entries.
    
    \item For elementary matrices \(\det C1 = 1\), \(\det C2 = -1\), \(\det C3 = \lambda\).
    
    \item For \(A \in \Reals^{n\times n}\, \det(\lambda A) = \lambda^n \det(A)\) 

\end{itemize}

\subsection{Proof of the multiplication of determinants}

For \( A, B \in K^{n \times n} \), it holds that

\[
    \det(AB) = \det(A)\det(B).
\]

\textbf{Proof:}

If \( \det(A) = 0 \) or \( \det(B) = 0 \), 
then \( \rank(L_A) < n \) or \( \rank(L_B) < n \).

Thus, \( \rank(L_{AB}) < n \), 
and the statement is trivial.

So let us assume that \(A\) and \(B\) are 
invertible, i.e., \( \rank(A) = \rank(B) = n \).

We now assume that in such a case, a matrix can 
always be transformed into reduced row 
echelon form using the row operations 
from Gaussian elimination.

Then, \( A = C_k \cdot \cdots \cdot C_1 \) for certain elementary matrices \( C_i \), and therefore:

\begin{align*}
    \det(AB) &= \det(C_k \cdot \cdots \cdot C_1 B) \\
    &= \det(C_k)\det(C_{k-1} \cdot \cdots \cdot C_1 B) \\
    &\quad \vdots \\
    &= \det(C_k) \cdots \det(C_1)\det(B) \\
    &= \det(C_k \cdot \cdots \cdot C_1)\det(B) \\
    &= \det(A)\det(B).
\end{align*}

Therefore, also if \(A\) is invertible, then

\[
    \det(A^{-1}) = {(\det(A))}^{-1}.
\]

\textbf{Proof:} 

It holds that
\[
    \det(A)\det(A^{-1}) = \det(AA^{-1}) = \det(E) = 1.
\]

\subsection{The Leibniz Formula}

Let \( A = (a_{ij}) \in K^{n \times n} \) be a square matrix over a field \( K \). The determinant of \(A\) is defined via the \textbf{Leibniz formula} as:

\[
    \det(A) = \sum_{\sigma \in S_n} \operatorname{sgn}(\sigma) \cdot a_{1\sigma(1)} a_{2\sigma(2)} \cdots a_{n\sigma(n)},
\]

where:

\begin{itemize}

    \item \( S_n \) is the set of all permutations of \( \{1, 2, \dots, n\} \),

    \item \( \operatorname{sgn}(\sigma) \) is the sign of the permutation \( \sigma \), equal to \( +1 \) for even permutations and \( -1 \) for odd ones.

\end{itemize}

\subsubsection{Derivation}

\textbf{Step 1: Expansion using multilinearity}

Each column vector \( a_j \) of the matrix \(A\) can be written as a linear combination of the standard basis vectors \( e_1, \dots, e_n \):

\[
    a_j = \sum_{i=1}^n a_{ij} e_i.
\]

Using the multilinearity of the determinant function (i.e., linearity in each column), we expand:

\[
    \det(a_1, a_2, \dots, a_n) = \det\left( \sum_{i_1} a_{i_1 1} e_{i_1}, \sum_{i_2} a_{i_2 2} e_{i_2}, \dots, \sum_{i_n} a_{i_n n} e_{i_n} \right).
\]

By multilinearity, this becomes a sum over all \(n\)-tuples \( (i_1, i_2, \dots, i_n) \):

\[
    = \sum_{i_1, i_2, \dots, i_n = 1}^n a_{i_1 1} a_{i_2 2} \cdots a_{i_n n} \cdot \det(e_{i_1}, e_{i_2}, \dots, e_{i_n}).
\]

\textbf{Step 2: Eliminate repeated indices}

If two indices \( i_j = i_k \) for \( j \neq k \), then the corresponding 
determinant is zero (because the determinant is alternating and thus, zero for repeated columns). 
Therefore, only terms with pairwise distinct indices remain. These correspond to permutations of \( \{1, 2, \dots, n\} \).

Let \( \sigma \in S_n \) be such a permutation. Then we can write:

\[
    \det(A) = \sum_{\sigma \in S_n} a_{\sigma(1) 1} a_{\sigma(2) 2} \cdots a_{\sigma(n) n} \cdot \det(e_{\sigma(1)}, e_{\sigma(2)}, \dots, e_{\sigma(n)}).
\]

\textbf{Step 3: Sign of the permutation}

Since the determinant of \( (e_{\sigma(1)}, \dots, e_{\sigma(n)}) \) is \( \operatorname{sgn}(\sigma) \), we obtain:

\[
    \det(A) = \sum_{\sigma \in S_n} \operatorname{sgn}(\sigma) \cdot a_{\sigma(1)1} a_{\sigma(2)2} \cdots a_{\sigma(n)n}.
\]

Equivalently, by re-indexing:

\[
    \det(A) = \sum_{\sigma \in S_n} \operatorname{sgn}(\sigma) \cdot \prod_{i=1}^n a_{i \sigma(i)}.
\]

This formula expresses the determinant as a sum over all permutations of the indices \( \{1, \dots, n\} \), where each term is a product of matrix entries taken from different rows and columns, weighted by the sign of the corresponding permutation.

\subsection{Laplace's Method (Co-factor Expansion)}

The determinant of an \( n \times n \) matrix \( A = [a_{ij}] \) can be 
computed using the \emph{Laplace expansion}, also known as co-factor expansion. 
This method expands the determinant along a specific row or column.

Expansion Along Row \(i\):

\[
    \det(A) = \sum_{j=1}^{n} {(-1)}^{i+j} a_{ij} \det(M_{ij})
\]

Expansion Along Column \(j\):

\[
    \det(A) = \sum_{i=1}^{n} {(-1)}^{i+j} a_{ij} \det(M_{ij})
\]

Where:

\begin{itemize}

    \item \( a_{ij} \) is the element in the \(i\)-th row and \(j\)-th column of matrix \(A\).

    \item \( M_{ij} \) is the \emph{minor} of \( a_{ij} \), i.e., the determinant of the submatrix formed by removing the \(i\)-th row and \(j\)-th column from \(A\).

    \item \( {(-1)}^{i+j} \) is the \emph{sign factor}, determining the sign of each co-factor term.

\end{itemize}

\textbf{Steps:}

1.\textbf{ Choose a Row or Column:} Select any row or column of the matrix.  It's often easiest to choose one with many zeros.

 2.\textbf{ For Each Element:} For each element, \(a_{ij}\), in the chosen row or column:

    \textbf{Find the Minor, \(M_{ij}\):} The minor \(M_{ij}\) is the determinant of the sub-matrix formed by deleting the 
    \indent \(i\)-th row and the \(j\)-th column of the original matrix.

    \textbf{Find the Co-factor, \(C_{ij}\):} The co-factor \(C_{ij}\) is the minor multiplied by a sign factor:
        
        \[
            C_{ij} = {(-1)}^{i+j} M_{ij}
        \]
        The term  \({(-1)}^{i+j}\)  gives a checkerboard pattern of signs:
        
        \[
            \begin{bmatrix}
            + & - & + & - & \cdots \\
            - & + & - & + & \cdots \\
            + & - & + & - & \cdots \\
            - & + & - & + & \cdots \\
            \vdots & \vdots & \vdots & \vdots & \ddots
            \end{bmatrix}
        \]

 3.\textbf{ Calculate the Determinant:} The determinant of the matrix, \(A\), is the sum of 
 the products of the elements in the chosen row or column and their corresponding co-factors.

    \textbf{Expansion along the \(i\)-th row:}
        \[
            \det(A) = \sum_{j=1}^{n} a_{ij} C_{ij} = a_{i1}C_{i1} + a_{i2}C_{i2} + \cdots + a_{in}C_{in}
        \]

    \textbf{Expansion along the \(j\)-th column:}
        \[
            \det(A) = \sum_{i=1}^{n} a_{ij} C_{ij} = a_{1j}C_{1j} + a_{2j}C_{2j} + \cdots + a_{nj}C_{nj}
        \]
        Both expansions give the same result.

\textbf{Example (\(3\times3\) Matrix):}

Let
\[
    A = \begin{bmatrix}
    a_{11} & a_{12} & a_{13} \\
    a_{21} & a_{22} & a_{23} \\
    a_{31} & a_{32} & a_{33}
    \end{bmatrix}
\]

Expanding along the first row:

1. \(a_{11}\):  \(M_{11} = \det \begin{bmatrix} a_{22} & a_{23} \\ a_{32} & a_{33} \end{bmatrix}\),  \(C_{11} = +M_{11}\)

2. \(a_{12}\):  \(M_{12} = \det \begin{bmatrix} a_{21} & a_{23} \\ a_{31} & a_{33} \end{bmatrix}\),  \(C_{12} = -M_{12}\)

3. \(a_{13}\):  \(M_{13} = \det \begin{bmatrix} a_{21} & a_{22} \\ a_{31} & a_{32} \end{bmatrix}\),  \(C_{13} = +M_{13}\)

Therefore,
\[
\det(A) = a_{11}C_{11} + a_{12}C_{12} + a_{13}C_{13}
\]

\textbf{Example of Determinant Calculation}

Let's calculate the determinant of the matrix:

\[
    A =
    \begin{bmatrix}
    2 & 1 & 3 \\
    1 & 0 & 2 \\
    0 & 1 & 1
    \end{bmatrix}
\]

Using the rule of Sarrus for \(3 \times 3\) matrices:

\begin{align*}
    \det(A) &= 2 \cdot 0 \cdot 1 + 1 \cdot 2 \cdot 3 + 3 \cdot 1 \cdot 1 - (3 \cdot 0 \cdot 0 + 1 \cdot 2 \cdot 2 + 2 \cdot 1 \cdot 1) \\
    &= 0 + 6 + 3 - (0 + 4 + 2) \\
    &= 9 - 6 = 3
\end{align*}
 
Thus, the determinant of matrix \(A\) is \(\det(A) = 3\).

 Now consider the following \(4 \times 4\) matrix:

\[
    A = 
    \begin{bmatrix}
    2 & 1 & 3 & 2 \\
    4 & 0 & -1 & 3 \\
    -2 & 3 & 1 & 5 \\
    1 & -1 & 0 & 2
    \end{bmatrix}
\]

For a \(4 \times 4\) matrix, we can use Laplace Method along the first row:

\begin{align*}
    \det(A) &= a_{11}C_{11} + a_{12}C_{12} + a_{13}C_{13} + a_{14}C_{14} \\
    &= 2 \cdot \det\begin{bmatrix} 0 & -1 & 3 \\ 3 & 1 & 5 \\ -1 & 0 & 2 \end{bmatrix} 
    - 1 \cdot \det\begin{bmatrix} 4 & -1 & 3 \\ -2 & 1 & 5 \\ 1 & 0 & 2 \end{bmatrix} \\
    &\quad + 3 \cdot \det\begin{bmatrix} 4 & 0 & 3 \\ -2 & 3 & 5 \\ 1 & -1 & 2 \end{bmatrix} 
    - 2 \cdot \det\begin{bmatrix} 4 & 0 & -1 \\ -2 & 3 & 1 \\ 1 & -1 & 0 \end{bmatrix}
\end{align*}

\[
    \det(A) = 145
\]

\subsection{Determinant of the reduced echelon form}

The determinant has another trait that states that if your matrix \(A\) is REF then
its determinant is given by the product of the main diagonal. 

\subsection{Gaussian Method for finding determinants}

Another method for finding determinants is using the row operations we already knew plus also column operations 
but with the catch that for the column case we have to take into consideration how
operating with columns affect the determinant, in contrast to the row operations, except for scaling
a row by some \(\lambda\) in this case we would need to undo the change by multiplying the determinant
by \(\frac{1}{\lambda}\).

\textbf{Example: }

We are given the matrix:

\[
    A = \begin{bmatrix}
    1 & 1 & 1 & 0 \\
    1 & 1 & 0 & 1 \\
    1 & 0 & 1 & 1 \\
    0 & 1 & 1 & 1
    \end{bmatrix}
\]

We apply Gaussian elimination:

Swap \( R_2 \leftrightarrow R_3 \):

\[
    \Rightarrow
    \begin{bmatrix}
    1 & 1 & 1 & 0 \\
    1 & 0 & 1 & 1 \\
    1 & 1 & 0 & 1 \\
    0 & 1 & 1 & 1
    \end{bmatrix}
    \rightarrow
    \begin{bmatrix}
    1 & 1 & 1 & 0 \\
    0 & -1 & 0 & 1 \\
    0 & 0 & -1 & 1 \\
    0 & 1 & 1 & 1
    \end{bmatrix}
    \rightarrow
    \begin{bmatrix}
    1 & 1 & 1 & 0 \\
    0 & -1 & 0 & 1 \\
    0 & 0 & -1 & 1 \\
    0 & 0 & 1 & 2
    \end{bmatrix}
    \rightarrow
    \begin{bmatrix}
    1 & 1 & 1 & 0 \\
    0 & -1 & 0 & 1 \\
    0 & 0 & -1 & 1 \\
    0 & 0 & 0 & 3
    \end{bmatrix}
\]

Now the matrix is upper triangular. The determinant is the product of diagonal elements times \( -1 \) for one row swap:

\[
    \det(A) = {(-1)}^1 \cdot (1) \cdot (-1) \cdot (-1) \cdot (3) = -3
\]

\[
    \det(A) = -3
\]

