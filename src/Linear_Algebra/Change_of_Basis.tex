\section{Changig between Basis}

Given a vertor \(\vec{x} = \lambda_1 \vec{v}_1 + \cdots + \lambda_n \vec{v}_n\) 
with \(\vec{v}_1, \dots, \vec{v}_n\)
being a the basis vector, we can write \(\vec{x}\) in term of its components \((a, b, c, \dots)^T\).

Now imagine having another basis \(\vec{b}_1, \dots, \vec{b}_n\) and a the same vector \(\vec{x}\)
with all of its components. The key here is that we want to know how our original
vector is described in with other basis.

So our vector \(\vec{x}\) is described:

\[\lambda_1 \vec{v}_1 + \cdots + \lambda_n \vec{v}_n = \mu_1 \vec{b}_1 + \cdots + \mu_n \vec{b}_n\]

in the corresponding bases.

Now lets write then a matrix vector multiplication
\[
(\vec{v}_1, \dots, \vec{v}_ n) 
\begin{pmatrix} \lambda_1 \\ \vdots \\ \lambda_n \end{pmatrix}
 =
(\vec{b}_1, \dots, \vec{b}_ n) 
\begin{pmatrix} \mu_1 \\ \vdots \\ \mu_n \end{pmatrix}
\]

Now lets use the following notation for the Basis and the coefficient vectors.

\begin{center}
    \boxed{P_B (\vec{x})_B = P_C (\vec{x})_C}
\end{center}

Here \(P_{X}\) is the matrix of the generic basis \(X\) vectors and \((\vec{x})_X\) the coefficients of the
vector \(\vec{x}\) described by that basis.

Now that convertion of basis is just a matter of solving for 
our desired vector by multipliying
by the inverse of the corresponding basis matrix.

\textbf{Example:}

Lets find \(\vec{x}\) in terms of tha basis \(C\)

\[P_{C}^{-1} P_B (\vec{x})_B = (\vec{x})_C\]

Now let us understan what really is happening in the step where we multiply by the inverse matrix.

\begin{align*}
P_{C}^{-1} P_B &= P_{C}^{-1}(\vec{b}_1, \dots, \vec{b}_n) \\
&= (P_{C}^{-1}\vec{b}_1, \dots, P_{C}^{-1}\vec{b}_n)\\
&= ((\vec{b}_1)_C, \dots, (\vec{b}_n)_C)
\end{align*}

Now it is clear that the inverse is just taking our basis vector and tranforming
them in to basis vector with respect to \(C\).

For the sake of completness here some importan Theoremes and definitions:

\subsection{Theoremes and definitions concerning the change of basis}

In this section we will use a slightly different notation. Here \(K_X\) is 
the \textbf{coefficient} vector with respect to a basis also called the coordinates
of the vector. And \(\varphi_X\) is the \(P_X\) that is the matrix of the basis vectors with
respect to \(X\).

\textbf{I} Let \( V \) be a \( K \)-vector space with a 
basis \( B = (v_1, \ldots, v_n) \).  
Then there exists exactly one isomorphism 
\[
\varphi_B : K^n \to V
\]
such that
\[
\varphi_B(e_i) = v_i, \quad \text{for } 1 \leq i \leq n.
\]

\textbf{II} The isomorphism \( \varphi_B \) from the previous 
Theorem is called the \emph{coordinate mapping},  
and for \( v \in V \), we define
\[
K_B (\vec{v}) := \varphi_B^{-1} (v) \in K^n
\]
as the \emph{coordinates of \( v \) with respect to \( B \)}. Or like above 
\(P_{B}^{-1} P_C (\vec{x})_C = (\vec{x})_B\)

This is the previously discussed \(\vec{v} = P_{B}^{-1} \vec{v}\) 
that takes a vector from the other basis and translates its into our basis.
 
\textbf{III} We define the \emph{Traslation} of a vector to another
basis as:

\[
T_{B}^{A} = \varphi_{B}^{-1} \circ \varphi_A
\]

With \(\varphi_A\) begin the matrix of basis vectors of the new coordinate
system and \(\varphi_{B}^{-1}\) the inverse of the starting basis 
Or \(P_{B}^{-1} \circ P_C\).

Another way of finding this transition matrix is to write our basis vectors as a linear
combination of the basis vector of the other base and then use Gauß-Jordan to find the coefficients.
the result of the corresponding matrices are going to make our transition matrix.

\[
\begin{tikzcd}
& V & \\
K^n \arrow[ur, "\varphi_A"] \arrow[rr, "T_{B}^{A} = \varphi_B^{-1} \circ \varphi_A"'] & & K^n \arrow[ul, "\varphi_B"']
\end{tikzcd}
\]

 \textbf{IV} Let \( v \in V \) be arbitrary, 
with \( K_A(v) = (x_1, \ldots, x_n)^T \) 
and \( K_B(v) = (y_1, \ldots, y_n)^T \).  
Then the following holds:
\[
\begin{pmatrix}
y_1 \\
\vdots \\
y_n
\end{pmatrix}
=
T_B^A
\begin{pmatrix}
x_1 \\
\vdots \\
x_n
\end{pmatrix}
\]
If the coordinates of \( v \) with respect to \( A \) are known, then the matrix \( T_B^A \) can be used to compute the coordinates of \( v \) with respect to \( B \).
\[ T_B^A = B^{-1}A \].

 \textbf{V} Let \( V \) and \( W \) be finitely 
generated \( K \)-vector spaces with 
bases \( A \) and \( B \), respectively, and let \( f \in \mathrm{Hom}(V, W) \).
Then we have
\[
M_B^A(f) = \varphi_B^{-1} \circ f \circ \varphi_A
\]

In plain english the matrix that represents \( f \) with respect to 
the bases \( A \) and \( B \)
 is obtained by "coordinatizing" \( V \) (via \( \varphi_A \)), 
 applying \( f \), and then transforming the result into coordinates of 
 \( W \) (via \( \varphi_B^{-1} \)).

\begin{itemize}[label=\(-\)]
    \item First, \( \varphi_A \) turns a vector \( v \in V \) 
    into its coordinate vector in \( K^n \). This is just the matrix of the 
    base of \(K^n\).
    \item Then \( f \) acts on the original vector \( v \). This us our linear transformation matrix in our
    basis
    \item  Finally, \( \varphi_B^{-1} \) translates the result \( f(v) \) into a 
coordinate vector in \( K^m \) (according to basis \( B \)).
\end{itemize}

Thus, \( M_B^A(f) \) tells you how the coordinates change through \( f \).

 \textbf{VI} Let \( V \) and \( W \) be finitely generated vector spaces with bases \( A \) and \( A' \) for \( V \), and \( B \) and \( B' \) for \( W \), respectively.  
Furthermore, let \( f: V \to W \) be a linear map. Then the following holds:
\[
M_{B'}^{A'}(f) = T_{B'}^{B} \cdot M_{B}^{A}(f) \cdot \left( T_{A'}^{A} \right)^{-1}.
\]

Now here’s what this means:

\begin{itemize}[label=\(-\)]
\item \( M_{B}^{A}(f) \) is the matrix representing \( f \) relative to the original bases \( A \) (for \( V \)) and \( B \) (for \( W \)).
\item \( M_{B'}^{A'}(f) \) is the matrix representing \( f \) relative to the new bases \( A' \) and \( B' \).
\item\( T_{A'}^{A} \) is the change of basis matrix from \( A' \) to \( A \).
\item \( T_{B'}^{B} \) is the change of basis matrix from \( B \) to \( B' \).
\end{itemize}

The formula says that to itepret a linear transformation in terms of the
new coordinates we have to:
\begin{enumerate}
    \item Take a vector of the new coordinates
    \item Muliply it the matrix of its basis vectors. (Translate to our basis)
    \item Apply the linear transformation
    \item Inverse the change of basis by taking the inverse of the 
    matrix in step 2.
\end{enumerate}

\newpage