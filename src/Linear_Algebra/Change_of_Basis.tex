\newpage
\section{Changing between Basis}

To change from one basis to another of the same space, the question we are asking is How to write our basis vector 
in terms of the other basis. Because after that any linear combination of our basis vector will be directly translated 
to the other basis, just by writing them in terms of the other basis. So the matrix made of the new basis vectors 
\(T\) times a vector of coefficient in our systems of coordinates gives us the same vector but described in the new basis.
And to translate to our language we just have to multiply the vector in the other basis by \(T^{-1}\).

Finally, imagine that you have a matrix that represents some linear transformation in our language, but we want to transform 
a vector in another language. We need to first take that vector in the foreign basis, apply \(T^{-1}\), apply the transformation \emph{M} 
and finally multiply by \(T\) to return to the new basis.

Now to formalize these ideas:

Given a vector \(\vec{x} = \lambda_1 \vec{v}_1 + \cdots + \lambda_n \vec{v}_n\) 
with \(\vec{v}_1, \dots, \vec{v}_n\)
being the basis vectors, we can write \(\vec{x}\) in terms of its components \({(a, b, c, \dots)}^T\).

Now imagine having another basis \(\vec{b}_1, \dots, \vec{b}_n\) and the same vector \(\vec{x}\)
with all of its components. The key here is that we want to know how our original
vector is described in terms of other basis. More precisely what are its coordinates

So our vector \(\vec{x}\) is described:

\[
    \lambda_1 \vec{v}_1 + \cdots + \lambda_n \vec{v}_n = \mu_1 \vec{b}_1 + \cdots + \mu_n \vec{b}_n,
\]

in the corresponding bases.

Now let us write then a matrix vector multiplication

\[
    (\vec{v}_1, \dots, \vec{v}_ n) 
    \begin{pmatrix} \lambda_1 \\ \vdots \\ \lambda_n \end{pmatrix}
    =
    (\vec{b}_1, \dots, \vec{b}_ n) 
    \begin{pmatrix} \mu_1 \\ \vdots \\ \mu_n \end{pmatrix}
\]

Now let us use the following notation for the Basis and the coefficient vectors.

\[
    P_B {(\vec{x})}_B = P_C {(\vec{x})}_C
\]

Here \(P_{X}\) is the matrix of the basis \(X\) vectors and \({(\vec{x})}_X\) the coefficients of the
vector \(\vec{x}\) described by that basis.

Now that conversion of basis is just a matter of solving for 
our desired vector by multiplying
by the inverse of the corresponding basis matrix.

\textbf{Example I:}

Let us find \(\vec{x}\) in terms of that basis \(C\)

\[
    P_{C}^{-1} P_B {(\vec{x})}_B = {(\vec{x})}_C
\]

\textbf{Example II:}

Give are the vector \(\vec{u}_1 = {(1,2)}^T\) and \(\vec{u}_2 = {(3, 3)}^T\)
and our canonical basis vector \(\imath\) and \(\jmath\). The
vector in the canonical basis has the coordinates (2, 1). 

We set them equal

\[
    \lambda_1 \vec{u}_1 + \lambda_2 \vec{u}_2 = \mu_1 \imath + \mu_2 \jmath 
\]

Let us write \(\vec{u}_1\) and \(\vec{u}_2\) in terms of the canonical basis.

\[
    \vec{u}_1 = \imath + 2 \jmath
\]
\[
    \vec{u}_2 = 3 \imath + 3 \jmath
\]

Now we can substitute this values in the original expression

\[
    \lambda_1 (\imath + 2 \jmath) + \lambda_2 (3 \imath + 3 \jmath) = \mu_1 \imath + \mu_2 \jmath 
\]

\[
    \lambda_1 \imath + \lambda_1 2 \jmath + \lambda_2 3 \imath + \lambda_2 3 \jmath = \mu_1 \imath + \mu_2 \jmath 
\]

\[
    (\lambda_1  +  \lambda_2 3)\imath + (\lambda_1 2  + \lambda_2 3) \jmath = \mu_1 \imath + \mu_2 \jmath 
\]

Now we can write this in matrix form

\[
    \begin{bmatrix}
        1 & 3 \\
        2 & 3 \\
    \end{bmatrix} \begin{pmatrix}
        \lambda_1 \\ \lambda2
    \end{pmatrix} = \begin{pmatrix}
        \mu_1 \\ \mu_2
    \end{pmatrix}
\]

This equation relates the coefficient in the standard basis to the new basis
coefficients

What we are really doing here is taking the basis vectors of \(B\) and writing then as a linear combination
of the basis vectors of \(C\). This will give us \(n\) matrices that we can solve.

To go the other way around we take the inverse of the matrix whose columns are the basis vectors of \(B\)
relative to the new basis \(C\)

Now let us understand what really is happening in the step where we multiply by the inverse matrix.

\begin{align*}
    P_{C}^{-1} P_B &= P_{C}^{-1}(\vec{b}_1, \dots, \vec{b}_n) \\
    &= (P_{C}^{-1}\vec{b}_1, \dots, P_{C}^{-1}\vec{b}_n)\\
    &= ({(\vec{b}_1)}_C, \dots, {(\vec{b}_n)}_C)
\end{align*}

Now it is clear that the inverse is just taking our basis vector and transforming
them in to basis vector with respect to \(C\).

\subsection{Translation under transformation}

For linear transformation we use the formula.

\[
    A^{-1} M A \vec{v}
\]

Where \(\vec{v}\) is a vector in the other basis, \(A\) the
other basis,\emph{M} the linear transformation and \(A^{-1}\) the inverse of
the basis vector of \(A\).

We interpret this as taking a vector of the new basis, translating it to our language,
performing the transformation and then returning to the new basis.

For the sake of completeness here some important theorems and definitions:

\subsection{Theorems and definitions concerning the change of basis}

In this section we will use a slightly different notation. Here \(K_X\) is 
the \emph{coefficient} vector with respect to a basis also called the coordinates
of the vector. And \(\varphi_X\) is the \(P_X\) that is the matrix of the basis vectors with
respect to \(X\).

\subsubsection{Theorem I} 

Let \( V \) be a \( K \)-vector space with a basis \( B = (v_1, \ldots, v_n) \).  
Then there exists exactly one isomorphism 

\[
    \varphi_B : K^n \to V
\]

such that

\[
    \varphi_B(e_i) = v_i, \quad \text{for } 1 \leq i \leq n.
\]

\subsubsection{Theorem II} 

The isomorphism \( \varphi_B \) from the previous theorem is called the \emph{coordinate mapping},  
and for \( v \in V \), we define

\[
    K_B (\vec{v}) := \varphi_B^{-1} (v) \in K^n
\]

as the \emph{coordinates of \( v \) with respect to \(B\)}.

\subsubsection{Theorem III} 

We define the \emph{Translation} of a vector to another basis as:

\[
    T_{B}^{A} = \varphi_{B}^{-1} \circ \varphi_A
\]

\[
    \begin{tikzcd}
    & V & \\
    K^n \arrow[ur, "\varphi_A"] \arrow[rr, "T_{B}^{A} = \varphi_B^{-1} \circ \varphi_A"'] & & K^n \arrow[ul, "\varphi_B"']
    \end{tikzcd}
\]

\subsubsection{Theorem IV} 

Let \( v \in V \) be arbitrary, with \( K_A(v) = {(x_1, \dots, x_n)}^T \) 
and \( K_B(v) = {(y_1, \dots, y_n)}^T \).  Then the following holds:

\[
    \begin{pmatrix}
    y_1 \\
    \vdots \\
    y_n
    \end{pmatrix}
    =
    T_B^A
    \begin{pmatrix}
    x_1 \\
    \vdots \\
    x_n
    \end{pmatrix}
\]

If the coordinates of \( v \) with respect to \(A\) are known, then the matrix \( T_B^A \) can be used to compute the coordinates of \( v \) with respect to \(B\).

\[ 
    T_B^A = B^{-1}A. 
\]

\[
    \begin{tikzcd}
        K^n \arrow[rrrr, "M_{B}^{A}"] \arrow[rd, "\varphi_A"] &                   &  &   & K^m \arrow[ld, "\varphi_B"'] \\
                                                              & V \arrow[rr, "f"] &  & W &                             
        \end{tikzcd}
\]

\subsubsection{Theorem V}
 
Let \( V \) and \( W \) be finitely generated \( K \)-vector spaces with 
bases \(A\) and \(B\), respectively, and let \( f \in \mathrm{Hom}(V, W) \).
Then we have

\[
    M_B^A(f) = \varphi_B^{-1} \circ f \circ \varphi_A
\]

Thus, \( M_B^A(f) \) tells you how the coordinates change through \(f\).

\subsubsection{Theorem VI} 

Let \( V \) and \( W \) be finitely generated vector spaces with bases \(A\) and \( A' \) 
for \( V \), and \(B\) and \( B' \) for \( W \), respectively.  Furthermore, let \( f: V \to W \) be 
a linear map. Then the following holds:

\[
    M_{B'}^{A'}(f) = T_{B'}^{B} \cdot M_{B}^{A}(f) \cdot {\left( T_{A'}^{A} \right)}^{-1}.
\]

\[
    \begin{tikzcd}
        K^n \arrow[dd, "T_{A'}^{A}"] \arrow[rrrr, "M_{B}^{A}"] \arrow[rd, "\varphi_A"] &                   &  &   & K^m \arrow[dd, "T_{B'}^{B}"] \arrow[ld, "\varphi_B"'] \\
                                                                                       & V \arrow[rr, "f"] &  & W &                                                       \\
        K^n \arrow[rrrr, "M_{B'}^{A'}"] \arrow[ru, "\varphi_{A}'"']                    &                   &  &   & K^m \arrow[lu, "\varphi_{B'}"]                       
    \end{tikzcd}
\]

The formula says that to interpret a linear transformation in terms of the
new coordinates we have to:

\begin{enumerate}
    \item Take a vector of the new coordinates
    \item Multiply it the matrix of its basis vectors. (Translate to our basis)
    \item Apply the linear transformation
    \item Inverse the change of basis by taking the inverse of the 
    matrix in step 2.
\end{enumerate}
