\section{Orthogonality}

\subsection{Orthogonality and Projection}

Let \( a, b \in V \). The vectors \( a \) and \( b \) are orthogonal to each other if

\[
\langle a, b \rangle = 0.
\]

This is written as \( a \perp b \).

With the same proof as in Theorem 2.25, the Pythagorean Theorem holds

\[
\|a + b\|^2 = \|a\|^2 + \|b\|^2
\]

for \( a \perp b \) in all unitary vector spaces.

For the orthogonal projection \( p_b(a) \) of a vector \( a \) onto \( b \), with \( b \neq 0 \), the formula in every unitary vector space is:

\[
p_b(a) = \frac{\langle a, b \rangle}{\langle b, b \rangle} b.
\]

\subsection{Orthogonal Projection and Orthogonal Complement}

Let \( U \) be a finitely generated subspace of \( V \) and \( a \in V \). A vector \( p_U(a) \in U \) is called the orthogonal projection of \( a \) onto \( U \) if

\[
a - p_U(a) \perp u \quad \forall u \in U \quad \text{(3.9)}.
\]

The question arises about well-definedness, i.e., whether such a vector \( p_U(a) \) always exists and if it is unique. The following concept helps in the discussion of uniqueness.

For \( M \subseteq V \), the \textbf{orthogonal complement} of \( M \) is defined as

\[
M^\perp = \{ v \in V \mid v \perp u \, \forall u \in M \}.
\]

\begin{itemize}[label=\(-\)]
    \item \( M^\perp \) is a subspace of \( V \).
    \item Let \( U \) be a subspace of \( V \). Then, we have \( U \cap U^\perp = \{0\} \).
\end{itemize}

\textbf{Proof:}
1.) We need to check the closure property. For \( u \in M \), \( x, y \in M^\perp \), and \( \lambda \in \mathbb{R} \), we have:

\[
\langle x + y, u \rangle = \langle x, u \rangle + \langle y, u \rangle = 0
\]
\[
\langle \lambda x, u \rangle = \lambda \langle x, u \rangle = 0.
\]

2.) Let \( a \in U \cap U^\perp \). Then, we have \( \langle a, u \rangle = 0 \) for all \( u \in U \), because \( a \in U^\perp \), and in particular, \( \langle a, a \rangle = 0 \) since \( a \in U \), which implies \( a = 0 \).

\subsection{Orthogonality of the basis to the Complement}

Let \( U \) be as before, and let \( (u_1, \ldots, u_m) \) be a basis of \( U \). For \( v \in V \), we have:

\[
v \in U^\perp \quad \text{if and only if} \quad \langle v, u_i \rangle = 0 \quad \forall 1 \leq i \leq m.
\]

\subsection{How to find the orthogonal complement}
Let \( U \) be a finitely generated subspace of \( V \) with basis \( (u_1, \ldots, u_m) \). To find the orthogonal complement \( U^\perp \), we can solve the system of equations:
\[
\langle v, u_i \rangle = 0 \quad \forall 1 \leq i \leq m.
\]
This system can be expressed in matrix form as \( A \cdot v = 0 \), where \( A \) is the matrix whose rows are the vectors \( u_i \) and \( v \) is the vector we want to find in \( U^\perp \).

\subsection{Orthogonal and Orthonormal Systems}
Let \( B = (v_1, \ldots, v_m) \) be an \( m \)-tuple of vectors in \( V \setminus \{0\} \).
\begin{itemize}[label=\(-\)]
    \item \( B \) is called an orthogonal system in \( V \) if all the vectors \( v_i \) are pairwise orthogonal.
    \item An orthogonal system is called an orthonormal system if, in addition, \( \|v_i\| = 1 \) for all \( i = 1, \ldots, m \).
    \item An orthogonal system that forms a basis of \( V \) is called an orthogonal basis of \( V \).
    \item An orthonormal system that forms a basis of \( V \) is called an orthonormal basis of \( V \).
\end{itemize}

\textbf{Notation:}
\[
\text{OG-System} \quad \text{OG-Basis} \quad \text{ON-System} \quad \text{ON-Basis}
\]

Using the Kronecker delta symbol \( \delta_{i,j} \):

\[
\delta_{i,j} =
\begin{cases}
1, & \text{if } i = j, \\
0, & \text{if } i \neq j,
\end{cases}
\]

we have \( \langle v_i, v_j \rangle = \delta_{i,j} \) for any orthonormal system.

\subsection{Writing a vector in terms of the Orthogonal Basis}
\textbf{Example:} The vectors
\[
a_1 =
\begin{pmatrix}
1 \\
0 \\
0
\end{pmatrix}, \quad
a_2 =
\begin{pmatrix}
0 \\
1 \\
0
\end{pmatrix}
\]
are orthogonal and normalized, but do not form a basis. Thus, they constitute an \textbf{orthonormal system}.

\textbf{Example:} The vectors
\[
a_1 =
\begin{pmatrix}
\frac{1}{\sqrt{2}} \\
\frac{1}{\sqrt{2}} \\
0
\end{pmatrix}, \quad
a_2 =
\begin{pmatrix}
-\frac{1}{\sqrt{2}} \\
\frac{1}{\sqrt{2}} \\
0
\end{pmatrix}, \quad
a_3 =
\begin{pmatrix}
0 \\
0 \\
1
\end{pmatrix}
\]
form a basis, are orthogonal and normalized. Therefore, they form an \textbf{orthonormal basis}.

\subsection{Linear Independence of Orthogonal Systems}
An orthogonal system \( (v_1, \ldots, v_n) \) is linearly independent.

\textit{Proof:} Let \( \sum_{i=1}^n \lambda_i v_i = 0 \). Then, for any \( v_j \), it follows that
\[
\left\langle \sum_{i=1}^n \lambda_i v_i, v_j \right\rangle = \sum_{i=1}^n \lambda_i \langle v_i, v_j \rangle = 0.
\]
Due to the orthogonality of the system, all terms vanish except \( \lambda_j \langle v_j, v_j \rangle \), so
\[
\lambda_j \langle v_j, v_j \rangle = 0.
\]
Since \( v_j \neq 0 \) in any orthogonal system, we have \( \langle v_j, v_j \rangle = \|v_j\|^2 > 0 \), implying \( \lambda_j = 0 \) for all \( 1 \leq j \leq n \). \qed

\subsection{Representation with Respect to an Orthogonal Basis}

Let \( B = (v_1, \ldots, v_n) \) be an orthogonal basis of a vector space \( V \). Then for every \( v \in V \), we have:
\[
v = \sum_{k=1}^n \frac{\langle v, v_k \rangle}{\langle v_k, v_k \rangle} v_k,
\]
that is, the coordinates of \( v \) with respect to the basis \( B \) are given by
\[
\left( \frac{\langle v, v_k \rangle}{\|v_k\|^2} \right)_{1 \leq k \leq n}^T.
\]


\subsection{Orthogonal Projection and Direct Sum Decomposition}

Let \( B = (v_1, \ldots, v_m) \) be an orthogonal system in \( V \), and let \( U = L(B) \) be the subspace of \( V \) spanned by \( B \).

\begin{itemize}[label=\(-\)]
    \item For every \( v \in V \), the orthogonal projection of \( v \) onto \( U \) is given by:
    \[
    p_U(v) = \sum_{i=1}^{m} \frac{\langle v, v_i \rangle}{\langle v_i, v_i \rangle} v_i.
    \]

    \item Every vector \( v \in V \) can be uniquely written as the sum \( v = p_U(v) + w \), with \( w \in U^\perp \). In this case, we have:
    \[
    w = v - p_U(v).
    \]

    \item \( V = U \oplus U^\perp \), which means that every vector in \( V \) can be uniquely decomposed into a sum of a vector from \( U \) and a vector from \( U^\perp \).

    \item If \( \dim(V) = n \), then:
    \[
    \dim(U) + \dim(U^\perp) = n \quad \text{for every subspace} \, U.
    \]
\end{itemize}

\subsection{The Gram-Schmidt Process}
The Gram-Schmidt process is a method for orthonormalizing a set of vectors in an inner product space. Given a finite set of linearly independent vectors \( (v_1, \ldots, v_n) \), the process generates an orthonormal basis \( (u_1, \ldots, u_n) \) as follows:
\begin{enumerate}
    \item Set \( u_1 = \frac{v_1}{\|v_1\|} \).
    \item For \( k = 2, \ldots, n \):
    \begin{enumerate}
        \item Set \( w_k = v_k - \sum_{j=1}^{k-1} \langle v_k, u_j \rangle u_j \).
        \item Set \( u_k = \frac{w_k}{\|w_k\|} \).
    \end{enumerate}
    \item The resulting set \( (u_1, \ldots, u_n) \) is an orthonormal basis of the subspace spanned by \( (v_1, \ldots, v_n) \).
\end{enumerate}
\textbf{Note:} The Gram-Schmidt process can be applied to any finite set of linearly independent vectors in an inner product space, and it is particularly useful for constructing orthonormal bases in Euclidean spaces.
\textbf{Example:} Let \( v_1 = (1, 0, 0) \), \( v_2 = (1, 1, 0) \), and \( v_3 = (1, 1, 1) \). The Gram-Schmidt process yields:
\[
u_1 = \frac{v_1}{\|v_1\|} = (1, 0, 0), \quad
u_2 = \frac{v_2 - \langle v_2, u_1 \rangle u_1}{\|v_2 - \langle v_2, u_1 \rangle u_1\|} = \left(0, 1, 0\right), \quad
\]
\[
u_3 = \frac{v_3 - \langle v_3, u_1 \rangle u_1 - \langle v_3, u_2 \rangle u_2}{\|v_3 - \langle v_3, u_1 \rangle u_1 - \langle v_3, u_2 \rangle u_2\|} = \left(0, 0, 1\right).
\]
Thus, the orthonormal basis is \( (1, 0, 0), (0, 1, 0), (0, 0, 1) \).

\subsection{Existence of Orthonormal Bases and Orthogonal Projections}

\begin{itemize}[label=\(-\)]

    \item Every finitely generated unitary vector space has an orthonormal basis.

    \item I is, in general, false for vector spaces that are not finitely generated.

    \item We now provide the existence proof of the orthogonal projection.

    \item Let \( V \) be a unitary vector space and \( U \) a finitely generated subspace. Then for every \( v \in V \), the orthogonal projection \( p_U(v) \) of \( v \) onto \( U \) exists.
    
    \item Let \( V \) be a finitely generated unitary vector space and \( U \) any subspace. Then we have \( V = U \oplus U^\perp \), and
    \[
    \dim(V) = \dim(U) + \dim(U^\perp).
    \]
    
    \item Every hyperplane in \( \mathbb{R}^n \) admits a normal form; the normal vector is unique up to scalar multiplication.
    
    \item Let \( V \) be as above and let \( v_1, \ldots, v_m \in V \). If it is possible to construct orthonormal vectors \( w_1, \ldots, w_m \) from them using the Gram-Schmidt process, then \( (v_1, \ldots, v_m) \) are linearly independent.
    
\end{itemize}


\subsubsection{Visualization of the Gram-Schmidt process in 2 dimension}

\begin{center}
\begin{tikzpicture}

    \draw[<->] (5,0) -- (-5, 0);
    \draw[<->] (0,5) -- (0, -5);
    \draw[->]  (0,0) -- (2,2)node[below]{\(v1\)};
    \draw[->]  (0,0) -- (1, 3)node[above]{\(v2\)};
    \draw[-, orange] (2,2) -- (1, 3)node[right]{\(v2 - \alpha v1\)};
    \draw[->,red]  (0,0) -- (-1, 1)node[below]{\(w2\)};
    \draw[->,blue]  (0,0) -- (1,1)node[below]{\(w1\)};

\end{tikzpicture}
\end{center}

\subsection{Best Approximation}

Let \(V\) be a unitary vector space and \(U\) a finitely generated subspace of \(V\).
A vector \(v* \in U\) is called Best Approximation in \(U\) on \(v\), if:
\[
\|v* - v\| = \inf_{x \in U}\|x - v\|
\]

 also

\[
\|v* - P_u(v)\| = \inf_{u \in U}\|u - v\|
\]

\newpage

