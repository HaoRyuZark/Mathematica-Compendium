\newpage
\section{Matrices}

In this section, we explore the fundamental concepts of matrices, their operations, and important properties that form the foundation of linear algebra.

\subsection{Definition of a Matrix}

A matrix is a rectangular array of numbers, symbols, or expressions arranged in rows and columns. Formally, an \(m \times n\) matrix \(A\) consists of \(mn\) elements \(a_{ij}\) where \(i = 1, 2, \ldots, m\) and \(j = 1, 2, \ldots, n\):

\begin{equation*}
A = 
\begin{pmatrix}
a_{11} & a_{12} & \cdots & a_{1n} \\
a_{21} & a_{22} & \cdots & a_{2n} \\
\vdots & \vdots & \ddots & \vdots \\
a_{m1} & a_{m2} & \cdots & a_{mn}
\end{pmatrix}
\end{equation*}

The set of all \(m \times n\) matrices with real entries is denoted by \(\mathbb{R}^{m \times n}\). Special cases include:
\begin{itemize}[label=\(-\)]
    \item Square matrix: matrix with the same number of rows and columns (\(m = n\))
    \item Column vector: \(m \times 1\) matrix
    \item Row vector: \(1 \times n\) matrix
    \item Identity matrix \(I_n\): An \(n \times n\) matrix with ones on the main diagonal and zeros elsewhere
    \item Zero matrix: A matrix where all entries are zero
\end{itemize}

\subsection{Matrix Addition and Subtraction}

Matrix addition and subtraction are defined for matrices of the same dimensions.

\subsubsection*{Addition}

For matrices \(A, B \in \mathbb{R}^{m \times n}\), their sum \(C = A + B\) is defined as:
\begin{equation*}
c_{ij} = a_{ij} + b_{ij} \quad \text{for all } i = 1, 2, \ldots, m \text{ and } j = 1, 2, \ldots, n
\end{equation*}

\subsubsection*{Subtraction}

Similarly, the difference \(C = A - B\) is defined as:
\begin{equation*}
c_{ij} = a_{ij} - b_{ij} \quad \text{for all } i = 1, 2, \ldots, m \text{ and } j = 1, 2, \ldots, n
\end{equation*}

Matrix addition satisfies the following properties:
\begin{align*}
A + B &= B + A \quad \text{(Commutativity)} \\
(A + B) + C &= A + (B + C) \quad \text{(Associativity)} \\
A + O &= A \quad \text{(Identity element)} \\
A + (-A) &= O \quad \text{(Inverse element)}
\end{align*}
where \(O\) is the zero matrix.

\subsection{Matrix Multiplication}

Matrix multiplication is defined between matrices where the number of columns in the first matrix equals the number of rows in the second matrix.

For \(A \in \mathbb{R}^{m \times p}\) and \(B \in \mathbb{R}^{p \times n}\), their product \(C = AB \in \mathbb{R}^{m \times n}\) is defined as:
\begin{equation*}
c_{ij} = \sum_{k=1}^{p} a_{ik}b_{kj} = a_{i1}b_{1j} + a_{i2}b_{2j} + \cdots + a_{ip}b_{pj}
\end{equation*}

Matrix multiplication satisfies the following properties:
\begin{align*}
A(BC) &= (AB)C \quad \text{(Associativity)} \\
A(B+C) &= AB + AC \quad \text{(Left distributivity)} \\
(A+B)C &= AC + BC \quad \text{(Right distributivity)} \\
AI_n &= A \quad \text{and} \quad I_m A = A \quad \text{(Identity)}
\end{align*}

Note that matrix multiplication is generally not commutative, i.e., \(AB \neq BA\) in most cases.

\subsection{The Transpose of a Matrix}

The transpose of a matrix \(A \in \mathbb{R}^{m \times n}\), denoted \(A^T \in \mathbb{R}^{n \times m}\), is obtained by interchanging rows and columns:
\begin{equation*}
(A^T)_{ij} = a_{ji} \quad \text{for all } i = 1, 2, \ldots, n \text{ and } j = 1, 2, \ldots, m
\end{equation*}

Properties of the transpose include:
\begin{align*}
(A^T)^T &= A \\
(A + B)^T &= A^T + B^T \\
(AB)^T &= B^T A^T \\
(\alpha A)^T &= \alpha A^T \quad \text{for any scalar } \alpha
\end{align*}

Special matrices related to the transpose include:
\begin{itemize}[label=\(-\)]
    \item Symmetric matrix: \(A = A^T\)
    \item Skew-symmetric matrix: \(A = -A^T\)
\end{itemize}

\subsection{The Equivalence of Matrices}

Two matrices \(A\) and \(B\) are said to be equivalent if one can be transformed into the other through a finite sequence of elementary row operations. We write \(A \sim B\) to denote this equivalence.

The elementary row operations are:
\begin{itemize}[label=\(-\)]
    \item Interchanging two rows: \(R_i \leftrightarrow R_j\)
    \item Multiplying a row by a non-zero scalar: \(R_i \mapsto \alpha R_i\) where \(\alpha \neq 0\)
    \item Adding a multiple of one row to another: \(R_i \mapsto R_i + \alpha R_j\) where \(i \neq j\)
\end{itemize}

Matrix equivalence is an equivalence relation, satisfying reflexivity, symmetry, and transitivity. Equivalent matrices represent the same linear system in different bases.

\subsubsection{Row Echelon Form (REF)}

A matrix is in row echelon form if:
\begin{itemize}[label=\(-\)]
    \item All rows consisting entirely of zeros are at the bottom of the matrix.
    \item The leading entry (first non-zero element) of each non-zero row is to the right of the leading entry of the row above it.
    \item All entries in a column below a leading entry are zeros.
\end{itemize}

\subsubsection{Reduced Row Echelon Form (RREF)}
A matrix is in reduced row echelon form if:
\begin{itemize}[label=\(-\)]
    \item It is in row echelon form.
    \item Each leading entry is 1.
    \item Each leading entry is the only non-zero entry in its column.
\end{itemize}

The Gauß-Jordan elimination algorithm proceeds as follows:
\begin{itemize}[label=\(-\)]
    \item Start with the leftmost non-zero column.
    \item Find the pivot (non-zero element) in this column. If necessary, swap rows to move a non-zero element to the pivot position.
    \item Divide the pivot row by the pivot value to make the pivot equal to 1.
    \item Eliminate all other entries in the pivot column by subtracting appropriate multiples of the pivot row.
    \item Cover the pivot row and column, and repeat steps 1-4 on the submatrix until all rows are processed.
    \item For RREF, eliminate all entries above each pivot as well.
\end{itemize}

\subsection{The Inverse of a Matrix and Its Properties}

For a square matrix \(A \in \mathbb{R}^{n \times n}\), the inverse matrix \(A^{-1}\) (if it exists) satisfies:
\begin{equation*}
A A^{-1} = A^{-1} A = I_n
\end{equation*}

\subsubsection{Properties of the Inverse}
\begin{align*}
(A^{-1})^{-1} &= A \\
(AB)^{-1} &= B^{-1}A^{-1} \\
(A^T)^{-1} &= (A^{-1})^T \\
\det(A^{-1}) &= \frac{1}{\det(A)}
\end{align*}

\subsubsection{Finding the Inverse}
There are several methods to find the inverse of a matrix:

\emph{Gauß-Jordan Method} 

Form the augmented matrix \([A|I_n]\) and apply Gauß-Jordan elimination to transform it into \([I_n|A^{-1}]\):
\begin{enumerate}
    \item Create the augmented matrix \([A|I_n]\)
    \item Apply row operations to transform the left side into \(I_n\)
    \item The right side will be \(A^{-1}\)
\end{enumerate}

\emph{Adjoint Method}

For an \(n \times n\) matrix \(A\):
\begin{equation*}
A^{-1} = \frac{1}{\det(A)} \text{adj}(A)
\end{equation*}
where \(\text{adj}(A)\) is the adjoint (or adjugate) of \(A\), defined as the transpose of the cofactor matrix.

A matrix is invertible if and only if its determinant is non-zero. Such matrices are called non-singular or regular matrices.

\subsection{The Rank of a Matrix and How to Find It}

The rank of a matrix \(A\), denoted \(\text{rank}(A)\) or \(\text{rg}(A)\), is the dimension of the column space (or equivalently, the row space) of \(A\).

Equivalent definitions of rank include:
\begin{itemize}[label=\(-\)]
    \item The maximum number of linearly independent columns of \(A\)
    \item The maximum number of linearly independent rows of \(A\)
    \item The order of the largest non-zero minor of \(A\)
    \item The number of non-zero rows in any row echelon form of \(A\)
\end{itemize}

\subsubsection{Finding the Rank}

To find the rank of a matrix:
\begin{enumerate}
    \item Transform the matrix into row echelon form using Gauß-Jordan elimination
    \item Count the number of non-zero rows in the resulting matrix
\end{enumerate}

Properties of rank include:
\begin{align*}
&\text{rank}(A) \leq \min(m,n) \text{ for } A \in \mathbb{R}^{m \times n} \\
&\text{rank}(A^T) = \text{rank}(A) \\
&\text{rank}(AB) \leq \min(\text{rank}(A), \text{rank}(B)) \\
&\text{rank}(A+B) \leq \text{rank}(A) + \text{rank}(B)
\end{align*}

For a square matrix \(A \in \mathbb{R}^{n \times n}\), the following are equivalent:
\begin{itemize}[label=\(-\)]
    \item \(A\) is invertible
    \item \(\text{rank}(A) = n\)
    \item \(\det(A) \neq 0\)
    \item The columns of \(A\) are linearly independent
    \item The rows of \(A\) are linearly independent
    \item \(Ax = 0\) has only the trivial solution \(x = 0\)
\end{itemize}

\subsection{The Definitions of Column Space, Row Space, and Null Space}

These fundamental spaces associated with a matrix \(A \in \mathbb{R}^{m \times n}\) provide important insights into its structure.

\subsubsection{Column Space}
The column space of \(A\), denoted \(\text{Col}(A)\), is the span of the columns of \(A\):
\begin{equation*}
\text{Col}(A) = \{\vec{y} \in \mathbb{R}^m : \vec{y} = A\vec{x} \text{ for some } \vec{x} \in \mathbb{R}^n\}
\end{equation*}

This is also called the range or image of the linear transformation represented by \(A\). The dimension of the column space equals the rank of \(A\).

\subsubsection{Row Space}
The row space of \(A\), denoted \(\text{Row}(A)\), is the span of the rows of \(A\):
\begin{equation*}
\text{Row}(A) = \text{Col}(A^T)
\end{equation*}

The dimension of the row space also equals the rank of \(A\).

\subsubsection{Null Space}
The null space (or kernel) of \(A\), denoted \(\text{Null}(A)\) or \(\text{Ker}(A)\), is the set of all vectors that \(A\) maps to zero:
\begin{equation*}
\text{Null}(A) = \{\vec{x} \in \mathbb{R}^n : A\vec{x} = \vec{0}\}
\end{equation*}

The dimension of the null space is called the nullity of \(A\), denoted \(\text{nullity}(A)\).

\subsubsection{Left Null Space}
The left null space of \(A\) is the null space of \(A^T\):
\begin{equation*}
\text{Null}(A^T) = \{\vec{y} \in \mathbb{R}^m : A^T\vec{y} = \vec{0}\} = \{\vec{y} \in \mathbb{R}^m : \vec{y}^T A = \vec{0}^T\}
\end{equation*}

The Rank-Nullity Theorem connects these spaces:
\begin{equation*}
\text{rank}(A) + \text{nullity}(A) = n
\end{equation*}

To find a basis for these spaces:
\begin{itemize}[label=\(-\)]
    \item Column space: Take the linearly independent columns of \(A\)
    \item Row space: Take the non-zero rows from any row echelon form of \(A\)
    \item Null space: Solve the homogeneous system \(A\vec{x} = \vec{0}\) and express the general solution in terms of free variables
\end{itemize}
\subsection{Examples of Matrix Operations}

In this subsection, we provide detailed examples of Gauß-Jordan elimination and matrix multiplication to illustrate these fundamental matrix operations.

\subsubsection{Example of Matrix Multiplication}

Consider the matrices \(A\) and \(B\) given by:

\begin{equation*}
A = 
\begin{pmatrix}
2 & 3 & 1 \\
1 & 0 & -2
\end{pmatrix} \in \mathbb{R}^{2 \times 3}
\quad \text{and} \quad
B = 
\begin{pmatrix}
1 & 2 \\
-1 & 3 \\
4 & 0
\end{pmatrix} \in \mathbb{R}^{3 \times 2}
\end{equation*}

To compute the product \(C = AB \in \mathbb{R}^{2 \times 2}\), we calculate each entry \(c_{ij}\) using the formula:

\begin{equation*}
c_{ij} = \sum_{k=1}^{3} a_{ik} b_{kj}
\end{equation*}

Let's calculate each entry of \(C\):

\begin{align*}
c_{11} &= a_{11}b_{11} + a_{12}b_{21} + a_{13}b_{31} \\
&= 2 \cdot 1 + 3 \cdot (-1) + 1 \cdot 4 \\
&= 2 - 3 + 4 = 3
\end{align*}

\begin{align*}
c_{12} &= a_{11}b_{12} + a_{12}b_{22} + a_{13}b_{32} \\
&= 2 \cdot 2 + 3 \cdot 3 + 1 \cdot 0 \\
&= 4 + 9 + 0 = 13
\end{align*}

\begin{align*}
c_{21} &= a_{21}b_{11} + a_{22}b_{21} + a_{23}b_{31} \\
&= 1 \cdot 1 + 0 \cdot (-1) + (-2) \cdot 4 \\
&= 1 + 0 - 8 = -7
\end{align*}

\begin{align*}
c_{22} &= a_{21}b_{12} + a_{22}b_{22} + a_{23}b_{32} \\
&= 1 \cdot 2 + 0 \cdot 3 + (-2) \cdot 0 \\
&= 2 + 0 + 0 = 2
\end{align*}

Therefore, the product \(C = AB\) is:
\begin{equation*}
C = AB = 
\begin{pmatrix}
3 & 13 \\
-7 & 2
\end{pmatrix}
\end{equation*}

Let's verify that matrix multiplication is not generally commutative by attempting to compute \(BA\):
\\\\
Since \(B\) is a \(3 \times 2\) matrix and \(A\) is a \(2 \times 3\) matrix, the product \(BA\) would be a \(3 \times 3\) matrix. However, this calculation cannot be performed since the number of columns in \(B\) (which is 2) does not equal the number of rows in \(A\) (which is 2). Thus, \(BA\) is undefined, demonstrating that matrix multiplication is not always commutative.
\\\\
\textbf{Example of Gauß-Jordan Elimination}
\\\\
We'll use Gauß-Jordan elimination to solve the linear system:
\begin{align*}
2x + y - z &= 8 \\
-3x - y + 2z &= -11 \\
x + y + z &= 3
\end{align*}

First, we set up the augmented matrix:
\begin{equation*}
\begin{pmatrix}
2 & 1 & -1 & | & 8 \\
-3 & -1 & 2 & | & -11 \\
1 & 1 & 1 & | & 3
\end{pmatrix}
\end{equation*}

Now we apply Gauß-Jordan elimination to transform this into reduced row echelon form:
\\\\
\textbf{Step 1:} We'll choose the first element in the first row as our pivot. Let's first swap row 1 and row 3 to get a simpler pivot:

\begin{equation*}
\begin{pmatrix}
1 & 1 & 1 & | & 3 \\
-3 & -1 & 2 & | & -11 \\
2 & 1 & -1 & | & 8
\end{pmatrix}
\end{equation*}

\textbf{Step 2:} Eliminate the first elements in rows 2 and 3:
\\\\
Row 2 + 3 \(\times\) Row 1:
\begin{equation*}
\begin{pmatrix}
1 & 1 & 1 & | & 3 \\
0 & 2 & 5 & | & -2 \\
2 & 1 & -1 & | & 8
\end{pmatrix}
\end{equation*}

Row 3 - 2 \(\times\) Row 1:
\begin{equation*}
\begin{pmatrix}
1 & 1 & 1 & | & 3 \\
0 & 2 & 5 & | & -2 \\
0 & -1 & -3 & | & 2
\end{pmatrix}
\end{equation*}

\textbf{Step 3:} Make the pivot in row 2 equal to 1 by dividing the entire row by 2:
\begin{equation*}
\begin{pmatrix}
1 & 1 & 1 & | & 3 \\
0 & 1 & \frac{5}{2} & | & -1 \\
0 & -1 & -3 & | & 2
\end{pmatrix}
\end{equation*}

\textbf{Step 4:} Eliminate the second element in rows 1 and 3:
\\\\
Row 1 - Row 2:
\begin{equation*}
\begin{pmatrix}
1 & 0 & -\frac{3}{2} & | & 4 \\
0 & 1 & \frac{5}{2} & | & -1 \\
0 & -1 & -3 & | & 2
\end{pmatrix}
\end{equation*}

Row 3 + Row 2:
\begin{equation*}
\begin{pmatrix}
1 & 0 & -\frac{3}{2} & | & 4 \\
0 & 1 & \frac{5}{2} & | & -1 \\
0 & 0 & -\frac{1}{2} & | & 1
\end{pmatrix}
\end{equation*}

\textbf{Step 5:} Make the pivot in row 3 equal to 1 by multiplying the entire row by -2:
\begin{equation*}
\begin{pmatrix}
1 & 0 & -\frac{3}{2} & | & 4 \\
0 & 1 & \frac{5}{2} & | & -1 \\
0 & 0 & 1 & | & -2
\end{pmatrix}
\end{equation*}

\textbf{Step 6:} Eliminate the third element in rows 1 and 2:
\\\\
Row 1 + \(\frac{3}{2} \times\) Row 3:
\begin{equation*}
\begin{pmatrix}
1 & 0 & 0 & | & 1 \\
0 & 1 & \frac{5}{2} & | & -1 \\
0 & 0 & 1 & | & -2
\end{pmatrix}
\end{equation*}

Row 2 - \(\frac{5}{2} \times\) Row 3:
\begin{equation*}
\begin{pmatrix}
1 & 0 & 0 & | & 1 \\
0 & 1 & 0 & | & -1 + 5 = 4 \\
0 & 0 & 1 & | & -2
\end{pmatrix}
\end{equation*}

The matrix is now in reduced row echelon form:
\begin{equation*}
\begin{pmatrix}
1 & 0 & 0 & | & 1 \\
0 & 1 & 0 & | & 4 \\
0 & 0 & 1 & | & -2
\end{pmatrix}
\end{equation*}

This corresponds to the system:
\begin{align*}
x &= 1 \\
y &= 4 \\
z &= -2
\end{align*}

Therefore, the solution to the original system is \(x = 1\), \(y = 4\), and \(z = -2\).
\\\\
\textbf{Example of Finding the Inverse of a Matrix using Gauß-Jordan Elimination}
\\\\
Let's find the inverse of the matrix:
\begin{equation*}
A = 
\begin{pmatrix}
2 & 1 & 1 \\
3 & 2 & 1 \\
2 & 1 & 2
\end{pmatrix}
\end{equation*}

We form the augmented matrix \([A|I_3]\):
\begin{equation*}
\begin{pmatrix}
2 & 1 & 1 & | & 1 & 0 & 0 \\
3 & 2 & 1 & | & 0 & 1 & 0 \\
2 & 1 & 2 & | & 0 & 0 & 1
\end{pmatrix}
\end{equation*}

Now we apply Gauß-Jordan elimination:
\\\\
\textbf{Step 1:} Make the first pivot equal to 1 by dividing the first row by 2:
\begin{equation*}
\begin{pmatrix}
1 & \frac{1}{2} & \frac{1}{2} & | & \frac{1}{2} & 0 & 0 \\
3 & 2 & 1 & | & 0 & 1 & 0 \\
2 & 1 & 2 & | & 0 & 0 & 1
\end{pmatrix}
\end{equation*}

\textbf{Step 2:} Eliminate the first element in rows 2 and 3:
\\\\
Row 2 - 3 \(\times\) Row 1:
\begin{equation*}
\begin{pmatrix}
1 & \frac{1}{2} & \frac{1}{2} & | & \frac{1}{2} & 0 & 0 \\
0 & \frac{1}{2} & -\frac{1}{2} & | & -\frac{3}{2} & 1 & 0 \\
2 & 1 & 2 & | & 0 & 0 & 1
\end{pmatrix}
\end{equation*}

Row 3 - 2 \(\times\) Row 1:
\begin{equation*}
\begin{pmatrix}
1 & \frac{1}{2} & \frac{1}{2} & | & \frac{1}{2} & 0 & 0 \\
0 & \frac{1}{2} & -\frac{1}{2} & | & -\frac{3}{2} & 1 & 0 \\
0 & 0 & 1 & | & -1 & 0 & 1
\end{pmatrix}
\end{equation*}

\textbf{Step 3:} Make the second pivot equal to 1 by multiplying the second row by 2:
\begin{equation*}
\begin{pmatrix}
1 & \frac{1}{2} & \frac{1}{2} & | & \frac{1}{2} & 0 & 0 \\
0 & 1 & -1 & | & -3 & 2 & 0 \\
0 & 0 & 1 & | & -1 & 0 & 1
\end{pmatrix}
\end{equation*}

\textbf{Step 4:} Eliminate the second element in row 1 and the third element in row 2:
\\\\
Row 1 - \(\frac{1}{2} \times\) Row 2:
\begin{equation*}
\begin{pmatrix}
1 & 0 & 1 & | & 2 & -1 & 0 \\
0 & 1 & -1 & | & -3 & 2 & 0 \\
0 & 0 & 1 & | & -1 & 0 & 1
\end{pmatrix}
\end{equation*}

Row 2 + Row 3:
\begin{equation*}
\begin{pmatrix}
1 & 0 & 1 & | & 2 & -1 & 0 \\
0 & 1 & 0 & | & -4 & 2 & 1 \\
0 & 0 & 1 & | & -1 & 0 & 1
\end{pmatrix}
\end{equation*}

\textbf{Step 5:} Eliminate the third element in row 1:
\\\\
Row 1 - Row 3:
\begin{equation*}
\begin{pmatrix}
1 & 0 & 0 & | & 3 & -1 & -1 \\
0 & 1 & 0 & | & -4 & 2 & 1 \\
0 & 0 & 1 & | & -1 & 0 & 1
\end{pmatrix}
\end{equation*}

The right side of the augmented matrix now gives us \(A^{-1}\):
\begin{equation*}
A^{-1} = 
\begin{pmatrix}
3 & -1 & -1 \\
-4 & 2 & 1 \\
-1 & 0 & 1
\end{pmatrix}
\end{equation*}

To verify, we can check that \(AA^{-1} = I_3\):
\begin{align*}
AA^{-1} &= 
\begin{pmatrix}
2 & 1 & 1 \\
3 & 2 & 1 \\
2 & 1 & 2
\end{pmatrix}
\begin{pmatrix}
3 & -1 & -1 \\
-4 & 2 & 1 \\
-1 & 0 & 1
\end{pmatrix} \\
&= 
\begin{pmatrix}
2 \cdot 3 + 1 \cdot (-4) + 1 \cdot (-1) & 2 \cdot (-1) + 1 \cdot 2 + 1 \cdot 0 & 2 \cdot (-1) + 1 \cdot 1 + 1 \cdot 1 \\
3 \cdot 3 + 2 \cdot (-4) + 1 \cdot (-1) & 3 \cdot (-1) + 2 \cdot 2 + 1 \cdot 0 & 3 \cdot (-1) + 2 \cdot 1 + 1 \cdot 1 \\
2 \cdot 3 + 1 \cdot (-4) + 2 \cdot (-1) & 2 \cdot (-1) + 1 \cdot 2 + 2 \cdot 0 & 2 \cdot (-1) + 1 \cdot 1 + 2 \cdot 1
\end{pmatrix} \\
&= 
\begin{pmatrix}
6 - 4 - 1 & -2 + 2 + 0 & -2 + 1 + 1 \\
9 - 8 - 1 & -3 + 4 + 0 & -3 + 2 + 1 \\
6 - 4 - 2 & -2 + 2 + 0 & -2 + 1 + 2
\end{pmatrix} \\
&= 
\begin{pmatrix}
1 & 0 & 0 \\
0 & 1 & 0 \\
0 & 0 & 1
\end{pmatrix} = I_3
\end{align*}

This confirms that we have correctly found the inverse of matrix \(A\).

\subsection{Linear Maps as matrices and Their Properties}

Let \(V\) and \(W\) be vector spaces over the field \(K\).
\\\\
Let \(v_1, \dots, v_n \in V\) and \(w_1, \dots, w_n \in W\). If \((v_1, \dots, v_n)\) forms a basis of \(V\), then there exists a unique \(f \in \text{Hom}(V, W)\) with \(f(v_i) = w_i\), \(1 \leq i \leq n\). The map \(f\) has the following properties:

1.  \(\text{Im}(f) = \text{span}(f(v_1), \dots, f(v_n))\).

2.  \(f\) is injective \(\Leftrightarrow\) \(w_1, \dots, w_n\) are linearly independent.

 Let \(V\) and \(W\) be two \(K\)-vector spaces, \(B_V = (v_1, \dots, v_n)\) a basis of \(V\) and \(B_W = (w_1, \dots, w_m)\) a basis of \(W\), and let \(f : V \rightarrow W\) be linear. Then there exists a unique matrix \(M_{B_V}^{B_W}(f) = (a_{ij}) \in K^{m \times n}\) with
\[
f(v_j) = \sum_{i=1}^{m} a_{ij}w_i \quad \forall j = 1, \dots, n
\]

\subsection{Example of an exercise}

\begin{enumerate}
    \item Determination of the kernel
    \item Determination of the dimension of the kernel
    \item Determination of the rank (dimension formula)
    \item Determination of the image
\end{enumerate}


\textbf{Example:} 

Given
\[
f\begin{pmatrix}
x_1 \\
x_2 \\
x_3
\end{pmatrix} =
\begin{pmatrix}
2x_1 + x_2 \\
x_1 - x_2 + x_3 \\
4x_1 - x_2 + 2x_3
\end{pmatrix} .
\]

It should be shown that \(f\) is linear, and \(\ker(f)\), \(\text{Im}(f)\) and their dimensions should be determined.

A direct proof of linearity is easily possible. Instead, we give the transformation matrix \(A\). The images of the (canonical) basis vectors are
\[
f\begin{pmatrix}
1 \\
0 \\
0
\end{pmatrix} =
\begin{pmatrix}
2 \\
1 \\
4
\end{pmatrix} , \quad
f\begin{pmatrix}
0 \\
1 \\
0
\end{pmatrix} =
\begin{pmatrix}
1 \\
-1 \\
-1
\end{pmatrix} , \quad
f\begin{pmatrix}
0 \\
0 \\
1
\end{pmatrix} =
\begin{pmatrix}
0 \\
1 \\
2
\end{pmatrix} .
\]
We obtain
\[
A =
\begin{pmatrix}
2 & 1 & 0 \\
1 & -1 & 1 \\
4 & -1 & 2
\end{pmatrix} .
\]
But now it must be shown that indeed \(f(x) = Ax \quad \forall x \in \mathbb{R}^3\) holds, by, 
for example, calculating both \(Ax\) and \(f(x)\) for a 
general \(x\) and showing equality: Here, with \(x = (x_1, x_2, x_3)^T\)
\[
A \cdot x =
\begin{pmatrix}
2 & 1 & 0 \\
1 & -1 & 1 \\
4 & -1 & 2
\end{pmatrix}
\cdot
\begin{pmatrix}
x_1 \\
x_2 \\
x_3
\end{pmatrix} =
\begin{pmatrix}
2x_1 + x_2 \\
x_1 - x_2 + x_3 \\
4x_1 - x_2 + 2x_3
\end{pmatrix} .
\]
This obviously corresponds to \(f(x)\), so that by Theorem 4.36, the map \(f\) is linear. To determine the kernel, one has to solve the system of linear equations
\[
\begin{array}{cccc}
2 & 1 & 0 & 0 \\
1 & -1 & 1 & 0 \\
4 & -1 & 2 & 0
\end{array}
\]
which corresponds to the equation \(Ax = f(x) = 0\). Gaussian elimination yields \(x_3 = \lambda'\); \(x_2 = \frac{2}{3} \cdot \lambda'\); \(x_1 = -\frac{1}{3} \cdot \lambda'\), thus with \(\lambda = \frac{1}{3} \lambda'\):
\[
\ker(f) =
\left\{
x = \lambda
\begin{pmatrix}
-1 \\
2 \\
3
\end{pmatrix}
\, \middle| \, \lambda \in \mathbb{R}
\right\}
\]
It follows that \(\dim(\ker(f)) = 1\) and because of \(\dim(V) = 3\) from the dimension 
formula \(\dim(\text{Im}(f)) = 2\). By a corallary, \(\text{Im}(f)\) corresponds to the linear 
span of the columns of the matrix. One chooses consequently \(\dim(\text{Im}(f))\) column vectors, e.g., the first ones, and tests if they are linearly independent. In the concrete case, this is obvious, because the second column is 
not a multiple of the first. It follows therefore
\[
\text{Im}(f) =
\left\{
x \, \middle| \, x = \lambda
\begin{pmatrix}
2 \\
1 \\
4
\end{pmatrix} + \mu
\begin{pmatrix}
1 \\
-1 \\
-1
\end{pmatrix}
, \quad \lambda, \mu \in \mathbb{R}
\right\}
\]

