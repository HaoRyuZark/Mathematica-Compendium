\section{Algebraic Structures}

\subsection{Introduction}

Algebraic structures are mathematical systems consisting of a set equipped with one or more operations that satisfy certain axioms. They provide a unified language to study various objects in mathematics, from numbers and matrices to functions and vector spaces. Understanding these structures is fundamental in abstract algebra and has applications in computer science, cryptography, coding theory, and physics.

\subsection{Operations: Internal and External}

An \textbf{internal composition law} is a binary operation that takes two elements from a set and returns another element in the same set. Formally, for a set $S$ and operation $\circ$, we have:
\[
\circ: S \times S \rightarrow S
\]

An \textbf{external composition law} involves a second set acting on the structure, such as scalar multiplication in vector spaces:
\[
\cdot: K \times V \rightarrow V
\]
where $K$ is a field and $V$ is a vector space.

\subsection{Properties of Operations}

Let $\ast$ be a binary operation on a set $S$. The most important properties include:

\begin{itemize}
    \item \textbf{Associativity:} $(a \ast b) \ast c = a \ast (b \ast c)$ for all $a,b,c \in S$
    \item \textbf{Commutativity:} $a \ast b = b \ast a$ for all $a,b \in S$
    \item \textbf{Identity Element:} There exists $e \in S$ such that $a \ast e = e \ast a = a$ for all $a \in S$
    \item \textbf{Inverse Element:} For every $a \in S$, there exists $a^{-1} \in S$ such that $a \ast a^{-1} = a^{-1} \ast a = e$
    \item \textbf{Distributivity:} $a \circ (b \bullet c) = (a \circ b) \bullet (a \circ c)$ and/or $(b \bullet c) \circ a = (b \circ a) \bullet (c \circ a)$
\end{itemize}

\subsection{Homomorphisms and Isomorphisms}

Let $(G, \oplus)$ and $(H, \oplus')$ be two algebraic structures.

\begin{itemize}
    \item A \textbf{homomorphism} is a function $\varphi: G \rightarrow H$ such that:
    \[
    \varphi(a \oplus b) = \varphi(a) \oplus' \varphi(b), \quad \forall a,b \in G
    \]
    
    \item An \textbf{isomorphism} is a bijective homomorphism. If such a map exists, we say the structures are \textbf{isomorphic}, written as $G \cong H$.
\end{itemize}

\subsection{Common Algebraic Structures}

The following table lists common algebraic structures along with their notation and defining properties. Let $\oplus$ denote the additive operation and $\odot$ the multiplicative one:

\begin{center}
\renewcommand{\arraystretch}{1.4}
\begin{tabular}{|c|c|l|}
\hline
\textbf{Name} & \textbf{Notation} & \textbf{Properties} \\
\hline
\textbf{Semigroup} & $(S, \oplus)$ & Associative \\
\hline
\textbf{Monoid} & $(M, \oplus)$ & Associative, Identity element \\
\hline
\textbf{Group} & $(G, \oplus)$ & Associative, Identity, Inverses \\
\hline
\textbf{Abelian Group} & $(A, \oplus)$ & Group + Commutativity \\
\hline
\textbf{Ring} & $(R, \oplus, \odot)$ & \begin{tabular}[c]{@{}l@{}}$(R, \oplus)$ is an abelian group,\\ $(R, \odot)$ is a semigroup,\\ Distributivity: $a \odot (b \oplus c) = a \odot b \oplus a \odot c$\end{tabular} \\
\hline
\textbf{Commutative Ring} & $(R, \oplus, \odot)$ & Ring + $(R, \odot)$ is commutative \\
\hline
\textbf{Field} & $(K, \oplus, \odot)$ & \begin{tabular}[c]{@{}l@{}}Commutative Ring +\\ $(K \setminus \{0\}, \odot)$ is an abelian group\end{tabular} \\
\hline
\textbf{Vector Space} & $(V, \oplus, \cdot)$ & \begin{tabular}[c]{@{}l@{}}$(V, \oplus)$ is an abelian group,\\ $\cdot: K \times V \to V$ (scalar mult.),\\ Distributivity, associativity, identities\end{tabular} \\
\hline
\end{tabular}
\newpage
\end{center}
