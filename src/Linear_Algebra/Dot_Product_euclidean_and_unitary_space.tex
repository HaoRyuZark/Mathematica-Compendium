\newpage
\section{Dot Product, Euclidean and Unitary Space}

\subsection{Scalar Product}

Let \(V\) be a vector space over a field \(K\). A mapping \(\langle \cdot, \cdot \rangle : V \times V 
\to K\) is called an inner product (or inner product) if the following conditions are satisfied:

\emph{Symmetry}

For all \(a, b \in V\):

\[
    \langle a, b \rangle = 
    \begin{cases}
    \langle b, a \rangle & \text{if } K = \Reals, \\
    \overline{\langle b, a \rangle} & \text{if } K = \Complex.
    \end{cases}
\]

\emph{Linearity in the First Argument}

For all \(a, b, c \in V\):

\[
    \langle a, b + c \rangle = \langle a, b \rangle + \langle a, c \rangle
\]

and

\[
    \langle a + b, c \rangle = \langle a, c \rangle + \langle b, c \rangle.
\]

\emph{Homogeneity in the First Argument}

For all \(\alpha \in K\), we have:

\[
    \langle \alpha a, b \rangle = \alpha \langle a, b \rangle = 
    \begin{cases}
    \langle a, \alpha b \rangle & \text{if } K = \Reals, \\
    \langle a, \alpha b \rangle & \text{if } K = \Complex.
    \end{cases}
\]

\emph{Positive Definiteness}

For all \(a \in V \setminus \{0\}\):

\[
    \langle a, a \rangle > 0,
\]

and

\[
    \langle 0, 0 \rangle = 0.
\]

It also can be interpreted as a linear transformation that takes a vector and maps it to a real number
 via matrix vector multiplication. Also, a more practical way of thinking about the 
 \emph{dot product} is the question: How much are two vectors pointing in the same direction?

\subsection{Trigonometric Definition}

Think about the orthogonal projection as a triangle. Now, remember that to find the adjacent side \(x\) 
you need to take \(x = \cos(\theta)h\), which is the value \(\alpha\) in our projection formula 
\(p_a (b) = \frac{\langle a, b\rangle}{\langle a, a\rangle} \alpha a\). 
Here we add \(\|a\|\|b\|\) to make this function linear.

\[
    \langle a, b \rangle = \|a\| \|b\| \cos (\theta)
\]

\subsection{Standard Scalar Product for Complex Number}

Let \( a = {(a_i)}_{i=1}^n \) and \( b = {(b_i)}_{i=1}^n \) be vectors in \( \Complex^n \). The standard 
inner product is defined by

\[
    \langle a, b \rangle := \sum_{i=1}^n a_i b_i.
\]

\subsection{Scalar Product on \texorpdfstring{\( C[a, b] \)}{}}

Let \( f, g \in C[a, b] \). The inner product on \( C[a, b] \) is defined by

\[
    \langle f, g \rangle := \int_a^b f(x) \cdot g(x) \, dx.
\]

\subsection{Euclidean and Unitary Vector Spaces}

A real vector space equipped with an inner product is called a \emph{Euclidean vector space}, while a 
complex vector space with an inner product is called a \emph{unitary vector space}.

\subsection{Norms in Vector Spaces}

Let \( V \) be a \( K \)-vector space and \( a, b \in V \). A function \( \| \cdot \| : V \to \Reals \) 
is called a norm if and only if the following conditions hold:

\begin{itemize}

    \item \( \|a\| \in \Reals \),

    \item \( \|a\| \geq 0 \),

    \item \( \|a\| = 0 \iff a = 0 \),

    \item \( \forall \lambda \in K, \ \| \lambda a \| = |\lambda| \| a \| \),

    \item \( \| a + b \| \leq \| a \| + \| b \| \).

\end{itemize}

\subsubsection{Induced Norm by a Scalar Product}

As in the special case \( V = \Reals^n \), an inner product induces a norm.
In a unitary (or Euclidean) space, the inner product induces a (standard) norm defined by

\[
    \| \cdot \| = \sqrt{\langle \cdot, \cdot \rangle}.
\]

\subsection{Cauchy-Schwarz Inequality in Unitary Vector Spaces}

In all unitary vector spaces \( V \), the Cauchy-Schwarz inequality holds:

\[
    | \langle a, b \rangle | \leq \| a \| \| b \| \quad \forall a, b \in V.
\]

\subsubsection{Proof of the Triangle Inequality}

Both sides of the triangle inequality are real and, in particular, non-negative. 
Therefore, it is sufficient to prove that the squares of both sides satisfy the 
desired inequality, i.e., we need to show:

\[
    \langle a + b, a + b \rangle \leq {(\|a\| + \|b\|)}^2.
\]

First, we expand the left-hand side:

\[
    \langle a + b, a + b \rangle = \langle a, a \rangle + \langle a, b \rangle + \langle b, a \rangle + 
    \langle b, b \rangle.
\]

Since \( \langle b, a \rangle = \langle a, b \rangle \), we have:

\[
    \langle a, b \rangle + \langle b, a \rangle = 2 \, \text{Re} \langle a, b \rangle.
\]

Now, we know that the absolute value of a complex number is always greater than or equal to its 
real part, so:

\[
    2 \, \text{Re} \langle a, b \rangle \leq 2 |\langle a, b \rangle|.
\]

Using the Cauchy-Schwarz inequality, we can further bound this by:

\[
    2 \, \text{Re} \langle a, b \rangle \leq 2 \|a\| \|b\|.
\]

Thus, we have:

\[
    \langle a + b, a + b \rangle \leq \langle a, a \rangle + 2 \|a\| \|b\| + \langle b, b \rangle.
\]

Using the definition of the norm, \( \|a\|^2 = \langle a, a \rangle \) and \( \|b\|^2 = \langle b, b 
\rangle \), we obtain:

\[
    \langle a + b, a + b \rangle \leq \|a\|^2 + 2 \|a\| \|b\| + \|b\|^2.
\]

This is exactly the expansion of \( {(\|a\| + \|b\|)}^2 \), which completes the proof.

\QED

\subsection{Definiteness of Inner Products}

The concept of the \emph{inner product} plays a major rule in linear algebra, and there are various 
manifestations of this element. We are going to see how inner products are constructed based on the 
properties we are interested in. Those being linearity for the columns, symmetry and positive definiteness.

Let us start with a matrix \(A \in \Reals^{n \times n}\) and a map 
\((\centerdot , \centerdot)_A \Reals^n \times \Reals^n : \to \Reals \) more specific 
\((x,y)_A := \langle x, Ay\rangle\) using the standard Euclidean inner product.

For our first property, the linearity; it is granted for every matrix \(A = (a_1, a_2, \dots, a_n)\) 
because of the rules of matrix-vector multiplication. For example for the basis vectors \(e_i, e_j\) 
we get 

\[
    (e_i, e_j)_A = \langle e_i, Ae_j\rangle = e_{i}^{T} A e_j = e_{i}^{T} a_j = a_{ij} 
\]

Therefore, \((e_i, e_j)_A = (e_j, e_i)_A \iff a_{ij} = a_{ji}\). This means that \(A\) has to be symmetric.
As a side note, the choice for the Euclidean inner product for the structure of our map is optional. Any
kind of inner product would also do the job. This implies that each inner product can be described via a 
matrix.

\subsection{Definiteness and Quadratic Forms}

Given a symmetric \(A \in \Reals^{n \times n}\) then:

\begin{enumerate}
    
    \item The map \(x \to \langle x, Ax\rangle\) is called the \emph{Quadratic Form}
    
    \item \(A\) is \emph{positive definite} if \(\langle x, Ax \rangle > 0 \forall x \in \Reals^n 
          \backslash \{0\}\)

    \item \(A\) is \emph{negative definite} if \(\langle x, Ax \rangle < 0 \forall x \in \Reals^n 
          \backslash \{0\}\)
    
    \item \(A\) is \emph{positive semidefinite} if \(\langle x, Ax \rangle \ge 0 \forall x \in \Reals^n 
          \backslash \{0\}\)

    \item \(A\) is \emph{negative semidefinite} if \(\langle x, Ax \rangle \le 0 \forall x \in \Reals^n 
          \backslash \{0\}\)

    \item \(A\) is \emph{indefinite} if \(\exists x,y \in \Reals^n  \backslash \{0\}: \langle x, Ax\rangle 
          < 0 \land \langle y, Ay \rangle > 0 \)
    
\end{enumerate}

\((\centerdot , \centerdot)_A\) is only an inner product if \(A\) is symmetric and positive definite.

\textbf{Example:}

Given is the term \(x_{1}^{2} - x_{3}^{2} + x_1 x_4\), show that it is a quadratic form.

First write the formula for the quadratic form \(\langle x, Ax\rangle\) where \(A\) is real symmetric 
matrix

\[
    \left\langle 
    \begin{bmatrix}
        x_1 \\ x_2 \\ x_3 \\ x_4
    \end{bmatrix}
    ,
    \begin{bmatrix}
        a & b & c & d \\
        e & f & g & h \\
        i & j & k & l \\
        m & n & o & q  
    \end{bmatrix}
    \begin{bmatrix}
        x_1 \\ x_2 \\ x_3 \\ x_4
    \end{bmatrix}
    \right\rangle
\]

\[
    \left\langle 
    \begin{bmatrix}
        x_1 \\ x_2 \\ x_3 \\ x_4
    \end{bmatrix}
    ,
    \begin{bmatrix}
        ax_1  + bx_2 + cx_3 + dx_4\\ 
        ex_1  + fx_2 + gx_3 + hx_4\\ 
        ix_1  + jx_2 + kx_3 + lx_4\\ 
        mx_1  + nx_2 + ox_3 + px_4 
    \end{bmatrix}
    \right\rangle
\]

\[
    x_1(ax_1  + bx_2 + cx_3 + dx_4) + 
    x_2(ex_1  + fx_2 + gx_3 + hx_4) + 
    x_3(ix_1  + jx_2 + kx_3 + lx_4) +  
    x_4(mx_1  + nx_2 + ox_3 + px_4) 
\]

Now we have to compare the coefficients with the original expression 
\(x_{1}^{2} + (0)x_{2}^{2}  - x_{3}^{2} + x_1 x_4\).

\begin{align*}
    &ax_{1}^{2} + bx_1x_2 + cx_1x_3 + dx_4x_4 + \\ 
    &ex_2x_1  + fx_{2}^2 + gx_2x_3 + hx_2x_4 +  \\
    &ix_3x_1  + jx_3x_2 + kx_{3}^{2} + lx_3x_4) + \\
    &mx_4x_1  + nx_4x_2 + ox_4x_3 + px_{4}^{2}    
\end{align*}

Note that the only quadratic terms present in the expression are \(x_{1}^{2}\) and \(-x_{3}^{2}\) while the 
others are zero. This means that they have to disappear in the main diagonal. Also, the only term 
that \(x_1\) is also multiplying is \(x_4\) therefore all the others have to be zero like all 
other non-quadratic terms which are not present. Thus 

\[
    \begin{bmatrix}
        1 & 0 & 0 & d \\
        0 & 0 & 0 & 0 \\
        0 & 0 & -1 & 0 \\
        m & 0 & 0 & 0  
    \end{bmatrix}
\]

Now the question is what are \(d\) and \(m\) because \(A\) has to be symmetric, but they can not be 1 
because then we would have \(2x_1 x_4\). Because of that, we will take the only available option 
\(\frac{1}{2}\).

\[
    \begin{bmatrix}
        1 & 0 & 0 & \frac{1}{2} \\
        0 & 0 & 0 & 0 \\
        0 & 0 & -1 & 0 \\
        \frac{1}{2} & 0 & 0 & 0  
    \end{bmatrix}
\]

And now we are done.

\subsection{Minors}

For a matrix \(A \in \Reals^{n \times n}\) the starting from the left \(k \times k\)-submatrices 
\(A_k = (a_{ij})_{i,j=1}^{k}\) of \(A\). The \(D_k  = \det(A_k)\) is called a \emph{main minor} or 
\emph{minor} of \(A\).

\subsubsection{Minors and Definiteness}

Let \(A\) be real, square symmetric matrix, then

\begin{itemize}
    
    \item \(A\) is positive definite if all main minors are positive.

    \item \(A\) is negative definite if all main minors are negative.

    \item \(A\) is positive semidefinite if all main minors are greater or equal to zero but not all zero. 

    \item \(A\) is negative semidefinite if all main minors are less or equal to zero but not all zero.

    \item \(A\) is indefinite if all main minors are equal to zero.

\end{itemize}

\subsection{Connection to Symmetric Matrices}

For a real symmetric matrix \(A\) the following statements are equivalent.

\begin{itemize}
    
    \item \(A\) is positive definite.

    \item All eigenvalues of \(A\) are positive.

    \item All main minors of \(A\) are positive.

\end{itemize}

Continuing, \(A\) is positive semidefinite, if all eigenvalues of \(A\) are not negative and the 
main minors are all not positive.

\subsection{Sign of the \texorpdfstring{\(D_k\)}{}}

For the main minors \(D_k\) of a negative definite matrix it applies that 

\[
    (-1)^k D_k = (-1)^k \det(A_k) = \det(-A_k) > 0,
\]

because \(-A\) is positive definite.

\subsection{Negative Quadratic Form}

If \(A\) is negative definite then 

\[
    -\langle x, Ax \rangle = \langle x -Ax \rangle > 0
\]

\subsection{Theorems for Non-squares matrices}

\subsubsection{Theorem I}

Given \(m \ge n\) and \(A \in \Reals^{m \times n}\) with \(rg(A) = n\). Then \(A^T A\) is positive 
definite. For \(rg(A) < n\) is \(A^T A\) positive semidefinite.

\textbf{Proof:}

Because of \((A^T A)^T = A^T A\) is \(A^T A\) symmetric. Therefore, it applies that 

\[
    \langle x, A^T A x \rangle = x^T A^T A x  = (Ax)^T Ax = \langle Ax, Ax \rangle = \|Ax\|^2 \ge 0 
\]

Also, \(A^T A\) is positive semidefinite, and it is also invertible for \(rg(A) = n\). If there were an 
\(x \ne 0\) such that \(\|Ax\| = 0\) this would imply that \(Ax = 0\), also \(A^T (Ax) = 0\) therefore 
\(x \in ker(A^T A)\). And \(A^T A\) would not be invertible. Finally, 
\(\langle x, A^T Ax \rangle > 0 \forall x \ne 0\),  and \(A^T A\) is symmetric and positive definite.

\subsection{Theorem II}

Given \(m \le n\) and \(A \in \Reals^{m \times n}\). Then \(A^T A\) is positive semidefinite and even 
positive definite if \(rg(A) = m\).

