\section{Elementary Matrix}

We are going to define 3 matrices that can be obtained from the Elementary Row
Operations on the \emph{Identity Matrix} and calle them \emph{Elementary Matrices}.
\\\\
\emph{C1: Row Addtion}

\[
C_1 := 
\begin{pmatrix}
1      &        &        &        &        \\
       & \ddots &        &        &        \\
       &        & 1      &        &        \\
       &        & \lambda & 1     &        \\
       &        &        &        & \ddots \\
\end{pmatrix}
\in K^{n \times n}
\quad \text{or} \quad
C_1 := 
\begin{pmatrix}
1      &        &        &        &        \\
       & \ddots &        &        &        \\
       &        & 1      & \lambda &        \\
       &        &        & 1      &        \\
       &        &        &        & \ddots \\
\end{pmatrix}
\in K^{n \times n}
\]


\emph{C2: Row swap}

\[
C_2 := 
\begin{pmatrix}
1      &        &        &        &        \\
       & \ddots &        &        &        \\
       &        & 0      & 1      &        \\
       &        & 1      & 0      &        \\
       &        &        &        & \ddots \\
\end{pmatrix}
\in K^{n \times n}
\]


\emph{C3: Row Scaling}
\[
C_3 := 
\begin{pmatrix}
1      &        &        &        &        \\
       & \ddots &        &        &        \\
       &        & \lambda &       &        \\
       &        &        & \ddots &        \\
       &        &        &        & 1      \\
\end{pmatrix}
\in K^{n \times n}
\]

Each of them also encodes the row Operations from where they came from.
So now Gaußian Elimination can be encoded as matrix multiplication.

\subsection{Invertibility}

Because every row operation is invertible now we can also claim that
Elementary Matrices are invertible and thy also represented that inverted row operation.


\subsection{determinants}

The determinants corresponding to each matrix are:

\[\det C1 = 1\]

\[\det C2 = -1\]

\[\det C3 = \lambda\]

\newpage