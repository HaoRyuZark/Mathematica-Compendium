\newpage
\section{Eigenvectors and Eigenvalues}

\emph{Eigenvector}

An eigenvector of a square matrix \(A\) is a non-zero vector \(\vec{v}\) that, when multiplied by \(A\), results in a vector that is a scalar multiple of itself. In other words, the direction of the vector \(\mathbf{v}\) remains unchanged (up to scaling) when the linear transformation represented by \(A\) is applied to it.

\emph{Eigenvalue}

The scalar multiple, denoted by \(\lambda\), is called the eigenvalue associated with the eigenvector \(\mathbf{v}\). It represents the factor by which the eigenvector is scaled when transformed by the matrix \(A\).
\vspace{\baselineskip}

Mathematically, the relationship between a square matrix \(A\), an eigenvector \(\vec{v}\), and its corresponding eigenvalue \(\lambda\) is 
expressed by the following equation:

\[
A\vec{v} = \lambda\vec{v}
\]

The way we find a method to find them is by manipulating the equation above

\begin{align*}    
A\vec{v} &= \lambda\vec{v}\\
A\vec{v} - \lambda\vec{v} &= \vec{0}\\
A\vec{v} - \lambda I \vec{v} &= \vec{0}\\
(A\vec{v} - \lambda I) \vec{v} &= \vec{0}
\end{align*}

Here note that if \((A\vec{v} - \lambda I)\) were invertible then we could multiply both sides by its inverse 
and then get \(\vec{v} = 0\). But we do not want that therefore, \((A\vec{v} - \lambda I)\) must be linear transformation 
whose determinant is \(0\) which means that it can not ve invertible.

\begin{align*}    
(A\vec{v} - \lambda I) \vec{v} &= \vec{0}\\
\det (A\vec{v} - \lambda I)  &= \vec{0}
\end{align*}

\subsection{Theorems of Eigenvalues and Eigenvectors}

\subsubsection{Theorem I}

Given an eigenvalue \(\lambda\) of \(f\) and \(v_1, \dots, v_n\) being the eigenvectors of \(f\) with \(\lambda\) 
then \(v \in L(v_1, \dots, v_k) \backslash \{0\}\) is an eigenvector of \(f\) with \(\lambda\).
\vspace{\baselineskip}

\subsubsection{Theorem II}

For \(\lambda \in \mathbb{C}\) is \(Eig(f;\lambda):= \{c \in V | f(v) = \lambda v\}\) the \emph{Eigenspace} of 
\(f\) with \(\lambda\) and, it is a subspace of \(V\).

\subsubsection{Theorem III}

For \(\lambda \ne \gamma \quad Eig(f;\lambda) \cap Eig(f;\gamma) = \{0\}\).

\subsubsection{Theorem IV}

Eigenvectors with different eigenvalues are linearly independent.

\subsection{Theorem V}

\(Eig(f;\lambda) = ker(A - \lambda E)\)


\subsection{How to find the Eigenvectors and Eigenvalues}

To find the eigenvalues and eigenvectors of a square matrix \(A\), we solve the eigenvalue equation:
\vspace{\baselineskip}

\textbf{1. Form the characteristic equation:}

    Rewrite the equation \(A\vec{v} = \lambda\vec{v}\) as 
    \((A - \lambda I)\vec{v} = \vec{0}\), where \(\lambda I\) is the identity matrix times \(\lambda\) because 
    this matrix encodes the multiplication by some scalar. 
    
    To have a non-trivial solution \(0\) for \(\vec{v}\), the matrix \((A - \lambda I)\) must be 
    singular like said before, which means its determinant must be zero because otherwise \(\vec{v}\) would be 0. Thus, 
    we have the characteristic equation:
    \[
    \det(A - \lambda I) = 0
    \]

\textbf{2. Solve for the eigenvalues:}

Solve the characteristic equation for \(\lambda\). The solutions \(\lambda_1, \lambda_2, \dots, \lambda_n\) 
are the eigenvalues of the matrix \(A\).
\vspace{\baselineskip}

\textbf{3. Find the eigenvectors:}

For each eigenvalue \(\lambda_i\), substitute it back into the equation \((A - \lambda_i I)\vec{v} = \vec{0}\) and solve 
for the vector \(\vec{v}\). The non-zero solutions for \(\vec{v}\) are the eigenvectors corresponding to the 
eigenvalue \(\lambda_i\). Note that we are going to get infinite solutions thus, we have to write our vector with dependence on the free 
parameters.
\vspace{\baselineskip}

\textbf{Example:}
\vspace{\baselineskip}

Find the eigenvalues and vectors of 

\[
A = \begin{pmatrix}
    1 & 1 \\
    4 & 1 \\
\end{pmatrix}
\]

Let us build the characteristic equation and solve it

\[
\det 
\begin{pmatrix}
    1 - \lambda & 1 \\
    4 & 1 - \lambda \\
\end{pmatrix}
= 0
\]

\begin{align*}
(1 - \lambda)^2 - 4 &= 0 \\
1 - 2\lambda + \lambda^2 - 4 &= 0 \\
\lambda^2 - 2\lambda - 3 &= 0 \\
(\lambda - 3) (\lambda + 1) &= 0 
\end{align*}

We have got the eigenvalues \(\lambda_1 = 3\) and \(\lambda_2 = -1\). Now let us continue by 
substituting our eigenvalues in the original equation.

\begin{align*}
    \begin{pmatrix} 1 - \lambda_2 & 1 \\ 4 & 1 - \lambda-2 \\ \end{pmatrix} &= \vec{0} \\
    \begin{pmatrix} 1 - 2 & 1 \\ 4 & 1 - 2 \\ \end{pmatrix} &= \vec{0} \\
    \begin{pmatrix} 2 & 1 \\ 4 & 2 \\ \end{pmatrix} &= \vec{0} 
\end{align*}

Using Gaussian elimination we get: 

\[
\begin{pmatrix} 
    2 & 1 \\ 
    0 & 0 \\ 
\end{pmatrix} 
\]

Thus, let \(v_2 = -2\) and \(v_1 = \frac{2}{2} = 1\). This give us the first eigenvector and all of it 
scalar versions.

\[
e_1 = \left\{ c \begin{pmatrix} 1 \\ -2 \end{pmatrix} | c \in \mathbb{R}\right\}
\]

The process for the other eigenvalue is exactly the same.


\subsection{How to diagonalize a matrix}

Diagonalizing a matrix involves finding a diagonal matrix that is similar to the given matrix. 
A square matrix \(A\) is diagonalizable if there exists an invertible matrix \(X\) made of eigenvectors of 
\(A\) such that \(X^{-1}AX = D\), where \(D\) is a diagonal matrix made of eigenvalues.

The order of the eigenvalue and eigenvectors matters if you choose \(\lambda_1, \lambda_2, \dots, \lambda_n\) then the 
order of the eigenvectors \(v_1, v_2, \dots, v_n\) must also correspond. This process only works if \(A\) 
has unique eigenvalues or if they are duplicates they have to be linearly independent.
\vspace{\baselineskip}

The process of diagonalization is as follows:
\vspace{\baselineskip}

\textbf{1. Find the eigenvalues and eigenvectors of \(A\).}
\vspace{\baselineskip}

\textbf{2. Form the matrix \(X\):}

Create a matrix \(X\) whose columns are the linearly independent eigenvectors of \(A\).
\vspace{\baselineskip}

\textbf{3. Form the diagonal matrix \(D\):} 

Create a diagonal matrix \(D\) whose diagonal entries are the eigenvalues of \(A\), corresponding to the order of the eigenvectors in \(P\). That is, if the \(i\)-th column of \(P\) is the eigenvector corresponding to the eigenvalue \(\lambda_i\), then the \(i\)-th diagonal entry of \(D\) is \(\lambda_i\).
\vspace{\baselineskip}

\textbf{4. Verify the diagonalization:}
    
Check that \(X^{-1}AX = D\).
\vspace{\baselineskip}

A matrix \(A\) is diagonalizable if and only if it has \(n\) linearly independent eigenvectors, where \(n\) is the size of the matrix.
\vspace{\baselineskip}

\textbf{Example: }
\vspace{\baselineskip}

Consider the matrix
\[
A = \begin{pmatrix}
-3 & -4 \\
5 & 6
\end{pmatrix}
\]


The characteristic equation is

\[
\det(A - \lambda I) 
= \det 
\begin{pmatrix}
        -3 - \lambda & -4 \\
    5 & 6 - \lambda
\end{pmatrix} 
= (-3 - \lambda)(6 - \lambda) + 20 = \lambda^2 - 3\lambda + 2 = 0    
\]
\[
(\lambda - 1)(\lambda - 2)
\]

Solving for \(\lambda\), we get \(\lambda_1 = 1\) and \(\lambda_2 = 2\).
\vspace{\baselineskip}

For \(\lambda_1 = 1\):

\[
   (A - I)\vec{v} = \begin{pmatrix}
    -4 & -4 \\
    5 & 5
    \end{pmatrix} \begin{pmatrix}
    x \\
    y
    \end{pmatrix} = \begin{pmatrix}
    0 \\
    0
    \end{pmatrix}
\]

Note that this is telling us that \(x = -y\). Now choose \(x = -1\) and \(y = 1\), which gives us the vector 

\[
\vec{x} = \begin{pmatrix}
    -1 // 1
\end{pmatrix}
\]

For \(\lambda_2 = 2\) the process is similar, so we are going to skip it. We get the vector

\[
\vec{y} = \begin{pmatrix}
    -\frac{4}{5} \\ 1
\end{pmatrix}
\]

Now we will complete the process by writing \(D, X, \text{ and } X^{-1}\).

\[
D = \begin{pmatrix}
    1 & 0 \\
    0 & 2
\end{pmatrix}
X = \begin{pmatrix}
    -1  & -\frac{4}{5} \\
    1   &   1 
\end{pmatrix}
X^{1} = \begin{pmatrix}
    -1  & \frac{4}{5} \\
    -1   &   -1 
\end{pmatrix}
\]






