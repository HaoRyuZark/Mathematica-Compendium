\newpage
\section{Logarithms}

The \emph{Logarithm} of a number \(x\) 
with base \(b\) is the exponent to which the base must be raised to produce that number. It is denoted as:

\[
    \log_b(x) = y \iff b^y = x
\]

where \(b > 0\), \(b \neq 1\), and \(x > 0\).

\[
    \log_2(8) = 3 \quad \text{because} \quad 2^3 = 8
\]

We can also approximate a logarithm by dividing our target \(x\) by the basis \(b\) repeatedly until we get \(1\) or less.

\subsection{Properties of Logarithms}
\begin{itemize}
  \item \(\log_b(xy) = \log_b(x) + \log_b(y)\)
  \item \(\log_b\left(\frac{x}{y}\right) = \log_b(x) - \log_b(y)\)
  \item \(\log_b(x^k) = k \cdot \log_b(x)\)
  \item \(\log_b(b) = 1\)
  \item \(\log_b(1) = 0\)
  \item \(\log_{b^k}(x^w) = \frac{1}{k} \cdot \log_b(x)\)
  \item \(\log_b\left(\frac{1}{x}\right) = -\log_b(x)\)
  \item \(\log_b(b^x) = x\)
  \item \(\log_b(x) = \frac{\log_k(x)}{\log_k(b)}\) for any positive \(k \neq 1\)
  \item \(e^{\ln(x)} = x\)
  \item If \(0 < a < 1\) then \(\ln(a)\) is a negative number. 
\end{itemize}

\subsection{Fundamental Identity of Logarithms}
\[
a^{\log_a(x)} = x
\]

\subsection{Change of Basis Formula}
\[
\log_b(x) = \frac{\log_k(x)}{\log_k(b)}
\]
where \(k\) is any positive number different from 1.

\subsection{The Chain Rule}
\[
    \log_y(a) \log_a(b) = \log_y(b)
\]

\subsection{The derivative of the Natural Logarithm}

\[
f'(x) = \lim_{h \to 0} \frac{f(x + h) - f(x)}{h}
\]

Let \(f(x) = \ln(x)\), then

\begin{align*}
&f'(x) = \lim_{h \to 0} \frac{\ln(x + h) - \ln(x)}{h}\\
&= f'(x) = \lim_{h \to 0} \frac{\ln\left(\frac{x + h}{x}\right)}{h}\\
&= f'(x) = \lim_{h \to 0} \frac{\ln\left(1 + \frac{h}{x}\right)}{h}\\
&= f'(x) = \lim_{h \to 0} \ln{\left(1 + \frac{h}{x}\right)}^{\frac{1}{h}}
\end{align*}
 
Now let \(n = \frac{h}{x}\)

\begin{align*}
f'(x) &= \lim_{n \to 0} \ln{\left(1 + n\right)}^{\frac{x}{h} \frac{1}{x}}\\
f'(x) &= \lim_{n \to 0} \frac{1}{x} \ln{\left(1 + n\right)}^{\frac{1}{n}}\\
f'(x) &= \frac{1}{x} \ln \left(\lim_{n \to 0} {(1 + n)}^{\frac{1}{n}}\right)\\
f'(x) &= \frac{1}{x}
\end{align*}

Therefore, the derivative of the natural logarithm is:
\[
\frac{d}{dx} \ln(x) = \frac{1}{x}
\]
\QED

\subsection{Demonstration of the Properties}

\subsubsection{Product}

Let \(a^n a^m = a^{n + m}\) and \(a^n = x\) and \(a^m = y\) then

\[
\log_a x = n \text{ and } \log_a y = m \rightarrow \log_a (xy) = n + m
\]
\[
\log_a xy = \log_a x + \log_a y
\]

\QED

\subsubsection{Quotient}

Let \(\frac{a^n}{a^m} = a^{n - m}\) and \(a^n = x\) and \(a^m = y\) then like in the previous proof
\[
\log_a \left(\frac{x}{y}\right) = \log_a x - \log_a y
\]
\QED 

\subsubsection{Power}

Let \({(a^n)}^m = a^{nm}\) and \(a^n = x\) and \(a^{nm} = y\)
\[
\log_x y = m \text{ and } \log_a x = n \text{ thus, } \log_a y = (\log_a x )m
\]
\QED

\subsubsection{Power Rule General Case:}

Let \({(a^n)}^m = b^x\) then \(\log_{a^n} b^x = m\) now let us manipulate the original expression

\begin{align*}
\log_a {(a^n)}^m &= \log_a b^x\\
nm \log_a a &= x \log_a b\\
m = \frac{x}{n} \log_a b &=  \log_{a^n} b^x 
\end{align*}

\QED
