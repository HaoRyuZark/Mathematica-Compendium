\newpage
\section{Recursion}

Recursion is term used to refer to a thing define in terms
of itself. An example the factorial \(n! = (n - 1)\)! with \(0! = 1\) being the
base case where the recursion stops.

\subsection{From Recursive to Closed Form}

A sequence defined recursively can often be rewritten in an explicit or closed form.

\emph{Recursive form} defines \( a_n \) based on previous terms (e.g., \( a_n = a_{n-1} + 2 \)).

\emph{Closed form} expresses \( a_n \) directly in terms of \(n\), with no reference to previous terms.

\subsubsection{General Method (when linear):}

\begin{enumerate}

    \item Calculate some terms and watch the difference or quotient between them
          maybe the difference is clear, and you can build a geometric or arithmetic sequence directly.

    \item Identify the type (e.g., linear, homogeneous, constant coefficients).

    \item Solve the characteristic equation (if linear).
    
    \item Use initial conditions to determine constants.

\end{enumerate}

\subsection{Linear Recursion}

A linear recurrence has the form:

\begin{align*}
    a_n & = \lambda a_{n -1} + c \\
    a_1 & = \lambda a_0 + c \implies a_2 = \lambda (\lambda a_0 + c) + c \\
    \implies a_n & = \lambda^n a_0 + (1 + \lambda + \lambda^2 + \cdots + \lambda^{n - 1})c
\end{align*}

Therefore,

\[
    a_n = \lambda^n a_0 + \frac{\lambda^n - 1}{\lambda - 1}c
\]

\subsection{Recursive Inequalities}

A \emph{recursive inequality} bounds a sequence rather than defines it exactly.
\vspace{\baselineskip}

\textbf{Example:}
\vspace{\baselineskip}

Suppose

\[
    a_{n+1} \le \frac{1}{2} a_n + 3
\]

To analyze:

\begin{itemize}

    \item Guess a bound (e.g., show \( a_n \le M \) for all \(n\))

    \item Use induction or iteration to justify convergence or boundedness

    \item Compare with a simpler sequence (e.g., geometric series)

\end{itemize}

These techniques are common in analysis, approximation, and numerical algorithms.

\subsection{Fibonacci Numbers and Closed Form}

The Fibonacci sequence is defined recursively:

\[
    F_0 = 0, \quad F_1 = 1, \quad F_n = F_{n-1} + F_{n-2}
\]

The characteristic equation is:

\[
    x^2 - x - 1 = 0 \Rightarrow x = \varphi = \frac{1 + \sqrt{5}}{2}, \quad \psi = \frac{1 - \sqrt{5}}{2}
\]

The closed form (Binet’s Formula) is:

\[
    F_n = \frac{1}{\sqrt{5}} \left( \varphi^n - \psi^n \right)
\]

\begin{itemize}
    
    \item \( \varphi \approx 1.618 \) is the golden ratio
    
    \item This formula demonstrates exponential growth of \( F_n \)

\end{itemize}

