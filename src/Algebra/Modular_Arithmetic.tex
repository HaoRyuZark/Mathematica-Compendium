\newpage
\section{Modular Arithmetic}

This is basically Wheel Math or division with residual \(b : n \text{residual} = a\)

\[
    a \cong b \mod n \iff n | (a - b) \land  a,b \in \Integers
\]

\subsection{Properties}

\emph{Addition}

\[
    (a + b) \mod n = (a \mod n + b \mod n) \mod n
\]

\emph{Multiplication}

\[
    (a * b) \mod n = (a \mod n * b \mod n) \mod n
\]

\emph{Exponentiation}

\[
    (a^m) \mod n = ({(a \mod n)}^n) \mod n
    \]

\emph{Division}

\[
    (\frac{a}{b}) \mod n \ne  \frac{a \mod n}{b \mod n}
\]

\subsection{General Division Rules}

\subsubsection{Sum Rules}

If the number \( a \) can be expressed as a sum and \( b \) is a common divisor of all the terms, then \( b \) is also a divisor of \( a \). \\
\vspace{\baselineskip}

\textbf{Example:} 
\vspace{\baselineskip}

\( 742 = 700 + 42 \). Then 7, as a common divisor of 700 and 42, is a divisor of 742.

\subsubsection{Product Rule}

If the number \( a \) can be written as a product of two factors and \( b \) is a divisor of one of these factors, then \( b \) is also a divisor of \( a \). \\
\vspace{\baselineskip}

\textbf{Example:}
\vspace{\baselineskip}

\( 1500 = 15 \cdot 100 \). Then 3 and 5 are divisors of 15, and 2, 4, 5, 10, 25, and 50 are divisors of 100—thus, all are divisors of 1500.

\subsubection{Divisor Pair Rules}

If the number \( a \) has \( b \) as a divisor, then \( \frac{a}{b} \) is also a divisor of \( a \). \\
\vspace{\baselineskip}

\textbf{Example:}
\vspace{\baselineskip}
 
3 is a divisor of 315. Then \( 105 = \frac{315}{3} \) is also a divisor of 315.

\subsubsection{Divisor Rule}

If the number \( a \) has \( b \) as a divisor, then every divisor of \( b \) is also a divisor of \( a \). \\
\vspace{\baselineskip}

\textbf{Example:}
\vspace{\baselineskip}
 
45 is a divisor of 450. Then 3, 5, 9, and 15 (divisors of 45) are also divisors of 450.

\subsubsection{Difference Rule}

If the number \( a \) can be expressed as a difference and \( b \) is a common divisor of the minuend and subtrahend, then \( b \) is also a divisor of \( a \). \\
\vspace{\baselineskip}

\textbf{Example:}
\vspace{\baselineskip}

\( 686 = 700 - 14 \). Then 7, as a common divisor of 700 and 14, is a divisor of 686.

\subsubsection{Relative Prime Rule}

If the number \( a \) has two relatively prime numbers \( b \) and \( c \) as divisors, then their product \( b \cdot c \) is also a divisor of \( a \). \\
\vspace{\baselineskip}

\textbf{Example:}
\vspace{\baselineskip}

1540 has the relatively prime numbers 7 and 11 as divisors. Then \( 77 = 7 \cdot 11 \) is also a divisor of 1540.

\subsection{Using Modular Arithmetic to find the n digit of a big number}

What is the last digit of the number \(13^{2023}\)

Note that \(13 \mod 10 = 3\) and  \(13^2 \mod 10 =  ((13 \mod 10)  (13\mod 10)) \mod 10 = 9 \)

We can continue this process, and we will notice a pattern. After some iteration we will now that

\[(13^{2020} 13) \mod 10 = 7\]

\subsection{Gaussian Brackets}

The Gaussian brackets refer to two closely related functions used to round real numbers to integers in specific ways:

\emph{Floor Function}
\vspace{\baselineskip}

The floor function, denoted by \(\lfloor x \rfloor\), gives the greatest integer less than or equal to a real number \(x\). \\
\vspace{\baselineskip}
    
\textbf{Example:}
\vspace{\baselineskip}
 
\(\lfloor 3.7 \rfloor = 3\), \(\lfloor -1.2 \rfloor = -2\)

\emph{Ceiling Function} 
    
The ceiling function, denoted by \(\lceil x \rceil\), gives the smallest integer greater than or equal to a real number \(x\). \\
\vspace{\baselineskip}

\textbf{Example:}
\vspace{\baselineskip}
 
\(\lceil 3.7 \rceil = 4\), \(\lceil -1.2 \rceil = -1\)
\vspace{\baselineskip}

These functions are particularly useful: in number theory, computer science 
and rounding operations, where precise integer bounds of real numbers are required.

