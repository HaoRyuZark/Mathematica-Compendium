\newpage
\section{Proportionality and the Rule of Three}

\subsection{Proportionality}

Two quantities are said to be \emph{proportional} if their ratio remains constant.

\subsubsection{Direct Proportionality}

Two quantities \(a\) and \(b\) are in \emph{direct proportion} if:

\[
    \frac{a}{b} = k \quad \Rightarrow \quad a = k \cdot b
\]

Where \(k\) is the constant of proportionality.
\vspace{\baselineskip}

\textbf{Example:}
\vspace{\baselineskip}
 
If 2 pencils cost 1, then 4 pencils cost 2. The ratio is constant: \(\frac{2}{1} = \frac{4}{2}\).

\subsubsection{Inverse Proportionality}

Two quantities \(a\) and \(b\) are in \emph{inverse proportion} if their product is constant:

\[
    a \cdot b = k
\]

\textbf{Example:}
\vspace{\baselineskip}

If 4 workers finish a job in 6 hours, then 2 workers would need 12 hours:

\[
    4 \cdot 6 = 2 \cdot 12 = 24
\]

\subsection{The Rule of Three (Simple)}

The \emph{Rule of Three} is a method to find a fourth value when three values are known and a 
proportional relationship is assumed.

\subsubsection{Direct Rule of Three (Simple)}

Given: \(a : b = c : x\), solve for \(x\):

\[
    x = \frac{b \cdot c}{a}
\]

\textbf{Example:}
\vspace{\baselineskip}
 
If 3 apples cost 6, how much do 5 apples cost?

\[
    x = \frac{6 \cdot 5}{3} = 10
\]

\subsubsection{Inverse Rule of Three (Simple)}

If the relationship is inverse:

\[
    a : b = x : c \quad \Rightarrow \quad x = \frac{a \cdot c}{b}
\]

\textbf{Example:}
\vspace{\baselineskip}
 
If 5 people finish a task in 8 hours, how long will 10 people need?

\[
    x = \frac{5 \cdot 8}{10} = 4
\]

\subsection{The Rule of Three (Compound)}

The \emph{Compound Rule of Three} (or \emph{composed rule of three}) involves more than two variables.
\vspace{\baselineskip}

\textbf{Example:}
\vspace{\baselineskip}
 
If 4 machines produce 120 items in 5 hours, how many items will 6 machines produce in 8 hours?
\vspace{\baselineskip}

\textbf{Step 1:} Set up proportionally:

\[
    \text{Items} \propto \text{Machines} \quad (\text{direct}) \\
    \text{Items} \propto \text{Time} \quad (\text{direct})
\]

\textbf{Step 2:} Adjust the quantity:

\begin{align*}
    \text{Initial: } 4 \text{ machines, } 5 \text{ hrs } \rightarrow 120 \text{ items} \\
    \text{New: } 6 \text{ machines, } 8 \text{ hrs } \rightarrow x \text{ items}
\end{align*}

\textbf{Step 3:} Use proportionality:

\[
    x = 120 \cdot \frac{6}{4} \cdot \frac{8}{5} = 120 \cdot 1.5 \cdot 1.6 = 288
\]

\textbf{Answer:} 288 items.
\vspace{\baselineskip}

\textbf{Summary Table}

\begin{center}
    \begin{tabular}{|l|l|}
    \hline
    \textbf{Type} & \textbf{Formula} \\
    \hline
    Direct Proportion & \(x = \frac{b \cdot c}{a}\) \\
    Inverse Proportion & \(x = \frac{a \cdot c}{b}\) \\
    Compound Rule of Three & Multiply by all direct and divide by inverse ratios \\
    \hline
    \end{tabular}
\end{center}

