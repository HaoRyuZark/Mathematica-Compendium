\section{The Fundamental Theorem of Algebra}

Given any positive integer \(n \geq 1\) and any choice of complex numbers \(a_0, a_1, \ldots, a_n\), such that \(a_n \neq 0\), the polynomial equation
\[
	a_n z^n + \cdots + a_1 z + a_0 = 0 \tag{1}
\]
has at least one solution \(z \in \mathbb{C}\).

\subsection*{Gist of the Proof}

For readers familiar with Newton’s method for solving equations, one starts with a reasonably close approximation to a root, then adjusts the approximation by moving closer in an appropriate direction. We will employ the same strategy here, showing that if one assumes that the argument where the polynomial function achieves its minimum absolute value is not a root, then there is a nearby argument where the polynomial function has an even smaller absolute value, contradicting the assumption that the argument of the minimum absolute value is not a root.

\subsection{Definitions and Axioms}
In the following, \(p(z)\) will denote the \(n\)-th degree polynomial
\[
	p(z) = p_0 + p_1 z + p_2 z^2 + \cdots + p_n z^n,
\]
where the coefficients \(p_i\) are any complex numbers, with neither \(p_0\) nor \(p_n\) equal to zero (otherwise the polynomial is equivalent to one of lesser degree).

We will utilize a fundamental completeness property of real and complex numbers, namely that a continuous function on a closed set achieves its minimum at some point in the domain. This can be taken as an axiom, or can be easily proved by applying other well-known completeness axioms, such as the Cauchy sequence axiom or the nested interval axiom.

\subsection{Theorem 1}
Every polynomial with real or complex coefficients has at least one complex root.
\\\\
\textbf{Proof:}

Suppose that \(p(z)\) has no roots in the complex plane. First note that for large \(z\), say \(|z| > 2 \max_i |p_i/p_n|\), the \(z^n\) term of \(p(z)\) is greater in absolute value than the sum of all the other terms. Thus, given some \(B > 0\), for any sufficiently large \(s\), we have \(|p(z)| > B\) for all \(z\) with \(|z| \geq s\). We will take \(B = 2|p(0)| = 2|p_0|\).
\\\\
Since \(|p(z)|\) is continuous on the interior and boundary of the circle with radius \(s\), it follows by the completeness axiom that \(|p(z)|\) achieves its minimum value at some point \(t\) in this circle. But since \(|p(0)| < \frac{1}{2} |p(z)|\) for all \(z\) on the circumference of the circle, it follows that \(|p(z)|\) achieves its minimum at some point \(t\) in the interior.
\\\\
Now rewrite the polynomial \(p(z)\) by translating the argument \(z\) by \(t\), thus producing a new polynomial
\[
	q(z) = p(z + t) = q_0 + q_1 z + q_2 z^2 + \cdots + q_n z^n,
\]
and similarly translate the circle. Presumably the polynomial \(q(z)\), defined on some circle centered at the origin, has a minimum absolute value \(M > 0\) at \(z = 0\). Note that \(M = |q(0)| = |q_0|\).
\\\\
Our proof strategy is to construct some point \(x\), close to the origin, such that \(|q(x)| < |q(0)|\), thus contradicting the assumption that \(|q(z)|\) has a minimum nonzero value at \(z = 0\).
\\\\
\textbf{Construction of \(x\) such that \(|q(x)| < |q(0)|\)}

Let the first nonzero coefficient of \(q(z)\) following \(q_0\) be \(q_m\), so that
\[
	q(z) = q_0 + q_m z^m + q_{m+1} z^{m+1} + \cdots + q_n z^n.
\]
We choose
\[
	x = r \left(-\frac{q_0}{q_m}\right)^{1/m},
\]
where \(r\) is a small positive real value, and \(\left(-\frac{q_0}{q_m}\right)^{1/m}\) denotes any \(m\)-th root of \(\left(-\frac{q_0}{q_m}\right)\).
\\\\
Unlike the real numbers, in the complex number system the \(m\)-th roots of any complex number are guaranteed to exist. If \(z = z_1 + i z_2\), then the \(m\)-th roots of \(z\) are given by
\[
	\left\{ R^{1/m} \cos\left(\frac{\theta + 2k\pi}{m}\right) + i R^{1/m} \sin\left(\frac{\theta + 2k\pi}{m}\right) \,\bigg|\, k = 0, 1, \ldots, m-1 \right\},
\]
where \(R = \sqrt{z_1^2 + z_2^2}\) and \(\theta = \arctan(z_2 / z_1)\).
\\\\
\textbf{Proof:} 

\[|q(x)| < |q(0)|\]

With the definition of \(x\), we can write
\[
	q(x) = q_0 - q_0 r^m + q_{m+1} r^{m+1} \left(-\frac{q_0}{q_m}\right)^{(m+1)/m} + \cdots + q_n r^n \left(-\frac{q_0}{q_m}\right)^{n/m} = q_0 - q_0 r^m + E,
\]
where the extra terms \(E\) can be bounded as follows. Assume \(q_0 \leq q_m\), and define \(s = r \left|\frac{q_0}{q_m}\right|^{1/m}\). Then
\[
	|E| \leq r^{m+1} \max_i |q_i| \left|\frac{q_0}{q_m}\right|^{(m+1)/m} (1 + s + s^2 + \cdots + s^{n - m - 1}) \leq \frac{r^{m+1} \max_i |q_i|}{1 - s} \left|\frac{q_0}{q_m}\right|^{(m+1)/m}.
\]
Thus \(|E|\) can be made arbitrarily small compared to \(|q_0 r^m| = |q_0| r^m\) by choosing \(r\) small enough. For example, select \(r\) so that \(|E| < \frac{|q_0| r^m}{2}\). Then:
\[
	|q(x)| = |q_0 - q_0 r^m + E| < |q_0 - \frac{q_0 r^m}{2}| = |q_0| \left(1 - \frac{r^m}{2}\right) < |q_0| = |q(0)|,
\]
which contradicts the assumption that \(|q(z)|\) has a minimum nonzero value at \(z = 0\).
\QED

\subsection{Theorem 2}
Every polynomial of degree \(n\) with real or complex coefficients has exactly \(n\) complex roots, when counting multiplicities.
\\\\
\textbf{Proof}

If \(\alpha\) is a root of the polynomial \(p(z)\) of degree \(n\), then by dividing \(p(z)\) by \((z - \alpha)\), we get:
\[
	p(z) = (z - \alpha) q(z) + r,
\]
where \(q(z)\) has degree \(n - 1\) and \(r\) is a constant. But since \(p(\alpha) = r = 0\), we conclude:
\[
	p(z) = (z - \alpha) q(z).
\]
Continuing by induction, we conclude that the original polynomial \(p(z)\) has exactly \(n\) complex roots, counted with multiplicities.

\QED
\newpage
