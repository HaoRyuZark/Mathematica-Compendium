\newpage
\section{Partial Fractions}

\subsection{The Simplest Case}

In the most common partial fraction decomposition, we split:

\[
    \frac{N(x)}{(x - a_1)(x - a_2)\cdots(x - a_d)}
\]

into a sum of the form:

\[
    \frac{A_1}{x - a_1} + \cdots + \frac{A_d}{x - a_d}
\]

We now show that this decomposition can always be achieved, under the assumption that the \(a_i\) are 
all different and \(N(x)\) is a polynomial of degree at most \(d - 1\).

\subsubsection{Lemma 1}

Let \(N(x)\) and \(D(x)\) be polynomials of degree \(n\) and \(d\), respectively, with \(n \leq d\). 
Suppose that \(a\) is not a root of \(D(x)\). Then there exists a polynomial \(P(x)\) of degree \(< d\) 
and a number \(A\) such that:

\[
    \frac{N(x)}{D(x)(x - a)} = \frac{P(x)}{D(x)} + \frac{A}{x - a}
\]

\textbf{Proof:} 

Let \(z = x - a\). Define:

\[
    \tilde{N}(z) = N(z + a), \quad \tilde{D}(z) = D(z + a)
\]

Then:

\[
    \frac{\tilde{N}(z)}{\tilde{D}(z)z} = \frac{\tilde{P}(z)}{\tilde{D}(z)} + \frac{A}{z}
    \Rightarrow \frac{\tilde{P}(z)z + A\tilde{D}(z)}{\tilde{D}(z)z}
\]

We equate:

\[
    \tilde{P}(z)z + A\tilde{D}(z) = \tilde{N}(z)
\]

Choosing \(A = \frac{\tilde{N}(0)}{\tilde{D}(0)}\), the constant terms match. The remainder has no 
constant term and is divisible by \(z\), so:

\[
    \tilde{P}(z)z = \tilde{N}(z) - A\tilde{D}(z)
\]

Thus, \(\tilde{P}(z)\) is a polynomial of degree \(< d\).

\subsubsection{Recursive Decomposition}

Now, consider:

\[
    \frac{N(x)}{(x - a_1)(x - a_2)\cdots(x - a_d)}
\]

Apply Lemma 1 recursively:

\[
    \frac{N(x)}{(x - a_1)\cdots(x - a_d)} = \frac{A_1}{x - a_1} + \frac{P(x)}{(x - a_2)\cdots(x - a_d)}
\]

Then:

\[
    \frac{P(x)}{(x - a_2)\cdots(x - a_d)} = \frac{A_2}{x - a_2} + \frac{Q(x)}{(x - a_3)\cdots(x - a_d)}
\]

Continue until:

\[
    \frac{N(x)}{(x - a_1)\cdots(x - a_d)} = \frac{A_1}{x - a_1} + \cdots + \frac{A_d}{x - a_d}
\]

\subsection{Lemma 2}

Let \(N(x)\) and \(D(x)\) be polynomials of degree \(n\) and \(d\) respectively, with \(n < d + m\).
Suppose that \(a\) is NOT a zero of \(D(x)\). Then there is a polynomial \(P(x)\) of degree \(p < d\) and
numbers \(A_1, \dots, A_m\) such that

\[
    \frac{N(x)}{D(x) {(x-a)}^m} = \frac{P(x)}{D(x)} + \frac{A_1}{x-a} + \frac{A_2}{{(x-a)}^2} + \cdots + 
    \frac{A_m}{{(x-a)}^m}
\]

\textbf{Proof:} 

To save writing, let \(z = x - a\). Then \(\tilde{N}(z) = N(z + a)\) and \(\tilde{D}(z) = D(z + a)\)
are polynomials of degree \(n\) and \(d\) respectively, \(\tilde{D}(0) = D(a) \neq 0\) and we have to find a
polynomial \(\tilde{P}(z)\) of degree \(p < d\) and numbers \(A_1, \dots, A_m\) such that

\begin{align*}
    \frac{\tilde{N}(z)}{\tilde{D}(z) z^m} &= \frac{\tilde{P}(z)}{\tilde{D}(z)} + \frac{A_1}{z} + \frac{A_2}{z^2} + \cdots + \frac{A_m}{z^m} \\
    &= \frac{\tilde{P}(z) z^m + A_1 z^{m-1} \tilde{D}(z) + A_2 z^{m-2} \tilde{D}(z) + \cdots + A_m \tilde{D}(z)}{\tilde{D}(z) z^m}
\end{align*}

or equivalently, such that

\[
    \tilde{P}(z)z^m + A_1 z^{m-1} \tilde{D}(z) + A_2 z^{m-2} \tilde{D}(z) + \cdots + A_{m-1} z 
    \tilde{D}(z) + A_m \tilde{D}(z) = \tilde{N}(z)
\]

Now look at the polynomial on the left-hand side. Every single term on the left-hand side,
except for the very last one, \(A_m \tilde{D}(z)\), has at least one power of \(z\). So the constant 
term on the left-hand side is exactly the constant term in \(A_m \tilde{D}(z)\), which is 
\(A_m \tilde{D}(0)\). The constant term on the right-hand side is \(\tilde{N}(0)\). So the constant terms 
on the left and right-hand sides are the same if we choose \(A_m = \frac{\tilde{N}(0)}{\tilde{D}(0)}\). 
Recall that \(\tilde{D}(0) \neq 0\). Now move \(A_m \tilde{D}(z)\) to the right-hand side.

\[
    \tilde{P}(z)z^m + A_1 z^{m-1} \tilde{D}(z) + A_2 z^{m-2} \tilde{D}(z) + \cdots + A_{m-1} z 
    \tilde{D}(z) = \tilde{N}(z) - A_m \tilde{D}(z)
\]

The constant terms in \(\tilde{N}(z)\) and \(A_m \tilde{D}(z)\) are the same, so the right-hand side 
contains no constant term and the right-hand side is of the form \(\tilde{N}_1(z)z\) with \(\tilde{N}_1\) 
a polynomial of degree at most \(d + m - 2\). (Recall that \(\tilde{N}\) is of degree at most 
\(d + m - 1\) and \(\tilde{D}\) is of degree at most \(d\).) Divide the whole equation by \(z\).

\[
    \tilde{P}(z)z^{m-1} + A_1 z^{m-2} \tilde{D}(z) + A_2 z^{m-3} \tilde{D}(z) + \cdots + A_{m-1} 
    \tilde{D}(z) = \tilde{N}_1(z)
\]

Now, we can repeat the previous argument. The constant term on the left-hand side, which
is exactly \(A_{m-1} \tilde{D}(0)\) matches the constant term on the right-hand side, which is 
\(\tilde{N}_1(0)\) if we choose \(A_{m-1} = \frac{\tilde{N}_1(0)}{\tilde{D}(0)}\). With this choice of 
\(A_{m-1}\)

\[
    \tilde{P}(z)z^{m-1} + A_1 z^{m-2} \tilde{D}(z) + A_2 z^{m-3} \tilde{D}(z) + \cdots + A_{m-2} z 
    \tilde{D}(z) = \tilde{N}_1(z) - A_{m-1} \tilde{D}(z) = \tilde{N}_2(z)z
\]

With \(\tilde{N}_2\) a polynomial of degree at most \(d + m - 3\). Divide by \(z\) and continue. 
After \(m\) steps like this, we end up with

\[
    \tilde{P}(z)z = \tilde{N}_{m-1}(z) - A_1 \tilde{D}(z)
\]

After having chosen \(A_1 = \frac{\tilde{N}_{m-1}(0)}{\tilde{D}(0)}\). There is no constant term on the 
right side so that \(\tilde{N}_{m-1}(z) - A_1 \tilde{D}(z)\) is of the form \(\tilde{N}_m(z)z\) with 
\(\tilde{N}_m\) a polynomial of degree \(d - 1\). Choosing \(\tilde{P}(z) = \tilde{N}_m(z)\) completes 
the proof.

Now back to

\[
    \frac{N(x)}{{(x-a_1)}^{n_1} \times \cdots \times {(x-a_d)}^{n_d}}
\]

Apply Lemma 2, with \(D(x) = {(x - a_2)}^{n_2} \times \cdots \times {(x - a_d)}^{n_d}\), \(m = n_1\) and 
\(a = a_1\). It says

\[
    \frac{N(x)}{{(x-a_1)}^{n_1} \times \cdots \times {(x-a_d)}^{n_d}} = \frac{P(x)}{{(x-a_2)}^{n_2} 
    \times \cdots \times {(x-a_d)}^{n_d}} + \frac{A_{1,1}}{x-a_1} + \frac{A_{1,2}}{{(x-a_1)}^2} + \cdots + 
    \frac{A_{1,n_1}}{{(x-a_1)}^{n_1}}
\]

Apply Lemma 2 a second time, with \(D(x) = {(x - a_3)}^{n_3} \times \cdots \times {(x - a_d)}^{n_d}\), 
\(N(x) = P(x)\), \(m = n_2\) and \(a = a_2\). And so on. Eventually, we end up with

\[
    \frac{N(x)}{{(x-a_1)}^{n_1} \times \cdots \times {(x-a_d)}^{n_d}} = \left[ \frac{A_{1,1}}{x-a_1} + 
    \cdots + \frac{A_{1,n_1}}{{(x-a_1)}^{n_1}} \right] + \cdots + \left[ \frac{A_{d,1}}{x-a_d} + \cdots + 
    \frac{A_{d,n_d}}{{(x-a_d)}^{n_d}} \right]
\]
