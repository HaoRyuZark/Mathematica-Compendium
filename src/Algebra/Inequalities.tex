\newpage
\section{Inequalities with and without Absolute Value}

\emph{Inequalities} are like normal \emph{equalities} but with ranges of solutions 
instead of one particular solution. The inequalities without absolute value are straightforward 
to solve, just like equations with the particularity of when dividing both sides by a negative number 
the relation \(<, > \ge, \le\) has to be converted to its opposite.
\vspace{\baselineskip}

The with absolute value is similar but here we have to account for different cases depending on the 
expression inside the \(|\circ|\).
\vspace{\baselineskip}

\textbf{Example:}
\vspace{\baselineskip}

Solve \(\|x + 1 \| \le 10 + \|2x - 6\|\)
\vspace{\baselineskip}

\textbf{Main cases:}

\[
    F_1: x + 1 \ge 0 \implies x \ge 1
\]

\[
    F_2: x + 1 < 0 \implies x < -1
\]

\textbf{Subcases:}

\[
    F_{1,1}: 2x - 6 \ge 0  \implies x \ge 3
\]

\begin{align*}
    x + 1 &\ge 10 + 2x - 6 \\
    x &\ge -3 \\
    \mathcal{L} &= [3; \infty )
\end{align*}


\[
    F_{1,2}: 2x - 6 < 0  \implies x < 3
\]

\begin{align*}
    x + 1 &\ge 10 - (2x - 6) \\
    x &\ge 5 \\
    \mathcal{L} &= (-\infty; 3]
\end{align*}

\[
    F_{2,1}: 2x - 6 \ge 0 \implies x \ge 3
\]

\begin{align*}
   -(x + 1 ) &\ge 10 + 2x - 6 \\
    x &\ge \frac{-5}{3} \\
    \mathcal{L} &= \emptyset
\end{align*}

\[
    F_{2,2}: 2x - 6 < 0 \implies x < 3
\]

\begin{align*}
   -(x + 1 ) &\ge 10 - (2x - 6) \\
    x &\le 17 \\
    \mathcal{L} &= (- \infty, 3)
\end{align*}

\textbf{Union of the solutions:}

\[
    \mathcal{L} = 
    (-\infty; 3] \cup 
    [3; \infty ) \cup 
    \emptyset    \cup
    (- \infty, 3) = (- \infty, \infty)
\]

\subsection{Reciprocal Inequality}

\[
    a < b < c \implies \frac{1}{a} > \frac{1}{b} > \frac{1}{c}
\]

\subsection{Bernoulli Inequality}

Given \(x \in \Reals, x > -1, x \ne 0\). Then for \(n \in \Naturals, n \ge 2\)

\[
    (1 + x)^n > 1 + nx
\]

\textbf{Proof:}

For \(n = 2\)

\[
    (1 + x)^2 > 1 + 2x
\]

Then for \(n + 1\) with \(n > 2\)

\begin{align*}
    (1 + x)^{n + 1} &> 1 + (n + 1)x \\
    (1 + x)^n &> 1 + (n + 1)x
\end{align*}

\QED

