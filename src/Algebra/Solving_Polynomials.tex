\newpage
\section{Solving Polynomial Equations}

In this section, we discuss the solution formulas for polynomial equations of degrees 2 and 3: the PQ 
formula, the ABC formula, and the Cubic formula. We also derive each of them step by step.

\subsection{The PQ Formula}

The PQ formula solves quadratic equations of the form:

\[
    x^2 + px + q = 0
\]

\subsubsection{Derivation}

To derive the PQ formula, we complete the square:

\begin{align*}
    x^2 + px + q &= 0 \\
    x^2 + px &= -q \\
    x^2 + px + {\left(\frac{p}{2}\right)}^2 &= -q + {\left(\frac{p}{2}\right)}^2 \\
    {\left(x + \frac{p}{2}\right)}^2 &= {\left(\frac{p}{2}\right)}^2 - q \\
    x + \frac{p}{2} &= \pm \sqrt{{\left(\frac{p}{2}\right)}^2 - q} \\
    x &= -\frac{p}{2} \pm \sqrt{{\left(\frac{p}{2}\right)}^2 - q}
\end{align*}

\subsubsection{PQ formula:}

\[
    x = -\frac{p}{2} \pm \sqrt{{\left(\frac{p}{2}\right)}^2 - q}
\]

\subsection{The Quadratic Formula}

The general quadratic equation is:

\[
    ax^2 + bx + c = 0 \quad\text{with } a \ne 0
\]

\subsubsection{Derivation of the PQ-Formula}

We normalize the equation by dividing through by \(a\) and complete the square:

\begin{align*}
    ax^2 + bx + c &= 0 \\
    x^2 + \frac{b}{a}x + \frac{c}{a} &= 0 \\
    x^2 + \frac{b}{a}x &= -\frac{c}{a} \\
    x^2 + \frac{b}{a}x + {\left(\frac{b}{2a}\right)}^2 &= -\frac{c}{a} + {\left(\frac{b}{2a}\right)}^2 \\
    {\left(x + \frac{b}{2a}\right)}^2 &= \frac{b^2 - 4ac}{4a^2} \\
    x + \frac{b}{2a} &= \pm \frac{\sqrt{b^2 - 4ac}}{2a} \\
    x &= \frac{-b \pm \sqrt{b^2 - 4ac}}{2a}
\end{align*}

\subsubsection{ABC formula:}

\[
    x = \frac{-b \pm \sqrt{b^2 - 4ac}}{2a}
\]

\subsection{The Cubic Formula}

To solve a general cubic equation:

\[
    ax^3 + bx^2 + cx + d = 0,
\]

we first reduce it to a depressed cubic using a substitution.

\textbf{Step 1: Depress the cubic}

Let \(x = t - \frac{b}{3a}\), then the equation becomes:

\[
    t^3 + pt + q = 0
\]

with:

\[
    p = \frac{3ac - b^2}{3a^2}, \quad q = \frac{2b^3 - 9abc + 27a^2d}{27a^3}
\]

\textbf{Step 2: Solve the depressed cubic using Cardano’s method}

Assume a solution of the form:

\[
    t = u + v
\]

Then substitute and simplify:

\[
    {(u+v)}^3 + p(u+v) + q = 0
\]

Expanding and setting:

\[
    u^3 + v^3 + (3uv + p)(u+v) + q = 0
\]

To eliminate the \((u+v)\) term, set:

\[
    3uv + p = 0 \quad \Rightarrow \quad uv = -\frac{p}{3}
\]

Now:

\[
    u^3 + v^3 = -q
\]

Let:

\[
    u^3 = A, \quad v^3 = B \quad \Rightarrow \quad A + B = -q, \quad AB = -\frac{p^3}{27}
\]

These are the roots of the quadratic:

\[
    z^2 + qz - \frac{p^3}{27} = 0
\]

Solve for \(A\) and \(B\), then take cube roots to get \(u\) and \(v\). The final solution is:

\[
    x = u + v
\]

\subsubsection{Cardano’s Formula (for depressed cubic)}

\[
    x = \sqrt[3]{-\frac{q}{2} + \sqrt{{\left(\frac{q}{2}\right)}^2 + {\left(\frac{p}{3}\right)}^3}} + 
    \sqrt[3]{-\frac{q}{2} - \sqrt{{\left(\frac{q}{2}\right)}^2 + {\left(\frac{p}{3}\right)}^3}}
\]

This formula gives one real root. The other roots (if real) can be found using trigonometric or complex 
methods depending on the discriminant.

