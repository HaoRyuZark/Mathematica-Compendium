\newpage
\section{Factorization Techniques}

Factorization is the process of writing a mathematical expression as a product 
of its factors. This is a fundamental technique in algebra used to simplify expressions, 
solve equations, and analyze functions.

\subsection{Common Factor}

Factor out the greatest common divisor (GCD) of all terms.\newline
\textbf{Example:}
\[
6x^2 + 9x = 3x(2x + 3)
\]

\subsection{Difference of Squares}

A difference of squares follows the identity:
\[
a^2 - b^2 = (a - b)(a + b)
\]

\textbf{Example:}
\[
x^2 - 16 = (x - 4)(x + 4)
\]


\subsection{Sum of Cubes:}
\[
a^3 + b^3 = (a + b)(a^2 - ab + b^2)
\]

\subsection{Difference of Cubes:}
\[
a^3 - b^3 = (a - b)(a^2 + ab + b^2)
\]

\textbf{Example:}
\[
x^3 + 8 = (x + 2)(x^2 - 2x + 4)
\]


\subsection{Trinomial: Special Case \texorpdfstring{\(x^2 + bx + c\)}{x² + bx + c}}

This is the case where \(a = 1\). Find two numbers whose product is \(c\) and sum is \(b\).\newline
\textbf{Example:}
\[
x^2 + 5x + 6 = (x + 2)(x + 3)
\]


\subsection{Trinomial: General Form \texorpdfstring{\(ax^2 + bx + c\)}{ax² + bx + c}}

Like in the previous case we are going to use a similar procedure but we start by
multiplying and diving by \(\frac{a}{a}\) and then proceed to do the rest. Look at the example

\[
3x^2 -5x - 2 = \frac{3(3x^2 -5x - 2)}{3} = \frac{{(3x)}^2 -5(3x) - 6}{3} = \frac{(3x- 6)(3x+1)}{3} 
\]
\[
= (x- 2)(3x +1)
\]

\subsection{Perfect Square Trinomial}

These follow the identities:

\[
{(a + b)}^2 = a^2 + 2ab + b^2
\]
\[
{(a - b)}^2 = a^2 - 2ab + b^2
\]

\textbf{Example:}
\[
x^2 + 6x + 9 = {(x + 3)}^2
\]

\subsection{Substitution}

Substitute a more complex expression with a single variable, factor, then back-substitute.
\newline
\textbf{Example:}
\[
x^4 + 2x^2 + 1 \Rightarrow \text{Let } y = x^2 \Rightarrow y^2 + 2y + 1 = {(y + 1)}^2 \Rightarrow {(x^2 + 1)}^2
\]

\subsection{Rationalization of Radicals}

Rationalizing removes radicals from the denominator.
\newline
\textbf{Example (Single Radical):}
\[
\frac{1}{\sqrt{2}} = \frac{\sqrt{2}}{2}
\]

\textbf{Example (Binomial):}
\[
\frac{1}{\sqrt{3} + 1} = \frac{\sqrt{3} - 1}{(\sqrt{3} + 1)(\sqrt{3} - 1)} = \frac{\sqrt{3} - 1}{2}
\]

\subsection{Horner’s Method}

Used to divide a polynomial by a binomial of the form \((x - r)\).
\newline
\textbf{Steps:}
\begin{enumerate}
    \item Write coefficients of the polynomial.
    \item Bring down the first coefficient.
    \item Multiply it by \(r\), add to next coefficient.
    \item Repeat until the remainder.
\end{enumerate}

\textbf{Example:}
Divide \(P(x) = x^3 - 6x^2 + 11x - 6\) by \(x - 1\):

\[
\begin{array}{r|rrrr}
1 & 1 & -6 & 11 & -6 \\
  &   & 1 & -5 & 6 \\
\hline
  & 1 & -5 & 6 & 0 \\
\end{array}
\Rightarrow Q(x) = x^2 - 5x + 6
\]

\subsection{Long Division of Polynomials}

Use the same algorithm as numerical long division.

\textbf{Example:}

\bigskip
\polylongdiv{X^3+X^2+0X-1}{X-1}
