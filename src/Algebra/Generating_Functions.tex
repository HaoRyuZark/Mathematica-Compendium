\newpage
\section{Generating Functions}

\emph{Generating functions} are a central tool in combinatorics and discrete mathematics. At their 
core, a generating function is a formal power series in one or more variables, 
typically used to encode sequences of numbers.

\subsection{Definition}

Given a sequence \( {\{a_n\}}_{n \ge 0} \), its (ordinary) generating function is defined as:

\[
    G(x) = \sum_{n=0}^{\infty} a_n x^n
\]

Here, \( x \) is an indeterminate, and the series is usually treated formally, meaning we do not 
worry about convergence unless the context requires it.

\subsection{Applications}

Generating functions are used to:

\begin{itemize}

    \item Encode and manipulate sequences algebraically.

    \item Solve recurrence relations.

    \item Find closed-form expressions for sequences.

    \item Prove combinatorial identities.

    \item Model and solve counting problems in partitions, compositions, and permutations.

\end{itemize}

For example, if \( a_n \) counts the number of ways to form a sum \( n \) using coins of certain 
denominations, the generating function helps derive a formula for \( a_n \) or analyze its properties.

\subsection{Connection to Subsets}

There is a deep similarity between generating functions and the process of forming subsets. Consider the 
generating function for the set \( \{1, x\} \):

\[
    {(1 + x)}^n = \sum_{k=0}^n \binom{n}{k} x^k
\]

This expansion encodes all subsets of an \( n \)-element set, where each term \( x^k \) corresponds to 
selecting a subset of size \( k \). The coefficient \( \binom{n}{k} \) gives the number of such subsets.
\vspace{\baselineskip}

In this way, generating functions can be viewed as a powerful algebraic analogue of subset selection and 
counting, with the exponents representing sizes or weights, and the coefficients representing counts.
\vspace{\baselineskip}

Generating functions serve as a bridge between algebra and combinatorics, allowing abstract operations to 
yield concrete combinatorial results.

\subsection{Closed form of the Fibonacci Sequence}

Let us use generating function to find a closed form for the Fibonacci number. 
\vspace{\baselineskip}

We know that \(F_0 = 1\) and \(F_1 = 1\) and for \(n \ge 2\) \(F_n = F_{n-1} + F_{n-2}\)
\vspace{\baselineskip}

Now let us use the following function
\[
    F(x) = \sum_{n = 0}^{\infty} F_n x^{n}
\]

Which is going to have the Fibonacci numbers as the coefficients of the power series.

\[
    1 + x + 2x^2 + 3x^3 + 5x^4 + \dots
\]

But before that we have to look at when our recursive formula is defined.
In this case \(n = 2\). Thus, we have to write our expression in the following way.

\begin{align*}
    F(x) &= 1 + x + \sum_{n = 2}^{\infty} F_n x^{n}\\
    &= 1 + x + \sum_{n = 2}^{\infty} F_{n -1} x^{n} + \sum_{n = 2}^{\infty} F_{n - 2}x^n 
\end{align*}

Here we are going to adjust our expression by making our \(n\) be equal.

\begin{align*}
    F(x) &= 1 + x + \sum_{n = 2}^{\infty} F_{n -1} x^{n} + \sum_{n = 2}^{\infty} F_{n - 2}x^n \\ 
    F(x) &= 1 + x + x \left( \sum_{n = 2}^{\infty} F_{n -1} x^{n-1}\right) + x^2 \left( \sum_{n = 2}^{\infty} F_{n -2} x^{n-2}\right)
\end{align*}

And now we are going to reset our sum to start from zero by adding a smart zero to our term with 
\(n - 1\). This allows us to shift the sum to start from 0 because when \(n = 1\) for 
\(F_{n - 1}x^{n - 1}\) turns into \(F_0 x^0\).

\begin{align*}
    F(x) &= 1 + x + x \left( -F_{0}x^0 + F_{0}x^0 + \sum_{n = 2}^{\infty} F_{n-1} x^{n-1}\right) + x^2 \left( \sum_{n = 2}^{\infty} F_{n} x^{n}\right)\\
    F(x) &= 1 + x + x \left( -F_{0}x^0 + \sum_{n = 1}^{\infty} F_{n-1} x^{n-1}\right) + x^2 \left( \sum_{n = 2}^{\infty} F_{n} x^{n}\right)\\
    F(x) &= 1 + x + x \left( \sum_{n = 0}^{\infty} F_{n} x^{n}\right) + x^2 \left( \sum_{n = 2}^{\infty} F_{n} x^{n}\right)\\
    F(x) &= 1 + x + x F(x) - x + x^2 F(x)
\end{align*}

Which at end simplifies to:

\[
    F(x) = \frac{1}{1 - x - x^2}
\]

If you start with \(F_0 = 0\) then the result  is

\[
    F(x) = \frac{x}{1 - x - x^2}
\]


Now we can use partial fraction decomposition to make this expression even more explicit
by using the following expression where \(\alpha\) and \(\beta\) are zeroes of the polynomial in the
denominator.

\[
    \frac{1}{\left(1 - \alpha x\right)\left(1 - \beta x\right)}
\]

The zeros of the polynomial are:

\[
    x_1 = \frac{1 + \sqrt{5}}{2} \text{ and }  x_2 = \frac{1 - \sqrt{5}}{2}    
\]


Now we can write our expression like:
\[
    F(x) = \frac{1}{1 - x - x^2} = \frac{1}{\left(1 - x \cdot \frac{1 + \sqrt{5}}{2}\right)\left(1 - x 
    \cdot \frac{1 - \sqrt{5}}{2}\right)}
\]

Which after performing the partial fraction decomposition left us with:

\[
    F(x) = \left( {\left( \frac{1}{\sqrt{5}} (1 + \sqrt{5}) \right)}^n - {\left( \frac{1}{\sqrt{5}} 
    (1 - \sqrt{5}) \right)}^n \right)
\]

\QED


