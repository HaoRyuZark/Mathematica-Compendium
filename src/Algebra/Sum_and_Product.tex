\section{Sum and Product Notation}

In mathematics, the sum and product notations are compact ways to represent repeated addition and multiplication, respectively. These notations are essential for working with sequences, series, and algebraic expressions.

\subsection{Sum Notation \texorpdfstring{\(\sum\)}{∑}}


The summation symbol \(\sum\) represents the addition of a sequence of terms:
\[
\sum_{i = m}^{n} a_i = a_m + a_{m+1} + \dots + a_n
\]
where \(i\) is the index of summation, \(m\) is the lower bound, and \(n\) is the upper bound.

\begin{itemize}[label=\(-\)]
    \item \textbf{Linearity:}
    \[
    \sum_{i = m}^{n} (a_i + b_i) = \sum_{i = m}^{n} a_i + \sum_{i = m}^{n} b_i
    \]
    \[
    \sum_{i = m}^{n} c \cdot a_i = c \cdot \sum_{i = m}^{n} a_i
    \]
    \item \textbf{Splitting:}
    \[
    \sum_{i = m}^{n} a_i = \sum_{i = m}^{k} a_i + \sum_{i = k+1}^{n} a_i \quad (m \le k < n)
    \]
\end{itemize}

\subsubsection{Change of Index}

Let \(j = i + k\), then:
\[
\sum_{i = m}^{n} a_i = \sum_{j = m + k}^{n + k} a_{j - k}
\]

\textit{Example:}
\[
\sum_{i = 1}^{4} a_i = \sum_{j = 2}^{5} a_{j - 1}
\]

\subsubsection{Power Sums and Their Formulas}

\begin{align*}
\sum_{i = 1}^{n} i &= \frac{n(n+1)}{2} \\
\sum_{i = 1}^{n} i^2 &= \frac{n(n+1)(2n+1)}{6} \\
\sum_{i = 1}^{n} i^3 &= \left[\frac{n(n+1)}{2}\right]^2 \\
\sum_{i = 1}^{n} i^k &= \text{(Higher-order polynomial in \(n\))}
\end{align*}

\paragraph{Derivation of \texorpdfstring{\(\sum_{i=1}^{n} i^2\)}{∑i²}}

We use the method of finite differences or induction. Assume a quadratic form:
\[
\sum_{i=1}^{n} i^2 = An^3 + Bn^2 + Cn
\]
Plug in small values of \(n\) (e.g., 1, 2, 3), solve the system of equations to find:
\[
A = \frac{1}{3}, \quad B = \frac{1}{2}, \quad C = \frac{1}{6}
\Rightarrow \sum_{i = 1}^{n} i^2 = \frac{n(n+1)(2n+1)}{6}
\]

\subsubsection{Telescoping Sum}

A telescoping sum is a sum where intermediate terms cancel out, leaving only the first and last terms.

\textit{Example:}
\[
\sum_{i=1}^{n} \left( \frac{1}{i} - \frac{1}{i+1} \right)
= 1 - \frac{1}{2} + \frac{1}{2} - \frac{1}{3} + \dots + \frac{1}{n} - \frac{1}{n+1}
= 1 - \frac{1}{n+1}
\]

\subsubsection{Geometric Series}

\paragraph{Finite Geometric Series:}
For a geometric sequence \(a, ar, ar^2, \dots, ar^{n-1}\):
\[
\sum_{i = 0}^{n - 1} ar^i = a \cdot \frac{1 - r^n}{1 - r}, \quad r \ne 1
\]

\paragraph{Infinite Geometric Series:}
If \(|r| < 1\), then:
\[
\sum_{i = 0}^{\infty} ar^i = \frac{a}{1 - r}
\]

\subsection{Product Notation \texorpdfstring{\(\prod\)}{∏}}

The product notation \(\prod\) represents repeated multiplication:
\[
\prod_{i = m}^{n} a_i = a_m \cdot a_{m+1} \cdot \dots \cdot a_n
\]


\begin{itemize}[label=\(-\)]
    \item \textbf{Multiplicativity:}
    \[
    \prod_{i = m}^{n} (a_i \cdot b_i) = \left( \prod_{i = m}^{n} a_i \right) \cdot \left( \prod_{i = m}^{n} b_i \right)
    \]
    \item \textbf{Power Rule:}
    \[
    \prod_{i = m}^{n} a^k = a^{k(n - m + 1)}
    \]
\end{itemize}

\subsubsection{Change of Index}

Let \(j = i + k\), then:
\[
\prod_{i = m}^{n} a_i = \prod_{j = m + k}^{n + k} a_{j - k}
\]

\textbf{Example:}
\[
\prod_{i = 1}^{3} a_i = \prod_{j = 2}^{4} a_{j - 1}
\]

\subsubsection{Telescoping Product}

A telescoping product occurs when consecutive terms simplify or cancel.

\textbf{Example:}
\[
\prod_{i = 1}^{n} \frac{i}{i+1} = \frac{1}{2} \cdot \frac{2}{3} \cdot \frac{3}{4} \cdots \frac{n}{n+1} = \frac{1}{n+1}
\]


\subsection{Proof of the geometric Series}

Let \(S_n = \sum_{k = 0}^{n}aq^k\) with \(|q| < 1\).
\\\\
Then \(S_n q = a(\sum_{k = 0}^{n + 1})q^k = a\left( \sum_{k= 0}^{n} q^k + q^{n + 1}\right)\)
\[
a + S_nq = \sum_{k = 0}^{n}aq^k + aq^{n + 1}  = S_n + aq^{n + 1}
\]
Finally we subtract \(S_n\) and \(a\) and after that divide by \(q - 1\)
\[
S_n = \frac{a - aq^{n + 1}}{q - 1}
\]
\newpage