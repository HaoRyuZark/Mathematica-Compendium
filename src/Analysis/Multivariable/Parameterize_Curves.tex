\newpage
\section{Parameterized Curves}

\subsection{Definition}

A \emph{parameterized curve} is a mathematical object where each point on the curve is described as a 
function of one or more parameters, typically denoted by \(t\). Formally, a parameterized curve in 
\( \Reals^n \) is a continuous function:

\[
    \mathbf{r}(t) = (x(t), y(t), z(t), \ldots)
\]

Where \(t\) ranges over some interval \( I \subseteq \Reals \), and each component function 
(e.g., \( x(t) \), \( y(t) \)) is real-valued.

\subsection{Purpose}

The purpose of parameterizing a curve is to provide a flexible and detailed way of describing geometric 
objects, particularly when they cannot be easily expressed as a simple function \( y = f(x) \). 
Parameterizations allow us to:

\begin{itemize}

    \item Describe curves that loop, cross themselves, or have vertical segments.

    \item Compute quantities like arc length, tangent vectors, and curvature.

    \item Easily model motion along a curve, where \(t\) can represent time.

\end{itemize}

\subsection{How to Parameterize a Function}

To parameterize a curve or a function:

\begin{enumerate}

    \item Introduce a new parameter \(t\), often representing time or distance along the curve.

    \item Express the coordinates \( (x, y) \) (or \( (x, y, z) \)) as functions of \(t\).

    \item Ensure that as \(t\) varies over a chosen interval, the set of points \( (x(t), y(t)) \) 
         traces the desired curve.

\end{enumerate}

Sometimes, parameterizations are natural (e.g., circles using trigonometric functions), while in other 
cases, one must carefully design a suitable parameterization based on the curve's properties.

\textbf{Example:}

\textbf{Parameterizing a Circle:}  

Consider the unit circle defined by the equation:

\[
    x^2 + y^2 = 1
\]

A natural parameterization uses the trigonometric functions sine and cosine:

\[
    x(t) = \cos(t), \quad y(t) = \sin(t)
\]

For \( t \in [0, 2\pi] \).  

As \(t\) varies from \( 0 \) to \( 2\pi \), the point \( (x(t), y(t)) \) traces out the entire circle 
exactly once.

\subsection{Arclength and t Parameter}

Recall that the arclength of a curve in 3 dimensions is

\[
    \int_{a}^{b} \sqrt{{f'(t9)}^2 + {g'(t)}^2 + {h'(t)}^2} dt,
\]

which is norm of our tangent vector.

\[
    \int_{a}^{b} \sqrt{|\vec{v}(t)|} dt
\]

\subsubsection{Arclength Parameter}

This is given by

\[
    s(t) = \int_{a}^{t} \sqrt{|\vec{\tau}(t)|} dt,
\]

where \(\tau\) is the length parameter.

This parameter is intrinsic to the curve, and represents one choice because there only one valid
interpretation of the length. The downside is that this is hard to compute.

\subsubsection{The t Parameter}

For the \(t\) parameter, we can say that this is intrinsic to the object along the curve,
it can be interpreted in multiple ways therefore, many choices and is easy to compute. 

