\newpage
\section{Parameterized Curves}

\subsection{Definition}
A \emph{parameterized curve} is a mathematical object where each point on the curve is described as a function of one or more parameters, typically denoted by \( t \). Formally, a parameterized curve in \( \mathbb{R}^n \) is a continuous function:
\[
\mathbf{r}(t) = (x(t), y(t), z(t), \ldots)
\]
where \( t \) ranges over some interval \( I \subseteq \mathbb{R} \), and each component function (e.g., \( x(t) \), \( y(t) \)) is real-valued.

\subsection{Purpose}
The purpose of parameterizing a curve is to provide a flexible and detailed way of describing geometric objects, particularly when they cannot be easily expressed as a simple function \( y = f(x) \). Parameterizations allow us to:
\begin{itemize}[label=\(-\)]
    \item Describe curves that loop, cross themselves, or have vertical segments.
    \item Compute quantities like arc length, tangent vectors, and curvature.
    \item Easily model motion along a curve, where \( t \) can represent time.
\end{itemize}

\subsection{How to Parameterize a Function}
To parameterize a curve or a function:
\begin{enumerate}
    \item Introduce a new parameter \( t \), often representing time or distance along the curve.
    \item Express the coordinates \( (x, y) \) (or \( (x, y, z) \)) as functions of \( t \).
    \item Ensure that as \( t \) varies over a chosen interval, the set of points \( (x(t), y(t)) \) traces the desired curve.
\end{enumerate}
Sometimes, parameterizations are natural (e.g., circles using trigonometric functions), while in other cases, one must carefully design a suitable parameterization based on the curve's properties.
\vspace{\baselineskip}

\textbf{Example:}
\vspace{\baselineskip}

\textbf{Parameterizing a Circle:}  
Consider the unit circle defined by the equation:
\[
x^2 + y^2 = 1
\]
A natural parameterization uses the trigonometric functions sine and cosine:
\[
x(t) = \cos(t), \quad y(t) = \sin(t)
\]
for \( t \in [0, 2\pi] \).  
As \( t \) varies from \( 0 \) to \( 2\pi \), the point \( (x(t), y(t)) \) traces out the entire circle exactly once.

\subsection{Arclength and t Parameter}

Recall that the arclength of a curve in 3 dimensions is

\[
\int_{a}^{b} \sqrt{{f'(t9)}^2 + {g'(t)}^2 + {h'(t)}^2} dt
\]

which is norm of our tangent vector.

\[
\int_{a}^{b} \sqrt{|\vec{v}(t)|} dt
\]

\subsubsection{Arclength Parameter}

This is given by

\[
s(t) = \int_{a}^{t} \sqrt{|\vec{\tau}(t)|} dt
\]

where \(\tau\) is the length parameter.

