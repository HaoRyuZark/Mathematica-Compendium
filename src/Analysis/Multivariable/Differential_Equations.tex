\section{Differential Equations}

A Differential Equation is an equation that relates a function (the dependent varible)
with a varible or multiple variables (the independent variables) and its derivative.

They can be categorized in \emph{Ordinary} and \emph{Partial} Differential Equations.

\subsection{Ordinary Differential Equations}

An equation of the form 

\[
y^n = f(x, y, x', y', \dots, y^{n -1})
\]

is called \emph{Ordinary Explicit Differential Equations of order} \(n\)

If \(y^n = 0\) then it is \emph{implicit}

If the exponent of the dependent varible is 1 and it
is build like a linear equation then it is categorized as \emph{Linear}

\subsection{Initial Value Problems and Solutions}

The specification of an explicit differential equation and the values
\[
x_0,\quad y(x_0) = y_0,\quad y'(x_0) = y_1,\quad \dots,\quad y^{(n-1)}(x_0) = y_{n-1}
\]
is called an \emph{initial value problem (IVP)}.

An \emph{n}-times differentiable function \( y(x) \) that satisfies 
the explicit differential equation is called the \emph{general solution} of the 
differential equation. If it also satisfies the conditions of the initial value problem 
from the previous definition, it is called the \emph{particular solution} of the IVP.

\subsection{Volterra Integral}

The function \(y(x)\) is a solution of the IVP \(y' = f(x,y),\quad y(x_0) = y_0\) if and only if

\[y(x) = y_0 + \int_{x_0}^{x} f(t, y(t))dt\]

\textbf{Proof:}

\[\int_{x_0}^{x} f(t, y(t))dt = \int_{x_0}^{x} y'\]
\[y(x) = y(x_0)\]

The other way would be

\[\left(\int_{x_0}^{x} f(t, y(t))dt + y_0\right)' = f(x, y(x)) = y' \]

\QED

\subsection{The Picard-Lindelöf Iteration Method}

We can approximate the desired solution using the Volterra integral equation as follows. Start with an arbitrary initial function, for example a constant — preferably the initial value:
\[
g_0(x) = y_0
\]
Then form the next approximation to the desired function \( y(x) \) as:
\[
g_{n+1}(x) = y_0 + \int_{x_0}^{x} f(t, g_n(t)) \, dt
\]

\subsection{Integration case}

To solve a ODE of the form \(y' = f(x)\) it is as easy as integrating \(f(x)\).

\subsection{Separation of Variables}

An equation \(F(x,y,y',\dots) = 0\) is a called an equation of \emph{separable variables} if it can
be written in one of the following forms:

\begin{itemize}[label=\(-\)]
    \item \(\frac{dy}{dx} = f(x) g(y)\)
    \item \(\frac{dy}{dx} = \frac{f(x)}{g(y)}\)
     \item \(\frac{dy}{dx} = \frac{g(y)}{f(x)}\)
\end{itemize}

\subsubsection{Steps to solve a ODE of separable variables}

\begin{enumerate}
    \item Solve for the derivative
    \item Check that it is indeed an ODE of separable variables
    \item Separate the variables
    \item Integrate with respect to the independent variable
\end{enumerate}

\textbf{Example:}

Given is \((y - 2)y' = 3x - 5\)

It is an ODE of separable varible
\[y' = \frac{3x - 5}{y - 2}\]

Now we separate the variables and Integrate

\[\int (y - 2)y' dx = \int 3x - 5 dx\]

Note that \(y' = \frac{dy}{dx}\) and \( \frac{dy}{dx} dx = dy\) (only as a convinient fiction :D 
in reality we are changing the variables) thus

\[\int (y - 2)dy = \int 3x - 5 dx\]
\[\frac{y^2}{2} - 2y + c = \frac{3x^2}{2} - 5x + c\]

The solution we get is a implicit solution. To make it look nice solve for \(y\)
if you can.

\subsubsection*{Understanding the fiction}

The previous fiction we used can explained in the following manner.

Given \(y' = g(y)*f(x)\) separate the variables and let \(h(y) = \frac{1}{g(y)}\).

\[h(y)y' = f(x)\]

This looks kinda like the \emph{chain rule} we already know from \emph{differentiation} and
also remember that \(y\) a function of the form \(y(x)\) thus

\[\int h(y(x)) y'(x)dx = \int f(x) dx\]

Use \emph{U-Substitution}

\[\int h(u) du = \int f(x) dx\]

\[H(y(x)) = F(x) + c\]

\[y(x) = H^{-1}(F(x) + c)\]

\QED

\subsection{Geometry of ODE}

You can imagine the plot of an ODE as a \emph{Slope field} where the slope (derivative) of a function
obeys a certain condition, this being the differential equation \(\frac{dy}{dx} = expr\).

A point on this field is called an \emph{Initial Condition} and the curve that goes through
that point in the slope field is called an \emph{Integral curve} which a solution to the differential
equation with respect to that initial condition.

\begin{center}
\begin{tikzpicture}[
    declare function={f(\x) = 2*\x;} % Define which function we're using
]
\begin{axis}[
    MaoYiyi, title={$\dfrac{\mathrm{d}y}{\mathrm{d}x}=2x$}
]
\addplot3 (x,y,0);
\addplot {x^2+0.15}; % You need to find the antiderivative yourself, unfortunately. Good exercise!
\end{axis}
\end{tikzpicture}
\end{center}

\subsection{Existence and Uniqueness}

Let \( f(x, y) \) be continuous on a rectangle \( [a, b] \times \mathbb{R} \). Then the differential equation
\[
y' = f(x, y)
\]
has at least one solution. A unique solution is only guaranteed under additional conditions.

A function \( f : \mathbb{R}^2 \to \mathbb{R} \) is called \emph{Lipschitz continuous with respect to \( y \)} if there exists a constant \( L > 0 \) such that
\[
|f(x, y_1) - f(x, y_2)| \leq L |y_1 - y_2|
\quad \text{for all } (x, y_1), (x, y_2) \in D.
\]

Let \( x_0 \in [a, b] \), and let \( f : [a, b] \times \mathbb{R} \to \mathbb{R} \) be continuous and bounded, and Lipschitz continuous in \( y \). Then the initial value problem
\[
y' = f(x, y), \quad y(x_0) = y_0
\]
has a unique solution on \( [a, b] \).

Or in easier words

If \(f\) and \(\frac{\partial f}{\partial y}\) are continuous near \((x_0, y_0)\) then
there is a unique solution on an interval \(\alpha < x_0 < \beta\) to the IVP

\[y' = f(x,y), \quad y(x_0) = y_0\]

\subsection{How to deal with initial conditions}

After solving a differential equation we may have the problem that certain initial
conditions were set at the start, thus making our general solution not specific.
\\\\
Solving this problem is pretty easy, because we just have to solve for the constant terms
in our general solution with respect to our initial conditions. Like for example \(y(0) = 1\)
we would plug 0 for all \(x\)'s in our general solution the whole equation will be set equal to 1.

\subsection{Solving DE with Substitution}

In some cases a complex function can be simplified via a substitution.

\subsubsection{Case I}

\[y' = f(ax + by(x) + c)\]

With a substitution like \(z(x) = ax + by(x) + c\) the righthand side will become \(f(z)\)
and for the lefthand side we get \(y(x) = \frac{z(x) - ax - c}{b}\).
\\\\
Now we can substitute in the original equation and we get

\[y'(x) = \frac{1}{b} (z'(x) - a) = f(z)\]

\[z'(x) = a + b f(z)\]

Now we can integrate and find the solution and then return to our original variables.
\\\\
\textbf{Example:}

\[y' = (x + y(x))^2\]

Here \(a = 1\), \(b = 1\), \(c = 0\) and \(f(z) = z^2\)

\textbf{1. Substitute}
\[
z(x) = x + y(x)
\]
\[y(x) = z(x) - x\]
\[y'(x) = z'(x) - 1\]

\textbf{2. Plug in the DE}
\[y' = (x + y(x))^2\]
\[z'(x) - 1 = z(x)^2\]

\textbf{3. Integrate}

\[z'(x) = 1 + z(x)^2\]
\[\frac{z'(x)}{1 + z(x)^2}\]
\[\int \frac{1}{1 + z^2} dz = \int 1 dx\]
\[\arctan (z) = x + c\]
\[z = \tan(x + c)\]

\textbf{4. Return to the orignial variables}

\[x + y(x) = \tan(x + c)\]
\[y(x) = \tan(x + c) - x\]

\subsubsection{Case II}

\[y' = f(\frac{y}{x})\]

For this to be valid \(z(x) = \frac{y}{x}\) our \(y(x)\) needs to look like \(y(x) = z(x)x\) and so \(y' = z(x) + z'(x)x\).
\\\\
Now we substitute in the differential equation unsing \(f(z) = z(x) + z'(x)x\)

\[
z'(x) = \frac{f(z) - z(x)}{x}
\]

Now we can integrate and then return to our orignial variables. \(y(x) = z(x) x\) where \(z(x)\) is our
result of the integration.
\\\\
\textbf{Example:}
\[y' = \frac{y(x)}{x} + 1\]

This in converted to \(y' = f(z) = z + 1\)
\\\\
\textbf{1. Substitute}

\[z(x) = \frac{y(x)}{x}\]
\[y(x) = z(x)x\]
\[y'(x) = z(x) + z'(x)x\]

\textbf{2. Plug in the DE}

\[z(x) + z'(x)x = z(x) + 1\]

\textbf{3. Integrate}

\[z'(x)x = 1\]
\[\int z' dz = \int \frac{1}{x} dx\]
\[z = (\ln(x) + c)x\]

\subsection{Linear Differential Equations}

An ODE is called linar if all occurences of the dependent varible and its
derivatives have a exponent of at most 1.

\[a_n(x)y^n + a_{n - 1}y^{n -1} + \cdots + a_1(x)y' + a_0(x)y = b(x)\]

If such an Equation has also the property \(b(x) = 0\) it is called \emph{Homogeneous}

\subsubsection{How to solve a Linear Differential Equation}

In this case we will focus on first Order. ODE.

Write it in the standard form \(y' + p(x)y = f(x)\)

Now use \emph{Integrating Factor Method}

\subsection{Integrating Factor Method}

We are going to multiply our expression by function \(r(x)\) 

\[r(x)y' + r(x)p(x)y = r(x)f(x)\]

Now we would like to write the left side as \(\frac{d}{dx} y r(x)\) because now if we
were to integrate it would take us to the expression in the right.

When we evaluate this new expression we get that

\[\frac{d}{dx} y r(x) = y'r(x) + r'(x)y\]

whihc is not quite what we actually have but it is close. Now note that we can say

\[r'(x) = r(x)p(x)\]

This is a differential equation of separable variables thus we can

\[\frac{r'(x)}{r(x)} = p(x)\]

\[\int \frac{r'(x)}{r(x)}dx = \int p(x)dx\]

\[ \ln(r(x)) = \int p(x)\]

afer using \(e\)

\[ r(x) = e^{\int p(x)}\]

Now we have found a \(r(x)\) and now that we know that such a function exists we can return 
to our initial problem and say

\[\frac{d}{dx}r(x)y = r(x)f(x)\]
\begin{center}
    \boxed{y = \frac{1}{r (x) } \int r (x) f (x) dx}
\end{center}

and 

\[r(x) = e^{\int p (x) }\]

\textbf{Example}

Give are \(y' + 4y = e^{-x}\) and \(y(0) = \frac{4}{3}\)

Here \(4 = p(x)\) and \(e^{-x} = f(x)\)

Now lets use the formula for \(r(x) = e^{\int p (x)}\)

this gives us \(r(x) = e^{\int 4dx} = e^{4x}\)

Now recall the what we saw earlier

\[e^{4x}y' + e^{4x}4y = e^{4x}e^{-x} = e^{3x}\]

Our left side is just \(\frac{d}{dx} e^{4x}y\)

Now lets complete the exercise with our last formula by integrating both sides

\[\frac{d}{dx} e^{4x}y = e^{3x}\]
\[y = \frac{1}{e^4x}\int e^{3x}\]
\[y = \frac{1}{e^4x} \frac{1}{3}e^{3x} + c\]

Which is our general solution

To find our specific solution we just plug 0 in to our general solution that gives us
\(\frac{1}{3} + c = \frac{4}{3}\) thus \(c = 1\)

\subsection{Homogeneous Differential Equations}

A differential equation 

\[M(x,y)dx + N(x,y)dy = 0\]

is considered \emph{Homogeneous} if and only if
\(M(x)\) and \(N(x)\) are \emph{homogeneous} and they have the same degree.
\\\\
The equation \(y' + f(x)y = 0\) is called \emph{homogeneous linear differential equation of first order}.

\subsubsection{Homogeneous Equation}

A function \(f(x,y)\) is called \emph{homogeneous} if and only if it can be written

\[f(xt, yt) = t^n f(x,y)\]

\textbf{Example: }

\[f(x,y) = x^3 + y^3\]
\[f(tx, ty) = (tx)^3 + (ty)^3\]
\[t^3 (x^3 + y^3) = t^3 f(x,y)\]

It is important to note that the key is the each term have a toltal degree of \(n\).
For example: \(x^2y + x3y^2\)

\subsubsection{Homogeneous Functions Theoreme}

If \(f(x,y)\) is homogeneous of degree \(0\) in \(x\) and \(y\) then 
\(f\) is a function of \(\frac{y}{x}\)

Or \(f(x,y,y')\) is homogeneous if it can be written as \(y' = f(\frac{x}{y})\) or \(y' = f(\frac{y}{x})\)

\subsection{Steps for solving Homogeneous DE}

\begin{enumerate}
    \item Write in the form \(M(x,y)dx + N(x,y)dy = 0\)
    \item Check if it is homogeneous
    \item Change the variables to \(y = ux\) or \(x = uy\) depending on the situation
    \item Solve using the separable variables method
    \item Rewrite the answer in term of the original variables again
\end{enumerate}

\textbf{Example:}

Give is the function \((x -y)dx = - xdy\)

\textbf{Step 1.} We write it in the desired form \((x - y)dx + xdy = 0\)

\textbf{Step 2.} We notice that both are of degree \(1\). This time we skip the rigorous check.

\textbf{Step 3.} We are going to choose the simpler function to do the change of variable 
depending on the \emph{differential!} 

The simpler function is \(xdy\) thus we are to change \(y\)

\[y = ux\]
\[dy = ux' + xu'\]

and 

\[u = \frac{y}{x}\]

Now substitute

\[(x - ux)dx + x(udx + xdu) = 0\]
\[xdx - uxdx + xudx + x^{2}du = 0\]
\[(x - ux + xu)dx + x^{2}du = 0\]
\[xdx + x^{2}du = 0\]
\[xdx = - x^{2}du\]
\[dx = - xdu\]

\textbf{Step 4.} Now we can integrate both sides
\[\int \frac{1}{x}dx = \int du\]
\[\ln x + c = -u + c\]

\textbf{Step 5.} No we use \(u = \frac{y}{x}\) to return to our original variables

\[\ln x + c = -\frac{y}{x} + c\]
\[-x\ln x - xc = y\]

\subsection{Inhomogenous Equation}

The equation \(y' + f(x)y = g(x)\) is called \emph{linear inhomogenous differential equation of first order}
. \(g(x)\) is called the \emph{distortion function}. The correspondent IVP is called
\emph{linear} IVP.

\subsection{Solutions of LDE}

Every linear IVP with continuous functions \(f(x)\) and \(g(x)\) plus a bounded \(f(x)\) 
has exactly one solution.
\\\\
\textbf{Proof:}

\begin{align*}
y' = -f(x)y + g(x) \text{ therefore } |-f(x)y_1 + g(x) + -f(x)y_2 + g(x) | &= |f(x)||y_1 - y_2|\\
                                                                        &\le M |y_1 - y_2|
\end{align*}

Therefore the function is Lipschitz continuous, thus a solution is possible.
\QED


\subsubsection{Solving of Inhomogenous DE}

Recall that for the homogeneous case \(y' + f(x)y = 0\) case we use 
\(y = c e^{\int -f(x) dx}\) but now we have \(y' + f(x)y = g(x)\), thus lets use the following
approach

\[y = c(x) e^{\int -f(x) dx}\]

Now use the product rule to derivate

\begin{align*}
y' &= c'(x) e^{\int -f(x) dx} + c(x)e^{\int -f(x) dx}(-(-f(x)))\\
   &= c'(x)e^{\int -f(x) dx} - f(x)y
\end{align*}

thus 

\[y' + f(x)y = c'(x)e^{\int -f(x) dx}\]

therefore \(g(x) = c'(x)e^{\int -f(x) dx}\)
\\\\
Then we can get a solution for \(c(x)\) from

\[
    c'(x) = g(x)e^{\int f(x) dx}
\]


\subsection{Bernoulli Equation}
This is an equation of the form 

\[
y' + P(x)y = Q(x)y^n
\]

Lets use a substitution \(u = y^{1 - n}\) then \(u' = (1 - n)y^{-n} y'\). Now this kinda similar to the original equation
divided by \(y^n\) that looks like

\[
y^{-n}y' + P(x)y^{1 -n} = Q(x)
\]

Notice that \(y^{-n} = \frac{u'}{(1-n)y'}\) thus after substitution

\[
\frac{1}{1 - n}u' + P(x)u = Q(x)
\]

Now our equation is \emph{linear}.
\\\\
\textbf{Example: }

Given is \(y' -5y = \frac{-5}{2}xy^3\).

\[
u = y^{-3}  \text{ and } u' = -2y^{-3}y'
\]

\[
y^{-3} = \frac{u'}{-2y'}
\]

And our equation after dividing by \(y^{3}\)

\[
\frac{1}{y^3}y' - \frac{5}{y^3}y = -\frac{5}{2}x 
\]

Now substitute in the original equation

\[
- \frac{1}{2} u' - 5u = \frac{-5}{2}x
\]

Now we can use the \emph{Integrating Factor Method}. First bring to the standard form.
\(y' + p(x)y = f(x)\)

\[
  u' + 10u = 5x 
\]

Remember that \(r(x) = e^{\int p(x)dx} = e^{\int 10 dx} = e^{10x}\) therefore we have

\[
e^{10x}u' + 10ue^{10x} = e^{10x}5x
\]

Now lets integrate

\[
\frac{d}{dx} \left[e^{10x}u\right] = e^{10x}5x
\]

\[
e^{10x}u = \int e^{10x} 5x dx = 5x \frac{e^{10x}}{10} - \int e^{10x}5dx
\]

\[
e^{10x}u = \int e^{10x} 5x dx = 5x \frac{e^{10x}}{10} - \frac{5}{10}\int e^{10x}dx
\]

\[
e^{10x}u = \frac{xe^{10x}}{2} - \frac{5}{100} e^{10x} + c
\]

\[
u = \frac{x}{2} - \frac{1}{20} + \frac{c}{e^{10x}} = y^{-2}
\]


\subsection{Autonomous Equations}

An \emph{Autonomous Differential Equation} only depends on the dependent
variable \(y\).\textbf{Example: } \(\frac{dy}{dt} = (1 + y)(1 -y)\).

\subsubsection{Equilibrium Solutions}
Values where \(f(y) = 0\) are \emph{Equilibrium Solutions}.

\subsubsection*{Asymptotically stable}
An equilibrium solution \(y = a\) is \emph{asymptotically stable} if solutions
that start near \(a\) tend towards \(a\) as \(t \to \infty\).

\subsubsection*{Asymptotically unstable}
An equilibrium solution \(y = a\) is \emph{asymptotically unstable} if solutions
that start near \(a\) leave as \(t \to \infty\)


\newpage