\newpage
\section{Multi-variable Differentiation}

\subsection{Partial Derivatives}

Let \( f : U \subseteq \mathbb{R}^n \to \mathbb{R} \) and \( \vec{X}_0 = (x_1^{(0)}, \dots, x_n^{(0)}) \in U \). Then \( f \) is \emph{partially differentiable with respect to \( x_i \)} at \( \vec{X}_0 \) if the limit:

\[
\frac{\partial f}{\partial x_i}(\vec{X}_0) := \lim_{h \to 0} \frac{f(x_1^{(0)}, \dots, x_i^{(0)} + h, \dots, x_n^{(0)}) - f(\vec{X}_0)}{h}
\]

exists. This derivative measures the rate of change of \( f \) in the direction of the \( x_i \)-axis while keeping all other variables fixed.

A function is \emph{(partially) differentiable} at \( \vec{X}_0 \) if all partial derivatives \( \frac{\partial f}{\partial x_i}(\vec{X}_0) \) exist.

\subsection{Differentiability}
We call a function \(f\) differentiable at \((x_1, x_2, \cdots, x_n)\) if and only if:

\[
f(x_1 + \Delta x_1, x_2 + \Delta x_2, \dots, x_n + \Delta x_n) - f(x_1, x_2, \dots, x_n) = 
\]

\[
\frac{\partial f(x_1, y_2, \dots, x_n)}{\partial x_1} \Delta x_1 + \frac{\partial f(x_1, x_2, \dots, x_n)}{\partial x_2} \Delta x_2 + \cdots + E_1(\Delta x_1) + E_2(\Delta x_n) + \cdots + E_n(\Delta x_n)
\]

where \(\lim_{\Delta x_1 \to 0}\frac{E_1(\Delta x_1)}{\Delta x_1} = \lim_{\Delta x_2 \to 0}\frac{E_2(\Delta x_2)}{\Delta x_2} = \cdots = 0\)

\subsection{The Gradient}

Let \( f : U \subseteq \mathbb{R}^n \to \mathbb{R} \) be differentiable at \( \vec{X}_0 = (x_1^{(0)}, \dots, x_n^{(0)}) \). 
The \emph{gradient} of \( f \) at \( \vec{X}_0 \) is:

\[
\nabla f(\vec{X}_0) = \begin{pmatrix}
\frac{\partial f}{\partial x_1}(\vec{X}_0) \\
\vdots \\
\frac{\partial f}{\partial x_n}(\vec{X}_0)
\end{pmatrix}
\]

\subsubsection{Origin of the formula}

Suppose along the curve \(r(t) = x_1(t)\hat{v_1} + x_2(t)\hat{v_2} + \cdots + x_n(t)\hat{v_n}\) that \(f(x_1(t), \dots, x_n(t)) = C\)

\[
\frac{\partial}{\partial t}f(x_1(t), \dots, x_n(t)) = \frac{\partial}{\partial t}C
\]

\[
\frac{\partial f}{\partial x_1}\frac{dx_1}{dt} + \cdots + \frac{\partial f}{\partial x_n}\frac{dx_n}{dt} = 0
\]

\[
\langle \left(\frac{\partial f}{\partial x_1}\hat{v_1} + \cdots + \frac{\partial f}{\partial x_n}\hat{v_n}\right) , \left( \frac{dx_1}{dt}\hat{v_1} + \cdots + \frac{dx_n}{dt}\hat{v_n}\right)\rangle = 0
\]

From the formula we see that: \(\nabla f\) is the normal and \(\frac{d\vec{r}}{dt}\) the tangent vector of the curve

\QED

\subsubsection{Gradient Operations:}
\begin{itemize}[label=\(-\)]
\item \( \nabla(f + g) = \nabla f + \nabla g \)
\item \( \nabla(\lambda f) = \lambda \nabla f \) for \( \lambda \in \mathbb{R} \)
\item \( \nabla(fg) = f \nabla g + g \nabla f \)
\end{itemize}

The \(\frac{\operatorname{grad}f}{\|\operatorname{grad}f\|}\) points in the direction of the steepest increase of \( f \) and
 \(- \frac{\operatorname{grad}f}{\|\operatorname{grad}f\|}\) in the lowest increase.

\subsection{The Tangent Plane}

The tangent plane to a differentiable surface \( z = f(x, y) \) at the point \( (x_0, y_0) \) is the plane that best approximates the surface near that point. It is given by the linearization of \( f \):

\[
T(x, y) = f(x_0, y_0) + f_x(x_0, y_0)(x - x_0) + f_y(x_0, y_0)(y - y_0)
\]

This plane touches the graph of the function at a single point\\
and shares its slope in both \( x \)- and \( y \)-directions.

It can also be written in the parameterized form:

\[
T(x,y) = \begin{pmatrix} x_0\\ y_0\\ z_0\end{pmatrix} + \lambda 
\begin{pmatrix} 1\\ 0 \\ f_x(x_0,y_0)\end{pmatrix} + \mu \begin{pmatrix}
0 \\ 1 \\ f_y(x_0, y_0)
\end{pmatrix}
\]

\subsubsection{Generalization}
\[
T(\vec{X}) = f(\vec{x_0}) + \sum_{i = 1}^{n} f_{x_i}(\vec{x_0})(x_i - x_{i}^0)
\]
\[
 = f(\vec{x_0}) + \langle \nabla f, (\vec{x} - \vec{x_0})\rangle
\]

\textbf{Example: \( f(x, y) = 4x^2y + xy^2 + 1 \)}
\vspace{\baselineskip}

Compute partial derivatives:
\[
f_x(x, y) = 8xy + y^2, \quad f_y(x, y) = 4x^2 + 2xy
\]

At the point \( (1, 1) \):
\[
f(1, 1) = 4{(1)}^2(1) + 1{(1)}^2 + 1 = 6 \\
f_x(1, 1) = 8(1)(1) + 1 = 9, \quad f_y(1, 1) = 4{(1)}^2 + 2(1)(1) = 6
\]

Thus, the tangent plane is:
\[
T(x, y) = 6 + 9(x - 1) + 6(y - 1)
\Rightarrow T(x, y) = 9x + 6y - 9
\]

\subsection{The Directional Derivative}

Let \( f : \mathbb{R}^n \to \mathbb{R} \) be differentiable at \( \vec{x}_0 \), 
and let \( \vec{v} \in \mathbb{R}^n \) be a direction vector with \( \|\vec{v}\| = 1 \). The 
\emph{directional derivative} of \( f \) at \( \vec{x}_0 \) in the direction \( \vec{v} \) is defined as:

\[
D_{\vec{v}}f(\vec{x}_0) := \lim_{h \to 0} \frac{f(\vec{x}_0 + h\vec{v}) - f(\vec{x}_0)}{h}
\]

\textbf{Formula:}
\[
D_{\vec{v}}f(\vec{x}_0) = \langle \nabla f(\vec{x}_0), \vec{v} \rangle
\]

\emph{Schwarz Inequality (Cauchy–Schwarz):}
\[
|\langle \vec{a}, \vec{b} \rangle| \le \|\vec{a}\| \cdot \|\vec{b}\|
\]

This implies:
\[
|D_{\vec{v}}f(\vec{x}_0)| \le \|\nabla f(\vec{x}_0)\|
\]

\subsubsection{Derivation of the Directional Derivative Formula}

Let \( f(x, y) \) be differentiable, and let \( \vec{v} = (e_1, e_2) \) be a direction vector. We aim to derive the formula for the directional derivative \( D_{\vec{v}} f(x_0, y_0) \).

We begin by considering the tangent plane to the surface \( z = f(x, y) \) at the point \( (x_0, y_0, f(x_0, y_0)) \). This plane is spanned by the vectors:

\[
\vec{u}_1 = 
\begin{pmatrix}
1 \\
0 \\
f_x(x_0, y_0)
\end{pmatrix}, \quad
\vec{u}_2 = 
\begin{pmatrix}
0 \\
1 \\
f_y(x_0, y_0)
\end{pmatrix}
\]

These correspond to directional derivatives along the \( x \)- and \( y \)-axes, respectively.

Now consider the direction vector \( \vec{v} = (e_1, e_2) \). Lift it into 3D space (onto the tangent plane) as:

\[
\vec{u}_3 =
\begin{pmatrix}
e_1 \\
e_2 \\
D_{\vec{v}}f
\end{pmatrix}
\]

Since all three vectors lie in the same plane, the determinant of the matrix formed by these vectors as columns must vanish:

\[
\det(\vec{u}_1, \vec{u}_2, \vec{u}_3) = 0
\]

Explicitly:
\[
\det
\begin{pmatrix}
1 & 0 & e_1 \\
0 & 1 & e_2 \\
f_x(x_0, y_0) & f_y(x_0, y_0) & D_{\vec{v}}f
\end{pmatrix} = 0
\]

Expanding the determinant gives:
\[
f_x(x_0, y_0) \cdot e_2 - f_y(x_0, y_0) \cdot e_1 + D_{\vec{v}}f = 0
\]

Solving for \( D_{\vec{v}}f \), we obtain:
\[
D_{\vec{v}}f = f_x(x_0, y_0) \cdot e_1 + f_y(x_0, y_0) \cdot e_2
\]

\textbf{Vector notation:}
\[
D_{\vec{v}}f = \langle \nabla f(x_0, y_0), \vec{v} \rangle
\]

This is the desired result: the directional derivative equals the dot product of the gradient of \( f \) and the direction vector \( \vec{v} \).

\subsection{The Total Differential}

If \( f : \mathbb{R}^n \to \mathbb{R} \) is differentiable at \( \vec{x}_0 \), then the 
\emph{total differential} of \( f \) at \( \vec{x}_0 \) is the linear approximation:

\[
df = \sum_{i=1}^n \frac{\partial f}{\partial x_i}(\vec{x}_0) \, dx_i = \langle \nabla f(\vec{x}_0), d\vec{x} \rangle
\]

This expresses how small changes in the input variables propagate into changes in the function value.


\subsection{Absolute and Relative Error}

Given an approximate value \( \tilde{x} \) for the exact value \( x \):

\begin{itemize}[label=\(-\)]
\item \emph{Absolute error:} \(\Delta z_{\max} \le |f_{x_1}||\Delta x_1| + \cdots + |f_{x_n}||\Delta x_n| \)
\item \emph{Relative error:} \( \varepsilon = \frac{\Delta z_{\max}}{z}\) with dependence on the error of the inputs \(\frac{\Delta x}{x}\) etc.
\end{itemize}

In multi-variable contexts, similar formulas apply using vector norms.
\vspace{\baselineskip}

\textbf{Absolute Error Example}
\vspace{\baselineskip}

We consider the function:
\[
z = \sqrt{x^2 + y^2}
\]

with the measured values:
\[
x = 4 \, \text{cm}, \quad y = 3 \, \text{cm}
\]

Both \( x \) and \( y \) are measured with a precision of \( \Delta x = \Delta y = 0{,}1 \, \text{cm} \). The side length calculated is \( z = 5 \, \text{cm} \). We now determine the maximum possible absolute error in \( z \).

\textbf{Step 1: Partial derivatives of \( z \):}
\[
\frac{\partial z}{\partial x} = \frac{x}{\sqrt{x^2 + y^2}}, \quad
\frac{\partial z}{\partial y} = \frac{y}{\sqrt{x^2 + y^2}}
\]

\textbf{Step 2: Evaluate at the given point:}
\[
\frac{\partial z}{\partial x}(4, 3) = \frac{4}{5}, \quad
\frac{\partial z}{\partial y}(4, 3) = \frac{3}{5}
\]

\textbf{Step 3: Use the absolute error formula:}
\[
\Delta z \leq \left| \frac{\partial z}{\partial x} \right| \cdot \Delta x + \left| \frac{\partial z}{\partial y} \right| \cdot \Delta y
\]
\[
\Delta z \leq \frac{4}{5} \cdot 0{,}1 + \frac{3}{5} \cdot 0{,}1 = \frac{7}{5} \cdot 0{,}1 = 0{,}14 \, \text{cm}
\]

\textbf{Conclusion:}  
The result for \( z \) is accurate to within \( \boxed{0{,}14 \, \text{cm}} \).


\subsection{The Chain Rule}

Let \( f : \mathbb{R}^n \to \mathbb{R} \) be differentiable, and suppose \( \vec{x} = \vec{x}(t) \in \mathbb{R}^n \) is a differentiable path. Then:

\[
\frac{d}{dt} f(\vec{x}(t)) = \langle \nabla f(\vec{x}(t)), \vec{x}'(t) \rangle
\]

This is the chain rule in vector form.

Or to be more explicit, the small change in \(t\) makes a change in \(x_1, x_2, \cdots\) and the addition of all this changes
sum up to the total change in the original function.

\[
\frac{dz}{dt} = \frac{dz}{dx_1}\frac{dx_1}{dt} + \cdots + \frac{dz}{dx_n}\frac{dx_n}{dt}
\]

\textbf{Relative Error Example}

The relative error of a function \( z = f(x_1, \dots, x_n) \) is defined by:
\[
\varepsilon = \frac{\Delta z_{\max}}{z}
\]

Assuming \( z \) depends on variables with known relative measurement errors, and the function is differentiable, we estimate:

\[
\varepsilon = \frac{|\partial f / \partial x_1| \cdot \Delta x_1 + \cdots + |\partial f / \partial x_n| \cdot \Delta x_n}{f(x_1, \dots, x_n)}
\]

or more directly using relative errors:

\[
\varepsilon \approx \left| \frac{\Delta x_1}{x_1} \right| + \cdots + \left| \frac{\Delta x_n}{x_n} \right|
\]

\textbf{Example: Volume of a Cuboid}
\vspace{\baselineskip}

Let:
\[
z = f(a, b, c) = a \cdot b \cdot c
\]

Assume the side lengths \( a = b = c \) are measured with a uniform relative error \( \varepsilon_{\text{in}} = \frac{\Delta a}{a} = \frac{\Delta b}{b} = \frac{\Delta c}{c} = \varepsilon \)

Then the partial derivatives are:
\[
\frac{\partial z}{\partial a} = b c, \quad
\frac{\partial z}{\partial b} = a c, \quad
\frac{\partial z}{\partial c} = a b
\]

Applying the relative error formula:
\[
\frac{\Delta z_{\max}}{z} \leq \frac{bc \cdot \Delta a + ac \cdot \Delta b + ab \cdot \Delta c}{abc}
= \frac{\Delta a}{a} + \frac{\Delta b}{b} + \frac{\Delta c}{c} = 3 \varepsilon
\]

\textbf{Target accuracy:}  
If the final result \( z \) must be accurate within \( \varepsilon_z = 0{,}01 \) (i.e., 1), then we must satisfy:
\[
3 \varepsilon \leq 0{,}01 \Rightarrow \varepsilon \leq \frac{0{,}01}{3} = \boxed{0{,}0033} \quad \text{(or } 0{,}33\% \text{)}
\]

\textbf{Conclusion:}
\vspace{\baselineskip}

Each input measurement must be made with a maximum relative error of \( \boxed{0{,}33\%} \) to ensure the output \( z = abc \) is accurate to within 1%.

\subsubsection{Implicit Differentiation}

If a function \( F(x, y) = 0 \) defines \( y \) implicitly as a function of \( x \), and \( F \) is differentiable, then:

\[
\frac{dz}{dx} = -\frac{F_x}{F_z}
\]

This generalizes to higher dimensions using the total differential and the implicit function theorem.
\vspace{\baselineskip}

\textbf{Example:}
\vspace{\baselineskip}

Given is \(x^2 + y^2 + z^2 = 1\) differentiate with respect to \(x\).
See \(z\) or the target function as function in terms of the other variables. 
In this case \(z(x,y)\).

\[
\frac{d}{dx} x^2 + y^2 + z^2  -1 = 0
\]

\[
 2x + 2z \frac{dz}{dx}  = 0
\]

\[
\frac{dz}{dx} = - \frac{x}{z}
\]

\subsection{Divergence and Curl (Rotation)}

Let \( \vec{F} = (F_1, F_2, F_3) : \mathbb{R}^3 \to \mathbb{R}^3 \) be a vector field.

\subsubsection{Divergence:}
\[
\operatorname{div} \vec{F} = \langle\nabla, \vec{F}\rangle = \frac{\partial F_1}{\partial x_1} + \frac{\partial F_2}{\partial x_2} + \frac{\partial F_3}{\partial x_3}
\]

\subsubsection{Curl:}
\[
\operatorname{rot} \vec{F} = \nabla \times \vec{F} = \begin{pmatrix}
\frac{\partial F_3}{\partial x_2} - \frac{\partial F_2}{\partial x_3} \\
\frac{\partial F_1}{\partial x_3} - \frac{\partial F_3}{\partial x_1} \\
\frac{\partial F_2}{\partial x_1} - \frac{\partial F_1}{\partial x_2}
\end{pmatrix}
\]

These describe how the field spreads or rotates around a point.

Where vector field \( \operatorname{div}\vec{f} > 0\) then that region is a called a source
and where \(\operatorname{rot}\vec{f} = 0\) is called a whirlpool

\subsection{Schwarz’s Theorem (Clairaut’s Theorem)}

Let \( f : \mathbb{R}^n \to \mathbb{R} \) be twice continuously differentiable. Then for any \( i \ne j \), the mixed partial derivatives satisfy:

\[
\frac{\partial^2 f}{\partial x_i \partial x_j} = \frac{\partial^2 f}{\partial x_j \partial x_i}
\]

That is, the order of partial differentiation does not matter if all second derivatives are continuous.

\subsection{The Jacobian}

Let \( \vec{f} : \mathbb{R}^n \to \mathbb{R}^m \), where
\[
\vec{f}(\vec{x}) = \begin{pmatrix}
f_1(x_1, \dots, x_n) \\
\vdots \\
f_m(x_1, \dots, x_n)
\end{pmatrix}
\]

The \emph{Jacobian matrix} of \( \vec{f} \) is the \( m \times n \) matrix:
\[
J_{\vec{f}}(\vec{x}) = \begin{pmatrix}
\frac{\partial f_1}{\partial x_1} & \cdots & \frac{\partial f_1}{\partial x_n} \\
\vdots & \ddots & \vdots \\
\frac{\partial f_m}{\partial x_1} & \cdots & \frac{\partial f_m}{\partial x_n}
\end{pmatrix}
\]

In the case \( n = m \), the \emph{Jacobian determinant} \( \det J_{\vec{f}}(\vec{x}) \) 
describes local invertibility and orientation.


\subsection{Taylor's Theorem}

Let \( f : \mathbb{R}^n \to \mathbb{R} \) be \( k \)-times continuously differentiable at \( \vec{x}_0 \). Then the Taylor polynomial of degree 2 around \( \vec{x}_0 \) is:

\[
f(\vec{x}) \approx f(\vec{x}_0) + \langle \nabla f(\vec{x}_0), \vec{x} - \vec{x}_0 \rangle + \frac{1}{2} {(\vec{x} - \vec{x}_0)}^T H_f(\vec{x}_0)(\vec{x} - \vec{x}_0) + \cdots,
\]

where \( H_f(\vec{x}_0) \) is the \emph{Hessian matrix} of second partial derivatives.
\vspace{\baselineskip}

\textbf{Example:} 
\vspace{\baselineskip}

Let \( f(x, y) = e^y \sin(x + 2y) \). We expand around the point \( (a, b) = (0, 0) \).
\vspace{\baselineskip}

First, compute the necessary derivatives at \( (0, 0) \):

\begin{align*}
f(0, 0) &= \sin(0) = 0 \\
f_x &= e^y \cos(x + 2y) \quad \Rightarrow \quad f_x(0, 0) = 1 \\
f_y &= 2y e^y \cos(x + 2y) + e^y \sin(x + 2y) \quad \Rightarrow \quad f_y(0, 0) = 0 + 2 = 2 \\
f_{xx} &= -e^y \sin(x + 2y) \quad \Rightarrow \quad f_{xx}(0, 0) = 0 \\
f_{xy} = f_{yx} &= 2e^y \sin(x + 2y) + e^y \cos(x +2y) \quad \Rightarrow \quad f_{xy}(0, 0) = 3 \\
f_{yy} &= -3 e^y \sin(x + 2y) + 4e^y \cos(x + 2y) \quad \Rightarrow \quad f_{yy}(0, 0) = 0 + 4 = 4
\end{align*}


Now, the second order Taylor polynomial at \( (0, 0) \) is:

\begin{align*}
f(x, y) \approx\ & 0 + 1 \cdot x + 2 \cdot y \\
&+ \frac{1}{2} \cdot 0 \cdot x^2 + 3 \cdot x y + \frac{1}{2} \cdot 4 \cdot y^2 \\
=\, & x + 2y + 3xy + 2y^2
\end{align*}


\subsection{Relative Extrema}

To determine local extrema for \( f : \mathbb{R}^n \to \mathbb{R} \), follow this procedure:

\textbf{Step 1:} Find critical points by solving:
\[
\nabla f(\vec{x}) = 0
\]

\textbf{Step 2:} Compute the Hessian matrix:
\[
H_f(\vec{x}) = \left( \frac{\partial^2 f}{\partial x_i \partial x_j} \right)
\]

\textbf{Step 3:} Analyze the Hessian at critical points:
\begin{itemize}[label=\(-\)]
\item If \( H_f \) is positive definite \( \Rightarrow \) local minimum
\item If \( H_f \) is negative definite \( \Rightarrow \) local maximum
\item If \( H_f \) has mixed signs (indefinite) \( \Rightarrow \) saddle point
\end{itemize}

\begin{itemize}[label=\(-\)]
    \item If \(f_{xx}(x_0, y_0) > 0, d > 0 \implies\) positive definite
    \item If \(f_{xx}(x_0, y_0) < 0, d > 0 \implies\) negative definite
    \item If \(d < 0 \implies \) saddle point indefinite
    \item If \(d = 0 \implies \) Next derivative decides
\end{itemize}

\textbf{Example: \( f(x, y) = x^2 + y^2 \)}

\[
\nabla f = \begin{pmatrix} 2x \\ 2y \end{pmatrix} \Rightarrow \text{Critical point at } (0, 0)
\]

\[
H_f = \begin{pmatrix} 2 & 0 \\ 0 & 2 \end{pmatrix} \Rightarrow \text{positive definite}
\Rightarrow (0, 0) \text{ is a local minimum}
\]

\subsection{Finding Candidate Points for Zeros While Seeking Relative Extrema}

To find local minima or maxima of a function \( f : \mathbb{R}^n \to \mathbb{R} \), we first identify 
\emph{critical points}, which are the candidates for relative extrema. 
These are the points where the gradient of \( f \) vanishes or does not exist.
\vspace{\baselineskip}

\textbf{Step 1: Compute the Gradient}

The gradient of \( f \), denoted \( \nabla f \), collects all the first partial derivatives:

\[
\nabla f(x_1, \dots, x_n) =
\begin{pmatrix}
\frac{\partial f}{\partial x_1} \\
\vdots \\
\frac{\partial f}{\partial x_n}
\end{pmatrix}
\]

\textbf{Step 2: Solve the System \( \nabla f = 0 \)}

Find all points \( \vec{x}_0 \in \mathbb{R}^n \) such that:

\[
\nabla f(\vec{x}_0) = \vec{0}
\]

This is a nonlinear system of \( n \) equations in \( n \) variables. The solutions \( \vec{x}_0 \) 
are the \emph{candidate points for extrema}, and also referred to as the 
\textbf{stationary points} or \emph{critical points} of the function.
\vspace{\baselineskip}

You use the following techniques to find the points: 
\begin{itemize}[label=\(-\)]
    \item Factorization
    \item Addition of rows
    \item Substitution
\end{itemize}

There are more but which one to use depends on the kind problem.
\vspace{\baselineskip}

\textbf{Step 3: Analyze the Nature of the Critical Points}

Once all critical points \( \vec{x}_0 \) satisfying \( \nabla f(\vec{x}_0) = 0 \) are found, analyze each using
 the \emph{second derivative test} involving the Hessian matrix \( H_f \):

\[
H_f(\vec{x}) = \left( \frac{\partial^2 f}{\partial x_i \partial x_j} \right)
\]

Evaluate \( H_f \) at each critical point and use the eigenvalues or definiteness to classify:

\begin{itemize}[label=\(-\)]
    \item If \( f_{xx}(x_0, y_0) > 0 \) and \( D > 0 \), then \( f \) has a \emph{local minimum} at \( (x_0, y_0) \).
    \item If \( f_{xx}(x_0, y_0) < 0 \) and \( D > 0 \), then \( f \) has a \emph{local maximum} at \( (x_0, y_0) \).
    \item If \( D < 0 \), then \( (x_0, y_0) \) is a \emph{saddle point}.
    \item If \( D = 0 \), the \emph{second derivative test is inconclusive}; higher-order derivatives must be considered.
\end{itemize}

\subsubsection{Note: Non-differentiable Points}

If the function \( f \) is not differentiable at some point \( \vec{x}_0 \), but defined there, this point may also be a 
candidate for extremum and should be examined separately using limit-based analysis or directional behavior.

\subsection{Lagrange Multipliers}

Let \( f : \mathbb{R}^n \to \mathbb{R} \) be the function to optimize, subject to the constraint \( g(\vec{x}) = 0 \), where \( g : \mathbb{R}^n \to \mathbb{R} \).  

If we look at the constraint intersection with the original function we are going to notice that along the intersection
there are maxima and minima with their respective tangent lines
and gradients that are normal to them. We also are going to see that both the gradient of \(f\) and 
the gradient of \(g\) are just scalar version of each other so: 

This leads to the \emph{Lagrange system}:

\[
\begin{cases}
\nabla f(\vec{x}) = \lambda \nabla g(\vec{x}) \\
g(\vec{x}) = 0
\end{cases}
\]

For more conditions  we can use \(g_1, \dots, g_n\) and \(\lambda_1, \dots, \lambda_n\)

\[
\begin{cases}
\nabla f(\vec{x}) = \lambda_1 \nabla g_1(\vec{x}) + \cdots + \lambda_n \nabla g_n(\vec{c}) \\
g_1(\vec{x}) = 0 \\
\vdots \\
g_n(\vec{x}) = 0
\end{cases}
\]


We can also compact these equations in the \emph{Lagrangian}

\[
\mathcal{L}(x,y, \dots, \lambda_1, \dots, \lambda_n) = f(x, y, \dots) + \lambda_1 g_1(x, y, \dots) + \cdots + \lambda_n g_n(x, y, \dots)
\]

\subsubsection{Determinant method}

Sometimes it is a good way to solve a problem of this topic is to:

\begin{enumerate}
    \item Build the \emph{Lagrangian}
    \item \(\nabla L = 0\) and maybe compute the determinant to generate another equation to solve
    \item Get rid of \(\lambda\) via factorization, substitution, addition of rows (preferred), etc.
    \item Solve the system without \(\lambda\)
\end{enumerate}

\subsubsection{The Bordered Hessian}
The Bordered Hessian matrix is:
\[
H_B =
\begin{bmatrix}
0 & {\left( \nabla g(x_1, \dots, x_n) \right)}^T \\
\nabla g(x_1, \dots, x_n) & \nabla^2 \mathcal{L}(x, \dots, x_n, \lambda_1, \dots, \lambda_n)
\end{bmatrix}
\]
Where:
\begin{itemize}[label=\(-\)]
  \item \( \nabla  g(x_1, \dots, x_n) \) is the \( m \times n \) Jacobian matrix of the constraints.
  \item \( \nabla^2 \mathcal{L}(x, \dots, x_n, \lambda_1, \dots, \lambda_n) \) is the \( n \times n \) Hessian of the Lagrangian with respect to \( x \).
\end{itemize}

Or basically a Hessian of the Lagrangian starting with\(lambda_1\).
\vspace{\baselineskip}

\textbf{Example: Maximizing \( f(x, y) = xy \) under the constraint \( 2x + 2y = 40 \)}
\vspace{\baselineskip}

We want to find the maximum of the function:

\[
f(x, y) = xy
\]

subject to the constraint:

\[
g(x, y) = 2x + 2y - 40 = 0
\]

The gradient of the constraint is:

\[
\nabla g = \begin{pmatrix} 2 \\ 2 \end{pmatrix}
\]

Since \( \nabla g \ne 0 \), the rank condition is fulfilled, and no further critical points need to be analyzed.
\vspace{\baselineskip}

\textbf{Step 1: Build the Lagrangian}

\[
\mathcal{L}(x, y, \lambda) = f(x, y) + \lambda g(x, y) = xy + \lambda (2x + 2y - 40)
\]

\textbf{Step 2: Compute the necessary conditions}

\[
\frac{\partial \mathcal{L}}{\partial x} = y + 2\lambda = 0 \quad \Rightarrow \quad \lambda = -\frac{y}{2}
\]

\[
\frac{\partial \mathcal{L}}{\partial y} = x + 2\lambda = 0 \quad \Rightarrow \quad \lambda = -\frac{x}{2}
\]

Equating:

\[
-\frac{y}{2} = -\frac{x}{2} \Rightarrow x = y
\]

Using the constraint:

\[
2x + 2x = 40 \Rightarrow 4x = 40 \Rightarrow x = y = 10
\]

\textbf{Step 3: Hessian Test with Constraint}

We construct the bordered Hessian:
\[
H = \begin{pmatrix}
0 & 1 & 2 \\
1 & 0 & 2 \\
2 & 2 & 0
\end{pmatrix}
\]

Since \( \det(H) = 8 > 0 \), and the Lagrange conditions are satisfied, this indicates a \emph{local maximum}.
\vspace{\baselineskip}

\textbf{Conclusion:}
\vspace{\baselineskip}

The maximum value is:
\[
f(10, 10) = 10 \cdot 10 = \boxed{100}
\]


\subsection{The Tangent Vector}

Let \( \vec{X}(t) \in \mathbb{R}^n \) be a differentiable, parameterized curve. The \emph{tangent vector} at the point 
\( \vec{X}(t_0) \) is given by:

\[
\vec{X}'(t_0) = \begin{pmatrix}
    x_1 ' (t_0) \\ x_2 ' (t_0) \\ \vdots \\ x_n ' (t_0)
\end{pmatrix}
\]

This vector points in the direction in which the curve is moving at \( t_0 \), and its magnitude corresponds to the
 instantaneous speed. The unit tangent vector is:

\[
\vec{X}(t) = \frac{\vec{X}'(t)}{\|\vec{X}'(t)\|}
\]

The speed of the vector in \(\mathbb{R}^2\) is given by:

\[
\vec{X}(t) = \begin{pmatrix}
    t \\ f(t)
\end{pmatrix}
\]

and the tangent is:

\[
T(t) = \begin{pmatrix}
    t \\ f(t)
\end{pmatrix} + \lambda \begin{pmatrix}
    1 \\ f'(t)
\end{pmatrix}
\]

\subsubsection{Derivation of the Tangent Vector Formula}

Given a position vector function \(\vec{r}(t)\), the tangent vector to the curve described by \(\vec{r}(t)\) is 
obtained by taking the derivative of \(\vec{r}(t)\) with respect to time. This derivative is defined as the following limit:

\[
\frac{d\vec{r}}{dt} = \lim_{\Delta t \to 0} \frac{\vec{r}(t + \Delta t) - \vec{r}(t)}{\Delta t}
\]

Assuming that \(\vec{r}(t)\) is expressed in terms of the canonical basis vectors \(\hat{\imath}, \hat{\jmath}, \hat{k}\), 
we can write:

\[
\vec{r}(t) = x(t)\hat{\imath} + y(t)\hat{\jmath} + z(t)\hat{k}
\]

Then, the increment becomes:

\[
\vec{r}(t + \Delta t) = x(t + \Delta t)\hat{\imath} + y(t + \Delta t)\hat{\jmath} + z(t + \Delta t)\hat{k}
\]

So the difference in the numerator of the derivative is:

\begin{align*}
\vec{r}(t + \Delta t) - \vec{r}(t) &= \left[x(t + \Delta t) - x(t)\right]\hat{\imath} \\
&\quad + \left[y(t + \Delta t) - y(t)\right]\hat{\jmath} \\
&\quad + \left[z(t + \Delta t) - z(t)\right]\hat{k}
\end{align*}

Dividing by \(\Delta t\) and taking the limit:

\begin{align*}
\frac{d\vec{r}}{dt} &= \lim_{\Delta t \to 0} \frac{\vec{r}(t + \Delta t) - \vec{r}(t)}{\Delta t} \\
&= \left( \lim_{\Delta t \to 0} \frac{x(t + \Delta t) - x(t)}{\Delta t} \right) \hat{\imath} \\
&\quad + \left( \lim_{\Delta t \to 0} \frac{y(t + \Delta t) - y(t)}{\Delta t} \right) \hat{\jmath} \\
&\quad + \left( \lim_{\Delta t \to 0} \frac{z(t + \Delta t) - z(t)}{\Delta t} \right) \hat{k} \\
&= \frac{dx}{dt} \hat{\imath} + \frac{dy}{dt} \hat{\jmath} + \frac{dz}{dt} \hat{k}
\end{align*}

This vector \(\frac{d\vec{r}}{dt}\) points in the direction of motion and is tangent to the curve at each point \(t\).

