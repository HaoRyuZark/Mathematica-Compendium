\newpage
\section{Coordinate Systems}

Coordinate systems provide different ways of describing points in space, adapted to the geometry of the 
problem. Below are the most common systems used in multi-variable calculus.

\subsection{Cartesian Coordinates}

The \emph{Cartesian coordinate system} uses mutually orthogonal axes. A point in \( \Reals^2 \) 
is given by:

\[
    (x, y) \quad \text{and in } \Reals^3: \quad (x, y, z)
\]

This is the most natural system for rectangular domains and is the basis for most definitions in vector 
calculus. The coordinate axes are orthogonal and equidistant.


\subsection{Polar Coordinates}

In polar coordinates, a point \( (x, y) \in \Reals^2 \) is described by:

\[
    r \ge 0 \text{ (radius from origin)}, \quad \varphi \in [0, 2\pi) \text{ (angle from x-axis)}
\]

\textbf{Transformations:}

\[
    x = r \cos \varphi, \quad y = r \sin \varphi
\]

This comes from our right triangle with \(r = 1\) where

\[
    \cos \varphi = \frac{x}{r} \quad \sin \varphi = \frac{y}{r}
\]

\textbf{Inverse:}

\[
    r = \sqrt{x^2 + y^2}, \quad \varphi = \arctan\left(\frac{y}{x}\right)
\]

This system is useful for circular or radial symmetry. The Jacobian determinant is:

\[
    |J| = r
\]

\subsection{Cylindrical Coordinates}

Cylindrical coordinates extend polar coordinates by adding a height \emph{z} component:

\[
    (r, \varphi, z) \quad \text{where } r \ge 0, \ \varphi \in [0, 2\pi), \ z \in \Reals
\]

\textbf{Transformations:}

\[
    x = r \cos \varphi, \quad y = r \sin \varphi, \quad z = z
\]

\textbf{Inverse:}

\[
    r = \sqrt{x^2 + y^2}, \quad \varphi = \arctan\left(\frac{y}{x}\right), \quad z = z
\]

This coordinate system is ideal for objects with rotational symmetry about the \emph{z}-axis. 
The Jacobian determinant is:

\[
    |J| = r
\]


\subsection{Spherical Coordinates}

In spherical coordinates, a point \( (x, y, z) \in \Reals^3 \) is represented using:

\[
    \rho \ge 0 \text{ (radial distance)}, \quad \theta \in [0, \pi] \text{ (inclination)}, \quad 
    \varphi \in [0, 2\pi) \text{ (azimuth)}
\]

\textbf{Transformations:}

\[
    x = \rho \sin \theta \cos \varphi, \quad y = \rho \sin \theta \sin \varphi, \quad z = \rho \cos 
    \theta
\]

\textbf{Inverse:}

\[
    \rho = \sqrt{x^2 + y^2 + z^2}, \quad \theta = \arccos\left(\frac{z}{\rho}\right), \quad 
    \varphi = \arctan\left(\frac{y}{x}\right)
\]

Spherical coordinates are useful for spherical symmetry (e.g., in gravitational or electric fields). 
The Jacobian determinant is:

\[
    |J| = \rho^2 \sin \theta
\]


Each coordinate system is tailored to specific geometries and greatly simplifies integrals when used 
appropriately. Coordinate transformations require careful application of the Jacobian 
determinant to account for stretching or distortion of space.




