\newpage
\section{Sequences and Series}

A sequence is an ordered list of numbers, usually defined by a formula or a recurrence relation. It is 
one of the fundamental objects of study in mathematical analysis.

\subsection{Arithmetic Sequence}

An \emph{arithmetic sequence} is a sequence where the difference between consecutive terms is constant.

\[
    a_n = a_1 + (n - 1)d
\]

\begin{itemize}

    \item \(a_1\) is the first term

    \item \(d\) is the common difference \(d = a_{n + 1} - a_{n}\)

\end{itemize}

\textbf{Example:}
\vspace{\baselineskip}

\[
    a_n = 3 + (n - 1) \cdot 2 = 2n + 1 \quad \text{(Odd numbers)}
\]

\subsection{Geometric Sequence}

A \emph{geometric sequence} is a sequence where each term is obtained by multiplying the previous term 
by a fixed non-zero constant.

\[
    a_n = a_1 \cdot r^{n-1}
\]

\begin{itemize}

    \item \(a_1\) is the first term

    \item \(r\) is the common ratio \(r = \frac{a_{n + 1}}{a_n}\)

\end{itemize}

\textbf{Example:}
\vspace{\baselineskip}

\[
    a_n = 2 \cdot 3^{n-1} = 2, 6, 18, 54, \dots
\]

\subsection{Function Sequence}

A \emph{function sequence} is sequence composed of functions and its noted like:

\[
    {(f)}_n \text{with } x^t := \{ f: D\to R | \ f \text{ is a function}\}
\]

\subsection{Convergence}

A sequence \((a_n)\) \emph{converges} to a limit \(L\) if for every \(\varepsilon > 0\), there exists an 
integer \(n_0\) such that for all \(n \ge n_0\):

\[
    |\forall \varepsilon > 0\ \exists n \in \Naturals: \forall i \ge n\ |a_i - L| < \varepsilon
\]

In this case, we write:

\[
    \lim_{n \to \infty} a_n = L
\]

\subsection{Cauchy Sequence}

A sequence \((a_n)\) is a \emph{Cauchy sequence} if for every \(\varepsilon > 0\), there exists an 
integer \(n_0\) such that for all \(n, m \ge n_0\):

\[
    \forall \varepsilon > 0\ \exists n,m \in \Naturals: |a_n - a_m| < \varepsilon\ \text{with } n < m
\]

Every convergent sequence is a Cauchy sequence. In complete metric spaces (like \(\Reals\)), the 
converse also holds.
\vspace{\baselineskip}

\textbf{Example: Convergence Exercise}
\vspace{\baselineskip}

Let \(a_i = \frac{1}{i}\) and \(\varepsilon = 0.01\). We want to find \(i_0\) such that:

\[
    \left|\frac{1}{n} - 0\right| < \varepsilon \implies \frac{1}{i} \le \frac{1}{n} < \varepsilon \rightarrow \frac{1}{\varepsilon} < n
\]
\[
    \Rightarrow \frac{1}{i} < 0.01 \Rightarrow i > 100
    \Rightarrow i_0 = 101
\]

\subsection{Definition of \texorpdfstring{\(\varepsilon\)}{ε}-Neighborhood}

The \emph{\(\varepsilon\)-neighborhood} of a point \(a \in \Reals\) is the set:

\[
    B_\varepsilon(a) = \{x \in \Reals \mid |x - a| < \varepsilon\}
\]

This is the open interval \((a - \varepsilon, a + \varepsilon)\).

\subsection{Supremum, Infimum, Maximum and Minimum}

The \emph{supremum} of a set \(A \subseteq \Reals\) is the 
least upper bound: the smallest real number \(s\) such that \(a \le s\) for all \(a \in A\).

The \emph{infimum} of a set \(A\) is the greatest lower bound: 
the largest number \(t\) such that \(a \ge t\) for all \(a \in A\). The \emph{maximum} is the largest 
element in the set, if it exists.

The \emph{minimum} is the smallest element in the set, if it exists.
\vspace{\baselineskip}

\textbf{Example:}
\vspace{\baselineskip}

For \(A = (0,1)\):

\begin{itemize}

    \item \(\sup A = 1\) (not in \(A\))

    \item \(\inf A = 0\) (not in \(A\))

    \item \(\max A\) and \(\min A\) do not exist

\end{itemize}

\subsection{Limit and Accumulation Point}

\begin{itemize}

    \item A \emph{limit} of a sequence \((a_n)\) is the value the terms get arbitrarily close to as 
    \(n \to \infty\).

    \item An \emph{accumulation point} (or limit point) of a set \(A\) is a point \(x\) such that every 
    neighborhood of \(x\) contains infinitely many points of \(A\).

\end{itemize}


\textbf{Example:}
\vspace{\baselineskip}
 
The sequence \(a_n = {(-1)}^n + \frac{1}{n}\) has two accumulation points: \(1\) and \(-1\).

\subsection{Monotonicity by Difference and Quotient}

\emph{Monotonicity by Difference:}

\begin{itemize}

    \item If \(a_{n+1} - a_n \ge 0\) for all \(n\), then the sequence is non-decreasing.


    \item If \(a_{n+1} - a_n \le 0\), it is non-increasing.

\end{itemize}

\emph{Monotonicity by Quotient:} (Typically for positive sequences)

\begin{itemize}

    \item If \(\frac{a_{n+1}}{a_n} \ge 1\), then the sequence is non-decreasing.


    \item If \(\frac{a_{n+1}}{a_n} \le 1\), it is non-increasing.

\end{itemize}

\subsection{Divergence (Definition)}

A sequence \emph{diverges} if it does not converge. That is, there is no real number \(L\) such that:

\[
    \lim_{n \to \infty} a_n = L
\]

\begin{itemize}

    \item A sequence can diverge to \(\infty\) or \(-\infty\)

    \item A sequence can also oscillate without approaching any limit

\end{itemize}

\textbf{Example:}
\vspace{\baselineskip}
The sequence \(a_n = {(-1)}^n\) diverges since it oscillates between \(1\) and \(-1\).

\subsection{Subsequences and Their Properties}

A \emph{subsequence} of a sequence \((a_n)\) is a sequence \((a_{n_k})\) where \((n_k)\) is a strictly 
increasing sequence of natural numbers.

\textbf{Properties:}

\begin{itemize}

    \item Every subsequence of a convergent sequence converges to the same limit.

    \item A sequence converges if and only if all of its subsequences converge to the same limit.

    \item A bounded sequence always has at least one convergent subsequence (Bolzano-Weierstraß theorem).

\end{itemize}

\subsection{Series}

A \emph{series} is the sum of the terms of a sequence:

\[
    \sum_{n=1}^{\infty} a_n
\]

We define the partial sums \(S_n = \sum_{k=1}^n a_k\). If the sequence \((S_n)\) converges to a limit \(S\), then the series is said to converge:

\[
    \sum_{n=1}^{\infty} a_n = S
\]

\subsection{Power Series}

A \emph{power series} is a series of the form:

\[
    \sum_{n=0}^{\infty} a_n {(x - x_0)}^n
\]

Where \(x_0\) is the center of the expansion. Power series converge within a radius \emph{R}:

\[
    R = \frac{1}{\limsup\limits_{n \to \infty} \sqrt[n]{|a_n|}}
\]

\subsection{Point-wise Convergence}

A sequence of functions \((f_n)\) converges \emph{point-wise} to a function \(f\) on a set \(A\) if:

\[
    \forall x \in A, \quad \lim_{n \to \infty} f_n(x) = f(x)
\]

Point-wise convergence does not preserve properties like continuity or differentiability.

\subsection{Uniform Convergence}

A sequence of functions \((f_n)\) converges \emph{uniformly} to \(f\) on \(A\) if:

\[
    \forall \varepsilon > 0, \exists n_0 \in \Naturals \text{ such that } \forall n \ge n_0, \forall x \in A, \quad |f_n(x) - f(x)| < \varepsilon
\]

\emph{Key Property:} Uniform convergence preserves continuity, integration, and differentiation under 
certain conditions.

\subsection{Cauchy Criterion for Series}

A series \(\sum a_n\) converges if and only if for every \(\varepsilon > 0\) there exists \(n_0\) 
such that:

\[
    \left| \sum_{k = m}^{n} a_k \right| < \varepsilon \quad \forall n > m \ge n_0
\]

\subsection{Interval Nesting and the Interval Nesting Theorem}

Let \((I_n)\) be a sequence of closed intervals:

\[
    I_n = [a_n, b_n] \quad \text{with } I_{n+1} \subseteq I_n, \text{ and } \lim_{n \to \infty} (b_n - a_n) = 0
\]

Then, the intersection contains exactly one point:

\[
    \bigcap_{n=1}^{\infty} I_n = \{x\}
\]

This is useful in proving the existence of limits, roots, and fixed points.

\subsection{Weierstraß Approximation Theorem}

The \emph{Weierstraß Approximation Theorem} states:

Every continuous function \(f: [a, b] \rightarrow \Reals\) can be uniformly approximated by a 
polynomial \(P(x)\), i.e., for every \(\varepsilon > 0\) there exists a polynomial \(P(x)\) such that:

\[
    |f(x) - P(x)| < \varepsilon \quad \forall x \in [a, b]
\]

This theorem is foundational in numerical analysis and approximation theory.

\subsection{Zero Sequence}

A sequence \((a_n)\) is called a \emph{null sequence} or \emph{zero sequence} if:

\[
    \lim_{n \to \infty} a_n = 0
\]

Every zero sequence is convergent (to 0), and plays an essential role in convergence proofs and 
epsilon-delta arguments.

\subsection{The Monotony Principle}

Every bounded monotonic sequence converges.

\begin{itemize}

    \item If \((a_n)\) is non-decreasing and bounded above, then \(\lim a_n\) exists and equals 
    \(\sup \{a_n\}\).

    \item If \((a_n)\) is non-increasing and bounded below, then \(\lim a_n = \inf \{a_n\}\).

\end{itemize}

To prove the convergence of a sequence via this principle we follow the next steps

\begin{enumerate}

    \item Prove the Monotonicity by induction

    \item Calculate the limit

    \item Prove the limitedness via induction

    \item Write the conclusion

\end{enumerate}

\textbf{Example: } 
\vspace{\baselineskip}

\(a_{n + 1} = \sqrt{2a_n - 3} + 1\)
\vspace{\baselineskip}

Let us assume that the series is decreasing or \(a_{n + 1} < a_{n}\)

\begin{align*}
    a_{n + 2} &< a{n + 1} \\
    \sqrt{2a_{n + 1} -3} + 1 &< \sqrt{2a_{n} - 3} + 1 \\
    \sqrt{2a_{n + 1} -3} &< \sqrt{2a_{n} - 3} \\
    2a_{n + 1} - 3 &< 2a_{n} - 3 \\
    a_{n + 1} &< a_{n}
\end{align*}

\QED

Now let us calculated the limit

\[
    \lim_{n \rightarrow \infty} a_{n + 1} = \lim_{n \rightarrow \infty} \sqrt{2a_n - 3} + 1
\]

Because of the recursion we know that both \(a_{n + 1}\) and \(a_n\) have the same value, 
and we call it \(a\).

\[
    a = \sqrt{2a -3} +1
\]
\[
    a = 2
\]

We only have to prove the limitedness now:

For \(n = 0\) we have \(a_1 = 4 > 2\) then let us assume that the \(a\) is never going bellow 2.

\[
    a_{n + 1} = \sqrt{2a_n - 3} + 1 \ge \sqrt{2(2) - 3} + 1 = 2
\]

By setting \(a_n = 2\) we get the limit again

\QED

\subsection{Operations on Sequences}

\subsubsection{Sum of Sequences}

The sum of two sequences \((a_n)\) and \((b_n)\) is a new sequence \((c_n)\) where each term 
\(c_n\) is the sum of the corresponding terms of \((a_n)\) and \((b_n)\):

\[
    c_n = a_n + b_n, \quad \forall n \in \Naturals.
\]

If \(\lim_{n \to \infty} a_n = A\) and \(\lim_{n \to \infty} b_n = B\), then the limit of the sum sequence is:

\[
    \lim_{n \to \infty} (a_n + b_n) = A + B.
\]

\subsubsection{Multiplication of Sequences}

The product of two sequences \((a_n)\) and \((b_n)\) is a new sequence \((d_n)\) where each term \(d_n\) 
is the product of the corresponding terms of \((a_n)\) and \((b_n)\):

\[
    d_n = a_n \cdot b_n, \quad \forall n \in \Naturals.
\]

If \(\lim_{n \to \infty} a_n = A\) and \(\lim_{n \to \infty} b_n = B\), then the limit of the product sequence is:

\[
    \lim_{n \to \infty} (a_n \cdot b_n) = A \cdot B.
\]

\subsubsection{Absolute Value of a Sequence}

The absolute value of a sequence \((a_n)\) is a new sequence \((e_n)\) where each term \(e_n\) is the absolute value of the corresponding term of \((a_n)\):

\[
    e_n = |a_n|, \quad \forall n \in \Naturals.
\]

If \(\lim_{n \to \infty} a_n = A\), then the limit of the absolute value sequence is:

\[
    \lim_{n \to \infty} |a_n| = |A|.
\]

\subsubsection{Conjugate of a Complex Sequence}

If \((a_n)\) is a sequence of complex numbers, where \(a_n = x_n + i y_n\) with \(x_n, y_n \in \Reals\), then the conjugate of \((a_n)\) is a new sequence \((\overline{a_n})\) where each term \(\overline{a_n}\) is the complex conjugate of \(a_n\):

\[
    \overline{a_n} = x_n - i y_n, \quad \forall n \in \Naturals.
\]

If \(\lim_{n \to \infty} a_n = A = X + i Y\), then the limit of the conjugate sequence is:

\[
    \lim_{n \to \infty} \overline{a_n} = \overline{A} = X - i Y.
\]

\subsubsection{Complex Limit (Real and Imaginary Parts)}

For a sequence of complex numbers \((a_n)\) where \(a_n = x_n + i y_n\), the limit of the sequence 
\(\lim_{n \to \infty} a_n = A = X + i Y\) exists if and only if the limits of the real part sequence 
\((x_n)\) and the imaginary part sequence \((y_n)\) exist individually:

\[
    \lim_{n \to \infty} a_n = A \iff \left( \lim_{n \to \infty} \text{Re}(a_n) = \text{Re}(A) = X \quad \text{and} \quad \lim_{n \to \infty} \text{Im}(a_n) = \text{Im}(A) = Y \right).
\]

This means we can analyze the convergence of a complex sequence by examining the convergence of its real 
and imaginary parts separately.

\subsubsection{Asymptotic Equality}

Two sequences \({(a_n)}_{n \in \Naturals}\) and \({(b_n)}_{n \in \Naturals}\) are said to be 
asymptotically equal, denoted by \(a_n \sim b_n\), if

\[
    \lim_{n \to \infty} \frac{a_n}{b_n} = 1,
\]

provided that \(b_n \neq 0\) for all sufficiently large \(n\). Asymptotic equality implies that the sequences behave similarly as \(n\) approaches infinity.

