

\section{The Exponential Function and Relatives}

The exponential function and its related functions such as sine, cosine, and the hyperbolic functions are fundamental in both pure and applied mathematics. This section explores these functions through their series expansions and interrelationships.

\subsection{Exponential Function as a Series}

The exponential function \(e^x\) is defined as an infinite series:
\[
e^x = \sum_{n=0}^{\infty} \frac{x^n}{n!}
= 1 + \frac{x}{1!} + \frac{x^2}{2!} + \frac{x^3}{3!} + \cdots
\]
This series converges for all real and complex values of \(x\).

\subsection{Sine and Cosine Definitions with \texorpdfstring{\(e\)}{e}}

Using Euler's identities and Taylor series, sine and cosine can be defined in terms of exponential functions:
\[
\sin x = \frac{e^{ix} - e^{-ix}}{2i}, \quad
\cos x = \frac{e^{ix} + e^{-ix}}{2}
\]

These definitions are consistent with their Taylor expansions:
\[
\sin x = \sum_{n=0}^{\infty} (-1)^n \frac{x^{2n+1}}{(2n+1)!}, \quad
\cos x = \sum_{n=0}^{\infty} (-1)^n \frac{x^{2n}}{(2n)!}
\]

\subsection{Euler's Formula}

One of the most elegant formulas in mathematics is Euler’s formula:
\[
e^{ix} = \cos x + i\sin x
\]

This identity provides a deep connection between exponential and trigonometric functions and is valid for all real (and complex) \(x\).

\textbf{Special Case:}
\[
e^{i\pi} + 1 = 0
\]
This is known as Euler's Identity and is often celebrated as one of the most beautiful equations in mathematics.

\subsection{Reason for the Infinite Series of \texorpdfstring{\(e\)}{e} (Differentiation)}

The exponential series:
\[
e^x = \sum_{n=0}^{\infty} \frac{x^n}{n!}
\]
has the remarkable property that its derivative is the same function:
\[
\frac{d}{dx} e^x = \sum_{n=1}^{\infty} \frac{n x^{n-1}}{n!} = \sum_{n=1}^{\infty} \frac{x^{n-1}}{(n-1)!}
\]
Now replace the index \(m = n - 1\):
\[
= \sum_{m=0}^{\infty} \frac{x^m}{m!} = e^x
\]
This shows that \(e^x\) is the unique function (up to constant multiples) that is equal to its own derivative.

\subsection{Hyperbolic Functions with \texorpdfstring{\(e\)}{e}}

The hyperbolic functions are analogs of the trigonometric functions but defined using the exponential function:

\begin{align*}
\sinh x &= \frac{e^x - e^{-x}}{2} \\
\cosh x &= \frac{e^x + e^{-x}}{2} \\
\tanh x &= \frac{\sinh x}{\cosh x} = \frac{e^x - e^{-x}}{e^x + e^{-x}}
\end{align*}

These functions satisfy identities similar to those of trigonometric functions, for example:
\[
\cosh^2 x - \sinh^2 x = 1
\]

They naturally arise in the study of differential equations, special relativity, and the geometry of hyperbolas.

\subsection{Inverse of the Hyperbolic Trigonometric Functions}

\begin{itemize}[label=\(-\)]
    \item \(\sinh^{-1}(x) = \ln(x + \sqrt{x^2 + 1})\)
    \item \(\cosh^{-1}(x) = \ln(x + \sqrt{x^2 - 1})\)
    \item \(\tanh^{-1}(x) = \frac{1}{2} \ln\left(\frac{1 + x}{1 - x}\right)\)
    \item \(\coth^{-1}(x) = \frac{1}{2 \ln\left(\frac{x + 1}{x - 1}\right)}\)
    \item \(\operatorname{sech}^{-1} = \ln\left(\frac{1 + \sqrt{1 - x^2}}{x}\right)\)
    \item \(\operatorname{csch}^{-1} = \ln\left( \frac{1}{x} + \frac{\sqrt{1 + x^2}}{|x|}\right)\) 
\end{itemize}
\newpage