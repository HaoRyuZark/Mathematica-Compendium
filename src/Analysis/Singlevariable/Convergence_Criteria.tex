\newpage
\section{Convergence Criteria}

In this section, we present various tools used to determine whether an infinite series or sequence converges. These criteria are essential for analyzing the behavior of sequences and series in analysis.

\subsection{Quotient (Ratio) Test}

Let \(\sum a_n\) be a series with positive terms, and suppose the limit
\[
L = \lim_{n \to \infty} \left| \frac{a_{n+1}}{a_n} \right|
\]
exists.

\begin{itemize}[label=\(-\)]
\item If \(L < 1\), the series converges absolutely.
\item If \(L > 1\) or \(L = \infty\), the series diverges.
\item If \(L = 1\), the test is inconclusive.
\end{itemize}

\textbf{Proof of the Ratio Test (for \(L < 1\))}

Let \(r < 1\) such that \(L < r < 1\). Then for sufficiently large \(n\), we have:
\[
\left| \frac{a_{n+1}}{a_n} \right| \le r \Rightarrow a_{n+1} \le r a_n
\]

By induction:
\[
a_{n+k} \le a_n r^k
\]
So,
\[
\sum_{n=N}^\infty a_n \le a_N \sum_{k=0}^\infty r^k = \frac{a_N}{1 - r}
\]
Hence the series converges by comparison with a geometric series.

\subsection{Zero Sequence Criterion}

If a series \(\sum a_n\) converges, then:
\[
\lim_{n \to \infty} a_n = 0
\]
The converse is not true — the harmonic series is a classic counterexample.

\subsection{Majorant (Comparison Test)} 

If \(0 \le a_n \le b_n\) for all \(n\) and \(\sum b_n\) converges, then \(\sum a_n\) converges.

\subsection{Minorant} 

If \(0 \le b_n \le a_n\) and \(\sum b_n\) diverges, then \(\sum a_n\) diverges.

\subsection{Root Test}

Let \(\sum a_n\) be a series with non-negative terms, and suppose:
\[
L = \limsup_{n \to \infty} \sqrt[n]{|a_n|}
\]

\begin{itemize}[label=\(-\)]
\item If \(L < 1\), the series converges absolutely.
\item If \(L > 1\), the series diverges.
\item If \(L = 1\), the test is inconclusive.
\end{itemize}

\textbf{Proof of the Root Test (for \(L < 1\))}

Let \(r\) be such that \(L < r < 1\). Then for large \(n\):
\[
\sqrt[n]{|a_n|} \le r \Rightarrow |a_n| \le r^n
\]
Since \(\sum r^n\) converges, by comparison \(\sum |a_n|\) converges absolutely.

\subsection{Leibniz Criterion (Alternating Series Test)}

Let \(\sum (-1)^n a_n\) be a series with:
\begin{itemize}[label=\(-\)]
\item \(a_n\) decreasing: \(a_{n+1} \le a_n\)
\item \(\lim a_n = 0\)
\end{itemize}
Then the series converges.

\textbf{Example:}
\[
\sum_{n=1}^\infty \frac{(-1)^{n+1}}{n}
\]

\subsection{Compactness Criterion}

A sequence \((x_n)\) in \(\mathbb{R}^n\) has a convergent subsequence if and only if it is bounded. This is a consequence of the \textbf{Bolzano-Weierstraß theorem}, which is a form of sequential compactness in \(\mathbb{R}^n\).

\subsection{Cauchy Criterion I}

A sequence \((a_n)\) converges if and only if it is a \emph{Cauchy sequence}:
\[
\forall \varepsilon > 0, \exists n_0 \in \mathbb{N}, \forall n, m \ge n_0 \Rightarrow |a_n - a_m| < \varepsilon
\]

This is an internal (non-limit based) criterion for convergence, valid in complete metric spaces.

\subsection{Cauchy Criterion II}
A series \(a_n = \sum_{i = 0}^{k}x_i\) converges \(\iff \forall d \in \mathbb{R} d > 0 
\exists N \in \mathbb{N} \forall n,m > N: \left| \sum_{i = 0}^{n}x_i + \sum_{i=0}^{m}\right|\)
\[= \left|\sum_{i = 0}^{n}x_i\right| < d\]


\subsection{The Cauchy Product}

Let \(\sum a_n\), \(\sum b_n\) be convergent series. The \emph{Cauchy product} is:
\[
\sum_{n=0}^\infty c_n, \quad \text{where } c_n = \sum_{k=0}^n a_k b_{n-k}
\]

If \(\sum a_n\) and \(\sum b_n\) converge absolutely, then their Cauchy product converges absolutely and:
\[
\sum c_n = \left( \sum a_n \right) \left( \sum b_n \right)
\]

\subsubsection{Cauchy Product Example}

Prove that \(e^x  e^y = e^{x + y}\)\\
Note that \(e^x = \sum_{n = 0}^{\infty} \frac{x^n}{n!}\) and \(e^y = \sum_{n = 0}^{\infty} \frac{y^n}{n!}\)

Therefore

\[
  e^x e^y = \sum_{n = 0}^{\infty} \frac{x^n}{n!} \sum_{n = 0}^{\infty} \frac{y^n}{n!}
\]

And because of the absolute convergence
\[
  e^x e^y = \sum_{n = 0}^{\infty}\sum_{k = 0}^{n} \frac{x^k}{n!} \frac{y^{n - k}}{(n-k)!}
\]

Finally
\[
  e^x e^y = \sum_{n = 0}^{\infty}\sum_{k = 0}^{n}x^k y^{n - k} \binom{n}{k} \frac{1}{n!} = \sum_{n = 0}^{\infty} \frac{(x + y)^n}{n!}
\]
\QED
\subsection{Geometric Series}

\[
\sum_{n=0}^\infty ar^n = \frac{a}{1 - r} \quad \text{for } |r| < 1
\]

If \(|r| \ge 1\), the series diverges.

\subsection{Harmonic Series}

\[
\sum_{n=1}^{\infty} \frac{1}{n}
\]

This series diverges, even though its terms approach zero.

\textbf{Proof:} Group terms:
\[
\left(\frac{1}{2}\right) + \left(\frac{1}{3} + \frac{1}{4}\right) + \left(\frac{1}{5} + \dots + \frac{1}{8}\right) + \dots > \sum_{k=1}^\infty \frac{1}{2}
\]

The lower bound is a divergent series, hence the harmonic series diverges.
