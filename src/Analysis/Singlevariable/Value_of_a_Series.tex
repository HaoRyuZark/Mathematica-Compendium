\newpage
\section{Calculating Series Values}

\subsection{Telescoping Sums and Partial Fraction Decomposition}

Telescoping sums are series where most of the terms cancel out, leaving only a few terms. Partial fraction decomposition is a technique used to rewrite rational functions as a sum of simpler fractions, which can often lead to telescoping sums.

\textbf{Example: \(\sum_{k=1}^{\infty} \frac{1}{4k^2 - 1}\)}

Consider the infinite series

\[
    S = \sum_{k=1}^{\infty} \frac{1}{4k^2 - 1}.
\]

We can decompose the fraction using partial fraction decomposition:

\[
    \frac{1}{4k^2 - 1} = \frac{1}{(2k - 1)(2k + 1)} = \frac{A}{2k - 1} + \frac{B}{2k + 1}.
\]

Multiplying both sides by \((2k - 1)(2k + 1)\) gives

\[
    1 = A(2k + 1) + B(2k - 1).
\]

To solve for \(A\) and \(B\), we can use the following values of \(k\):

\begin{itemize}

    \item For \(k = \frac{1}{2}\): \(1 = A(1 + 1) + B(0) \implies A = \frac{1}{2}\).

    \item For \(k = -\frac{1}{2}\): \(1 = A(0) + B(-1 - 1) \implies B = -\frac{1}{2}\).

\end{itemize}

Thus, we have

\[
    \frac{1}{4k^2 - 1} = \frac{1}{2} \left( \frac{1}{2k - 1} - \frac{1}{2k + 1} \right).
\]

Now we can write the series as a telescoping sum:

\[
    S = \frac{1}{2} \sum_{k=1}^{\infty} \left( \frac{1}{2k - 1} - \frac{1}{2k + 1} \right).
\]

Let \(S_n\) be the \(n\)-th partial sum:

\begin{align*}
    S_n &= \frac{1}{2} \left[ \left( \frac{1}{1} - \frac{1}{3} \right) + \left( \frac{1}{3} - \frac{1}{5} \right) + \left( \frac{1}{5} - \frac{1}{7} \right) + \cdots + \left( \frac{1}{2n - 1} - \frac{1}{2n + 1} \right) \right] \\
    &= \frac{1}{2} \left[ 1 - \frac{1}{2n + 1} \right].
\end{align*}

Taking the limit as \(n\) approaches infinity, we get

\[
    S = \lim_{n \to \infty} S_n = \lim_{n \to \infty} \frac{1}{2} \left( 1 - \frac{1}{2n + 1} \right) = \frac{1}{2} (1 - 0) = \frac{1}{2}.
\]

Therefore, the value of the series is \(\frac{1}{2}\).
