\newpage
\section{Important Theorems of Single Variable Calculus}

\subsection{Intermediate Value Theorem}

Let \(f\) be a continuous function on the closed interval \([a, b]\). If \( f(a) \ne f(b) \) 
and \emph{N} is any number between \( f(a) \) and \( f(b) \), then there exists \( c \in (a, b) \) 
such that \( f(c) = N \).
\vspace{\baselineskip}

\textbf{Proof:}  

Without loss of generality, assume \( f(a) < N < f(b) \).  
Define:

\[
    S = \{ x \in [a, b] \mid f(x) \le N \}
\]

Since \( a \in S \), \emph{S} is non-empty and bounded above by \(b\). Let \( c = \sup S \).

By the continuity of \(f\), we have:

\[
    \lim_{x \to c^-} f(x) = f(c) = \lim_{x \to c^+} f(x)
\]

Since \( f(x) \le N \) for all \( x < c \), and for any \( \varepsilon > 0 \), 
there exists \( x > c \) with \( f(x) > N \), then by the definition of supremum and continuity:

\[
    f(c) = N
\]

\QED

\begin{center}
\begin{tikzpicture}[scale=1]
\draw[->] (-0.5,0) -- (5.5,0) node[right] {\(x\)};
\draw[->] (0,-1) -- (0,4.5) node[above] {\(f(x)\)};
\draw[scale=1,domain=0.5:5,smooth,variable=\x,blue,thick] plot ({\x},{0.5*(\x-3)^2+1});
\draw[dashed] (1.2,0) -- (1.2,2.6);
\draw[dashed] (4.2,0) -- (4.2,1.82);
\draw[dashed] (0,1.82) -- (5.5,1.82);
\node at (5.3,1.65) {\(N\)};
\node at (1.2,-0.3) {\(a\)};
\node at (4.2,-0.3) {\(b\)};
\node at (3,-0.3) {\(c\)};
\filldraw (3,0.99) circle (1.5pt);
\end{tikzpicture}
\end{center}

\subsection{Extreme Value Theorem}

If \(f\) is continuous on the closed interval \([a, b] \), then \(f\) attains a maximum and a 
minimum value on \([a, b]\). That is, there exist \( c, d \in [a, b] \) such that:

\[
    f(c) \le f(x) \le f(d) \quad \text{for all } x \in [a, b]
\]

\textbf{Proof:}  

Since \(f\) is continuous on the compact interval \([a, b]\), the image \( f([a, b]) \subset \Reals \) 
is also compact. A compact subset of \( \Reals \) is closed and bounded, and hence 
attains its supremum and infimum.

Therefore, there exist \( c, d \in [a, b] \) such that:

\[
    f(c) = \min\{f(x): x \in [a, b]\}, \quad
    f(d) = \max\{f(x): x \in [a, b]\}
\]

\QED

\subsection{Rolle’s Theorem}
Let \(f\) be continuous on \([a, b] \), differentiable on \((a, b) \), and suppose \( f(a) = f(b) \).  
Then, there exists \( c \in (a, b) \) such that \( f'(c) = 0 \).
\vspace{\baselineskip}

\textbf{Proof:}  

By the Extreme Value Theorem, \(f\) attains a maximum or minimum at some point \( c \in [a, b] \).

If \( c \in (a, b) \), then by Fermat’s theorem, \( f'(c) = 0 \).  
If the maximum or minimum occurs at \(a\) or \(b\), since \( f(a) = f(b) \), \(f\) must be 
constant, so \( f'(x) = 0 \) everywhere on \((a, b)\), and in particular for some \( c \in (a, b) \).

\QED

\begin{center}
\begin{tikzpicture}[scale=1]
\draw[->] (-0.5,0) -- (5.5,0) node[right] {\(x\)};
\draw[->] (0,-1) -- (0,4) node[above] {\(f(x)\)};
\draw[scale=1,domain=0.5:5,smooth,variable=\x,blue,thick] plot ({\x},{-0.5*(\x-3)^2+3});
\node at (0.5,-0.3) {\(a\)};
\node at (5,-0.3) {\(b\)};
\node at (2.8,-0.3) {\(c\)};
\filldraw (2.8,3) circle (1.5pt);
\draw[dashed] (2.8,0) -- (2.8,3);
\end{tikzpicture}
\end{center}



\subsection{Mean Value Theorem}

Let \(f\) be continuous on \([a, b] \) and differentiable on \((a, b) \).  
Then there exists \( c \in (a, b) \) such that:

\[
    f'(c) = \frac{f(b) - f(a)}{b - a}
\]

\textbf{Proof:}  

Define the auxiliary function:

\[
    g(x) = f(x) - \left( \frac{f(b) - f(a)}{b - a} \right)(x - a)
\]

Then \( g(a) = f(a) \), \( g(b) = f(b) - m(b - a) = f(a) \Rightarrow g(a) = g(b) \)

Apply Rolle’s theorem to \( g(x) \):  
Since \( g(a) = g(b) \), and \( g \) is continuous and differentiable, there exists \( c \in (a, b) \) such that \( g'(c) = 0 \)

Then:

\[
    g'(x) = f'(x) - \left( \frac{f(b) - f(a)}{b - a} \right) \Rightarrow f'(c) = \frac{f(b) - f(a)}{b - a}
\]

\QED

\begin{tikzpicture}[thick]
    \path ( 1,4)        node[coordinate] (a1) {}
          (10,5)        node[coordinate] (b1) {}
          (a1) ++(0,-2) node[coordinate] (a2) {}
          (b1) ++(0,-2) node[coordinate] (b2) {};
  
    \path[draw,green] (a1) -- (b1);
  
    \path[draw,red] (a2) --
      node[coordinate,pos=0.05] (c1) {}
      node[coordinate,pos=0.2 ] (c2) {}
      node[coordinate,pos=0.4 ] (c3) {}
      (b2);
  
    \draw[densely dashed] (a1)
      .. controls +(0,0) and  (c1)   .. (c2)
      .. controls  (c3)  and +(-2,2) .. (b1);
  
    \foreach \point/\text in {a1/a , b1/b , c2/c}
      \draw[dotted]
          let \p1 = (\point)
        in
             (0  ,\y1) node[anchor=east ] {$f(\text)$}
          -- (\p1)
          -- (\x1,0  ) node[anchor=north] {$\text$};
  
    \draw[->] (-1.5, 0  ) -- (11,0  ) node[anchor=south east] {\textsf{x}};
    \draw[->] (   0,-1.5) -- ( 0,6.5) node[anchor=north west] {\textsf{y}};
  
  \end{tikzpicture}
