\section{Differentiation}

Differentiation is a core concept in calculus. It describes the rate at which a quantity changes and provides tools for analyzing and modeling change in real-world phenomena.

\subsection{Definition of the Derivative}

The derivative of a function \(f\) at a point \(x\) is defined as the limit:
\[
f'(x) = \lim_{h \to 0} \frac{f(x + h) - f(x)}{h}
\]

or

\[f'(x) = \lim_{h \to a}\frac{f(x) - f(a)}{x - a}\]
provided this limit exists.

Geometrically, \(f'(a)\) is the slope of the tangent line to the graph of \(f\) at \(x = a\).

\subsection{Derivative Rules}

\emph{Power Rule:}
\[
\frac{d}{dx} x^n = nx^{n - 1}, \quad \text{for } n \in \mathbb{R}
\]
\emph{Constant Rule:}
\[
\frac{d}{dx} c = 0, \quad \text{for constant } c
\]
\emph{Constant Multiple Rule:}
\[
\frac{d}{dx} [c \cdot f(x)] = c \cdot f'(x)
\]
\emph{Sum and Difference Rule:}
\[
\frac{d}{dx} [f(x) \pm g(x)] = f'(x) \pm g'(x)
\]
\emph{Product Rule:}
\[
\frac{d}{dx} [f(x)g(x)] = f'(x)g(x) + f(x)g'(x)
\]
\emph{Quotient Rule:}
\[
\frac{d}{dx} \left( \frac{f(x)}{g(x)} \right) = \frac{f'(x)g(x) - f(x)g'(x)}{[g(x)]^2}
\]
\emph{Chain Rule:}
\[
\frac{d}{dx} f(g(x)) = f'(g(x)) \cdot g'(x)
\]

\subsection{Extrema and Critical Points Test}

A function \(f\) has a critical point at \(x = c\) if:
\[
f'(c) = 0 \quad \text{or} \quad f'(c) \text{ does not exist}
\]

To determine the nature of the critical point:
\begin{itemize}[label=\(-\)]
\item \emph{First Derivative Test:} Check sign changes of \(f'\) around \(c\)
\item \emph{Second Derivative Test:}
\[
f''(c) > 0 \Rightarrow \text{local minimum} \\
f''(c) < 0 \Rightarrow \text{local maximum}
\]
\end{itemize}

\subsection{Trigonometric Derivatives}

\begin{align*}
\frac{d}{dx} \sin x &= \cos x \\
\frac{d}{dx} \cos x &= -\sin x \\
\frac{d}{dx} \tan x &= \sec^2 x \\
\frac{d}{dx} \cot x &= -\csc^2 x \\
\frac{d}{dx} \sec x &= \sec x \tan x \\
\frac{d}{dx} \csc x &= -\csc x \cot x
\end{align*}

\subsection{Derivatives of Inverse Trigonometric Functions}

\begin{align*}
\frac{d}{dx} \arcsin x &= \frac{1}{\sqrt{1 - x^2}} \\
\frac{d}{dx} \arccos x &= -\frac{1}{\sqrt{1 - x^2}} \\
\frac{d}{dx} \arctan x &= \frac{1}{1 + x^2} \\
\frac{d}{dx} arcot x &= -\frac{1}{1 + x^2} \\
\frac{d}{dx} arcsec x &= \frac{1}{|x|\sqrt{x^2 - 1}} \\
\frac{d}{dx} arccsc x &= -\frac{1}{|x|\sqrt{x^2 - 1}}
\end{align*}

\subsection{The Differential}

The differential \(dy\) of a function \(y = f(x)\) is defined as:
\[
dy = f'(x) \, dx
\]

This linear approximation estimates the change in \(y\) for a small change in \(x\).

\subsection{The Secant Equation and Graphical Meaning}

The \emph{secant line} through points \((x_0, f(x_0))\) and \((x_0 + h, f(x_0 + h))\) has slope:
\[
\frac{f(x_0 + h) - f(x_0)}{h}
\]

As \(h \to 0\), the secant line approaches the tangent line. Graphically, this means:
\[
\lim_{h \to 0} \text{slope of secant} = \text{slope of tangent} = f'(x) = \Delta_h
\]

The secant line equation is given by:

\[
L(x) = f(x) + \Delta_h f(x_0)(x - x_0)
\]

\subsection{Optimization via the Derivative}

To find local or global extrema:
\begin{enumerate}
    \item Compute \(f'(x)\)
    \item Solve \(f'(x) = 0\) to find critical points
    \item Use the first or second derivative test to classify
    \item Evaluate endpoints if optimizing over a closed interval
\end{enumerate}

\textbf{Example:} Maximize area \(A = x(10 - 2x)\)

\[
A = 10x - 2x^2, \quad A' = 10 - 4x, \quad A' = 0 \Rightarrow x = 2.5
\]

\subsection{Power Rule Derivation}

We derive the power rule:
\[
\frac{d}{dx} x^n = nx^{n - 1}
\]
for \(n \in \mathbb{N}\) using the limit definition:
\[
\frac{d}{dx} x^n = \lim_{h \to 0} \frac{(x + h)^n - x^n}{h}
\]
Use the Binomial Theorem:
\[
(x + h)^n = x^n + nx^{n-1}h + \cdots + h^n
\Rightarrow \text{Only the linear term survives after dividing by } h
\]

\subsection{Implicit Differentiation}

Given:
\[
9x^2 + 4y^2 = 25
\]
Differentiate both sides implicitly:
\[
18x + 8y \frac{dy}{dx} = 0
\Rightarrow \frac{dy}{dx} = -\frac{18x}{8y} = -\frac{9x}{4y}
\]

This method is used when \(y\) is not isolated and is defined implicitly.

\newpage