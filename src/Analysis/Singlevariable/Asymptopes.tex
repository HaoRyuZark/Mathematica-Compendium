\section{Asymptotes}

An \textbf{asymptote} of a curve is a line that the curve approaches but never touches as the input or output grows infinitely large in magnitude. There are three main types of asymptotes: vertical, horizontal, and oblique (slant). Each one describes a different long-term behavior of a function.

\subsection{Vertical Asymptotes}

\subsubsection*{Definition}

A \textbf{vertical asymptote} is a vertical line \(x = a\) where the function grows without bound as \(x\) approaches \(a\) from the left or right.

\[
\lim_{x \to a^-} f(x) = \pm \infty \quad \text{or} \quad \lim_{x \to a^+} f(x) = \pm \infty
\]

\subsubsection*{How to Find Vertical Asymptotes}

\begin{enumerate}
    \item Find values of \(x\) that make the denominator of a rational function zero.
    \item Eliminate removable discontinuities (common factors).
    \item For each remaining zero in the denominator, check if the limit tends to infinity.
\end{enumerate}

\textbf{Example:}
\[
f(x) = \frac{1}{x - 2} \quad \Rightarrow \quad x = 2 \text{ is a vertical asymptote}
\]

\subsection{Horizontal Asymptotes}

\subsubsection*{Definition}

A \textbf{horizontal asymptote} is a horizontal line \(y = L\) that the function approaches as \(x\) tends to infinity or negative infinity:

\[
\lim_{x \to \pm\infty} f(x) = L
\]

\subsubsection*{How to Find Horizontal Asymptotes (Rational Functions)}

\begin{enumerate}
    \item Compare the degrees of the numerator (\(n\)) and denominator (\(m\)):
    \begin{itemize}
        \item If \(n < m\): Horizontal asymptote at \(y = 0\)
        \item If \(n = m\): Horizontal asymptote at \(y = \frac{\text{leading coefficient of numerator}}{\text{leading coefficient of denominator}}\)
        \item If \(n > m\): No horizontal asymptote (check for slant)
    \end{itemize}
\end{enumerate}

\textbf{Example:}
\[
f(x) = \frac{2x^2 + 3}{x^2 - 1} \quad \Rightarrow \quad y = \frac{2}{1} = 2 \text{ is the horizontal asymptote}
\]

\subsection{Oblique (Slant) Asymptotes}

\subsubsection*{Definition}

An \textbf{oblique or slant asymptote} occurs when the degree of the numerator is exactly one more than the degree of the denominator.

\[
\lim_{x \to \infty} [f(x) - (mx + b)] = 0
\]

\subsubsection*{How to Find Oblique Asymptotes}

\begin{enumerate}
    \item Perform polynomial long division (or synthetic division) on the rational function.
    \item The quotient (ignoring the remainder) is the equation of the slant asymptote.
\end{enumerate}

\textbf{Example:}
\[
f(x) = \frac{x^2 + 1}{x - 1}
\]
Long division gives:
\[
\frac{x^2 + 1}{x - 1} = x + 1 + \frac{2}{x - 1}
\Rightarrow \text{Slant asymptote: } y = x + 1
\]

\subsection{Summary Table}

\begin{center}
\begin{tabular}{|l|l|l|}
\hline
\textbf{Type} & \textbf{Equation} & \textbf{Occurs When} \\
\hline
Vertical      & \(x = a\)           & Denominator \(\to 0\) (non-removable) \\
Horizontal    & \(y = L\)           & \(\lim_{x \to \pm \infty} f(x) = L\) \\
Oblique       & \(y = mx + b\)      & Degree numerator = degree denominator + 1 \\
\hline
\end{tabular}
\end{center}

\newpage