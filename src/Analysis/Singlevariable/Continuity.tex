\newpage
\section{Continuity}

Continuity describes functions whose values change smoothly, without abrupt jumps or breaks. 
This concept is fundamental in calculus, analysis, and fixed point theory.

\subsection{Definition of Continuity at a Point}

A function \(f\) is \emph{continuous} at a point \(x_0\) if:

\[
    \forall \varepsilon > 0, \exists \delta > 0 \text{ such that } |x - x_0| < \delta \Rightarrow |f(x) - f(x_0)| < \varepsilon
\]

Equivalently,

\[
    \lim_{x \to x_0} f(x) = f(x_0)
\]

\textbf{Example:} 
\vspace{\baselineskip}

\(\varepsilon \delta\) Exercise for \(f(x) = \frac{1}{x}\) at \(x_0 > 0\)
\vspace{\baselineskip}

We want to find \(\delta\) such that:

\[
    |x - x_0| < \delta \Rightarrow \left|\frac{1}{x} - \frac{1}{x_0}\right| < \varepsilon
\]

Start by rewriting:
\[
    \left|\frac{1}{x} - \frac{1}{x_0}\right| = \left|\frac{x_0 - x}{xx_0}\right| = \frac{|x - x_0|}{|x||x_0|}
\]

Assume:

\[
    x \in \left[\frac{x_0}{2}, \frac{3x_0}{2}\right] \Rightarrow |x| \ge \frac{x_0}{2}
\]

Then:

\[
    \left|\frac{1}{x} - \frac{1}{x_0}\right| \le \frac{|x - x_0|}{\frac{x_0}{2} \cdot x_0} = \frac{2}{x_0^2} |x - x_0|
\]

Choose:

\[
    \delta = \min\left\{\frac{x_0}{2}, \frac{x_0^2 \varepsilon}{2} \right\}
\]

\subsection{Rules of Continuity}

Let \(f\) and \(g\) be continuous at \(x_0\):

\begin{itemize}
    
    \item \(f + g\), \(f - g\), \(f \cdot g\) are continuous at \(x_0\)
    
    \item \(\frac{f}{g}\) is continuous at \(x_0\) if \(g(x_0) \ne 0\)
    
    \item Compositions: if \(g\) is continuous at \(x_0\), and \(f\) is continuous at \(g(x_0)\), 
    then \(f \circ g\) is continuous at \(x_0\)

\end{itemize}

\subsection{Lipschitz Continuity}

A function \(f: D \subset \Reals \to \Reals\) is \emph{Lipschitz continuous} if:

\[
    \exists L > 0 \text{ such that } |f(x) - f(y)| \le L |x - y| \quad \forall x, y \in D
\]

\begin{itemize}

    \item Every Lipschitz continuous function is uniformly continuous.

    \item The smallest such \(L\) is called the \emph{Lipschitz constant}.

\end{itemize}

\subsection{Banach Fixed Point Theorem}

Let \((X, d)\) be a complete metric space, and \(f: X \to X\) a \emph{contraction mapping}, i.e.,

\[
    \exists L < 1 \text{ such that } d(f(x), f(y)) \le L \cdot d(x, y)
\]

Then:

\begin{itemize}

    \item \(f\) has a unique fixed point \(x^* \in X\), such that \(f(x^*) = x^*\)

    \item Iterating \(x_{n+1} = f(x_n)\) converges to \(x^*\)

\end{itemize}

\emph{Interpretation and Meaning}

The Banach Fixed Point Theorem ensures:

\begin{itemize}

    \item Existence and uniqueness of solutions (fixed points)

    \item Convergence of approximation by iteration

    \item Powerful tool in numerical methods and differential equations

\end{itemize}

\subsubsection{A Priori and A Posteriori Approximations}

\emph{A priori estimate:}

\[
    |x_n - x^*| \le \frac{L^n}{1 - L} |x_1 - x_0|
\]

\emph{A posteriori estimate:}

\[
    |x_n - x^*| \le \frac{L}{1 - L} |x_n - x_{n-1}|
\]

\subsubsection{Steps for a Fixed Point Exercise}

Given \(f: [a, b] \to [a, b]\), to prove existence and convergence:

\begin{enumerate}

    \item \textbf{Show monotonicity:} \(f\) is increasing or decreasing.

    \item \textbf{Check interval preservation:} \(f([a, b]) \subseteq [a, b]\)

    \item \textbf{Check Lipschitz continuity:} Find \(L < 1\)

    \item \textbf{Fixed point iteration:} Choose \(x_0\), compute \(x_{n+1} = f(x_n)\)

    \item \textbf{Apply a priori or a posteriori bound}

\end{enumerate}

\textbf{Example: \(f(x) = \frac{1}{x} + 3\) on \([2, 5]\)}

\begin{itemize}

    \item \emph{Domain check:} \(x \in [2, 5] \Rightarrow \frac{1}{x} \in [0.2, 0.5] \Rightarrow f(x) \in [3.2, 3.5] \subset [2, 5]\)

    \item \emph{Lipschitz constant:}

        \[
            f'(x) = -\frac{1}{x^2} \Rightarrow |f'(x)| \le \frac{1}{2^2} = \frac{1}{4} < 1
            \Rightarrow \text{Lipschitz with } L = \frac{1}{4}
        \]

    \item \emph{Contraction verified:} By derivative bound

    \item \emph{Fixed point iteration:}

        \[
            x_0 = 3, \quad x_1 = f(3) = \frac{1}{3} + 3 = \frac{10}{3}, \quad x_2 = f(x_1), \dots
        \]

    \item \emph{Use approximation bounds:}

        \[
            |x_n - x^*| \le \frac{L^n}{1 - L} |x_1 - x_0|
        \]
\end{itemize}