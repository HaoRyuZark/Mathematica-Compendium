\section{Integration}

Integration is the reverse process of differentiation and is used to calculate areas, volumes, accumulated change, and more.

\subsection{Definition (Riemann Integral)}

Let \(f: [a, b] \to \mathbb{R}\) be a bounded function. We define the \emph{definite integral} as:
\[
\int_a^b f(x)\,dx = \lim_{\|P\| \to 0} \sum_{i=1}^n f(\xi_i)\Delta x_i
\]
where \(P = \{x_0, \dots, x_n\}\) is a partition, \(\xi_i \in [x_{i-1}, x_i]\), and \(\Delta x_i = x_i - x_{i-1}\).

\subsection{Upper and Lower Sums}

Let \(f\) be bounded on \([a, b]\) and \(P\) a partition:
\[
\underline{S}(f, P) = \sum_{i=1}^n m_i \Delta x_i, \quad
\overline{S}(f, P) = \sum_{i=1}^n M_i \Delta x_i
\]
with:
\[
m_i = \inf_{x \in [x_{i-1}, x_i]} f(x), \quad
M_i = \sup_{x \in [x_{i-1}, x_i]} f(x)
\]
If \(\sup \underline{S} = \inf \overline{S}\), then \(f\) is Riemann integrable.

\subsection{Concrete and Non-Concrete Integrals}

\begin{itemize}[label=\(-\)]
\item \emph{Concrete:} Can be explicitly evaluated with anti-derivatives.
\item \emph{Non-concrete:} Cannot be integrated in elementary terms (e.g., \(\int e^{-x^2} dx\)).
\end{itemize}

\subsection{Rules of Integration}

\begin{itemize}[label=\(-\)]
\item \(\int 0\,dx = C\)
\item \(\int c\,dx = cx\)
\item \(\int x^n\,dx = \frac{x^{n+1}}{n+1} + C \quad (n \ne -1)\)
\item \(\int \frac{1}{x}\,dx = \ln|x| + C\)
\item \(\int e^x\,dx = e^x + C\)
\item \(\int \sin x\,dx = -\cos x + C\)
\item \(\int \cos x\,dx = \sin x + C\)
\item Linearity: \(\int (af + bg)\,dx = a\int f\,dx + b\int g\,dx\)
\end{itemize}

\subsection{Substitution Rule (U-Substitution)}

If \(u = g(x)\) and \(f = F'\), then:
\[
\int f(g(x))g'(x)\,dx = \int f(u)\,du
\]

\textbf{Example:}
\[
\int \frac{1}{(x - 1)^4}\,dx, \quad u = x - 1 \Rightarrow \int \frac{1}{u^4} du = -\frac{1}{3u^3} + C = -\frac{1}{3(x - 1)^3} + C
\]

\subsection{Integration by Parts}

\[
\int u\,dv = uv - \int v\,du
\]

\emph{ILATE:} Use this order to choose \(u\):
\begin{enumerate}
    \item Inverse Trig
    \item Logarithmic
    \item Algebraic
    \item Trigonometric
    \item Exponential
\end{enumerate}

\emph{Cyclic Integrals:} Some integrals cycle back to the original (e.g., \(e^x \sin x\)).

\textbf{Example:} \(\int e^x \sin x\,dx\)

Let:
\[
u = \sin x,\ dv = e^x dx \Rightarrow du = \cos x\,dx,\ v = e^x
\Rightarrow \int e^x \sin x\,dx = e^x \sin x - \int e^x \cos x\,dx
\]
Repeat integration by parts to complete.

\subsection{Integration by Partial Fractions}

Applies to rational functions. Four cases:

\begin{itemize}[label=\(-\)]
\item Linear factors
\item Repeated linear factors
\item Irreducible quadratic factors
\item Repeated irreducible quadratic factors
\end{itemize}

\textbf{Example:}
\[
\int \frac{1}{x^2 - 1} dx = \int \left( \frac{1}{2(x - 1)} - \frac{1}{2(x + 1)} \right) dx = \frac{1}{2} \ln\left|\frac{x - 1}{x + 1}\right| + C
\]

\subsection{Trigonometric Substitution}

Trigonometric substitution is useful for evaluating integrals involving square roots of quadratic expressions.

\begin{itemize}[label=\(-\)]
    \item \(\sqrt{a^2 - x^2}\): use \(x = a \sin \theta\)
    \item \(\sqrt{a^2 + x^2}\): use \(x = a \tan \theta\)
    \item \(\sqrt{x^2 - a^2}\): use \(x = a \sec \theta\)
\end{itemize}

\textbf{Example 1: \( \int \frac{1}{\sqrt{4 - x^2}} \,dx \)}

We use: \( x = 2\sin \theta \Rightarrow dx = 2\cos \theta\, d\theta \)

\[
\sqrt{4 - x^2} = \sqrt{4 - 4\sin^2 \theta} = \sqrt{4\cos^2 \theta} = 2\cos \theta
\]

Substitute:
\[
\int \frac{1}{\sqrt{4 - x^2}} \,dx = \int \frac{1}{2\cos \theta} \cdot 2\cos \theta \,d\theta = \int d\theta = \theta + C
\]

Return to \( x \) using:
\[
x = 2\sin \theta \Rightarrow \theta = \arcsin\left(\frac{x}{2}\right)
\Rightarrow \boxed{\int \frac{1}{\sqrt{4 - x^2}} \,dx = \arcsin\left(\frac{x}{2}\right) + C}
\]


\textbf{Example 2: \( \int \frac{x^3}{\sqrt{x^2 + 4}} \,dx \)}

Use: \( x = 2\tan \theta \Rightarrow dx = 2\sec^2 \theta\, d\theta \)

\[
\sqrt{x^2 + 4} = \sqrt{4\tan^2 \theta + 4} = \sqrt{4\sec^2 \theta} = 2\sec \theta
\]

Substitute:
\[
\int \frac{x^3}{\sqrt{x^2 + 4}}\,dx = \int \frac{(2\tan \theta)^3}{2\sec \theta} \cdot 2\sec^2 \theta\, d\theta
= \int \frac{8\tan^3 \theta}{\sec \theta} \cdot 2\sec^2 \theta\, d\theta
\]

Simplify:
\[
8\int \tan^{2}\theta\sec\theta\tan\theta \,dx = 8\int (\sec^2\theta - 1)\sec\theta\tan\theta \,dx 
\]
\[
8\int (\sec^2\theta\sec\theta - \sec\theta)\tan\theta \,dx 
\]
\[
8\left(\int \sec^2\theta\sec\theta\tan\theta \,dx - \int\sec\theta\tan\theta \,dx \right) 
\]
\[
8\left(\int \sec^2\theta\sec\theta \,dx - \int\sec\theta\tan\theta \,dx \right) 
\]

For the first one let \(u = \sec^3\theta\) and \(du = 3\sec^2\theta\,d\theta\) 

\[
\int udu = \frac{\sec^3\theta}{3} + c
\]

For the second one

\[
\int \sec\theta\tan\theta = \sec\theta + c
\]

Together

\[8\left(  \frac{\sec^3\theta}{3} - \sec\theta + c\right)\]

Then solve the integral using trigonometric identities and reduction formulas. Return to \( x \) using:

\[
\tan\theta = \frac{x}{2}\ \sec\theta = \frac{\sqrt{x^2 + 4}}{2}
\]

This leaves us with

\[
    \frac{8}{3} \left(\frac{\sqrt{x^2 + 4}}{2}\right)^3 - 8 \left( \frac{\sqrt{x^2 + 4}}{2}\right) + c
\]

\textbf{Example 3: \( \int \frac{x^3}{\sqrt{x^4 - 4}} \,dx \)}

Use: \(x = 2\sec \theta, \ dx = 2\sec \theta \tan \theta \,d\theta \)

Then
\[
\int \frac{(2\sec\theta)^2 2\sec\theta \tan\theta \,d\theta}{2\tan\theta} = \int 8\sec^4 \theta\,d\theta
\]

\[
8\int(\tan^2\theta + 1)\sec^2\theta = 8\left(\int\tan^2\theta\sec^2\theta + \int \sec^2\theta \right)
\]

Now let \(u = \tan\theta\) and \(du = \sec^2\theta \,d\theta\)

\[
8\left( \int udu  + \int \sec^2\theta\,d\theta\right) = 
\frac{8}{3}\tan^3\theta + 8\tan\theta + c
\]

Now we return to x via our triangle:
\[\tan\theta = \frac{\sqrt{x^2 + 4}}{2}\]

\[
\frac{8}{3}\left(\frac{\sqrt{x^2 + 4}}{2}\right)^3 + 8 \frac{\sqrt{x^2 + 4}}{2} + c
\]
\subsection{Average Value of a Function}

\[
\bar{f} = \frac{1}{b - a} \int_a^b f(x)\,dx
\]

\subsection{Mean Value Theorem for Integrals}

If \(f\) is continuous on \([a, b]\), then:
\[
\exists c \in [a, b] \text{ such that } \int_a^b f(x)\,dx = f(c)(b - a)
\]

\subsection{Length, Area, Volume Formulas}

\emph{Arc length:}
\[
L = \int_a^b \sqrt{1 + [f'(x)]^2}\,dx
\]
\emph{Area between curves:}
\[
A = \int_a^b |f(x) - g(x)|\,dx
\]
\emph{Volume (disk):}
\[
V = \pi \int_a^b f(x)^2\,dx
\]
\emph{Shell:}
\[
V = 2\pi \int_a^b x f(x) \sqrt{1 + f'(x)^2}\,dx
\]

\subsection{Differentiation of Integrals with Variable Bounds}

If:
\[
F(x) = \int_{a(x)}^{b(x)} f(t)\,dt
\]
Then:
\[
F'(x) = f(b(x)) \cdot b'(x) - f(a(x)) \cdot a'(x)
\]

\subsection{Parameter-Dependent Integrals}

Let \(F(p) = \int_a^b f(x, p)\,dx\). If \(f\) is continuous and differentiable in \(p\), then:
\[
\frac{d}{dp} \int_a^b f(x, p)\,dx = \int_a^b \frac{\partial f}{\partial p}(x, p)\,dx
\]

\subsection{Leibniz Rule for Differentiating Under the Integral Sign}

\[
\frac{d}{dp} \int_{a(p)}^{b(p)} f(x, p)\,dx = f(b(p), p) \cdot b'(p) - f(a(p), p) \cdot a'(p) + \int_{a(p)}^{b(p)} \frac{\partial f}{\partial p}(x, p)\,dx
\]

\subsection{Improper Integrals (Infinite Bounds)}

\[
\int_a^\infty f(x)\,dx = \lim_{R \to \infty} \int_a^R f(x)\,dx
\]
\[
\int_{-\infty}^\infty f(x)\,dx = \int_{-\infty}^0 f(x)\,dx + \int_0^\infty f(x)\,dx
\]

\subsection{Improper Integrals (Unbounded Functions)}

\[
\int_a^b f(x)\,dx \text{ improper if } f \text{ has an infinite discontinuity on } [a, b]
\]

\textbf{Example:}
\[
\int_0^1 \frac{1}{\sqrt{x}}\,dx = \lim_{\varepsilon \to 0^+} \int_\varepsilon^1 \frac{1}{\sqrt{x}}\,dx = 2
\]

\subsection{Convergence Criteria for Improper Integrals}

\begin{itemize}[label=\(-\)]
\item \emph{Majorant:} If \(0 \le f(x) \le g(x)\) and \(\int g(x)\,dx\) converges, then so does \(\int f(x)\,dx\).
\item \emph{Minorant:} If \(f(x) \ge g(x) \ge 0\) and \(\int g(x)\,dx\) diverges, then \(\int f(x)\,dx\) diverges.
\end{itemize}

\subsection{Integral Test for Series}

Let \(f(n) = a_n\) where \(f\) is continuous, decreasing, and positive for \(n \ge N\):
\[
\sum_{n=N}^\infty a_n \text{ converges } \iff \int_N^\infty f(x)\,dx \text{ converges}
\]

\textbf{Example:} \(\sum \frac{1}{n^p}\) converges \(\iff p > 1\)

\newpage                                                