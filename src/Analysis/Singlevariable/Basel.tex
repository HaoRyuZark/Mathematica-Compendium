\newpage
\section{The Basel Problem}

The Basel Problem seeks to determine the exact value of the infinite series:

\[
    \sum_{n=1}^{\infty} \frac{1}{n^2}
\]

Leonhard Euler famously solved this in 1735, showing that:

\[
    \sum_{n=1}^{\infty} \frac{1}{n^2} = \frac{\pi^2}{6}
\]

\subsection{Euler's Proof via the Function \( \frac{\sin(x)}{x} \)}

Euler's approach involves expressing \( \frac{\sin(x)}{x} \) both as a power series and as an 
infinite product, then equating the coefficients of like powers of \emph{x}.
\vspace{\baselineskip}

\textbf{Step 1: Power Series Expansion}

The Taylor series expansion of \( \frac{\sin(x)}{x} \) around \( x = 0 \) is:

\[
    \frac{\sin(x)}{x} = 1 - \frac{x^2}{3!} + \frac{x^4}{5!} - \frac{x^6}{7!} + \cdots
\]

\textbf{Step 2: Infinite Product Representation}

The sine function has zeros at \( x = n\pi \) for all non-zero integers \emph{n}. Euler utilized this to 
express \( \frac{\sin(x)}{x} \) as an infinite product over its zeros:

\[
    \frac{\sin(x)}{x} = \prod_{n=1}^{\infty} \left(1 - \frac{x^2}{n^2\pi^2}\right)
\]

\textbf{Step 3: Equating Coefficients}

Expanding the infinite product, the coefficient of \( x^2 \) is:

\[
    -\sum_{n=1}^{\infty} \frac{1}{n^2\pi^2}
\]

From the power series, the coefficient of \( x^2 \) is \( -\frac{1}{6} \). Equating these coefficients:

\[
    -\sum_{n=1}^{\infty} \frac{1}{n^2\pi^2} = -\frac{1}{6}
\]

Multiplying both sides by \( -\pi^2 \) yields:

\[
    \sum_{n=1}^{\infty} \frac{1}{n^2} = \frac{\pi^2}{6}
\]

\subsection{Key Observations}

\begin{itemize}

    \item Euler's method ingeniously connects the zeros of \( \sin(x) \) with the sum of reciprocal squares.

    \item This approach laid foundational work for the development of complex analysis and the theory 
          of entire functions.

    \item The infinite product representation of sine was later rigorously justified by Weierstrass's 
          factorization theorem.

\end{itemize}
