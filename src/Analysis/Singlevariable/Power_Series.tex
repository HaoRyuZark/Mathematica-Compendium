\newpage
\section{Transforming Functions into Power Series}

Many functions can be represented as power series, which is useful for various applications in calculus and analysis. This section discusses some common techniques for transforming functions into power series.

\subsection{Common Cases}
Here are some common cases where functions can be easily transformed into power series:

\(\frac{1}{1 - x}\)

The function \(\frac{1}{1 - x}\) has the well-known geometric series representation:
\[
\frac{1}{1 - x} = \sum_{n=0}^{\infty} x^n, \quad |x| < 1.
\]

\(\frac{1}{1 - {(x - x_0)}^n}\)

A slight generalization is:
\[
\frac{1}{1 - {(x - x_0)}^n} = \sum_{k=0}^{\infty} {(x - x_0)}^{nk}, \quad |x - x_0| < 1.
\]

\(\frac{1}{1 - d{(x - x_0)}^n}\)
For a constant \(d\):
\[
\frac{1}{1 - d{(x - x_0)}^n} = \sum_{k=0}^{\infty} d^k {(x - x_0)}^{nk}, \quad |d{(x - x_0)}^n| < 1.
\]

\textbf{Example: \(\frac{x + 1}{3 - x}\)}

Let's find the power series representation of the function \(f(x) = \frac{x + 1}{3 - x}\).

\subsubsection{Partial Fractions}

First, we perform partial fraction decomposition. We can rewrite the function as:
\[
\frac{x + 1}{3 - x} = \frac{-(3 - x) + 4}{3 - x} = -1 + \frac{4}{3 - x}.
\]

\subsubsection{Manipulating the Expression}
Now, we manipulate the fraction to fit the form \(\frac{1}{1 - u}\):
\[
\frac{4}{3 - x} = \frac{4}{3(1 - \frac{x}{3})} = \frac{4}{3} \cdot \frac{1}{1 - \frac{x}{3}}.
\]

\subsubsection{Power Series Representation}
Using the geometric series formula, we have
\[
\frac{1}{1 - \frac{x}{3}} = \sum_{n=0}^{\infty} {\left( \frac{x}{3} \right)}^n, \quad \left| \frac{x}{3} \right| < 1.
\]
Thus,
\begin{align*}
\frac{x + 1}{3 - x} &= -1 + \frac{4}{3} \sum_{n=0}^{\infty} {\left( \frac{x}{3} \right)}^n \\
&= -1 + \frac{4}{3} \sum_{n=0}^{\infty} \frac{x^n}{3^n} \\
&= -1 + \sum_{n=0}^{\infty} \frac{4}{3^{n+1}} x^n, \quad |x| < 3.
\end{align*}
Therefore, the power series representation of the function is
\[
\frac{x + 1}{3 - x} = -1 + \sum_{n=0}^{\infty} \frac{4}{3^{n+1}} x^n, \quad |x| < 3.
\]
