\newpage
\section{The Gamma Function}

The Gamma function \( \Gamma(x) \) is defined for all \( x > 0 \) by the improper integral:
\[
\Gamma(x) = \int_0^\infty t^{x-1} e^{-t} \,dt
\]

It converges for all \( x > 0 \) and can be analytically continued to most of the complex plane (except non-positive integers, where it has poles).

\textbf{Special Case: Factorials}

For any natural number \( n \in \mathbb{N} \), we have:
\[
\Gamma(n) = (n - 1)!
\]

\subsection{Recurrence Relation (Proof)}

We show that:
\[
\Gamma(x + 1) = x \cdot \Gamma(x)
\]

\textbf{Proof:}

\[
\Gamma(x + 1) = \int_0^\infty t^x e^{-t} \,dt
\]

Integrate by parts:
\[
u = t^x, \quad dv = e^{-t} dt \Rightarrow du = x t^{x-1} dt, \quad v = -e^{-t}
\]

\[
\Gamma(x + 1) = \left[ -t^x e^{-t} \right]_0^\infty + \int_0^\infty x t^{x-1} e^{-t} dt = x \cdot \Gamma(x)
\]

The boundary term vanishes because:
\[
\lim_{t \to \infty} t^x e^{-t} = 0, \quad \lim_{t \to 0} t^x e^{-t} = 0 \quad (\text{for } x > 0)
\]

\subsection{Use Cases of the Gamma Function}

\begin{itemize}[label=\(-\)]
    \item \textbf{Generalized factorials:} For non-integer values, \( \Gamma(x) \) interpolates the factorial function.
    \item \textbf{Probability and statistics:} Appears in distributions such as the Gamma, Beta, and Chi-squared.
    \item \textbf{Complex analysis:} Integral representations, analytic continuation, and meromorphic structure.
    \item \textbf{Fourier and Laplace transforms:} Arises in solutions of integrals and transforms involving exponential decay.
    \item \textbf{Physics and engineering:} Occurs in problems with power-law and exponential decay relationships.
\end{itemize}

\textbf{Example: Gamma Distribution}

The Gamma distribution with shape \( \alpha > 0 \), rate \( \beta > 0 \) has the density:
\[
f(x) = \frac{\beta^\alpha x^{\alpha - 1} e^{-\beta x}}{\Gamma(\alpha)} \quad \text{for } x > 0
\]
