\newpage
\section{Probability}

\subsection{Basics}

\subsubsection{Probability}

\(E\) is the event we are studying and \(\Omega\) the total. Example \(\frac{1}{2}\)

\[
    P(E) = \frac{E}{\Omega}
\]

\subsubsection{Expected Result}

Here \(p\) represents the probability of something and \(x\) the expected reward

\[
    E(x) = p_1 x_1 + \cdots + p_n x_n
\]

\subsection{Standard Deviation}

\[
    \sigma = \sqrt{ \frac{ {(\overline{x} - x_1)}^2 + \cdots + {(\overline{x} - x_n)}^2 }{ n } }
\]


\subsection{Binomial Distribution}

\textbf{Formulas:}

\begin{itemize}

    \item \emph{Probability: } \(P(X = k) = \binom{n}{k} p^k {(1 - p)}^{n - k}\)

    \item \emph{Expected Result: } \(E(X) = n * p\)

    \item \emph{Standard Deviation: } \(\sqrt{E(X)(1-p)}\)

    \item  \emph{Variance: } \(E(x)(1-p)\)

\end{itemize}

\subsubsection{Continuous Probability}

\[
    \sum_{i = P(X=k)}^{P(X=n) (P(X=i))}
\]

\textbf{Formulas: }

\begin{itemize}

    \item \(P(X = a) = P(X = a)\)

    \item \(P(X \le a) = P(X \le a)\)

    \item \(P(X < a) = P(X \le a - 1)\)

    \item \(P(X > a) = 1 - P(X \le a)\)

    \item \(P(X \ge a) = 1 - P(X \le - 1)\)

    \item \(P(a \le X \le b) = P(X \le b) - P(X \le a)\)

\end{itemize}

\subsubsection{Sigma Rules}

\begin{itemize}[label = \(-\)]

    \item \(P(\mu - \sigma \le x \mu + \sigma) \approx 68,3\%\)

    \item \(P(\mu - 2\sigma \le x \mu + 2\sigma) \approx 95,4\%\)

    \item \(P(\mu - 3\sigma \le x \mu + 3\sigma) \approx 99,7\%\)

\end{itemize}

\subsection{Normal Distribution}

\begin{itemize}
    
    \item \emph{Probability: } \(\frac{1}{\sqrt{2\pi \sigma^2} e^{\frac{1}{2} 
          {\left(\frac{x - u}{\sigma}\right)}^2}}\)
    
    \item \emph{Expected Result: } \(E(x) = np = \mu\)
    
    \item \emph{Variance: } \(Var(x) = E(X) (1-p)\)
    
    \item \emph{Standard Deviation: } \(\sqrt{Var(x)}\)

\end{itemize}

\subsection{Conditional Probability}

Probability of \(a\) under the condition \(b\).

\[
    P_b (a) = \frac{P(b \cap a)}{P(b)}
\]

\subsubsection{Formula for the Total Probability}

\[
    P(a) = P_b (a) P(b) + P_{\neg b}(a) P(\neg b)
\]

\subsection{Bayes Theorem}

\[
    P(a | b) = \frac{P(b | a) P(a)}{P(b)}
\]

\subsection{Hyper-geometric Distribution}

\[
    P(X = k) = \frac{\binom{M}{K} \binom{N - M}{n - K}}{\binom{N}{n}}
\]

\textbf{Nomenclature:}

\begin{itemize}

    \item \(N\): Total number of elements

    \item \emph{M}: Elements with the trait \(A\)

    \item \(N - M\): Elements without the trait \(A\)

    \item \(n = k\): Number to elements to take

\end{itemize}

\textbf{Formulas:}

\begin{itemize}

    \item \emph{Expected Results: } \(E(x) = n \frac{M}{N}\)

    \item \emph{Variance: } \(Var(X) = E(x)\left(1 - \frac{M}{N}\right) \left(\frac{N - n}{N - 1}\right)\)

\end{itemize}

\subsection{The Birthday Paradox}

This is a small question that says: What is the probability of at least two people having 
their birthday on the same day in a group of \(x\) persons.

\[
    P(x) = 1 - \prod_{n = 0}^{x} \frac{365 - n}{365}
\]

This formula gives us the probability of total minus all persons having it on different days. Which
what we are looking for.