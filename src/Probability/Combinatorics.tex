\section{Combinatorics}

\subsection{Permutation}

Number of ways of ordering \(n\) distinct elements.

\[n!\]

\subsection{Permutation with repetition}

Number of ways of ordering \(n\) non distinct elements. Here \(m_1, \dots, m_n\)
is the number a specific item is repeaed in the original set.

\[\frac{n!}{m_1! \cdots m_n!}\]

\subsection{Variation}

Ways of put \(n\) objects in \(k\) slots with repetition.

\[n^k\]

\subsection{Variation without repetition}

Ways of put \(n\) objects in \(k\) slots without repetition.

\[\frac{n!}{(n -k)!}\]

\subsection{Combination I}

Ways of choose \(k\) objects of \(n\) elements.

\[\binom{n}{k} = \frac{n!}{k!(n-k)!}\]

Here \(n!\) is the number of permutations of the original set.
\((n-k)!\) has the function of eliminating the permutations of elements we are not interested in, and
\(k!\) is to eliminate the duplicates because for a combination the order does not matter, contrary to
the permutations.

\subsection{Combination II}

This focuses more not on the 'slots' but in the 'separations' between the slots.
More specific the number of ways to distribute \(k\) identical objects into \(n\) identical boxes. 

\[\binom{n + k - 1}{k} = \frac{(n + k - 1)!}{k!(n - 1)!}\]

\subsection{Disarray}

Number of permutation in which no object ends in the same initial spot. Also know as the \textbf{subfactorial}

\[!n = n! \sum_{k = 0}^{n} \left(\frac{(-1)^k}{k!}\right)\]

\subsection{Bell Numbers}

Number of ways of grouping \(n\) objects in an arbitrary number of slots.

\[B(N) = \sum_{k = 0}^{N-1}\binom{N}{K}B(K)\]

with \(B(1) = 1\) and \(B(2) = 2\)

\subsection{Ramanujan Numbers}

The same as the Bell Numbers but, all objects are equal. As an example the number of ways
to decompose a number.

\[R(N) \approx \frac{1}{4N\sqrt{3}} e^{2\pi \sqrt{\frac{N}{6}}}\]

\subsection{Stirling Numbers I}

Ways of permuting \(N\) items with \(K\) exchanges.

\[ \left[ N + 1 \atop K\right] = N\left[N \atop K\right] + \left[N \atop K - 1\right]\] 

\subsection{Stirling Numbers II}

Ways of grouping \(n\) items in \(k\) slots.

\[ \left\{ \begin{matrix} N \atop K \end{matrix}\right\} = \frac{1}{K!} \sum_{j=0}^{K}(-1)^{k - j}\binom{K}{j}j^N \]

\subsection{Lah Numbers}

Ways of building \(K\) lists with a set \(N\) elements.

\[\left\lfloor \begin{matrix} N \atop K \end{matrix}\right\rfloor  = \binom{N - 1}{K - 1}\frac{N!}{K!} = \sum_{j=0}^{K}\left[N \atop K\right]\left\{N \atop K\right\}\]

\subsection{Euler's Numbers}

Number of permutations of a set of \(N\) elements of different size in which
\(K\) elements are bigger than the their previous element.

\[\left\langle \begin{matrix} N \atop K \end{matrix}\right\rangle = \sum_{j = 0}^{k}(-1)^j \binom{N + 1}{j}(K +1 -j)^N\]

\subsection{Catalan Numbers}

This ones have different applications like the number of ways of the triangulations of
a polygon or the number of trees with \(N\) leafs.

\[\mathfrak{C}(N) = \binom{2N}{N} - \binom{2N}{N + 1}\]

or

\[\mathfrak{C}(N) = \sum_{k=1}^{n}(k - 1) \mathfrak{C}(N - k)\]

\subsection{Pascals Triangle}

This is a triangle build by the formula \(\binom{n + 1}{k + 1} = \binom{n}{k}\binom{n}{k + 1}\)
\begin{center}
    
\begin{tabular}{>{$n=}l<{$\hspace{12pt}}*{13}{c}}
0 &&&&&&&1&&&&&&\\
1 &&&&&&1&&1&&&&&\\
2 &&&&&1&&2&&1&&&&\\
3 &&&&1&&3&&3&&1&&&\\
4 &&&1&&4&&6&&4&&1&&\\
5 &&1&&5&&10&&10&&5&&1&\\
6 &1&&6&&15&&20&&15&&6&&1
\end{tabular}

\vspace{1cm}

Here the triangle but with the Binomial Coefficients
\smallskip

\begin{tabular}{>{$n=}l<{$\hspace{12pt}}*{13}{c}}
0 &&&&&&&\(\binom{0}{0}\)&&&&&&\\
1 &&&&&&\(\binom{1}{0}\)&&\(\binom{1}{1}\)&&&&&\\
2 &&&&&\(\binom{2}{0}\)&&\(\binom{2}{1}\)&&\(\binom{2}{2}\)&&&&\\
3 &&&&\(\binom{3}{0}\)&&\(\binom{3}{1}\)&&\(\binom{3}{2}\)&&\(\binom{3}{3}\)&&&\\
\end{tabular}
\end{center}

It can be used with the Binomial Theorem to get the Coefficients of a polynomial of degree \(n\).

Example:

\[(a + b)^2 = 1 a^2b^0 + 2a^1b^1 + 1a^0b^2\]

or

\[(a - b)^2 = 1 a^2b^0 - 2a^1b^1 + 1a^0b^2\]
