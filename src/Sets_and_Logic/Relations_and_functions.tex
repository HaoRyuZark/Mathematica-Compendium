\newpage
\section{Relations, Maps and Functions}

A \emph{relation} in mathematics is a connection or relationship between elements of two sets. It's formally
defined as a subset of the Cartesian product of the sets.
\vspace{\baselineskip}

For example, if we have sets \(A\) and \(B\), a relation \(R\) from \(A\) to \(B\) consists
of ordered pairs \((a,b)\) where \(a \in A\) and \(b \in B\), such that \(a\) is related to \(b\) according to some rule or property.
\vspace{\baselineskip}

Common types of relations include:
\begin{itemize}
	\item Functions (special relations where each input has exactly one output)
	\item Equivalence relations (reflexive, symmetric, and transitive)
	\item Partial orders (reflexive, antisymmetric, and transitive)
\end{itemize}

Relations can be represented using diagrams, matrices, or sets of
ordered pairs, and they're fundamental to many areas of mathematics including algebra, calculus, and discrete mathematics.

\subsection{Types of relations}

Let \(A\) be a set and \(X\) be a relation on \(A\).

\begin{itemize}
	\item \emph{Reflexive:} \(\forall a \in A: (a, a) \in X\) (or written as \(a \sim a\))

	\item \emph{Irreflexive:} \(\forall a \in A: (a, a) \not\in X\)

	\item \emph{Symmetric:} \(\forall a, b \in A: (a, b) \in X \to (b, a) \in X\) (or written as \((a \sim b) \Rightarrow (b \sim a)\))

	\item \emph{Antisymmetric:} \(\forall a, b \in A: (a, b) \in X \text{ and } (b, a) \in X \to a = b\)

	\item \emph{Transitive:} \(\forall a, b, c \in A: (a, b) \in X \text{ and } (b, c) \in X \to (a, c) \in X\) (or written as \((a \sim b) \text{ and } (b \sim c) \Rightarrow (a \sim c)\))

	\item \emph{Total:} \(\forall a,b \in A: a \neq b \to (a, b) \in X \text{ or } (b, a) \in X\)
\end{itemize}

\subsection{Equivalence relation}

An \emph{equivalence relation} is a relation that is symmetric, transitive and reflexive.
\vspace{\baselineskip}

\textbf{Example:}
\vspace{\baselineskip}

\[
	R:= \{ (a,b) \in \Naturals \times \Naturals: a = b\}
\]

\subsection{The Graph}

\[
	\text{graph}(f):= \{(x, f(x)): x \in X\}
\]

\subsection{The identity}

\[
	\text{id}(f):= idx:= \{(x, x): x \in X\}
\]

\subsection{Image}

The \emph{image (or range)} of a relation \(R\) from set \(A\) to set \(B\) is the set of all elements in \(B\) that are related to at least one element in \(A\). Formally, if \(R \subseteq A \times B\) is a relation, then the image of \(R\) is defined as:

\[
	\text{Im}(R) = \{b \in B \mid \exists a \in A \text{ such that } (a,b) \in R\}
\]

In other words, the image consists of all the output values that appear in the ordered pairs of the relation. For example, if \(R = \{(1,4), (2,5), (3,4), (2,6)\}\), then \(\text{Im}(R) = \{4, 5, 6\}\).

\subsection{Domain}

The \emph{domain} of a relation \(R\) from set \(A\) to set \(B\) is the set of all elements in \(A\) that are related to at least one element in \(B\). Formally, if \(R \subseteq A \times B\) is a relation, then the domain of \(R\) is defined as:

\[
	\text{Dom}(R) = \{a \in A \mid \exists b \in B \text{ such that } (a,b) \in R\}
\]

In other words, the domain consists of all the input values that appear in the ordered pairs of the relation. For example, if \(R = \{(1,4), (2,5), (3,4), (2,6)\}\), then \(\text{Dom}(R) = \{1, 2, 3\}\).

\subsection{Equivalence Class}

Let \(\sim\) be an equivalence relation on a set \(A\). For an element \(x \in A\), the \emph{equivalence class} of \(x\), denoted by \([x]\), is the set of all elements in \(A\) that are equivalent to \(x\). Formally, it is defined as:

\[
	[x] := \{y \in A \mid x \sim y\} \subseteq A
\]

In other words, the equivalence class of \(x\) contains all elements \(y\) in \(A\) such that \(x\) is related to \(y\) under the equivalence relation \(\sim\).

\subsection{Quotient Space}

Let \(\sim\) be an equivalence relation on a set \(A\). The \emph{quotient space} of \(A\) by \(\sim\), denoted by \(A/\sim\) (or sometimes \(A/R\)), is the set of all distinct equivalence classes of elements in \(A\). Formally, it is defined as:

\[
	A/\sim := \{[x] \mid x \in A\}
\]

The quotient space \(A/\sim\) is a partition of the original set \(A\) into disjoint equivalence classes. Each element of the quotient space is an equivalence class \([x]\), which itself is a subset of \(A\).

\subsection{Definition of a Map}

A map (or function) from a set \(A\) to a set \(B\), denoted as \(f: A \to B\), is a relation that associates each element of the set \(A\) with exactly one element of the set \(B\).

Formally, a function \(f: A \to B\) is a subset of \(A \times B\) such that for every \(a \in A\), there exists exactly one \(b \in B\) where \((a,b) \in f\). We typically write \(f(a) = b\) to indicate that \(f\) maps the element \(a\) to the element \(b\).
\vspace{\baselineskip}

\textbf{Example:}
\vspace{\baselineskip}

Let \(A = \{1, 2, 3\}\) and \(B = \{x, y, z\}\). A possible function \(f: A \to B\) could be defined as:

\[
	f(1) = x, \quad f(2) = y, \quad f(3) = z
\]

This function can also be represented as the set of ordered pairs \(\{(1,x), (2,y), (3,z)\}\).

\subsection{Composition of Maps}

The composition of two functions is the operation of applying one function to the result of another function. If we have functions 
\(f: A \to B\) and \(g: B \rightarrow C\), then the composition of \(g\) and \(f\), denoted as \(g \circ f\) (read as g composed with f), is a function from \(A\) to \(C\) defined by:

\[
	(g \circ f)(a) = g(f(a)) \quad \text{for all } a \in A
\]

The composition applies \(f\) first, then applies \(g\) to the result. 
Note that the co-domain of \(f\) must match the domain of \(g\) for the composition to be defined.
\vspace{\baselineskip}

\textbf{Example:}
\vspace{\baselineskip}

Let \(f: \Reals \to \Reals\) be defined by \(f(x) = x^2\) and \(g: \Reals \rightarrow \Reals\) be defined by \(g(x) = x+3\). Then:

\[
	(g \circ f)(x) = g(f(x)) = g(x^2) = x^2 + 3
\]

\[
	(f \circ g)(x) = f(g(x)) = f(x+3) = {(x+3)}^2 = x^2 + 6x + 9
\]

Note that \(g \circ f \neq f \circ g\) in general, which shows that function composition is not commutative.

\subsection{Types of Functions}

\subsubsection{Injective Functions}

An injective function (also called a one-to-one function) is a 
function that maps distinct elements from the domain to distinct elements in the co-domain.

Formally, a function \(f: A \to B\) is injective if for all \(a_1, a_2 \in A\):

\[
	a_1 \neq a_2 \to f(a_1) \neq f(a_2)
\]

Equivalently, using the contrapositive:

\[
	f(a_1) = f(a_2) \to a_1 = a_2
\]

\subsubsection{Surjective Functions}

A surjective function (also called an onto function) is a function whose image equals its co-domain, meaning that every element in the 
co-domain has at least one pre-image in the domain.

Formally, a function \(f: A \to B\) is surjective if:

\[
	\forall b \in B, \exists a \in A \text{ such that } f(a) = b
\]

\subsubsection{Bijection Functions}

A bijective function (also called a one-to-one correspondence) is a 
function that is both injective and surjective. In other words, every
 element in the co-domain is mapped to by exactly one element in the domain.

Formally, a function \(f: A \to B\) is bijective if it is both:

\begin{itemize}
	\item Injective: \(\forall a_1, a_2 \in A, a_1 \neq a_2 \to f(a_1) \neq f(a_2)\)
	\item Surjective: \(\forall b \in B, \exists a \in A \text{ such that } f(a) = b\)
\end{itemize}

Bijective functions establish a perfect pairing between elements of the domain and co-domain, where each element in the domain 
corresponds to exactly one element in the co-domain, and vice versa. A bijection allows us to define an inverse function \(f^{-1}: B \to A\).

\subsection{Propositions on Images and Pre-images under Set Operations}

Let \( f : X \to Y \) be a function.

\subsubsection{Union and Cut Sets}

\begin{itemize}
	\item For subsets \( A_1, A_2 \subseteq X \), we have:
	      \[
		      f(A_1 \cup A_2) = f(A_1) \cup f(A_2)
		      \quad \text{and} \quad
		      f(A_1 \cap A_2) \subseteq f(A_1) \cap f(A_2)
	      \]

	\item For subsets \( B_1, B_2 \subseteq Y \), we have:
	      \[
		      f^{-1}(B_1 \cup B_2) = f^{-1}(B_1) \cup f^{-1}(B_2)
		      \quad \text{and} \quad
		      f^{-1}(B_1 \cap B_2) = f^{-1}(B_1) \cap f^{-1}(B_2)
	      \]
\end{itemize}

\textbf{Proof:}

We prove the second part of 1 and the first part of 2.
\vspace{\baselineskip}

Let \( y \in f(A_1 \cap A_2) \). Then there exists \( x \in A_1 \cap A_2 \) such that 
\( f(x) = y \). Since \( x \in A_1 \) and \( x \in A_2 \), it follows that \( y \in f(A_1) \) 
and \( y \in f(A_2) \), hence \( y \in f(A_1) \cap f(A_2) \). Therefore, every element of \( f(A_1 \cap A_2) \) 
is also an element of \( f(A_1) \cap f(A_2) \), so:

\[
	f(A_1 \cap A_2) \subseteq f(A_1) \cap f(A_2)
 \]

Let \( x \in f^{-1} (B_1 \cup B_2) \). Then \( f(x) \in B_1 \cup B_2 \), which means \( f(x) \in B_1 \) 
or \( f(x) \in B_2 \). Thus, \( x \in f^{-1}(B_1) \) 
or \( x \in f^{-1}(B_2) \), which implies:
	      
\[
	 x \in f^{-1}(B_1) \cup f^{-1}(B_2)
\]

Hence, both sets contain the same elements and are therefore, equal:
	      
\[
	f^{-1}(B_1 \cup B_2) = f^{-1}(B_1) \cup f^{-1}(B_2)
\]

\QED

\subsubsection{Union and Cut of the whole Domain and Range}

Let \( f : X \to Y \) be a function.

\begin{itemize}
	\item Let \( \mathcal{F} \) be a collection of subsets of \( X \). Then:
	      \[
		      f\left( \bigcup_{A \in \mathcal{F}} A \right) = \bigcup_{A \in \mathcal{F}} f(A)
		      \quad \text{and} \quad
		      f\left( \bigcap_{A \in \mathcal{F}} A \right) \subseteq \bigcap_{A \in \mathcal{F}} f(A)
	      \]

	\item Let \( \mathcal{G} \) be a collection of subsets of \( Y \). Then:
	      \[
		      f^{-1}\left( \bigcup_{B \in \mathcal{G}} B \right) = \bigcup_{B \in \mathcal{G}} f^{-1}(B)
		      \quad \text{and} \quad
		      f^{-1}\left( \bigcap_{B \in \mathcal{G}} B \right) = \bigcap_{B \in \mathcal{G}} f^{-1}(B)
	      \]
\end{itemize}

\textbf{Proof (partial):}

We show the first statement of part (ii); the rest follows analogously.

Let \( x \in f^{-1} \left( \bigcup_{B \in \mathcal{G}} B \right) \). Then:

\[
	f(x) \in \bigcup_{B \in \mathcal{G}} B
	\quad \Leftrightarrow \quad
	\exists B \in \mathcal{G} \text{ such that } f(x) \in B
	\quad \Leftrightarrow \quad
	\exists B \in \mathcal{G} \text{ such that } x \in f^{-1}(B)
\]

Hence:

\[
	x \in \bigcup_{B \in \mathcal{G}} f^{-1}(B)
\]

It follows that:

\[
	f^{-1} \left( \bigcup_{B \in \mathcal{G}} B \right) = \bigcup_{B \in \mathcal{G}} f^{-1}(B)
\]

\QED

\subsection{Inverse of a Function}

The \emph{inverse} of a function \(f: A \to B\) is a function \(f^{-1}: B \to A\) that 
reverses the operation of \(f\). That is, if \(f\) maps an element \(a \in A\) to an 
element \(b \in B\), then the inverse function \(f^{-1}\) maps \(b\) back to \(a\).

Formally, a function \(f: A \to B\) has an inverse \(f^{-1}: B \to A\)  
if and only if \(f\) is bijective (both injective and surjective). The inverse function satisfies the following properties:

\[
	f^{-1}(f(a)) = a \quad \text{for all } a \in A
\]

\[
	f(f^{-1}(b)) = b \quad \text{for all } b \in B
\]

In other words, composing a function with its inverse yields the identity function. That is:

\[
	f^{-1} \circ f = id_A \quad \text{and} \quad f \circ f^{-1} = id_B
\]

Where \(id_A\) and \(id_B\) are the identity functions on sets \(A\) and \(B\), respectively.

\subsubsection{Steps to Find the Inverse of a Function}

To find the inverse of a function \(f (x)\), follow these steps:

\begin{enumerate}
	\item Replace \(f (x)\) with \(y\): \(y = f (x)\)
	\item Interchange the variables \(x\) and \(y\): \(x = f (y)\)
	\item Solve for \(y\) in terms of \(x\): \(y = f^{-1} (x)\)
	\item Verify that the resulting function is indeed the inverse by checking that \(f^{-1}(f(x)) = x\) and \(f (f^{-1} (x)) = x\)
\end{enumerate}

\textbf{Example:}
\vspace{\baselineskip}

Let's apply the steps above to find the inverse of \(f (x) = 2x + 3\).
\vspace{\baselineskip}

\textbf{Step 1: Replace \(f (x)\) with \(y\)}

\[
	y = 2x + 3
\]

\textbf{Step 2: Interchange the variables \(x\) and \(y\)}

\[
	x = 2y + 3
\]

\textbf{Step 3: Solve for \(y\) in terms of \(x\)}

\begin{align*}
	x               & = 2y + 3 \\
	x - 3           & = 2y     \\
	\frac{x - 3}{2} & = y
\end{align*}

So, the inverse function is:

\[
	f^{-1}(x) = \frac{x - 3}{2}
\]

\textbf{Step 4: Verify that \(f^{-1} (f (x)) = x\) and \(f(f^{-1} (x)) = x\)}

Let's verify \(f^{-1} (f (x)) = x\):

\begin{align*}
	f^{-1}(f(x)) & = f^{-1}(2x + 3)         \\
	             & = \frac{(2x + 3) - 3}{2} \\
	             & = \frac{2x}{2}           \\
	             & = x
\end{align*}

And let's verify \(f(f^{-1}(x)) = x\):

\begin{align*}
	f(f^{-1}(x)) & = f\left(\frac{x - 3}{2}\right)     \\
	             & = 2\left(\frac{x - 3}{2}\right) + 3 \\
	             & = (x - 3) + 3                       \\
	             & = x
\end{align*}

Since both compositions yield the identity function, \(f^{-1} (x) = \frac{x - 3}{2}\) is indeed the inverse of \(f (x) = 2x + 3\).

\subsubsection{Properties of the Inverse Function}

Let \(f:X\to Y\) be a function.

\begin{itemize}

	\item Assume \(x \sim y \) if \(f(x)= f(y)\) so is \(\sim\) an equivalence relation.

	\item Consider the quotient set \( X_f := X/\sim \), where \( \sim \) is the equivalence relation defined by \( x \sim x' \iff f(x) = f(x') \). Let \( q_f : X \to X_f \) be the canonical projection defined by \( q_f(x) = [x] \), and let \( \iota_f : f(X) \to Y \) be the inclusion map, \( y \mapsto y \). Then the function
	      \[
		      \hat{f} : X_f \to f(X), \quad [x] \mapsto f(x)
	      \]
	      is a bijection, and the original map \( f \) can be written as the composition
	      \[
		      f = \iota_f \circ \hat{f} \circ q_f.
	      \]
	      Here, \( q_f \) is surjective, \( \hat{f} \) is bijective, and \( \iota_f \) is injective. This yields the following commutative diagram:
	      
		  \begin{center}
				\begin{tikzcd}
					X \arrow[r, "f"] \arrow[d, "q_f"] & Y                         \\
					X_f \arrow[r, "\hat{f}"]          & f(X) \arrow[u, "\iota f"]
				\end{tikzcd}
	      \end{center}

\end{itemize}

\subsection{Transformations of a Function}

Transformations modify the appearance of a function's graph without altering its basic shape.
Here, we examine how different algebraic changes to a function \( f(x) \) affect its graph:
\vspace{\baselineskip}

\emph{Vertical Translation:} \( f(x) + a \)
	    
\begin{itemize}
	\item Shifts the graph \emph{upward} if \( a > 0 \), and \emph{downward} if \( a < 0 \).
	\item Each point on the graph moves vertically by \( a \) units.
\end{itemize}

\emph{Horizontal Translation:} \( f(x + a) \)

\begin{itemize}
	\item Shifts the graph \emph{left} if \( a > 0 \), and \emph{right} if \( a < 0 \).
	\item This is opposite of what might be expected: adding to \( x \) shifts the graph in the negative direction.
\end{itemize}

\emph{Vertical Scaling (Stretch/Compression):} \( a f(x) \)

\begin{itemize}
	\item If \( |a| > 1 \): the graph is \emph{stretched} vertically (taller and narrower).
	\item If \( 0 < |a| < 1 \): the graph is \emph{compressed} vertically (shorter and wider).
	\item If \( a < 0 \): includes a reflection across the \emph{x-axis}.
\end{itemize}

\emph{Horizontal Scaling (Stretch/Compression):} \( f(a x) \)

\begin{itemize}
	\item If \( |a| > 1 \): the graph is \emph{compressed} horizontally (narrower).
	\item If \( 0 < |a| < 1 \): the graph is \emph{stretched} horizontally (wider).
	\item If \( a < 0 \): includes a reflection across the \emph{y-axis}.
\end{itemize}

\emph{Reflection across the x-axis:} \( -f(x) \)
	
\begin{itemize}
	\item Flips the graph upside-down over the x-axis.
	\item Each point \( (x, y) \) becomes \( (x, -y) \).
\end{itemize}

\emph{Reflection across the y-axis:} \( f(-x) \)

\begin{itemize}
	\item Flips the graph left-to-right over the y-axis.
	\item Each point \( (x, y) \) becomes \( (-x, y) \).
\end{itemize}

