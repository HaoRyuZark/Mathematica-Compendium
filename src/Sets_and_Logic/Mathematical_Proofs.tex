\newpage
\section{Mathematical Proofs}

In this section I will provide with some examples of different types of proofs.

\subsection{Proof by Direct Argument}

For any integer \( n \), if \( n \) is even, then \( n^2 \) is even.
\vspace{\baselineskip}

We will prove this theorem by direct argument.
\vspace{\baselineskip}

Assume \( n \) is an even integer. Then we can write \( n = 2k \) for some integer \( k \).
\vspace{\baselineskip}

	Now, we compute \( n^2 \):
	\[
		n^2 = {(2k)}^2 = 4k^2 = 2(2k^2)
	\]
	This shows that \( n^2 \) is even, as it can be expressed as \( 2m \) where \( m = 2k^2 \) is an integer.
	Therefore, we conclude that if \( n \) is even, then \( n^2 \) is even.
\QED

\subsection{Proof by Contradiction}

If \( n \) is an integer such that \( n^2 \) is even, then \( n \) is even.
\vspace{\baselineskip}

We will prove this theorem by contradiction. Assume that \( n \) is an integer such that \( n^2 \) is even, but \( n \) is odd. Then we can write \( n = 2k + 1 \) for some integer \( k \).
\vspace{\baselineskip}

	Now, we compute \( n^2 \):
	\[
		n^2 = {(2k + 1)}^2 = 4k^2 + 4k + 1 = 2(2k^2 + 2k) + 1
	\]
	This shows that \( n^2 \) is odd, which contradicts our assumption that \( n^2 \) is even. Therefore, our assumption that \( n \) is odd must be false, and thus, \( n \) must be even.
\QED

\subsection{Proof by Induction}

For all \( n \in \mathbb{N} \), the sum of the first \( n \) positive integers is given by:
\[
	S(n) = 1 + 2 + 3 + \cdots + n = \frac{n(n+1)}{2}
\]
We will prove this theorem by induction on \( n \).
\vspace{\baselineskip}

	\textbf{Base Case:} For \( n = 1 \):
	\[
		S(1) = 1 = \frac{1(1+1)}{2}
	\]
	The base case holds.
\vspace{\baselineskip}

	\textbf{Inductive Step:} Assume that the statement holds for some \( n = k \), i.e., assume that:
	\[
		S(k) = 1 + 2 + 3 + \cdots + k = \frac{k(k+1)}{2}
	\]
	We need to show that the statement holds for \( n = k + 1 \):
	\[
		S(k+1) = S(k) + (k + 1)
	\]
	By the inductive hypothesis, we have:
	\[
		S(k+1) = \frac{k(k+1)}{2} + (k + 1)
	\]
	\[
		= \frac{k(k+1)}{2} + \frac{2(k + 1)}{2}
	\]
	\[      = \frac{k(k+1) + 2(k + 1)}{2}
	\]
	\[
		= \frac{(k + 1)(k + 2)}{2}
	\]
	Thus, the statement holds for \( n = k + 1 \).
	By the principle of mathematical induction, the statement holds for all \( n \in \mathbb{N} \).

\QED

\subsection{Proof by Exhaustion}

The only integer solutions to the equation \( x^2 + y^2 = 1 \) are \( (0, 1), (1, 0), (0, -1), (-1, 0) \).
\vspace{\baselineskip}

	We will prove this theorem by exhaustion. We will check all possible integer values of \( x \) and \( y \) such that \( x^2 + y^2 = 1 \).
\vspace{\baselineskip}

	The possible integer values for \( x \) and \( y \) are \( -1, 0, 1 \). We will check each case:
	\vspace{\baselineskip}

	If \( x = 0 \):
	\vspace{\baselineskip}

	Then \( y^2 = 1 \) gives \( y = 1 \) or \( y = -1 \).
	\vspace{\baselineskip}

	Solutions: \( (0, 1), (0, -1) \).
	\vspace{\baselineskip}

	If \( x = 1 \):
	\vspace{\baselineskip}

	Then \( y^2 = 0 \) gives \( y = 0 \).
	\vspace{\baselineskip}

	Solution: \( (1, 0) \).
	\vspace{\baselineskip}

	If \( x = -1 \):
	\vspace{\baselineskip}

	Then \( y^2 = 0 \) gives \( y = 0 \).
	\vspace{\baselineskip}

	Solution: \( (-1, 0) \).
\vspace{\baselineskip}

	Thus, the only integer solutions to the equation are:
	\[
		(0, 1), (1, 0), (0, -1), (-1, 0)
	\]

\QED

\subsection{Proof by Cases}

For any integer \( n \), \( n^2 \) is even if and only if \( n \) is even.
\vspace{\baselineskip}

	We will prove this theorem by cases.
\vspace{\baselineskip}

	\textbf{Case 1:} Assume \( n \) is even. Then we can write \( n = 2k \) for some integer \( k \).
\vspace{\baselineskip}

	Now, we compute \( n^2 \):
	\[
		n^2 = {(2k)}^2 = 4k^2 = 2(2k^2)
	\]
	This shows that \( n^2 \) is even.
\vspace{\baselineskip}

	\textbf{Case 2:} Assume \( n \) is odd. Then we can write \( n = 2k + 1 \) for some integer \( k \).
\vspace{\baselineskip}

	Now, we compute \( n^2 \):
	\[
		n^2 = {(2k + 1)}^2 = 4k^2 + 4k + 1 = 2(2k^2 + 2k) + 1
	\]
	This shows that \( n^2 \) is odd.
\vspace{\baselineskip}

	Since both cases have been considered, we conclude that \( n^2 \) is even if and only if \( n \) is even.
\QED

\subsection{Proof by Construction}

There exists an irrational number \( x \) such that \( x^2 \) is rational.
\vspace{\baselineskip}

	We will construct an irrational number \( x \) such that \( x^2 \) is rational.
\vspace{\baselineskip}

	Let \( x = \sqrt{2} \). We know that \( \sqrt{2} \) is irrational. Now, we compute \( x^2 \):
	\[
		x^2 = (\sqrt{2})^2 = 2
	\]
	Since \( 2 \) is a rational number, we have constructed an irrational number \( x = \sqrt{2} \) such that \( x^2 = 2 \) is rational.
	Therefore, the theorem is proved.

\QED

\subsection{Proof by Counterexample}

The statement all prime numbers are odd is false.
\vspace{\baselineskip}

	To prove this theorem, we will provide a counterexample.
\vspace{\baselineskip}

	The number \( 2 \) is a prime number, as its only divisors are \( 1 \) and \( 2 \). However, \( 2 \) is even, which contradicts the statement that all prime numbers are odd.
\vspace{\baselineskip}

	Therefore, the statement All prime numbers are odd is false.
\QED

\subsection{Proof by Contrapositive}

If \( n \) is an integer such that \( n^2 \) is odd, then \( n \) is odd.
\vspace{\baselineskip}

	We will prove this theorem by contrapositive. The contrapositive of the statement is: If \( n \) is an integer such that \( n \) is even, then \( n^2 \) is even.
\vspace{\baselineskip}

	Assume \( n \) is even. Then we can write \( n = 2k \) for some integer \( k \).
\vspace{\baselineskip}

	Now, we compute \( n^2 \):
	\[
		n^2 = {(2k)}^2 = 4k^2 = 2(2k^2)
	\]
	This shows that \( n^2 \) is even.
\vspace{\baselineskip}

	Since the contrapositive statement is true, the original statement If \( n^2 \) is odd, then \( n \) is odd is also true.
\QED

\subsection{Proof by Reduction to Absurdity}

The square root of \( 2 \) is irrational.
\vspace{\baselineskip}

	We will prove this theorem by reduction to absurdity. Assume that \( \sqrt{2} \) is rational. Then we can write:
	\[
		\sqrt{2} = \frac{p}{q}
	\]
	where \( p \) and \( q \) are integers with no common factors (i.e., the fraction is in the simplest form).
\vspace{\baselineskip}

	Squaring both sides gives:
	\[
		2 = \frac{p^2}{q^2}
	\]
	Rearranging gives:
	\[
		p^2 = 2q^2
	\]
	This implies that \( p^2 \) is even, and therefore, \( p \) must be even (since the square of an odd number is odd).
	Let \( p = 2k \) for some integer \( k \). Substituting this back into the equation gives:
	\[
		{(2k)}^2 = 2q^2
	\]
	\[
		4k^2 = 2q^2
	\]
	\[
		2k^2 = q^2
	\]
	This implies that \( q^2 \) is even, and therefore, \( q \) must also be even.
	Since both \( p \) and \( q \) are even, they have a common factor of \( 2 \), which contradicts our assumption that \( p \) and \( q \) have no common factors.
	Therefore, our assumption that \( \sqrt{2} \) is rational must be false, and thus, \( \sqrt{2} \) is irrational.

\QED

\subsection{Proof by Analogy}

The set of rational numbers is dense in the set of real numbers.

We will prove this theorem by analogy.
\vspace{\baselineskip}
	
Consider the set of rational numbers \( \mathbb{Q} \) and the set of real numbers \( \mathbb{R} \). The density of \( \mathbb{Q} \) in \( \mathbb{R} \) means that between any two real numbers, there exists a rational number.
For example, between the real numbers \( 1 \) and \( 2 \), we can find the rational number \( \frac{3}{2} = 1.5 \). Similarly, between any two real numbers \( a \) and \( b \) (where \( a < b \)), we can find a rational number \( r = \frac{a + b}{2} \).
\vspace{\baselineskip}

This shows that the set of rational numbers is dense in the set of real numbers.
Therefore, the theorem is proved by analogy.

\QED


