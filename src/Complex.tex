\section{Complex Numbers}

\subsection*{What are Complex Numbers?}

A complex number is a number of the form:
\[
z = a + bi,
\]
where \( a, b \in \mathbb{R} \), and \( i \) is the imaginary unit defined by \( i^2 = -1 \). The set of all complex numbers is denoted by \( \mathbb{C} \).

\subsection*{The Complex Plane}

Complex numbers can be represented graphically in the \textbf{complex plane}, where the horizontal axis represents the real part and the vertical axis the imaginary part.

\begin{center}
\setlength{\unitlength}{0.8cm}
\begin{picture}(6,6)
  \put(0,3){\vector(1,0){6}} 
  \put(3,0){\vector(0,1){6}} 
  \put(6.2,3){\makebox(0,0){Re}} 
  \put(3,6.2){\makebox(0,0){Im}} 
  \put(3,3){\circle*{0.15}}
  \put(4,4){\circle*{0.2}} 
  \put(4.2,4.2){\makebox(0,0){$1+i$}} 
  \put(3,3){\line(1,1){1}}
\end{picture}
\end{center}

The point \( 1+i \) is located at (1,1), showing 1 unit on the real axis and 1 unit on the imaginary axis.

\subsection*{Operations in Cartesian Coordinates}

Let \( z_1 = a + bi \) and \( z_2 = c + di \) be two complex numbers.

\begin{itemize}
  \item \textbf{Addition:} 
  \[
  z_1 + z_2 = (a + c) + (b + d)i
  \]
  
  \item \textbf{Multiplication:}
  \[
  z_1 \cdot z_2 = (ac - bd) + (ad + bc)i
  \]
  
  \item \textbf{Quotient:}
  \[
  \frac{z_1}{z_2} = \frac{(a + bi)}{(c + di)} \frac{(c - di)}{(c - di)} = \frac{(a + bi)(c - di)}{c^2 + d^2} = \frac{(ac + bd) + (bc - ad)i}{c^2 + d^2}
  \]
\end{itemize}

\subsection*{Polar Coordinates}

A complex number can also be expressed in polar form as:
\[
z = r(\cos \theta + i \sin \theta) = re^{i\theta},
\]
where:
\begin{align*}
r &= |z| = \sqrt{a^2 + b^2} \quad \text{(modulus)} \\
\theta &= \arg(z) = \tan^{-1}\left(\frac{b}{a}\right) \quad \text{(argument)}
\end{align*}

\textbf{Important:} The value of \( \theta \) depends on the quadrant where the complex number lies:
\begin{itemize}
  \item Quadrant I: \( a > 0, b > 0 \) — use \( \tan^{-1}(b/a) \)
  \item Quadrant II: \( a < 0, b > 0 \) — add \( \pi \) to \( \tan^{-1}(b/a) \)
  \item Quadrant III: \( a < 0, b < 0 \) — add \( \pi \) to \( \tan^{-1}(b/a) \)
  \item Quadrant IV: \( a > 0, b < 0 \) — use \( \tan^{-1}(b/a) \)
\end{itemize}

\subsection*{Multiplication and Division in Polar Coordinates}

Given:
\[
z_1 = r_1 e^{i\theta_1}, \quad z_2 = r_2 e^{i\theta_2},
\]

\begin{itemize}
  \item \textbf{Multiplication:}
  \[
  z_1 \cdot z_2 = r_1 r_2 e^{i(\theta_1 + \theta_2)}
  \]

  \item \textbf{Division:}
  \[
  \frac{z_1}{z_2} = \frac{r_1}{r_2} e^{i(\theta_1 - \theta_2)}
  \]
\end{itemize}

This polar form is especially useful in simplifying powers and roots of complex numbers using De Moivre’s Theorem.

\subsection*{Exponentiation and Roots (De Moivre's Theorem)}

Let \( z = r(\cos \theta + i \sin \theta) = re^{i\theta} \) be a complex number in polar form.

\subsubsection*{Exponentiation}

To raise \( z \) to the power \( n \in \mathbb{N} \), we use De Moivre’s Theorem:
\[
z^n = r^n (\cos(n\theta) + i \sin(n\theta)) = r^n e^{in\theta}
\]

\subsubsection*{Roots of Complex Numbers}

To find the \( n \)th roots of a complex number \( z = r e^{i\theta} \), we use the formula:
\[
z^{1/n} = r^{1/n} \left( \cos\left( \frac{\theta + 2k\pi}{n} \right) + i \sin\left( \frac{\theta + 2k\pi}{n} \right) \right), \quad k = 0, 1, \ldots, n-1
\]

This yields \( n \) distinct roots, each separated by an angle of \( \frac{2\pi}{n} \) in the complex plane.

\subsection*{Example: Solve \( z^4 = 1 + \sqrt{3}i \)}

\textbf{Step 1: Convert RHS to polar form.}

Let \( w = 1 + \sqrt{3}i \).  
Real part: \( a = 1 \), Imaginary part: \( b = \sqrt{3} \)

\begin{align*}
r &= |w| = \sqrt{1^2 + (\sqrt{3})^2} = \sqrt{1 + 3} = 2 \\
\theta &= \arg(w) = \tan^{-1} \left( \frac{\sqrt{3}}{1} \right) = \frac{\pi}{3}
\end{align*}

So,
\[
w = 2 \left( \cos\left( \frac{\pi}{3} \right) + i \sin\left( \frac{\pi}{3} \right) \right)
\]

\textbf{Step 2: Solve \( z^4 = w \Rightarrow z = w^{1/4} \)}

Using the root formula:
\[
z_k = 2^{1/4} \left( \cos\left( \frac{\pi + 2k\pi}{12} \right) + i \sin\left( \frac{\pi + 2k\pi}{12} \right) \right), \quad k = 0, 1, 2, 3
\]

So the four roots are:
\begin{align*}
z_0 &= 2^{1/4} \left( \cos\left( \frac{\pi}{12} \right) + i \sin\left( \frac{\pi}{12} \right) \right) \\
z_1 &= 2^{1/4} \left( \cos\left( \frac{5\pi}{12} \right) + i \sin\left( \frac{5\pi}{12} \right) \right) \\
z_2 &= 2^{1/4} \left( \cos\left( \frac{9\pi}{12} \right) + i \sin\left( \frac{9\pi}{12} \right) \right) = 2^{1/4} \left( \cos\left( \frac{3\pi}{4} \right) + i \sin\left( \frac{3\pi}{4} \right) \right) \\
z_3 &= 2^{1/4} \left( \cos\left( \frac{13\pi}{12} \right) + i \sin\left( \frac{13\pi}{12} \right) \right)
\end{align*}

These represent the four complex 4th roots of \( 1 + \sqrt{3}i \), equally spaced around the circle of radius \( 2^{1/4} \) in the complex plane.