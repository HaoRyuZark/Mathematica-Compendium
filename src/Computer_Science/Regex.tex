\newpage
\section{Regex}

\emph{Regex} is powerful tool for searching pattern inside a text which can be represented as state 
machines. In which our regular expression defines the transitions between the states.

\subsection{Basic Characters}

\begin{itemize}
  
    \item \texttt{a} : Matches character 'a' exactly.
  
    \item \texttt{abc} : Matches the string "abc" exactly.

\end{itemize}

\subsection{Meta Characters (Special Characters)}

\begin{itemize}

    \item \texttt{.} : Matches any single character except a newline.

    \item \texttt{\textbackslash} : Escape character (used to escape special characters, e.g., \texttt{\textbackslash.} to match a literal period).

    \item \texttt{\^} : Matches the start of a string.

    \item \texttt{\$} : Matches the end of a string.

    \item \texttt{|} : Alternation (logical OR, e.g., \texttt{cat|dog} matches either "cat" or "dog").

\end{itemize}

\subsection{Character Classes}

\begin{itemize}

    \item \texttt{[abc]} : Matches any one of 'a', 'b', or 'c'.
    
    \item \texttt{[\^}\texttt{abc]} : Matches any character except 'a', 'b', or 'c'.

    \item \texttt{[a-z]} : Matches any lowercase letter.

    \item \texttt{[A-Z]} : Matches any uppercase letter.

    \item \texttt{[0-9]} : Matches any digit.

    \item \texttt{[a-zA-Z0-9]} : Matches any alphanumeric character.

    \item \texttt{.} : Matches any character except newline.

    \item \texttt{\textbackslash d} : Matches any digit (equivalent to \texttt{[0-9]}).

    \item \texttt{\textbackslash D} : Matches any non-digit character.

    \item \texttt{\textbackslash w} : Matches any word character (letters, digits, underscore).

    \item \texttt{\textbackslash W} : Matches any non-word character.

    \item \texttt{\textbackslash s} : Matches any whitespace character (spaces, tabs, newlines).

    \item \texttt{\textbackslash S} : Matches any non-whitespace character.

\end{itemize}

\subsection{Quantifiers}

\begin{itemize}

    \item \texttt{*} : Matches 0 or more occurrences of the preceding character (e.g., \texttt{a*} matches "", "a", "aa", etc.).

    \item \texttt{+} : Matches 1 or more occurrences (e.g., \texttt{a+} matches "a", "aa", etc.).

    \item \texttt{?} : Matches 0 or 1 occurrence (e.g., \texttt{a?} matches "" or "a").

    \item \texttt{\{n\}} : Matches exactly \texttt{n} occurrences (e.g., \texttt{a\{3\}} matches "aaa").

    \item \texttt{\{n,\}} : Matches at least \texttt{n} occurrences (e.g., \texttt{a\{2,\}} matches "aa", "aaa", etc.).

    \item \texttt{\{n,m\}} : Matches between \texttt{n} and \texttt{m} occurrences (e.g., \texttt{a\{2,4\}} matches "aa", "aaa", or "aaaa").

\end{itemize}

\subsection{Grouping and Capturing}

\begin{itemize}

    \item \texttt{(abc)} : Capturing group, treats "abc" as a single unit.

    \item \texttt{(?:abc)} : Non-capturing group, groups "abc" without capturing.

    \item \texttt{\textbackslash n} : Refers to the \emph{n}th captured group (e.g., \texttt{(\textbackslash w)\textbackslash 1} matches doubled letters like "oo").

\end{itemize}

\subsection{Assertions and Anchors}

\begin{itemize}

    \item \texttt{\^} : Matches the start of a string.

    \item \texttt{\$} : Matches the end of a string.

    \item \texttt{\textbackslash b} : Matches a word boundary (e.g., \texttt{\textbackslash bword\textbackslash b} matches "word" but not "sword").

    \item \texttt{\textbackslash B} : Matches a non-word boundary.

    \item \texttt{(?=...)} : Positive look ahead (e.g., \texttt{a(?=b)} matches 'a' only if followed by 'b').

    \item \texttt{(?!...)} : Negative look ahead (e.g., \texttt{a(?!b)} matches 'a' only if NOT followed by 'b').

    \item \texttt{(?<=...)} : Positive look behind (e.g., \texttt{(?<=a)b} matches 'b' only if preceded by 'a').

    \item \texttt{(?<!...)} : Negative look behind (e.g., \texttt{(?<!a)b} matches 'b' only if NOT preceded by 'a').

\end{itemize}

\subsection{Advanced Constructs}

\begin{itemize}

    \item \texttt{(?(condition)yes|no)} : Conditional expression.

    \item \texttt{(?P<name>...)} : Named capturing group (e.g., \texttt{(?P<word>\textbackslash w+)}).

    \item \texttt{(?P=name)} : Refers to a named capturing group.

    \item \texttt{(?\#...)} : Comment within regex for readability.

    \item \texttt{(*UTF8)} : Enforces UTF-8 mode for international text processing.

    \item \texttt{(?>...)} : Atomic group (prevents backtracking within the group).

\end{itemize}

\subsection{Examples}

\textbf{Example: Email validation}

\begin{verbatim}
    ^[a-zA-Z0-9._%+-]+@[a-zA-Z0-9.-]+\.[a-zA-Z]{2,}$
\end{verbatim}

Matches valid email addresses.

\textbf{Example: Phone number validation}

\begin{verbatim}
    ^\(?\d{3}\)?[-.]?\d{3}[-.]?\d{4}$
\end{verbatim}

Matches US phone numbers like (123) 456-7890, 123-456-7890, or 123.456.7890.

\textbf{Example: URL validation}

\begin{verbatim}
    ^https?:\/\/(www\.)?[a-zA-Z0-9.-]+\.[a-zA-Z]{2,}(/\S*)?$
\end{verbatim}

Matches URLs like http://example.com or https://www.example.com/path.

\textbf{Example: Hex color code validation}

\begin{verbatim}
    ^#([A-Fa-f0-9]{6}|[A-Fa-f0-9]{3})$
\end{verbatim}

Matches hex color codes like \#FFF or \#A1B2C3.

\textbf{Example: Date format (YYYY-MM-DD) validation}

\begin{verbatim}
    ^\d{4}-\d{2}-\d{2}$
\end{verbatim}

Match dates formatted as 2024-03-01.
