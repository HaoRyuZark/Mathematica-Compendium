\newpage
\section{State Machines and Languages}

\subsection{Languages and Alphabets}

A \emph{language} \(L\) is a subset of words of an \emph{alphabet} \(\Sigma\) which is a finite 
non-empty set of symbols which always includes the \(\varepsilon\)-word, the empty word.

\subsection{Pumping Lemma}

For a language \(L\) which can be recognized by a state machine, there is a natural number \(n\) such 
that for each word in \(L\) with length greater than or equal to \(n\), the word \(w\) can be decomposed 
into three subwords \(x, y, z\).

\begin{itemize}
    
    \item \(|xy| \leq n\)
    
    \item \(y \neq \varepsilon\)
    
    \item \(xy^i z \in L\) for all \(i \geq 0\)

\end{itemize}

\textbf{Example:}

Let \(L = \{ a^n b^n \mid n \geq 0 \}\). This language is \textbf{not} regular, and the Pumping 
Lemma can be used to prove this by contradiction.

\subsection{State Machines}

A \emph{state machine} (or \emph{finite state machine}, FSM) is a computational model used to 
design algorithms, systems, or software by breaking them down into a finite number of \emph{states}, 
\emph{transitions}, \emph{inputs}, and \emph{outputs}. It behaves according to a current state and can 
transition to other states based on input conditions.

State machines are widely used in fields like software engineering, digital electronics, linguistics, 
and artificial intelligence to model dynamic systems.

\subsection{Types of State Machines}

\subsubsection{Deterministic Finite Automaton (DFA)}  

\emph{DFAs} are defined by a quintuple \((Q, \Sigma, \delta, q_0, F)\) which consists of \(Q\) states, 
an input alphabet \(\Sigma\), a transition function \(\delta: Q \times \Sigma \to Q\), a start state \(q_0 \in Q\) and a set of 
terminations states \(F \subset Q\).

This kind of automata has the particularity that each state has \emph{exactly one transition} for 
each possible input. Therefore, \emph{deterministic}.

\textbf{Example:}

\[
    \begin{array}{|c|c|c|}
    \hline
    \textbf{State} & a & b \\
    \hline
    q_0 & q_1 & q_0 \\
    q_1 & q_2 & q_1 \\
    q_2 & q_3 & q_0 \\
    q_3 & q_3 & q_3 \\
    \hline
    \end{array}
\]
\begin{center}
\begin{tikzpicture}[shorten >=1pt, node distance=2.5cm, on grid, auto]
  \node[state, initial] (q0) {\(q_0\)};
  \node[state] (q1) [right of=q0] {\(q_1\)};
  \node[state] (q2) [right of=q1] {\(q_2\)};
  \node[state, accepting] (q3) [right of=q2] {\(q_3\)};

  \path[->]
    (q0) edge [bend left] node {a} (q1)
         edge [loop above] node {b} ()
    (q1) edge [bend left] node {a} (q2)
         edge [loop above] node {b} ()
    (q2) edge [bend left] node {a} (q3)
         edge [bend left=40] node[below] {b} (q0)
    (q3) edge [loop above] node {a,b} ();
\end{tikzpicture}
\end{center}

Here our language is 

\[
    L = \{ w \in \{a,b\}^* \mid \text{the sequence of transitions on } w \text{ leads to state } q_3 \}
\]

\subsubsection{Non-Deterministic Finite Automaton (NFA)}  

An NFA is a state machine defined by a 5-tuple \((Q, \Sigma, \delta, q_0, F)\), where: 
\(Q\) is a finite set of states, \(\Sigma\) is the input alphabet, 
\(\delta: Q \times \Sigma \to \mathcal{P}(Q)\) is the transition function (mapping to a set of states),
\(q_0 \in Q\) is the start state and \(F \subset Q\) is the set of accepting states.

Unlike a DFA, an NFA allows:
\begin{itemize}
    
    \item Multiple possible next states for a given input.
    
    \item No transition at all.

\end{itemize}

\textbf{Example:}

\[
    \begin{array}{|c|c|c|}
    \hline
    \textbf{State} & a & b \\
    \hline
    q_0 & \{q_0, q_1\} & \{q_0\} \\
    q_1 & \{q_2\} & \{q_1\} \\
    q_2 & \{q_3\} & \emptyset \\
    q_3 & \{q_3\} & \{q_3\} \\
    \hline
    \end{array}
\]

\begin{center}
\begin{tikzpicture}[shorten >=1pt, node distance=2.5cm, on grid, auto]
  \node[state, initial] (q0) {\(q_0\)};
  \node[state] (q1) [right of=q0] {\(q_1\)};
  \node[state] (q2) [right of=q1] {\(q_2\)};
  \node[state, accepting] (q3) [right of=q2] {\(q_3\)};

  \path[->]
    (q0) edge [bend left] node {a} (q1)
         edge [loop above] node {a,b} ()
    (q1) edge [bend left] node {a} (q2)
         edge [loop above] node {b} ()
    (q2) edge [bend left] node {a} (q3)
    (q3) edge [loop above] node {a,b} ();
\end{tikzpicture}
\end{center}

Here our language is still:

\[
    L = \{ w \in \{a,b\}^* \mid \text{some computation path leads to } q_3 \}
\]

\subsubsection{Epsilon-NFA \texorpdfstring{(\(\varepsilon\)-NFA)}{}}  

An \(\varepsilon\)-NFA is a nondeterministic finite automaton that allows transitions on the empty string 
\(\varepsilon\), meaning the automaton can change states without consuming input.

It is defined as a 5-tuple \((Q, \Sigma, \delta, q_0, F)\), with the transition function:

\[
    \delta: Q \times (\Sigma \cup \{\varepsilon\}) \to \mathcal{P}(Q)
\]

This means the automaton have the following properties:

\begin{itemize}
    
    \item It can transition normally on \(a \in \Sigma\) or move to other states "for free" via 
    \(\varepsilon\)-transitions.
    
    \item \emph{Epsilon-Closure:} It is the set of all states reachable via \(\varepsilon\) of a particular 
    state plus itself. 

\end{itemize}

\textbf{Example:}

\[
    \begin{array}{|c|c|c|c|}
    \hline
    \textbf{State} & a & b & \varepsilon \\
    \hline
    q_0 & \emptyset & \{q_0\} & \{q_1\} \\
    q_1 & \{q_2\} & \emptyset & \emptyset \\
    q_2 & \emptyset & \{q_3\} & \emptyset \\
    q_3 & \{q_3\} & \{q_3\} & \emptyset \\
    \hline
    \end{array}
\]

\begin{center}
\begin{tikzpicture}[shorten >=1pt, node distance=2.5cm, on grid, auto]
  \node[state, initial] (q0) {\(q_0\)};
  \node[state] (q1) [right of=q0] {\(q_1\)};
  \node[state] (q2) [right of=q1] {\(q_2\)};
  \node[state, accepting] (q3) [right of=q2] {\(q_3\)};

  \path[->]
    (q0) edge [loop above] node {b} ()
         edge [bend left] node {\(\varepsilon\)} (q1)
    (q1) edge [bend left] node {a} (q2)
    (q2) edge [bend left] node {b} (q3)
    (q3) edge [loop above] node {a,b} ();
\end{tikzpicture}
\end{center}

The language is:

\[
    L = \{ w \in \{a,b\}^* \mid w \text{ contains the substring } ab \}
\]

via the path:

\[
    q_0 \xrightarrow{\varepsilon} q_1 \xrightarrow{a} q_2 \xrightarrow{b} q_3
\]

\subsubsection{Use Cases of State Machines}

\begin{itemize}

    \item \emph{UI Design}: Modeling different screens and user interactions.
    
    \item \emph{Networking}: Protocol handling (TCP/IP, HTTP states).
    
    \item \emph{Game Development}: Character states like walking, jumping, attacking.
    
    \item \emph{Parsing and Compilers}: Lexical analysis using DFAs and NFAs.
    
    \item \emph{Robotics}: Movement planning and task sequencing.
    
    \item \emph{Embedded Systems}: Control logic in devices (e.g., washing machines).

\end{itemize}

\subsection{Regular Expressions}

Regular expressions (regex) are a practical application of finite automata (DFA/NFA). They define 
search patterns for strings and are processed internally using state machines.

\textbf{Example:}

Regex: \texttt{a(b|c)*d}

\begin{itemize}
    
    \item \emph{Start State}: Awaiting input.
    
    \item \emph{Transition on 'a'}: Move to a state expecting \( b \) or \( c \).
    
    \item \emph{Loop State}: Repeats \( b \) or \( c \) any number of times.
    
    \item \emph{Final Transition}: Ends with a \( d \).

\end{itemize}

Regular expressions describe sets of strings using a small set of operations:

\begin{itemize}
    \item \textbf{Concatenation \((R.S)\)}: A string from \(R\) followed by a string from \(S\).
    \item \textbf{Union \((R + S)\)}: A string from either \(R\) or \(S\).
    \item \textbf{Kleene star \((R^*)\)}: Zero or more repetitions of strings from \(R\).
\end{itemize}

Special symbols:

\begin{itemize}
    \item \(\emptyset\): Matches no string.
    \item \(\varepsilon\): Matches the empty string.
    \item Literal characters (e.g., \(a, b, c\)): Match themselves.
\end{itemize}

Below are \(\varepsilon\)-NFAs (nondeterministic automata with \(\varepsilon\)-transitions) for some 
regular expressions:

\textbf{R + E}

\begin{center}
\begin{tikzpicture}[->, >=stealth', shorten >=1pt, node distance=2cm, on grid, auto]
    \node[state, initial] (q0) {};
    \node[state] (q1) [above right=of q0] {};
    \node[state] (q2) [below right=of q0] {};
    \node[state, accepting] (qf) [right=2.5cm of q0] {};

    \path (q0) edge node {\(\varepsilon\)} (q1)
               edge node[below] {\(\varepsilon\)} (q2)
          (q1) edge node {\(R\)} (qf)
          (q2) edge node {\(E\)} (qf);

\end{tikzpicture}
\end{center}

\textbf{R.E}

\begin{center}
\begin{tikzpicture}[->, >=stealth', shorten >=1pt, node distance=2cm, on grid, auto]
    \node[state, initial] (q0) {};
    \node[state] (q1) [right=of q0] {};
    \node[state] (q2) [right=of q1] {};
    \node[state, accepting] (q3) [right=of q2] {};

    \path (q0) edge node {\(R\)} (q1)
          (q1) edge node {\(\varepsilon\)} (q2)
          (q2) edge node {\(E\)} (q3);

\end{tikzpicture}
\end{center}

\textbf{R*}

\begin{center}
\begin{tikzpicture}[->, >=stealth', shorten >=1pt, node distance=2cm, on grid, auto]
    \node[state, initial] (q0) {};
    \node[state] (q1) [right=of q0] {};
    \node[state] (q2) [right=of q1] {};
    \node[state, accepting] (qf) [right=of q2] {};

    \path (q0) edge node {\(\varepsilon\)} (q1)
          (q1) edge node {\(R\)} (q2)
          (q2) edge node {\(\varepsilon\)} (qf)
               edge[bend left] node {\(\varepsilon\)} (q1)
          (q0) edge[bend right] node[below] {\(\varepsilon\)} (qf);

\end{tikzpicture}
\end{center}

\textbf{Example: }\texttt{(a+b)*}

\begin{center}
\begin{tikzpicture}[->, >=stealth', shorten >=1pt, node distance=2cm, on grid, auto]

    % States
    \node[state, initial] (q0) {};
    \node[state] (q1) [right=of q0] {};
    \node[state] (q2a) [above right=of q1] {};
    \node[state] (q2b) [below right=of q1] {};
    \node[state] (q3a) [right=of q2a] {};
    \node[state] (q3b) [right=of q2b] {};
    \node[state] (q4) [right=of q1, xshift=4cm] {};

    \node[state, accepting] (qf) [right=of q4] {};

    % Transitions
    \path (q0) edge node {\(\varepsilon\)} (q1)
          (q1) edge node {\(\varepsilon\)} (q2a)
                edge node[swap] {\(\varepsilon\)} (q2b)
          (q2a) edge node {\(a\)} (q3a)
          (q2b) edge node {\(b\)} (q3b)
          (q3a) edge node {\(\varepsilon\)} (q4)
          (q3b) edge node {\(\varepsilon\)} (q4)
          (q4) edge[bend right=95] node[above] {\(\varepsilon\)} (q1)
          (q0) edge[bend right=50] node[below] {\(\varepsilon\)} (qf)
          (q4) edge node {\(\varepsilon\)} (qf);

\end{tikzpicture}
\end{center}


\subsection{State Machine Conversions and Minimization}

\subsubsection{Non-Deterministic Finite Automaton (NFA) to Deterministic Finite Automaton (DFA)}

The process for the conversion is the so called \emph{Power-Set-Construction-Method} or 
\emph{Subset-Construction-Method}. 

\textbf{Steps:}

\begin{enumerate}

    \item Write the transition table of the given automata.

    \item Write a similar table with the transition as columns. 

    \item Start with \(q_0\) and find which states can be reached with the transitions. 
    If from our current state we can reach more than one state with our current transition, write the 
    possible states as a set, then Proceed to write the set as the next entry to check 
    (write in the next row). 

    \item We terminate the algorithm when we after evaluating our states we only get states we have already 
    visited.

\end{enumerate}

\textbf{Example:}

Convert the NFA to a DFA 

\begin{center}
\begin{tikzpicture}[shorten >=1pt, node distance=2.5cm, on grid, auto]
  \node[state, initial] (q0) {\(q_0\)};
  \node[state] (q1) [right of=q0] {\(q_1\)};
  \node[state, accepting] (q2) [right of=q1] {\(q_2\)};

  \path[->]
    (q0) edge [loop above] node {0,1} ()
         edge [bend left] node {0} (q1)
    (q1) edge [bend left] node {1} (q2);
\end{tikzpicture}
\end{center}

\textbf{Step 1: Transition Table}

\[
    \begin{array}{|c|c|c|}
    \hline
    \textbf{State} & 0 & 1 \\
    \hline
    q_0 & \{q_0, q_1\} & q_0  \\
    q_1 & \emptyset & q_2\\
    q_2 & \emptyset & \emptyset \\
    \hline
    \end{array}
\]

\textbf{Step 2: Table of new states}

\[
    \begin{array}{|c|c|c|}
    \hline
    \textbf{State} & 0              & 1    \\
    \hline
    q_0            & \{q_0, q_1\} & q_0  \\
    \{q_0, q_1\}   & \{\{q_0, q_1\}, \emptyset \} & \{q_0, q_2\}  \\
    \{q_0, q_2\}   & \{\{q_0, q_1\}, \emptyset \} &  \{q_0, \emptyset \}\\
    \hline
    \end{array}
\]

\textbf{Step 3: Construct DFA}

\begin{center}
\begin{tikzpicture}[shorten >=1pt, node distance=3.2cm, on grid, auto]
  \node[state, initial] (q0) {\(q_0\)};
  \node[state] (r) [right of=q0] {\(\{q_0, q_1\}\)};
  \node[state, accepting] (p) [right of=r] {\(\{q_0, q_2\}\)};

  \path[->]
    (q0) edge [loop above] node {1} ()
         edge [bend left] node {0} (r)
    (r)  edge [loop above] node {0} ()
         edge [bend left] node {1} (p)
    (p)  edge [bend left] node {0} (r)
         edge [bend left=40] node[below] {1} (q0);
\end{tikzpicture}
\end{center}

\subsubsection{Epsilon-NFA to DFA}

The process is very similar to the previous one. The catch is that we have to keep in mind that 
our initial state is not only \(q_0\) but also all other from \(q_0\) reachable states via epsilon.

\textbf{Steps:}

\begin{enumerate}
    
    \item Write the transition table of the automata for all transition including epsilon.

    \item Like for the case \emph{NFA to DFA}, write the respective 
          table, but for the second one leave out the epsilon.

    \item For filling the entries, the start is going to be a tuple of 
          the start state plus all other states that are reachable via epsilon from the start, if they 
          exist.

    \item Start filling the table by evaluating the transitions like for the previously discussed type 
          conversion with new the states transitions. For each transition visited like from \(q_x\) to 
          \(q_x\) via "a" you would also have to include the states reachable from \(q_y\) with \(\varepsilon\) 
          to your set of reachable states.

\end{enumerate}

\textbf{Example:} 

Convert to DFA
\begin{figure}[H]
\centering
\begin{tikzpicture}[,scale=0.9, transform shape, shorten >=1pt, node distance=1.5cm and 1.5cm, on grid, auto]
  \node[state, initial] (p) {\(p\)};
  \node[state] (q) [above right=of p] {\(q\)};
  \node[state, accepting] (r) [below right=of p] {\(r\)};

  \path[->]
    (p) edge[bend left=10] node[above left] {\(\varepsilon\),b} (q)
        edge[bend right=10] node[below left] {\(\varepsilon\),c} (r)
    (q) edge[loop above] node {c} ()
        edge[bend left=15] node[right] {b} (r)
        edge[bend left=25] node[below right] {a, c} (p);
\end{tikzpicture}
\end{figure}

\textbf{Step 1: Transition Table}

\[
    \begin{array}{|c|c|c|c|c|}
    \hline
    \textbf{State} & a & b & c & \varepsilon \\
    \hline
    p & \emptyset & q & r & \{q, r\} \\
    q & p & r & p,q & \emptyset \\
    r & \emptyset & \emptyset & \emptyset & \emptyset \\
    \hline
    \end{array}
\]

\textbf{Step 2: Table for the DFA}


\[
    \begin{array}{|c|c|c|c|c|}
    \hline
    \textbf{State} & a & b & c \\
    \hline
    \{p,q,r\} & \{p,q,r\} & \{q,r\} & \{p,q,r\} \\
    \{q,r\}& \{p,q,r\} & r & \{p,q,r\} \\
     r       &  \emptyset  & \emptyset & \emptyset \\
    \hline
    \end{array}
\]

\textbf{Step 3: Construct the Automaton}

\begin{center}
\begin{tikzpicture}[->,>=stealth',shorten >=1pt, auto, node distance=3cm, on grid, semithick]
  \node[state, initial] (A) {\(\{p,q,r\}\)};
  \node[state] (B) [right=of A] {\(\{q,r\}\)};
  \node[state] (C) [right=of B] {\(\{r\}\)};

  \path[->]
    (A) edge[loop above] node {a,c} (A)
        edge[bend left] node {b} (B)
    (B) edge[bend left] node {a,c} (A)
        edge[bend left] node {b} (C)
    (C) edge[loop right] node {a,b,c} (C);
\end{tikzpicture}
\end{center}

\subsection{Epsilon-NFA to NFA}




\subsubsection{NFA to Regex}

Converting a regular expression into a NFA is not really difficult, but the other direction is not 
so trivial. The most common approach to tackle this type of conversion is via a system of equations.

You are given some NFA with \(n\) states and \(m\) transitions. Set your system as follows:

\begin{enumerate}

    \item Write an equation \(L_i\) for each state \(q_i\) 
    Each equation \(L_i\) represents the set of strings (language) accepted when starting from state 
    \(q_i\). These equations will be defined recursively in terms of transitions to other states.

    \item For each \(L_i\), describe all transitions from state \(q_i\):  
    For every possible outgoing transition from \(q_i\) to some \(q_j\) on symbol(s) \(w\), add a term 
    \(w L_j\) to the equation. If \(q_i\) is an accepting state, also include \(\varepsilon\) (the empty 
    string).

    \textbf{Example:} If from state 1 there is a transition \(a \to 1\) and \(b \to 2\), and state 1 is 
    accepting, then  
    
    \[
        L_1 = a L_1 \cup b L_2 \cup \varepsilon
    \]

    \item Rewrite the equations using regular expression notation  
    Replace set notation with regex syntax: \(\cup \to +\),  concatenation is written without operator 
    (e.g., \(a L_1\)) and replace sets of symbols (e.g., \(\{a, b\}\)) with grouped regex (e.g., \((a + b)\))

    \textbf{Example:}  
    \[
        L_1 = a L_1 + b L_2 + \varepsilon
    \]

    \item Substitute equations recursively to reduce the system. There are two cases:  
    
    \begin{itemize}
        
        \item If an equation \(L_i\) does \textbf{not} reference itself (i.e., no recursion), substitute 
        it into the other equations directly.
        
        \item If an equation \(L_i\) \textbf{does} reference itself (e.g., \(L_i = r L_i + s\)), apply 
        \textbf{Arden's Lemma}:  
        
        \[
            \text{If } L = YL \cup X, \text{ then } L = Y^*X
        \]
    
    \end{itemize}

    \textbf{Example:}  \begin{enumerate}
    
    \item Write for each state the equation \(L_i\).
    
    \item Now treat each state as a start state and look for the sets of words this state accepts and lead 
          to a terminating state then combine them with \(\cup\). If the current state is also a terminating 
          state then \(\varepsilon\) must also be added. Example: \(L_1 = \{ab\}L_1 \cup b L_2\).

          A trick is to write the table with the transition and for each row (state) you write 
          \(L_{row} = transition_1 L_{state_x} + transition_2 L_{state_y} + \cdots \). Where the transition is 
          a set of letters of the alphabet.
          
    \item Repeat step 2 for all \(L_i\) and them replace the set of words with regular expressions and the 
           \(\cup\) with \(+\) Example:\(L_1 = (a+b)L_1 + b L_2\).
    
    \item Start substituting equations into other equations. Here there are two cases. Either, the  
          equation \(L_x, \dots\) depends only on other equations different from itself or  \(L_x, \dots\) also depends on 
          itself which causes infinite recursion. For the first case the problem solves itself by just 
          doing the substituting until we have only one equation which is going to be our regex, but for the 
          other case we will use \emph{Arden's Lemma}. 

\end{enumerate}

    \[
        L = (a + b) L + b \Rightarrow L = (a + b)^* b
    \]

    Continue substituting until all equations are resolved. The expression for the initial state gives 
    the regular expression for the NFA.

\end{enumerate}


\subsubsection{Arden's Lemma}

Given \(X,Y\) as formal language with \(\varepsilon \notin Y\). It applies that:

\[
    L = YL \cup X \text{ has exactly one solution: } L = Y^*X
\]

And the special case \(X = \emptyset\)

\[
L = YL + \emptyset = Y^* \emptyset = \emptyset
\]

\textbf{Proof:}

To show \(L = Y^*X\) is a solution and more specific the only solution.

\begin{align*}
    L &= YL \cup X\\
    Y^*X &= YY^*X \cup X \text{ This shows that it is indeed a solution}
\end{align*}

For the second part we can argue that if \(L\) is a solution for the equation then 

\[
    Y^*X \subseteq L \implies X \cup YX \cup YYX \cup \dots \subseteq L
\]

Here we always took longer words which are also part of the language and united them. For the second 
part we are going to use shorter words.

Assume that

\[
    L \subsetneqq Y^*X 
\]

Then there is the shortest word \(w\) as the word in \(L\) which is not in \(Y^*X\). But 
because \(L = YL \cup X\) \(w\) must be either a word in \(X\) which is not possible because of 
\(w \notin Y^*X\); or \(YL\). This implies that \(w\) can be decomposed into \(uv\) with 
\(u \in Y, u \neq \varepsilon, v \in L\). But this is a contradiction then \(w\) would be a word of the 
form \(Y^*X\) which not true, and this would imply that there is an even shorter word for the shortest word 
and so on. Therefore 

\[
    L \subseteq Y^*X  
\]

\QED

\textbf{Example:}

Convert to Regex

\begin{center}
\begin{tikzpicture}[->,>=stealth',shorten >=1pt, auto, node distance=2.8cm and 2cm, on grid, semithick]

  % Nodes (properly arranged in a box shape)
  \node[state, initial] (A) {1};
  \node[state, accepting] (B) [right=of A] {2};
  \node[state, accepting] (C) [below=of A] {3};
  \node[state] (D) [right=of C] {4};

  % Transitions
  \path
    (A) edge[loop above] node {a,b} ()
        edge[bend left] node {a} (B)
        edge [bend right] node[left] {b} (C)
    (B) edge[loop above] node {b} ()
        edge[bend left] node {b} (D)
    (C) edge[bend right] node[left] {a,b} (A)
        edge[bend left] node[below] {a} (D)
    (D) edge[bend left] node[right] {a} (B);

\end{tikzpicture}
\end{center}

\textbf{Step 1: Initial Equations}

\begin{align*}
    L_1 &= \{a,b\}L_1 \cup aL_2 \cup bL_3\\
    L_2 &= \varepsilon \cup bL_2 \cup b L_4 \\
    L_3 &= \{a,b\}L1 \cup \varepsilon \cup aL4 \\
    L4  &= aL_2
\end{align*}

\textbf{Step 2: Regex Equations}

\begin{align*}
L_1 &= (a+b)L_1 + aL_2 +bL_3\\
L_2 &= \varepsilon + bL_2 + b L_4 \\
L_3 &= (a+b)L1 + \varepsilon + aL4 \\
L4  &= aL_2
\end{align*}

\textbf{Step 3: Substitution}

\(L_4\) in \(L_3\) and \(L_2\)

\[
    L_3 = (a+b)L1 + \varepsilon + aaL_2
\]

\[
    L_2 = \varepsilon + bL_2 + baL_2
\]

Now we can solve \(L_2\) using the lemma.

\begin{align*}
    L_2 &= \varepsilon + bL_2 + baL_2
        &= \varepsilon + (b + ba)L_2 \\
        &= (b + ba)^*\varepsilon \\
        &= (b + ba)^*
\end{align*}

\(L_2\) in \(L_3\) and \(L_1\) 

\begin{align*}
    L_1 &= (a+b)L_1 + a(b + ba)^* +bL_3\\
    L_3 &= (a+b)L_1 + \varepsilon + aa(b + ba)^* 
\end{align*}

\(L_3\) in  \(L_1\)

\begin{align*}
    L_1 &= (a+b)L_1 + a(b + ba)^* +b((a+b)L1 + \varepsilon + aa(b + ba)^*)\\
        &= (a+b)L_1 + a(b + ba)^* +b(a+b)bL1 + b + baa(b + ba)^*\\
        &= (a+b)L_1  + b(a+b)L_1 + a(b + ba)^* + b + baa(b + ba)^*\\
        &= ((a+b)  + b(a+b))L_1 + a(b + ba)^* + b + baa(b + ba)^*\\
        &= ((a+b)  + b(a+b))L_1 + a(b + ba)^* + b + baa(b + ba)^*\\
        &= ((a+b)  + b(a+b))^*(a(b + ba)^* + b + baa(b + ba)^*)
\end{align*}

This is the answer but we can simplify. Here \(((a+ b) + b(a+b))\) is a word of \((a+b)^*\)

\begin{align*}
    L_1 &= ((a+b)  + b(a + b)^*(a(b + ba)^* + b + baa(b + ba)^*) \\
        &= (a+b)^* (a(b + ba)^* + b + baa(b + ba)^*)
\end{align*}

\subsection{Minimizing State Machines}

Minimization aims to create an equivalent DFA with the \emph{fewest number of states}. One widely used 
method is the \emph{table-filling algorithm}.

\subsubsection{Table Method for DFA Minimization}

\textbf{Steps:}

\begin{enumerate}
    
    \item \emph{List all pairs of states}: Create a table of all possible state pairs \((p, q)\).

    \item \emph{Mark Distinguishable Pairs}: Initially mark pairs where one is accepting, and the other is not.

    \item \emph{Iteratively Mark Pairs}: For each unmarked pair \((p, q)\), if a transition on any symbol leads to a marked pair, mark \((p, q)\).

    \item \emph{Equivalent States}: Unmarked pairs are equivalent and can be merged.

    \item \emph{Reconstruct DFA}: Build a new DFA with merged equivalent states and updated transitions.

\end{enumerate}

\textbf{Result}: A minimized DFA that accepts the same language with optimal state count.

