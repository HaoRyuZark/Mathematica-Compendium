\newpage
\section{Random Access Machine}

The \emph{Random Access Machine}, \emph{RAM} for short, is a mathematical model used in the analysis of algorithm. 
It helps us tackle the following questions when dealing with algorithms.

\begin{itemize}
   
    \item \emph{Correctness}: Does the algorithm compute what we want. 

    \item \emph{Termination}: Does the algorithm ever ends. 

    \item \emph{Speed}: How long does the algorithm need to execute.

    \item \emph{Memory Consumption}: How much memory does the algorithm need. 

\end{itemize}

The RAM consists of:

\begin{itemize}

    \item Program memory (read-only access)
    
    \item Instruction counter
    
    \item Main memory (read and write)
    
    \item Memory cell 0 as accumulator
    
    \item Memory cells hold integers
    
        \begin{itemize}
        
            \item No size limitation!
        
        \end{itemize}
        
    \item Input and output tape
    
        \begin{itemize}
        
            \item Arbitrarily many integers
            
            \item No random access!
        \end{itemize}

\end{itemize}

\vspace{\baselinekip}
