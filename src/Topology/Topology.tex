\section{Topology}

In this section, we introduce essential vocabulary used in topology. Each term is accompanied by a brief explanation and its formal mathematical definition.

\subsection{Introduction to topological nomenclature}

	\subsubsection{Open Set}
	      A subset \( U \subseteq X \) of a topological space is called 
		  \emph{open} if for every point \( x \in U \), there exists an \( \varepsilon > 0 \) 
		  such that the open ball \( B_\varepsilon(x) \subseteq U \). 
	      Intuitively, an open set contains none of its boundary points and every point has some 
		  wiggle room around it.

	 \subsubsection{Closed Set} 
	      A subset \( A \subseteq X \) is called \emph{closed} if its 
		  complement \( X \setminus A \) is open. Equivalently, \( A \) contains all its limit points. 
	      That is, \( A \) is closed if it includes its boundary.

	 \subsubsection{Interior Point}
	      A point \( x \in A \) is an \emph{interior point} of \( A \subseteq X \) if there 
		  exists \( \varepsilon > 0 \) such that \( B_\varepsilon(x) \subseteq A \). 
	      The set of all interior points of \( A \) is called the \emph{interior} of \( A \), 
		  denoted \( \mathrm{int}(A) \).

	 \subsubsection{Boundary Point}
	      A point \( x \in X \) is a \emph{boundary point} of a set \( A \subseteq X \) 
		  if every open ball around \( x \) contains both points in \( A \) and in \( X \setminus A \). 
	      The set of all boundary points is called the \emph{boundary} of \( A \), denoted \( \partial A \).

	 \subsubsection{Accumulation Point / Limit Point}
		 A point \( x \in X \) is an \emph{accumulation point} of a set \( A \subseteq X \)
		  if every open ball \( B_\varepsilon(x) \) contains a point of \( A \setminus \{x\} \). 
	      In other words, points of \( A \) cluster arbitrarily close to 
		  \( x \), even if \( x \notin A \).

	 \subsubsection{Isolated Point} 
	      A point \( x \in A \) is an \emph{isolated point} if there exists \( \varepsilon > 0 \)
		   such that \( B_\varepsilon(x) \cap A = \{x\} \). 
	      That is, \( x \) stands alone in \( A \) without other points of \( A \) nearby.

	 \subsubsection{Compact Set} 
	      A set \( K \subseteq X \) is \emph{compact} if every open cover of \( K \) has a finite subcover. 
	      In \(\mathbb{R}^n\), this is equivalent to \( K \) being closed and bounded (by the Heine–Borel theorem).

	 \subsubsection{Dense Set} 
	      A subset \( D \subseteq X \) is \emph{dense} in \( X \) if every point 
		  \( x \in X \) is either in \( D \) or is a limit point of \( D \). 
	      Equivalently, the closure of \( D \) is \( X \), i.e., \( \overline{D} = X \).

	 \subsubsection{Open Ball} (\( B_\varepsilon(x) \)) 
	      For a metric space \( (X, d) \), the \emph{open ball} centered at 
		  \( x \in X \) with radius \( \varepsilon > 0 \) is defined as: 
	      \[
		      B_\varepsilon(x) := \{ y \in X \mid d(x, y) < \varepsilon \}
	      \]
	      It represents the set of all points within distance \( \varepsilon \) from \( x \), excluding the boundary.

\newpage